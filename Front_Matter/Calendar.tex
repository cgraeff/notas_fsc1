\thispagestyle{plain}
\begin{fullwidth}
\begin{center}
{\noindent\LARGE\textsc{Calendário}} \\
\end{center}
\end{fullwidth}

\vspace{3cm}

\begin{marginfigure}[5cm]
\centering
Agosto\\
\begin{tikzpicture}
\calendar (mycal) [dates=2017-08-01 to 2017-08-last,week list] if (Saturday,Sunday) [gray];
\end{tikzpicture}
\end{marginfigure}
\begin{marginfigure}
\centering
Setembro\\
\begin{tikzpicture}
\calendar (mycal) [dates=2017-09-01 to 2017-09-last,week list] if (Saturday,Sunday) [gray];
\end{tikzpicture}
\end{marginfigure}
\begin{marginfigure}
\centering
Outubro\\
\begin{tikzpicture}
\calendar (mycal) [dates=2017-10-01 to 2017-10-last,week list] if (Saturday,Sunday) [gray];
\end{tikzpicture}
\end{marginfigure}
\begin{marginfigure}
\centering
Novembro\\
\begin{tikzpicture}
\calendar (mycal) [dates=2017-11-01 to 2017-11-last,week list] if (Saturday,Sunday) [gray];
\end{tikzpicture}
\end{marginfigure}
\begin{marginfigure}
\centering
Dezembro\\
\begin{tikzpicture}
\calendar (mycal) [dates=2017-12-01 to 2017-12-last,week list] if (Saturday,Sunday) [gray];
\end{tikzpicture}
\end{marginfigure}
%\draw[dotted] (mycal-2017-12-13) circle (7pt);

As aulas seguirão o planejamento abaixo. No calendário ao lado, estão circuladas as datas das provas.
\begin{center}
\begin{longtable}{ccp{70mm}}
\toprule
Aula & Data & Conteúdo \\
\midrule
\endhead
\bottomrule
\endfoot
1 &  06/03 & Apresentação da Disciplina: Informações de contato, horários de atendimento, conteúdo, métodos de ensino e avaliação, cronograma e plano de ensino. \\
\\
2 & 08/03 & Movimento Unidimensional: Posição e Deslocamento (Posição, Deslocamento, Deslocamento escalar, Posição como função do tempo), Velocidade (Velocidade média, Velocidade instantânea, Velocidades escalares média e instantânea), Aceleração (Aceleração média, Aceleração instantânea), Sentido dos eixos de referência e sinais das variáveis cinemáticas, Interpretação da área de gráficos (v × t, a × t) \\
\\
3 & 13/03 & Movimento Unidimensional: Movimentos com aceleração constante (Equações cinemáticas para movimentos com aceleração constante: Evolução temporal da velocidade, Evolução temporal da posição, Equação de Torricelli, Variáveis ausentes em cada equação), Aceleração da gravidade. \\
\\
4 & 15/03 & Vetores: Vetores e escalares, Representação geométrica de um vetor, Operações envolvendo vetores (Soma, Subtração), Outras propriedades, Sistemas de referência (bases) (Componentes vetoriais, Notação módulo-ângulo, Soma através de componentes, Vetores unitários, Equações e vetores unitários) \\
\\
5 & 20/03 & Movimento Bi e Tridimensional: Vetores posição e deslocamento (Posição, Deslocamento), Velocidade (Velocidade média, Velocidade instantânea), Aceleração (Aceleração média, Aceleração instantânea), Independência do movimento em eixos perpendiculares, Movimento de projéteis (Eixo x: Movimento com velocidade constante, Eixo y: Movimento com aceleração constante, Altura máxima, Alcance horizontal, Equação para a trajetória) \\
\\
6 & 22/03 & Movimento Bi e Tridimensional: Movimento circular (Aceleração centrípeta, Decomposição da aceleração em componentes tangencial e centrípeta)\\
\\
7 & 27/03 & Movimento Bi e Tridimensional: Movimento Relativo (pode ser agrupada com a aula anterior em caso de semana acadêmica) \\
\\
8 & 29/03 & Revisão para prova \\
\\
9 & 03/04 & Prova 1 \\
\\
10 & 05/04 & Dinâmica da partícula: Conceitos de força e massa, Princípio da Inércia segundo Galileu e segundo Newton, Segunda Lei de Newton (Relação entre força e aceleração, Relação entre massa e aceleração, Medidas de massa), Terceira Lei de Newton, Forças (Força gravitacional e força peso, Normal, Tensão) \\
\\
11 & 10/04 & Dinâmica da partícula: Forças (Atrito, Arrasto, Força elástica) \\
\\
12 & 12/04 & Dinâmica da partícula: Forças no movimento circular \\
\\
13 & 17/04 & Trabalho e Energia Cinética: Teorema Trabalho-Energia (Trabalho realizado pela força peso, Trabalho realizado por forças de atrito e arrasto, Trabalho de um conjunto de forças e trabalho em uma situação de equilíbrio) \\
\\
14 & 19/04 & Trabalho e Energia Cinética: Trabalho como a área de um gráfico F × x (Trabalho realizado por uma força elástica), Trabalho como a integral da força (Teorema fundamental do cálculo), Potência, Energia cinética e referenciais inerciais \\
15 & 24/04 & Revisão para prova \\
\\
16 & 26/04 & Prova 2 \\
\\
17 & 03/05 & Energia potencial e Energia Mecânica: Potencial (Energia potencial gravitacional, Energia potencial elástica, Potencial e trabalho, Energia mecânica), Condições para a existência de um potencial, Cálculo da força a partir de um potencial, Dependência da energia na escolha do referencial \\
\\
18 & 08/05 & Energia potencial e Energia Mecânica: Curvas de potencial (Pontos de equilíbrio e retorno, Potencial interatômico, Potencial de Woods-Saxon) \\
\\
19 & 10/05 & Energia potencial e Energia Mecânica: Atrito e Trabalho de forças externas, Conservação da energia \\
\\
20 & 15/05 & Primeira Recuperação: Substitutiva das provas 1 ou 2. Os alunos que optarem por não realizar a prova estão dispensados. \\
\\
21 & 17/05 & Centro de Massa: Centro de Massa (Torque, Centro de massa de um conjunto de partículas), Centro de Massa de um corpo extenso (Discretização de um corpo extenso, Distribuição contínua de massa, Técnicas de simetria) \\
\\
22 & 22/05 & Momento Linear: Segunda Lei de Newton para o Centro de Massa, Momento Linear e conservação do momento linear. Impulso. \\
\\
23 & 24/05 & Momento Linear: Forças durante uma colisão, Momento e energia cinética em colisões (Colisões inelásticas, Colisões elásticas) \\
\\
24 & 29/05 & Revisão para prova \\
\\
25 & 31/05 & Prova 3 \\
\\
26 & 05/06 & Rotações: Cinemática da Rotação (Variáveis cinemáticas para rotações, Sinais, Equações para aceleração angular constante, Relação entre variáveis de translação e de rotação) \\
\\
27 & 07/06 & Rotações: Dinâmica da rotação (Torque, Segunda lei de Newton para as rotações), Trabalho e energia cinética para rotações (Energia cinética de rotação, Teorema trabalho-energia para rotações) \\
\\
28 & 12/06 & Rotações: Cálculo do momento de inércia (Momento de inércia de um sistema de partículas, Propriedades do momento de inércia, Discretização, Momento de inércia de uma distribuição contínua, Teorema dos eixos paralelos). \\
\\
29 & 14/06 & Rotações: Rolamento (Características do rolamento, Forças no rolamento, Energia cinética no rolamento) \\
\\
30 & 19/06 & Momento angular: Caráter vetorial das variáveis da rotação, Velocidade e aceleração, Torque como o produto vetorial r × F, Momento angular e Segunda Lei de Newton, Momento angular para uma partícula que se desloca em linha reta, Momento angular de um sistema de partículas, Momento angular de um corpo rígido \\
\\
31 & 21/06 & Momento angular:  Conservação do momento angular \\
\\
32 & 26/06 & Momento angular: Precessão de um giroscópio (pode ser agrupada com a aula seguinte para acomodar a semana acadêmica) \\
\\
33 & 28/06 & Revisão para prova 4. \\
\\
34 & 03/07 & Prova 4 \\
\\
35 & 05/07 & Entrega das notas finais (sem recuperação). \\
\\
36 & 10/07 & Segunda Recuperação: Substitutiva das provas 3 ou 4. Os alunos que optarem por não realizar a prova estão dispensados. \\
\end{longtable}
\end{center}
\cleardoublepage
