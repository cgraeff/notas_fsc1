{\noindent\LARGE\textsc{Mar português}} \\

{ \it
\noindent{}Ó mar salgado, quanto do teu sal \\
São lágrimas de Portugal! \\
Por te cruzarmos, quantas mães choraram, \\
Quantos filhos em vão rezaram! \\
Quantas noivas ficaram por casar \\
Para que fosses nosso, ó mar! \\

\noindent{}Valeu a pena? Tudo vale a pena \\
Se a alma não é pequena. \\
Quem quer passar além do Bojador \\
Tem que passar além da dor. \\
Deus ao mar o perigo e o abismo deu, \\
Mas nele é que espelhou o céu. \\
}

\noindent{}Fernando Pessoa, em ``Mensagem''

\thispagestyle{empty}
\clearpage

\begin{fullwidth}
{\noindent\LARGE\textsc{Liberdade}} \\

{\it
\noindent{}Ai que prazer \\
Não cumprir um dever, \\
Ter um livro para ler \\
E não o fazer! \\
Ler é maçada, \\
Estudar é nada. \\
Sol doira \\
Sem literatura \\
O rio corre, bem ou mal, \\
Sem edição original. \\
E a brisa, essa, \\
De tão naturalmente matinal, \\
Como o tempo não tem pressa... \\

\noindent{}Livros são papéis pintados com tinta. \\
Estudar é uma coisa em que está indistinta \\
A distinção entre nada e coisa nenhuma. \\

\noindent{}Quanto é melhor, quanto há bruma, \\
Esperar por D.Sebastião, \\
Quer venha ou não! \\

\noindent{}Grande é a poesia, a bondade e as danças... \\
Mas o melhor do mundo são as crianças, \\

\noindent{}Flores, música, o luar, e o sol, que peca \\
Só quando, em vez de criar, seca. \\

\noindent{}Mais que isto \\
É Jesus Cristo, \\
Que não sabia nada de finanças \\
Nem consta que tivesse biblioteca... \\
}

\noindent{}Fernando Pessoa, em ``Cancioneiro''

\end{fullwidth}

\thispagestyle{empty}
\clearpage
