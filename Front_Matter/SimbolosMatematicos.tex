\thispagestyle{plain}
\begin{fullwidth}
\begin{center}
{\noindent\LARGE\textsc{Símbolos matemáticos}} \\
\end{center}
\end{fullwidth}

\begin{table*}[!ht]
\centering
\begin{tabular}{ccc}
\toprule
Símbolo & Significado & Exemplo\\
\midrule
$\equiv$ & definido como & $\vec{p} \equiv m \vec{v}$ \\
$\approx$ & aproximadamente & $g \approx \np[m/s^2]{9,8}$\\
$\propto$ & proporcional a & $a \propto F$ \\
$\sim$ & da ordem de & $G \sim 10^{-11}$ \\
$\geqslant$ & maior ou igual a & $a \geqslant b$ \\
$\leqslant$ & menor ou igual a & $b \leqslant a$ \\
$\gg$ & muito maior que & $a \gg b$ \\
$\ll$ & muito menor que & $a \ll c$ \\
\\
$\Delta$ & variação & $\Delta t$ \\
\\
$\vec{~}$ & vetor & $\vec{a}$ \\
$|~|$ & módulo, norma & $|-5| = 5$, $|\vec{a}| = \sqrt{a_x^2 + a_y^2}$
\\
$\cdot$ & produto, produto escalar & $a \cdot b$, $\vec{F}\cdot\vec{d}$ \\
$\times$ & produto, produto escalar & $a \times b$, $\vec{r}\times\vec{F}$ \\
\\
$\therefore$ & portanto & $\vec{F}_R = 0$, $\therefore \vec{a} = 0$\\
$\because$ & pois & $\vec{a} = 0$, $\because \vec{F}_R = 0$\\
\\
$\Rightarrow$ & implica & \\
$\to$ & tende a & $\Delta t \to 0$ \\
$\mapsto$ & mapeia uma variável em outra (def. de funções) & $y=f(x): x\mapsto y=x^2$ \\
\\
$\sum_{i=1}^n$ & somatório & $\sum_{i1}^{n} m_i = m_1 + m_2 + m_3 + \dots + m_n$ \\
$n!$ & fatorial & $n! = n \cdot (n-1) \cdot (n-2) \cdot \dots \cdot 2 \cdot 1$ \\
\\
$\frac{d}{dx}$ & derivada em relação a uma variável $x$ & $\frac{d}{dt} x(t)$ \\
$\int dx$ & integral na variável $x$ & $\int_a^b F(t)dt$ \\
\bottomrule
\end{tabular}
\end{table*}

\hfill
\pagebreak
