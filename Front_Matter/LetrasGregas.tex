\thispagestyle{plain}
\begin{fullwidth}
\begin{center}
{\noindent\LARGE\textsc{Letras gregas}} \\
\end{center}
\end{fullwidth}

\begin{table*}[!ht]
\centering
\begin{tabular}{cccc}
\toprule
Equações & Minúscula & Maiúscula & Nome \\
\midrule
$\alpha$ & \textalpha & \textAlpha & alfa \\
$\beta$ & \textbeta & Β & beta \\
$\gamma$, $\Gamma$ & \textgamma & \textGamma & gama \\
$\delta$, $\Delta$ & \textdelta & \textDelta & delta \\
$\epsilon$, $\varepsilon$ & \textepsilon & \textEpsilon & épsilon \\
$\zeta$ & \textzeta & \textZeta & zeta \\
$\eta$ & \texteta & \textEta & eta \\
$\theta$, $\vartheta$, $\Theta$ & \texttheta & \textTheta & téta \\
$\iota$ & \textiota & \textIota & iota \\
$\kappa$, $\varkappa$ & \textkappa & \textKappa & capa \\
$\lambda$, $\Lambda$ & \textlambda & \textLambda & lambda \\
$\mu$ & \textmugreek & \textMu & mi \\
$\nu$ & \textnu & \textNu & ni\\
$\xi$, $\Xi$ & \textxi & \textXi & csi \\
 & \textomikron & \textOmikron & ómicron \\
$\pi$, $\varpi$, $\Pi$ & \textpi & \textPi & pi \\
$\rho$, $\varrho$ & \textrho & \textRho & rô \\
$\sigma$, $\varsigma$, $\Sigma$ & \textsigma & \textSigma & sigma \\
$\tau$ & \texttau & \textTau & tau \\
$\upsilon$, $\Upsilon$ & \textupsilon & \textUpsilon & úpsilon \\
$\phi$, $\varphi$, $\Phi$ & \textphi & \textPhi & fi \\
$\chi$ & \textchi & X & qui \\ % por alguma razão, o \textChi não funciona, então copiei e colei um qui mesmo (eu acho)
$\psi$, $\Psi$ & \textpsi & \textPsi & psi \\
$\omega$, $\Omega$ & \textomega & \textOmega & ômega \\
% obsoletas
$\digamma$ &  &  & digama \\
\bottomrule
\end{tabular}
\caption[][5mm]{Note que muitas letras não são utilizadas em equações, pois são indistinguíveis de letras romanas.
}
\end{table*}

\clearpage
\thispagestyle{empty}
\selectlanguage{greek}
\noindent{}\Large{Ἰλιάς} \normalsize{} \\
\\
Mῆνιν ἄειδε θεὰ Πηληϊάδεω Ἀχιλῆος \\
οὐλομένην, ἣ μυρί᾽ Ἀχαιοῖς ἄλγε᾽ ἔθηκε, \\
πολλὰς δ᾽ ἰφθίμους ψυχὰς Ἄϊδι προΐαψεν \\
ἡρώων, αὐτοὺς δὲ ἑλώρια τεῦχε κύνεσσιν \\
οἰωνοῖσί τε πᾶσι, Διὸς δ᾽ ἐτελείετο βουλή, \\
ἐξ οὗ δὴ τὰ πρῶτα διαστήτην ἐρίσαντε \\
Ἀτρεΐδης τε ἄναξ ἀνδρῶν καὶ δῖος Ἀχιλλεύς.\selectlanguage{brazil}\footnote{\protect\url{https://www.perseus.tufts.edu/hopper/text?doc=Perseus:text:1999.01.0133}}

\vspace{2cm}
\noindent{}Ilíada\\
\\
Canta, ó deusa, a cólera de Aquiles, o Pelida \\
(mortífera!, que tantas dores trouxe aos Aqueus \\
e tantas almas valentes de heróis lançou ao Hades, \\
ficando seus corpos como presa para cães e aves \\
de rapina, enquanto se cumpria a vontade de Zeus), \\
desde o momento em que primeiro se desentenderam \\
o Atrida, soberano dos homens, e o divino Aquiles.\cite{Iliada}

\hfill
\pagebreak
