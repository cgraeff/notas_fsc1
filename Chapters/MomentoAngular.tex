%%%%%%%%%%%%%%%%%%%%%%%%%%%%%%%%%%%%%%%%%%%%%%%%%%%%%%%%%%%%
\chapter{Momento Angular}
\label{Chap:MomentoAngular}
%%%%%%%%%%%%%%%%%%%%%%%%%%%%%%%%%%%%%%%%%%%%%%%%%%%%%%%%%%%%
%\minitoc
%\clearpage

%%%%%%%%%%%%%%%%%%%%
\section{Introdução}
%%%%%%%%%%%%%%%%%%%%

{\it
Intro ...
}



%%%%%%%%%%%%%%%%%%%%%%%%%%%%%%%%%%%%%%%%%%%%%%%%%%%%%%%%%%%%
\section{Caráter vetorial das variáveis da rotação}
%%%%%%%%%%%%%%%%%%%%%%%%%%%%%%%%%%%%%%%%%%%%%%%%%%%%%%%%%%%%

\textbf{Agora começamos a considerar o que acontece quando estamos interessados em descrever uma rotação arbitrária no espaço, não uma rotação em torno de um eixo fixo.}

Não está claro pra mim qual é a de definir esses deslocamentos angulares infinitesimais como vetores. Serve só pra definir a velocidade?

%%%%%%%%%%%%%%%%%%%%%%%%%%%%%%%%%%%%
\subsection{Velocidade e aceleração}
%%%%%%%%%%%%%%%%%%%%%%%%%%%%%%%%%%%%

Quando discutimos as grandezas da translação, concluímos que posição, velocidade e aceleração eram grandezas vetoriais e tinham módulo, direção e sentido. Podemos atribuir um caráter vetorial à velocidade angular e à aceleração angular. Nesses casos, no entanto, a direção do vetor não nos dá a direção do movimento, mas a direção \emph{em torno} da qual o objeto gira.

Para definirmos tal direção de maneira única, utilizamos a regra da mão direita: ``seguramos'' o eixo em torno do qual o objeto gira de forma que os dedos (exceto o polegar) apontem no sentido de rotação. Fazendo isso, o polegar apontará na direção do vetor.

No caso da aceleração, apontamos a direção da variação da velocidade (na direção de $\vec{\omega}$ se o módulo da velocidade angular cresce e na direção contrária se o módulo decresce). Tanto $\vec{\omega}$ quanto $\vec{\alpha}$ obedecem a todos os requisitos para serem denominados vetores, inclusive à soma vetorial.

A posição e -- consequentemente -- o deslocamento angulares, no entanto, não podem ser tratados como vetores. Se tomarmos um livro e realizarmos dois deslocamentos angulares sucessivos de \degree{90} em torno dos eixos $x$ e $y$, a ordem em que eles forem realizados influenciará no resultado final, resultando em estados finais diferentes. Como a soma vetorial de $\vec{a} + \vec{b} = \vec{b} + \vec{a}$, percebemos que os deslocamentos angulares não podem ser tratados como vetores.

\comment{Pra mim isso não explica nada. Como podemos mostrar que duas velocidades angulares podem ser somadas? (Acho que o Teorema de Euler para rotação explica, mas como? ver isso e colocar essa explicação aqui). TODO Por que podemos tratar deslocamentos para pequenos ângulos como vetores e não para grandes ângulos? ver isso com cuidado}

%%%%%%%%%%%%%%%%%%%%%%%%%%%%%%%%%%%%%%%%%%%%%%%%%%%%%%%%%%%%%%%%%%
\subsection{Torque como o produto vetorial $\vec{r}\times\vec{F}$}
%%%%%%%%%%%%%%%%%%%%%%%%%%%%%%%%%%%%%%%%%%%%%%%%%%%%%%%%%%%%%%%%%%

Da mesma forma que $\vec{\omega}$ e $\vec{\alpha}$ são grandezas vetoriais, também é possível mostrar que o torque é uma grandeza vetorial. Analisando a expressão para o módulo do produto vetorial entre dois vetores $\vec{a}$ e $\vec{b}$:
\begin{equation}
  |\vec{a}\times\vec{b}| = ab\sen\phi,
\end{equation}
%
onde $\phi$ é o ângulo entre os dois vetores, e comparando-a com a Equação~\eqref{Eq:DefModTorque}, podemos escrever o torque como
\begin{align}
  |\vec{\tau}| &= F d \sen\phi \\
  &= |\vec{F}\times\vec{d}.
\end{align}
%
Na expressão acima, $\vec{F}$ é o vetor que descreve a força que gera o torque, enquanto $\vec{d}$ é o vetor que denota a posição do ponto onde a força é aplicada. A origem do vetor é o próprio ponto em torno do qual o objeto gira. Apesar de utilizarmos $\vec{d}$ até o momento, em geral posições são denotadas por $\vec{r}$. Assim, o torque pode ser definido como: \comment{a origem pode ser um ponto qualquer, porém quando formos tratar de corpos rígidos mais adiante, usaremos sempre um ponto no eixo de rotação ... como explicar isso direito?}
\begin{equation}\label{Eq:DefTorque}
  \vec{\tau} = \vec{r}\times\vec{F}.
\end{equation}

A direção do vetor torque pode ser facilmente compreendida ao se analisar a expressão para a Segunda Lei de Newton para a rotação,
\begin{equation}
  \tau = I \alpha.
\end{equation}
%
Sabendo que $I$ é uma grandeza escalar e que atribuimos um caráter vetorial para a aceleração angular, obrigatoriamente temos que o torque também tem uma caráter vetorial (pois uma das propriedades dos vetores é que a multiplicação de um escalar por um vetor resulta em um vetor). Assim, o torque assume a mesma direção que a aceleração angular:
\begin{equation}
  \vec{\tau} = I\vec{\alpha}.
\end{equation}
%
Dessa conclusão podemos tirar uma observação importante, justificando a escolha da ordem dos vetores na Equação~\eqref{Eq:DefTorque}: devido à regra da mão direita, se temos um eixo em torno do qual um objeto gira e a aceleração angular é positiva, o torque dado pelo produto vetorial~\eqref{Eq:DefTorque} acima deve correspondentemente ser positivo. Para isso, também devemos adotar a regra da mão direita para o produto vetorial. Analisando o diagrama ao lado, percebemos que a ordem do produto vetorial deve ser $\vec{r}\times\vec{F}$, caso contrário o sentido resultante para o torque seria oposto ao sentido da aceleração. Alternativamente podemos usar $\vec{\tau} = -\vec{F}\times\vec{r}$, já que $\vec{a}\times\vec{b} = - \vec{b}\times\vec{a}$.

%%%%%%%%%%%%%%%%%%%%%%%%%%%%%%%%%%%%%%%%%%%%%%%%%%%%%%
\section{Momento angular e Segunda Lei de Newton}
%%%%%%%%%%%%%%%%%%%%%%%%%%%%%%%%%%%%%%%%%%%%%%%%%%%%%%

Da mesma forma que temos o momento linear, para o caso das rotações temos o momento angular, definido como
\begin{equation}\label{Eq:DefMomAngular}
  \vec{\ell} = \vec{r}\times\vec{p},
\end{equation}
%
onde $r$ denota a posição de uma partícula qualquer e $p$ denota seu momento angular. Se derivarmos essa expressão em relação ao tempo, temos
\begin{align}
  \frac{d\vec{\ell}}{dt} &= \frac{d(\vec{r}\times\vec{p})}{dt} \\
  &= \frac{d\vec{r}}{dt}\times\vec{p} + \vec{r}\times\frac{d\vec{p}}{dt},
\end{align}
%
onde usamos a regra da cadeia. Notando que $d\vec{r}/dt = \vec{v}$, $\vec{p} = m \vec{v}$ e $d\vec{p}/dt = \vec{F}$, podemos escrever
\begin{equation}
  \frac{d\vec{\ell}}{dt} = m\vec{v}\times\vec{v} + \vec{r} \times \vec{F}.
\end{equation}
%
Finalmente, notando que o produto vetorial de dois vetores colineares é nulo, temos que $\vec{v}\times\vec{v} = 0$ e, portanto,
\begin{align}
  \frac{d\vec{\ell}}{dt} &= \vec{r} \times \vec{F} \\
  &= \vec{\tau}.
\end{align}

Mais uma vez obtivemos um resultado para o caso das rotações que tem um análogo no caso da translação: a equação acima mostra que a taxa de variação do momento angular no tempo é igual ao torque, o que é análogo à forma $\vec{F} = d\vec{p}/dt$ para a Segunda Lei de Newton. Portanto, temos uma nova forma para a Segunda Lei de Newton para Rotações:
\begin{equation}\label{Eq:SegLeiNewtonRotDLDT}
  \vec{\tau} = \frac{d\vec{\ell}}{dt}.
\end{equation}

\comment{$\tau$ e $\ell$ devem ser definidos em relaçao ao mesmo ponto}

%%%%%%%%%%%%%%%%%%%%%%%%%%%%%%%%%%%%%%%%%%%%%%%%%%%%%%%%%%%%%%%%%%%%%%%%%%%%
\subsection{Momento angular para uma partícula que se desloca em linha reta}
%%%%%%%%%%%%%%%%%%%%%%%%%%%%%%%%%%%%%%%%%%%%%%%%%%%%%%%%%%%%%%%%%%%%%%%%%%%%
\comment{mostrar que $r\sen\phi = d$ ($d$ é a distância mínima entre a reta em que a partícula se desloca e a origem)}

Assim como no caso do torque, o momento angular é calculado em relação a um ponto. Mesmo que a partícula se desloque em uma linha reta, sem executar uma rotação em torno de um ponto, podemos lhe atribuir um momento angular.

%%%%%%%%%%%%%%%%%%%%%%%%%%%%%%%%%%%%%%%%%%%%%%%%%%%%%%%%
\subsection{Momento angular de um sistema de partículas}
%%%%%%%%%%%%%%%%%%%%%%%%%%%%%%%%%%%%%%%%%%%%%%%%%%%%%%%%

O momento angular de um sistema de partículas pode ser calculado somando-se o momento angular das várias partículas que o constituem:
\begin{align}
  \vec{L} &= \vec{\ell}_1 + \vec{\ell}_2 + \vec{\ell}_3 + \dots + \vec{\ell}_N \\
  &= \sum_{i=1}^N \vec{\ell}_i.
\end{align}
%
Esta propriedade é característica dos vetores, e já a utilizamos para definir o momento linear do centro de massa $\vec{P}_{\textrm{CM}}$ como sendo a soma do momento linear das partículas que o constituem.

Se derivarmos a expressão acima em relação ao tempo, temos
\begin{align}
  \frac{d\vec{L}}{dt} &= \frac{d}{dt}\left(\sum_{i=1}^N\vec{\ell}_i\right) \\
  &= \sum_{i=1}^N \frac{d\vec{\ell}_i}{dt}.
\end{align}
%
De acordo com a Equação~\ref{Eq:SegLeiNewtonRotDLDT} para a Segunda Lei de Newton para Rotações, $d\vec{\ell}_i/dt = \vec{\tau}_i$, isto é, o torque que atua sobre a i-ésima partícula. No entanto, para um sistema de partículas que interagem através de forças, os torques devido a forças internas geram um par que se cancela na soma. Dessa forma, restarão somente os torques externos, logo
\begin{equation}\label{Eq:SegLeiNewtonRotSisPartDLDT}
  \vec{\tau}_R^{\textrm{Ext}} = \frac{d\vec{L}}{dt}.
\end{equation}

\textbf{demonstrar aditividade do momento angular de um corpo rígido}

%%%%%%%%%%%%%%%%%%%%%%%%%%%%%%%%%%%%%%%%%%%
\subsection{Conservação do momento angular}
%%%%%%%%%%%%%%%%%%%%%%%%%%%%%%%%%%%%%%%%%%%

A partir da Equação~\ref{Eq:SegLeiNewtonRotSisPartDLDT}, percebemos que se $\vec{\tau}_R^{\textrm{Ext}} = 0$, temos que $d\vec{L}/dt = 0$, ou seja,
\begin{equation}
  \vec{L} = \textrm{constante}.
\end{equation}
%
Temos, portanto, uma nova lei de conservação -- a \emph{conservação do momento angular} --. Assim como nos casos da \emph{conservação da energia} e da \emph{conservação do momento linear}, o fato de termos uma lei de conservação envolvendo o momento angular nos permitirá analisar sistemas sem sabem em detalhes o que ocorre entre dois instantes quaisquer. Se um evento ocorre de forma que $\vec{\tau}_R^{\textrm{Ext}} = 0$, temos que o momento angular antes e depois de tal evento é o mesmo:
\begin{equation}
  L_i = L_f.
\end{equation}
%
Logo, se temos informações sobre o sistema antes do evento, podemos relacioná-las ao estado final do sistema sem saber detalhes do que ocorreu durante o evento. Isso será muito útil na análise de várias situações.

\textbf{discutir torques internos e o fato de que eles não mudam o momento angular, precisa de um exemplo aqui, nem que tenha que calcular o momento angular do sistema realizando a soma para um aro, ou dois. Ou usar exemplo de momento de inércia variável, com uma skew rod com massa desprezível e duas massas na ponta, deixar o ângulo de inclinação variar e verificar qual a velocidade das partículas depois.}


%%%%%%%%%%%%%%%%%%%%%%%%%%%%%%%%%%%%%%%%%%%%%%%%%%%%%%%%
\subsection{Momento angular de um corpo rígido}
%%%%%%%%%%%%%%%%%%%%%%%%%%%%%%%%%%%%%%%%%%%%%%%%%%%%%%%%

\begin{marginfigure}
\centering
\begin{tikzpicture}[>=Stealth, scale = 1.4,
     interface/.style={
        % superfície
        postaction={draw,decorate,decoration={border,angle=-45,
                    amplitude=0.2cm,segment length=2mm}}},
    ]

%%% Figura superior

% Disco
\draw (0,0) ellipse (1.25 and 0.5);

\draw (-1.25,0) -- (-1.25,-0.4);
\draw (1.25,-0.4) -- (1.25,0);  

\draw (-1.25,-0.4) arc (180:360:1.25 and 0.5);
\draw[densely dotted] (-1.25,-0.4) arc (180:360:1.25 and -0.5);

% Eixo z
\draw[dashdotted,->] (0,0) -- (0,1.5) node[below left]{$z$};
\draw[dashdotted] (0,-2.5) -- (0,-0.9);
\draw[dotted] (0,-0.9) -- (0,0);
\draw[fill] (0,0) circle (0.4pt);

% Origem
\draw[fill] (0,-2) circle (0.6pt);

% trajetória dos pontos
\draw[dotted] (0,0) ellipse (0.84 and 0.28);

% Partícula 1
\draw[fill] (5:-0.81) coordinate (p1) circle (0.4pt);
%\draw[dashed] (p1) -- (0,0);
\draw[dashed,<-] (p1)  -- (0,-2) coordinate (origin);
\draw ($ (p1) !0.4! (origin) $) node[below left]{$\vec{r}_i$} -- (origin);

\coordinate (encontro) at (0,0.5);
\draw[->, thick] (p1) -- ($ (p1) !0.6! (encontro) $) node[below]{$\vec{\ell}_i$}; 

\draw[->] (p1) -- +(-40:0.4) node[right]{$\vec{p}_i$};

% Partícula 2
\draw[fill] (5:0.81) coordinate (p2) circle (0.4pt);
%\draw[dashed] (p2) -- (0,0);
\draw[dashed,<-] (p2) -- (0,-2);
\draw ($ (p2) !0.45! (origin) $) node[below right]{$\vec{r}_j$} -- (origin);    

\draw[->, thick] (p2) -- ($ (p2) !0.6! (encontro) $) node[below]{$\vec{\ell}_j$};

\draw[->] (p2) -- +(145:0.4);
\node (pj) at (0.76,0.27) {$\vec{p}_j$};

\end{tikzpicture}
\caption{. }
\end{marginfigure}


Se um corpo gira em torno de um eixo, podemos calcular seu momento angular dividindo-o em várias partes e tratando-o como um sistema de partículas. Na Figura ??? ao lado, o momento angular de uma das partículas que compõe o corpo é mostrado. Segundo a definição do momento angular para uma partícula, temos
\begin{align}
  \vec{\ell}_i &= \vec{r}\times\vec{p} \\
  &= r_i p_i \sen \degree{90} \\
  &= r_i m_i v_i,
\end{align}
%
onde as variáveis $r_i$, $m_i$ e $v_i$ se referem à posição, velocidade e massa da partícula em questão. Usamos ainda o índice $i$ pois calculamos o momento angular de somente uma partícula, porém vamos somar sobre as demais já que a expressão é a mesma para todas elas.

\begin{marginfigure}
\centering
\begin{tikzpicture}[>=Stealth,
     interface/.style={
        % superfície
        postaction={draw,decorate,decoration={border,angle=-45,
                    amplitude=0.2cm,segment length=2mm}}},
    ]

    \draw (0,0) coordinate (origin) -- node[left]{$\vec{r}_i$}(125:4);
    \draw[dotted] (125:4) coordinate (l) -- +(35:3) coordinate (top);
    \draw[thick, ->] (125:4) -- node[above]{$\vec{\ell}_i$}+(35:1.5) coordinate (p);
    \draw[dashed] (125:4) -- node[below]{$r_\perp^i$} +(2.4,0) coordinate (m);
    \draw[dashdotted] (top) -- (0,0);
    
    \pic[draw, "$\gamma_i$", angle eccentricity = 1.5]{angle = m--l--p};
    \pic[draw, "$\gamma_i$", angle eccentricity = 1.5]{angle = top--origin--l};
    \pic[draw, "$\cdot$", angle eccentricity = 0.5, angle radius = 3.1mm]{angle = origin--l--top};
\end{tikzpicture}
\caption{. }
\end{marginfigure}
Se o corpo for simétrico e homogêneo, podemos perceber facilmente que para toda partícula $P$ existe uma partícula $P'$ diametralmente oposta à primeira e que tem os mesmos valores para as componentes do momento angular para os eixos $x$ e $y$, porém com sentidos contrários. Logo, ao realizarmos a soma sobre todas as partículas, concluímos que tais componentes resultarão em zero, restando somente a componente $z$ do momento angular. Esta componente pode ser calculada utilizando o ângulo $\theta$ e obtemos
\begin{align}
  \ell_{iz} &= \ell_i \sen\theta \\
  &= r_i m_i v_i \sen\theta.
\end{align}
%
A distância $r_{\perp}$ pode ser escrita como
\begin{equation}
  \ell_{\perp} = r_i\sen\theta.
\end{equation}
%
Consequentemente, 
\begin{equation}
  \ell_{iz} = m_i r_{\perp}^2 \omega,
\end{equation}
%
onde utilizamos $v_i = \omega r_{\perp}$ (lembre-se que $\omega$ é constante e igual para todas as partículas que compôe um corpo rígido). Dessa forma, podemos escrever o momento angular total do corpo como
\begin{equation}
  L_z = \left[\sum_{i=1}^N m_i r_{\perp}^2\right] \omega.
\end{equation}
%
O termo entre colchetes nada mais é que o momento de inércia do corpo. Além disso, sabemos que só resta a componente $z$ do momento angular, porém esta é a mesma direção da velocidade angular. Logo
\begin{equation}
  \vec{L}_z = I \vec{\omega}.
\end{equation}

É importante notar que se o corpo não for simétrico, restará uma componente do plano $xy$ que mudará constantemente de direção. No caso de um objeto assimétrico sofrer uma rotação, portanto, deve haver um torque que é realizado por um agente externo -- pois $\vec{\tau} = d\vec{L} / dt$ e se $\vec{L}$ não é constante, então $\vec{\tau} \neq 0$ --. Se o objeto em questão está prezo por mancais, por exemplo, tais suportes exercem força sobre o corpo que geram torques, possibilitanto que o momento angular varie. No entanto, existem as reações a essas forças, que são exercidas pelo corpo sobre os suportes, sendo responsáveis pelas \emph{vibrações} características de um corpo assimétrico submetido a rotações.

\textbf{Discutir o fato de que $\ell$ de uma partícula pode ser calculado através de $I\omega$ se considerarmos o momento de inércia de uma partícula somente.}

%%%%%%%%%%%%%%%%%%%%%%%%%%%%%%%%%%%%%%%%%%%%%%%%%%%
\subsection{Rotação de um corpo rígido assimétrico}
%%%%%%%%%%%%%%%%%%%%%%%%%%%%%%%%%%%%%%%%%%%%%%%%%%%

Ainda podemos utilizar a expressão obtida para $L_z$ em um corpo simétrico. No entanto, temos outras componentes de momento angular agora, e elas dependem do tempo. Precisamos de um torque para que exista variação do momento angular. Exemplificar usando semi-cilindro, como os usados para gerar vibrações em celular, controle de video-game.

falar da skew rod?

%%%%%%%%%%%%%%%%%%%%%%%%%%%%%%%%%%%%%%%%%%%%%%%%%
\subsection{Momento angular de um corpo que rola}
%%%%%%%%%%%%%%%%%%%%%%%%%%%%%%%%%%%%%%%%%%%%%%%%%

Ver seção 6.7 do Kleppner

%%%%%%%%%%%%%%%%%%%%%%%%%%%%%%%%%%%%%%%%%%%%%%%%%%%%%%%%
\subsection{Conservação do momento angular (envolvendo corpo rígidos, renomear essa seção)}
%%%%%%%%%%%%%%%%%%%%%%%%%%%%%%%%%%%%%%%%%%%%%%%%%%%%%%%%

\emph{Discutir caráter vetorial, considerar momentos de inércia variáveis}

\textbf{Exemplos: aluno com alteres girando em cadeira, patinadora; forças de maré e rotação sincronizada (terra+lua, plutão+caronte)}

%%%%%%%%%%%%%%%%%%%%%%%%%%%%%%%%%%%%%%%%%%%%%%%%%%%%%%%%
\section{Precessão de um giroscópio}
%%%%%%%%%%%%%%%%%%%%%%%%%%%%%%%%%%%%%%%%%%%%%%%%%%%%%%%%
\comment{Fazer figuras,melhorar texto, descrições, o que é um giroscópio, falar como é impressionante, etc.}

A Precessão de um giroscópio é um exemplo claro da razão pela qual o torque é uma grandeza vetorial. Se não fosse esse o caso, o movimento não poderia ser explicado. Na figura ao lado, mostramos um desenho esquemático de um giroscópio, com as forças e torques que atuam sobre ele quando ele está parado. Verificamos que há um torque na direção $y$ e que -- ao liberarmos a movimentação do sistema -- será responsável por girar o giroscópio em torno desse eixo, dotando-o de um momento angular $\vec{L}$ também na direção de $y$.

No caso de o giroscópio já estar girando antes de o soltarmos, já teremos um momento angular inicial $\vec{L}$ na direção do eixo do disco do giroscópio. Sabemos que nesse caso 
\begin{equation}
  \vec{L} = I\vec{\omega}.
\end{equation}
%
Se mantivermos a velocidade do disco constante, temos que o momento angular deve ser constante. Se soltarmos o sistema, o peso continuará exercendo um torque igual ao da situação anterior, na direção de $y$. Como $\vec{\tau}$ é perpendicular a $\vec{L}$, ele não pode mudar o \emph{módulo} do momento angular, porém pode mudar sua \emph{direção}. De fato, sabendo que
\begin{equation}
  \vec{\tau} = \frac{d\vec{L}}{dt},
\end{equation}
%
podemos escrever
\begin{equation}
  d\vec{L} = \vec{\tau} dt,
\end{equation}
%
o que nos indica que a \emph{variação} do vetor momento angular tem a mesma direção que o torque. Logo, após um intervalo de tempo $dt$, temos que o giroscópio aponta em uma nova direção no espaço.

Podemos determinar a velocidade de precessão do giroscópio fazendo a seguinte análise: Sabemos que o módulo do torque é dado, nesse caso, por
\begin{equation}
  \tau = Mgr,
\end{equation}
%
e portanto,
\begin{equation}
  dL = Mgr\,dt.
\end{equation}
%
Além disso, analisando a figura ao lado, temos que o arco $s$ tem comprimento
\begin{equation}
  s = L \phi.
\end{equation}
%
Para um ângulo muito pequeno,
\begin{equation}
  ds = L d\phi.
\end{equation}
%
Logo,
\begin{align}
  d\phi &= \frac{dL}{L} \\
  &= \frac{Mgr\,dt}{I\omega},
\end{align}
%
e, consequentemente,
\begin{equation}
  \Omega \equiv \frac{d\phi}{dt} = \frac{Mgr}{I\omega}.
\end{equation}

% Num avião monomotor, quando o ele acelera para decolar, existe uma inclinação do eixo de rotação do motor/hélice. Em um certo momento, o avião vai passar a ficar alinhado em relação à pista (devido à força de sustentação? ou ação do piloto?). Nesse momento, ocorre uma uma precessão das partes rotativas (giroscópio), fazendo com que o avião ``puxe'' para a esquerda (devido ao fato de que o giroscópio gira em sentido horário do ponto de vista do piloto). % Vi isso no Aviões e Músicas
