\chapter{Trabalho e Energia Mecânica}
\label{Chap:Energia}
%%%%%%%%%%%%%%%%%%%%%%%%%%%%%%%%%%%%%%%%%
%\minitoc


%\clearpage

%%%%%%%%%%%%%%%%%%%%%%%%%%%%%%%%%%%%%%%%%

%%% Quando iniciamos este capítulo, os alunos começam a ver derivadas em cálculo. Já viram produto escalar em geometria

\begin{fullwidth}
{\it

Diversas situações podem ser bastante complicadas de se resolver utilizando as Leis de Newton. Verificaremos neste capítulo um método alternativo baseado em uma grandeza escalar: a energia. Apesar de ser um método que inicialmente é uma consequência das leis do movimento, ele se revela uma característica fundamental dos fenômenos naturais: a conservação da energia em um fenômeno físico é tida como um princípio geral da Física.

}
\end{fullwidth}

%%%%%%%%%%%%%%%%%%%%%
\section{Introdução} 
%%%%%%%%%%%%%%%%%%%%%

\begin{marginfigure}[5cm]
\centering
\begin{tikzpicture}[>=Stealth, scale = 1.2,
       interface/.style={
       % superfície
       postaction={draw,decorate,decoration={border,angle=-45,
                   amplitude=0.2cm,segment length=2mm}}}
    ]

    \draw[interface] (1,0) -- (-1,0);
    \draw[dotted] ([shift={(0,0)}]180:2) arc[radius=2, start angle=180, end angle= 290];
    
    \draw[densely dotted] (0,0) -- (180:18mm);
    \draw[pattern = north west lines, dotted] (180:2) circle (2mm);
    \draw[->, thick, dotted] (180:2) +(0,-0.2) -- +(0,-0.7) node[left]{$\vec{P}$};
    
    \draw[densely dotted] (0,0) -- (-150:18mm);
    \draw[pattern = north west lines, dotted] (-150:2) circle (2mm);
    \draw[->, thick, dotted] (-150:2) +(0,-0.2) -- +(0,-0.7) node[left]{$\vec{P}$};
    \draw[->, thick, dotted] (-150:18mm) -- (-150:14mm) node[above left]{$\vec{T}$};
    
    \draw[densely dotted] (0,0) -- (-120:18mm);
    \draw[pattern = north west lines, dotted] (-120:2) circle (2mm);
    \draw[->, thick, dotted] (-120:2) +(0,-0.2) -- +(0,-0.7) node[left]{$\vec{P}$};
    \draw[->, thick, dotted] (-120:18mm) --  node[left]{$\vec{T}$} (-120:12mm);
    
    \draw (0,0) -- (-90:18mm);
    \draw[pattern = north west lines] (-90:2) circle (2mm);
    \draw[->] (-90:2) +(0.2,0) -- node[below]{$\vec{v}$} +(1,0);
    \draw[->, thick] (-90:2) +(0,-0.2) -- +(0,-0.7) node[left]{$\vec{P}$};
    \draw[->, thick] (-90:18mm) -- (-90:11mm) node[below right]{$\vec{T}$};
    
    \draw[fill] (0,0) circle (1pt);
\end{tikzpicture}
\caption{O cálculo da velocidade de um pêndulo através das leis de Newton é muito complexo, pois a aceleração depende da posição em que o corpo se encontra.}
\end{marginfigure}

Em muitos casos a determinação de algumas grandezas através das Leis de Newton é uma tarefa bastante complexa. Se, por exemplo, estamos interessados em calcular a velocidade de um pêndulo após ele percorrer uma certa distância, temos uma aceleração que varia dependendo da posição: ao aplicarmos a Segunda Lei de Newton obtemos, considerando a Figura~\ref{Fig:PenduloCalcVelSegLei}, obtemos
\begin{description}
    \item[Eixo $x$:]
        \begin{align}
            F_R^x &= m a_x \\
            T - P_x &= m \frac{v^2}{R} \\
            T &= P_x + m\frac{v^2}{R};
        \end{align}
    \item[Eixo $y$:]
        \begin{align}
            F_R^y &= m a_y \\
            P_y &= m a_y \\
            mg \sen\theta &= m a_y;
        \end{align}
\end{description}
%
A aceleração no eixo $x$ é responsável pela alteração da direção da velocidade, sendo portanto igual a $v^2/R$. Já a componente da aceleração no eixo $y$ é responsável pela alteração da velocidade do corpo, logo
\begin{equation}
    a_y = g \sen\theta.
\end{equation}
%
A determinação da velocidade nesse problema exige o uso de técnicas de Cálculo, uma vez que a aceleração não é constante, tornando a solução em algo não trivial.

\begin{marginfigure}[-5cm]
\centering
\begin{tikzpicture}[>=Stealth, scale = 1.2,
     interface/.style={
        % superfície
        postaction={draw,decorate,decoration={border,angle=-45,
                    amplitude=0.2cm,segment length=2mm}}},
    ]
    
    \draw[interface] (1,0) -- (-1,0);
    
    \draw[pattern = north west lines] (-1,-1.73) coordinate (bob) circle (3mm);
    \draw (-0.85,-1.47) coordinate (fix) -- (0,0) coordinate (O);
    \draw[densely dotted] (O) -- +(0,-0.75) coordinate (Oi);
    \draw[fill] (bob) circle (1pt);
    \draw[->, thick] (bob) -- +(0,-0.65) coordinate (P) node[right] {$\vec{P}$};
    \draw[->, thick] (fix) -- node[below right]{$\vec{N}$} +(60:0.866);
    
    \draw[dashed,->] (bob) +(240:1.5) coordinate (Xi) -- +(60:1.5) node[left]{$x$};
    \draw[dashed,->] (bob) +(150:1.5) -- +(-30:1.5) coordinate (Y) node[below left]{$y$};
    
    \pic[draw, "$\theta$", angle eccentricity = 1.5]{angle = fix--O--Oi};
    \pic[draw, "$\theta$", angle eccentricity = 1.5]{angle = Xi--bob--P};
    
    \draw[dashdotted, <->, gray] (O) -- node[right]{$R$} +(-70:2);
    
    \draw[dotted] ([shift={(0,0)}]-150:2) arc[radius=2, start angle=-150, end angle= -60];
\end{tikzpicture}
\caption{Note que no eixo $y$, cuja direção é tangente à trajetória executada pelo corpo preso à extremidade do fio, existe uma componente da força peso. Tal componente será responsável pela alteração do módulo da velocidade.\label{Fig:PenduloCalcVelSegLei}}
\end{marginfigure}

Podemos encontrar uma maneira mais simples de resolver problemas como esse utilizando o conceito de \emph{energia}. Diferentemente das variáveis cinemáticas e das forças, a energia é uma grandeza escalar, como a massa. Em certas circunstâncias, verificaremos que a energia de um sistema é uma constante, o que simplifica muito o tratamento de um fenômeno qualquer pois permite que relacionemos as grandezas associadas a configurações diferentes do sistema físico em questão: se a energia é constante, a energia associada a uma disposição inicial em um tempo inicial é igual à energia associada a uma disposição final em um tempo final.

Verificaremos a seguir como podemos determinar a energia associada ao movimento de um corpo ---~a \emph{energia cinética}~---: uma alteração da velocidade de um corpo está ligada a uma força, através da aceleração. Determinaremos o que chamamos de \emph{trabalho} realizado pela força, que pode ser entendido como \emph{a quantidade de energia cedida ao corpo pela força exercida sobre ele}. A relação entre a variação da energia cinética e o trabalho é conhecida como \emph{Teorema Trabalho--Energia-cinética}

%%%%%%%%%%%%%%%%%%%%%%%%%%%%%%%%%%%%%%%%%%%%
\section{Teorema Trabalho--Energia-cinética}
%%%%%%%%%%%%%%%%%%%%%%%%%%%%%%%%%%%%%%%%%%%%

% O primeiro a falar em trabalho e energia cinética com o significado atual foi Coriolis (segundo o livro de Sistemas Dinâmicos de Luiz Henrique Alves Monteiro).

Se tomarmos um objeto que pode se mover ao longo de um fio esticado horizontalmente, submetido a uma força $F$ constante e que faz um ângulo $\theta$ com a direção do fio. Definindo um eixo $x$ ao longo do fio, podemos verificar através da Equação de Torricelli que a velocidade estará relacionada à distância percorrida pelo objeto através de
\begin{equation}
  v_f^2 = v_i^2 + 2 a \Delta x.
\end{equation}

\begin{marginfigure}
\centering
\begin{tikzpicture}[>=Stealth]
    \draw[dashdotted, ->] (0,0) -- (4.5,0) coordinate (x) node[below left]{$x$};
    
    \draw[pattern = north west lines] (0.5,0) coordinate (c1) circle (2mm);
    \draw[->, thick] (0.5,0)+(30:2mm) -- +(30:1.2) coordinate (f1) node[above]{$\vec{F}$};
    
    \draw[->] (0.5, -0.5) +(-0.25, 0) -- node[below]{$\vec{v}_i$}+(0.25, 0);
    
    \coordinate (c2) at (3.5, 0);
    \draw[pattern = north west lines, dotted] (c2) circle (2mm);
    \draw[->, thick, dotted] (3.5,0)+(30:2mm) -- +(30:1.2) coordinate (f2) node[above]{$\vec{F}$};
    \draw[->, dotted] (3.5, -0.5) +(-0.35, 0) -- node[below]{$\vec{v}_i$}+(0.35, 0);
    
    \pic[draw, "$\theta$", angle eccentricity = 1.5] {angle = c2--c1--f1};
    \pic[draw, "$\theta$", angle eccentricity = 1.5] {angle = x--c2--f2};
    
\end{tikzpicture}
\caption{Uma conta que pode deslisar por um fio é acelerada lateralmente por uma força $\vec{F}$ constante. Note que para que uma aceleração lateral seja possível, é necessário que haja uma força exercida pelo fio sobre a conta, de forma a equilibrar a componente de $\vec{F}$ perpendicular ao fio (eixo $x$).\label{Fig:ContaAndandoNumFio}}
\end{marginfigure}

\noindent{}Sabemos que se o fio impede o movimento no eixo perpendicular a ele, temos somente aceleração no eixo $x$, o que resulta em uma aceleração dada por $a = F_x/m$. Logo,
\begin{equation}
  v_f^2 = v_i^2 + 2 \frac{F_x}{m} \Delta x,
\end{equation}
%
o que pode ser reescrito como
\begin{equation}
  \frac{1}{2} m v_f^2 - \frac{1}{2} m v_i^2 = F_x \Delta x.
\end{equation}

Através da expressão acima, verificamos que durante o deslocamento do objeto existe uma variação entre os valores inicial e final de uma grandeza $K$ definida como
\begin{equation}
  K = \frac{1}{2} m v^2 \mathnote{Energia cinética}
\end{equation}
%
e que denominamos como \emph{energia cinética}. A variação de tal grandeza está relacionada ao produto da força e do deslocamento o que define uma grandeza $W$ denominada \emph{trabalho}:
\begin{equation}
  W = F_x \Delta x.
\end{equation}
%
O trabalho pode ser analisado em três situações distintas, relacionadas ao ângulo que a força $\vec{F}$ faz com a direção do deslocamento:
\begin{itemize}
    \item Se tivermos que $\theta < \degree{90}$, a força tende a acelerar o objeto e a energia cinética deve aumentar com o tempo;
    \item Se, por outro lado, $\theta > \degree{90}$, a força tende a desacelerar o objeto, diminuindo sua energia cinética;
    \item Finalmente, se $\theta = \degree{90}$, a força não é capaz de acelerar o objeto\footnote[][-1cm]{Lembre-se que o objeto está limitado a se deslocar no eixo $x$, portanto não podemos ter aceleração no eixo perpendicular ao deslocamento.}, deixando a energia cinética constante, o que implica em um trabalho nulo.
\end{itemize}
%
\begin{marginfigure}[2cm]
\centering
\begin{tikzpicture}[>=Stealth]
    \draw[->] (0,0) coordinate (origin) -- node[below]{$\vec{a}$} (1.5,0) coordinate (A);
    \draw[->] (0,0) -- node[above]{$\vec{b}$} (1,1) coordinate (B);
    
    \pic [draw, "$\theta$", angle eccentricity = 1.5]{angle = A--origin--B};
\end{tikzpicture}
\caption{Ângulo $\theta$ entre dois vetores para o cálculo do produto escalar.\label{Fig:AnguloParaProdutoEscalar}}
\end{marginfigure}

\noindent{}Essas três observações podem ser conciliadas se definirmos um vetor $\vec{d}$ que descreva o deslocamento do objeto (a direção e sentido de $\vec{d}$ são a do semi-eixo $x$ positivo; o módulo é dado por $\Delta x$) e tomarmos o \emph{produto escalar} com a força. O produto escalar toma dois vetores e resulta em um escalar. Podemos calcular este produto através da expressão
\begin{equation}
    \vec{a} \cdot \vec{b} = ab\cos\theta, \mathnote{Produto escalar}
\end{equation}
%
onde o ponto entre os vetores denota a operação de produto escalar e o ângulo $\theta$ é o menor ângulo entre os dois vetores (veja a Figura~\ref{Fig:AnguloParaProdutoEscalar}). Caso as componentes dos vetores em um sistema de referência sejam conhecidas, podemos calcular o produto escalar como
\begin{equation}
    \vec{a} \cdot \vec{b} = a_xb_x + a_yb_y + a_zb_z.
\end{equation}

\begin{marginfigure}[-15mm]
\centering
\begin{tikzpicture}[>=Stealth]
    \draw[loosely dotted, ->] (0,0) -- (4.5,0) coordinate (x);
    
    \draw[pattern = north west lines] (0.5,0) coordinate (c1) circle (2mm);
    \draw[->, thick] (0.5,0)+(30:2mm) -- +(30:1.2) coordinate (f1) node[above]{$\vec{F}$};
    
    \draw[->] (0.5, -0.5) +(-0.25, 0) -- node[below]{$\vec{v}_i$}+(0.25, 0);
    \draw[fill] (0.5,0) circle (1pt);
    \draw[fill] (3.5,0) circle (1pt);
    
    \coordinate (c2) at (3.5, 0);
    \draw[pattern = north west lines, dotted] (c2) circle (2mm);
    \draw[->, thick, dotted] (3.5,0)+(30:2mm) -- +(30:1.2) coordinate (f2) node[above]{$\vec{F}$};
    \draw[->, dotted] (3.5, -0.5) +(-0.35, 0) -- node[below]{$\vec{v}_i$}+(0.35, 0);
    
    \draw[->] (0.5,0) -- node[below]{$\vec{d}$} (3.5, 0);
    
    \pic[draw, "$\theta$", angle eccentricity = 1.5] {angle = c2--c1--f1};
    \pic[draw, "$\theta$", angle eccentricity = 1.5] {angle = x--c2--f2};
    
\end{tikzpicture}
\caption{O trabalho efetuado por uma força $\vec{F}$ durante um deslocamento $\vec{d}$ é calculado através de $W = \vec{F}\cdot\vec{d}$.\label{Fig:DefDeslocamentoDeterminacaoTrabalho}}
\end{marginfigure}

Assim, considerando os vetores $\vec{F}$ e $\vec{d}$ mostrados na Figura~\ref{Fig:DefDeslocamentoDeterminacaoTrabalho}, obtemos
\begin{equation}\label{Eq:TrabalhoForcaConstante}
  W = \vec{F}\cdot\vec{d}. \mathnote{Trabalho de uma força constante}
\end{equation}
%
Consequentemente, temos o seguinte resultado, conhecido como \emph{Teorema Trabalho--Energia-cinética}\footnote[][15mm]{Apesar de o nome dar a impressão de que esse resultado é extremamente notável, ele é uma consequência da 2\textordfeminine lei de Newton e pode ser deduzido a partir dela utilizando técnicas de cálculo vetorial (veja as Seções~\ref{Sec:SolPenduloLeisDeNewton} e~\ref{Sec:DetTeoremaTrabEnergiaCalculoVetorial}). De qualquer forma, o resultado é bastante útil.}:
\begin{equation}
  \Delta K = W. \mathnote{Teorema Trabalho--Energia-cinética}
\end{equation}

Essa observação é muito importante, pois permite que interpretemos diversos fenômenos físicos a partir do ponto de vista da \emph{energia}. O trabalho pode ser interpretado como um processo de \emph{transferência de energia}, logo, quando temos forças que realizam trabalho positivo, energia é transferida \emph{para o objeto}, aumentando sua energia cinética; quando o trabalho é negativo, energia é transferida \emph{do objeto} ---~retirada, transferida para outro agente~---, fazendo com que a energia do objeto diminua.

%%%%%%%%%%%%%%%%%%%%
\paragraph{Unidades}
%%%%%%%%%%%%%%%%%%%%

Tanto a energia cinética, quanto o trabalho tem unidades dadas por
\begin{align*}
    [K] &= \left[\frac{1}{2} mv^2\right] & [W] &= [\vec{F}\cdot\vec{d}] \\
    &=[m][v]^2 & &= [F][d] \\
    &=\rm{M}\frac{\rm{L}^2}{\rm{T}^2} & &= \rm{M}\frac{\rm{L}}{\rm{T}^2} \rm{L},
\end{align*}
%
o que, utilizando unidades do Sistema Internacional, resulta em
\begin{equation}
    \np[J]{1} \equiv \np[kg\cdot m^2/s^2]{1}.
\end{equation}
%
A unidade $\rm{J}$ é dominada \emph{joule} em homenagem a James Prescott Joule, físico que realizou estudos sobre energia e calor.

%%%%%%%%%%%%%%%%%%%%%%%%%%%%%
\section{Cálculo do trabalho}
%%%%%%%%%%%%%%%%%%%%%%%%%%%%%

A determinação do trabalho realizado por uma força está ligada não somente às características da própria força, mas também à situação específica em questão, uma vez que o trabalho depende também do deslocamento. Nas próximas seções verificaremos como determinar o trabalho em algumas situações que envolvem as forças estudadas no capítulo anterior. Iniciaremos utilizando a definição dada pela Equação~\ref{Eq:TrabalhoForcaConstante} e, em alguns casos, utilizaremos algumas propriedades dos vetores e do produto escalar, porém verificaremos que tal expressão não pode ser aplicada em todos os casos, uma vez que foi determinada assumindo que a força é constante. Antes disso, no entanto, vamos verificar algumas propriedades do trabalho que são consequências da própria definição em termos do produto vetorial.

%%%%%%%%%%%%%%%%%%%%%%%%%%%%%%%%%%%%%%%%%%%%%
\paragraph{Trabalho de um conjunto de forças}
%%%%%%%%%%%%%%%%%%%%%%%%%%%%%%%%%%%%%%%%%%%%%

Vamos considerar o caso em que várias forças atuam sobre um corpo. A aceleração nesse caso será dada pela \emph{força resultante}. Nesse caso, a variação da energia cinética ---~e, consequentemente, o trabalho~--- deve estar ligado a tal força:
\begin{equation}
    \Delta K = W_{F_R}.
\end{equation}
%
\begin{marginfigure}
\centering
\begin{tikzpicture}[>=Stealth,
     interface/.style={
        % superfície
        postaction={draw,decorate,decoration={border,angle=-45,
                    amplitude=0.2cm,segment length=2mm}}},
    ]
      
    \draw[dashdotted, ->](-1,0) -- (3.25,0) node[below left]{$x$};
    
    \draw[pattern = north west lines] (0,0) circle (0.25cm);
        
    \draw[fill] (0,0) circle (1pt);
    \draw[->, thick] (60:0.25) -- (60:1.25) node[above]{$\vec{F}_1$};
    \draw[->, thick] (-60:0.25) -- (-60:1.25) node[below]{$\vec{F}_2$};
    
    \draw[dotted] (2,0) circle (0.25cm);

    \draw[->](0,0) -- node[below, near end]{$\vec{d}$} +(1.95,0);
    \draw[fill] (2,0) circle (1pt);
    
    \begin{scope}[shift={(0,-3)}]
        
        \draw[dashdotted, ->](-1,0) -- (3.25,0) coordinate (x) node[below left]{$x$};
    
        \draw[pattern = north west lines] (0,0) coordinate (O) circle (0.25cm);
            
        \draw[fill] (0,0) circle (1pt);
        \draw[->, thick] (0:0.25) -- (0:1.25) node[above left]{$\vec{F}_R$} coordinate (F);
        
        \draw[dotted] (2,0) circle (0.25cm);

        \draw[->](0,0) -- node[below, near end]{$\vec{d}$} +(1.95,0);
        \draw[fill] (2,0) circle (1pt);
        
        \pic[draw, angle radius = 8mm] {angle = x--O--F};
    
    \end{scope}
    
\end{tikzpicture}
\caption{Quando um conjunto de forças atua sobre um objeto, a variação da energia cinética está ligada ao trabalho realizado pela força resultante $\vec{F}_R$. \label{Fig:TrabalhoDiversasForcas}}
\end{marginfigure}

Podemos utilizar a propriedade distributiva do produto escalar, isto é, o fato de que
\begin{equation}
    (\vec{a} + \vec{b})\cdot \vec{c} = \vec{a}\cdot\vec{c} + \vec{b} \cdot\vec{c}
\end{equation}
%
para escrever
\begin{align}
    W_{F_R} &= \vec{F}_R \cdot \vec{d} \\
    &= (\vec{F}_1 + \vec{F}_2 + \vec{F}_3 + \dots) \cdot \vec{d} \\
    &= \vec{F}_1 \cdot \vec{d} + \vec{F}_2 \cdot \vec{d} + \vec{F}_3 \cdot \vec{d} + \dots \\
    &= W_{F_1} + W_{F_2} +  W_{F_3} + \dots, 
\end{align}
%
ou seja, \emph{podemos simplesmente calcular o trabalho de cada força independentemente e o trabalho total será dado pela soma de tais trabalhos}. 

%%%%%%%%%%%%%%%%%%%%%%%%%%%%%%%%%%%%%%%%%%%%%%%%%%
\paragraph{Trabalho em uma situação de equilíbrio}
%%%%%%%%%%%%%%%%%%%%%%%%%%%%%%%%%%%%%%%%%%%%%%%%%%

Note que no caso de uma situação em que há movimento, porém a força resultante é nula, devemos ter uma variação nula da energia cinética, já que o equilíbrio de forças implica em uma aceleração nula. Dessa forma, concluímos que
\begin{align}
    W_{F_R} &= \Delta K \\
    \vec{F}_R \cdot \vec{d} & = 0 \\
    W_{F_1} + W_{F_2} +  W_{F_3} + \dots &= 0, 
\end{align}
%
ou seja, necessariamente temos que parte das forças realiza trabalho positivo, enquanto outra parte realiza trabalho negativo.

%%%%%%%%%%%%%%%%%%%%%%%%%%%%%%%%%%%%%%%%%%
\paragraph{Decomposição vetorial da força}
%%%%%%%%%%%%%%%%%%%%%%%%%%%%%%%%%%%%%%%%%%

Ao definir o trabalho de uma força constante, verificamos que somente a componente da força que atua na direção do deslocamento é relevante. Vetorialmente, podemos descrever tal propriedade se escrevermos a força $\vec{F}$ como a soma de dois vetores, um $\vec{F}_{\parallel}$ na direção do deslocamento e outro $\vec{F}_{\perp}$ perpendicular ao deslocamento:
\begin{equation}
    \vec{F} = \vec{F}_{\parallel} + \vec{F}_{\perp},
\end{equation}
%
\begin{marginfigure}[-3cm]
\centering
\begin{tikzpicture}[>=Stealth, scale = 1.1]
    \draw[dashed] (-0.5,0) -- (3,0) coordinate (x);
    
    \draw[fill] (0.5,0) coordinate (c1) circle (1mm);
    \draw[->, thick] (c1) -- node[below, near end]{$\vec{d}$} +(2,0);
    
    \draw[->, thick] (0.5,0) -- +(30:1.2) coordinate (f1) node[above]{$\vec{F}$};
    \draw[dotted] (f1) -- +(0, -0.6) (f1)-- +(-1.039,0);
    \draw[->, thick] (0.5,0) -- node[below]{$\vec{F}_{\parallel}$} +(1.039 , 0) coordinate (c2);
    \draw[->, thick] (0.5,0) -- node[left]{$\vec{F}_{\perp}$} +(0,0.6);
       
    \pic[draw, "$\theta$", angle eccentricity = 1.5] {angle = c2--c1--f1};

\end{tikzpicture}
\caption{Podemos decompor um vetor qualquer como a soma de dois vetores em direções arbitrárias. Utilizamos essa propriedade vetorial para descrever um vetor como a soma e duas componentes, uma na direção do deslocamento $\vec{d}$ e outra perpendicular a ele. Verificamos que somente a componente paralela é capaz de realizar trabalho. Considerando isso, em algumas situações, vamos nos preocupar em determinar somente essa componente.}
\end{marginfigure}
%
\noindent{}de onde podemos verificar que
\begin{align}
    W &= \vec{F}\cdot\vec{d} \\
    &= (\vec{F}_{\parallel} + \vec{F}_{\perp})\cdot\vec{d} \\
    &= \vec{F}_{\parallel} \cdot \vec{d} + \vec{F}_{\perp} \cdot \vec{d} \\
    &= \vec{F}_{\parallel} \cdot \vec{d},
\end{align}
%
onde usamos $\vec{F}_{\perp} \cdot \vec{d} = 0$. Veja que o ângulo entre $\vec{F}_{\parallel}$ e $\vec{d}$ é de ou \degree{0}, ou \degree{180}, o que nos leva a \footnote{Note que o que fizemos aqui foi só verificar que as propriedades vetoriais descrevem adequadamente algo que foi pressuposto na própria definição do trabalho de uma força constante, Equação~\ref{Eq:TrabalhoForcaConstante}.}
\begin{equation}\label{eq:TrabalhoDecompForca}
    W = \pm F_{\parallel} d.
\end{equation}

Ao analisarmos a Figura~\ref{Fig:ContaAndandoNumFio}, assumimos que o movimento do corpo está limitado ao eixo $x$. Isso só é possível se existir uma força exercida na interação do corpo com o fio, de maneira a equilibrar a componente da força $\vec{F}$ que é perpendicular ao eixo $x$. Se assumirmos que o corpo é uma esfera plástica perfurada e que o fio passa pelo furo, a força exercida na direção perpendicular ao fio é uma força normal. No entanto, temos que o seu trabalho é dado por
\begin{align}
    W_N &= \vec{F}\cdot\vec{d} \\
    &= \vec{N}\cdot\vec{d} \\
    &= 0.
\end{align}

%%%%%%%%%%%%%%%%%%%%%%%%%%%%%%%%%%%%%%%%%%%%%%%%%
\paragraph{Decomposição vetorial do deslocamento}
%%%%%%%%%%%%%%%%%%%%%%%%%%%%%%%%%%%%%%%%%%%%%%%%%

Assim como pudemos escrever o vetor $\vec{F}$ em termos de uma soma de outros vetores, podemos fazer o mesmo para o vetor deslocamento. Se, por exemplo, temos um corpo que sofre um deslocamento $\vec{d}$ dado por
\begin{equation}
    \vec{d} = \vec{d}_1 + \vec{d}_2 + \vec{d}_3 + \dots,
\end{equation}
%
então o trabalho de uma força $\vec{F}$ que atua sobre o corpo durante tal deslocamento é dado por
\begin{align}
    W &= \vec{F}\cdot\vec{d} \\
    &= \vec{F}\cdot(\vec{d}_1 + \vec{d}_2 + \vec{d}_3 + \dots) \\
    &= \vec{F}\cdot\vec{d}_1 + \vec{F}\cdot\vec{d}_2 + \vec{F}\cdot\vec{d}_3 + \dots,
\end{align}
%
isto é, o trabalho total pode ser determinado pela soma dos trabalhos realizados em cada segmento do deslocamento.

Uma consequência particularmente útil do resultado acima é o fato de que podemos separar o deslocamento em uma componente \emph{paralela} à força $\vec{F}$ e em uma componente \emph{perpendicular} à força. Nesse caso, temos que
\begin{equation}
    \vec{d} = \vec{d}_{\parallel} + \vec{d}_{\perp},
\end{equation}
%
de onde obtemos para o trabalho
\begin{align}
    W &= \vec{F}\cdot\vec{d} \\
    &= \vec{F}\cdot(\vec{d}_{\parallel} + \vec{d}_{\perp}) \\
    &= \vec{F}\cdot\vec{d}_{\parallel} + \vec{F}\cdot\vec{d}_{\perp}.
\end{align}
%
Claramente temos que o segundo termo à direita na equação acima é nulo, uma vez que os vetores $\vec{F}$ e $\vec{d}_{\perp}$ são perpendiculares por definição, o que leva a um produto escalar nulo entre eles. Logo,
\begin{equation}
    W = \vec{F}\cdot\vec{d}_{\parallel}.
\end{equation}
%
Note que, como o ângulo entre $\vec{F}$ é ou \degree{0}, ou \degree{180}, 
\begin{equation}\label{eq:TrabalhoDecompDesl}
    W = \pm F d_{\parallel}.
\end{equation}

Portanto, assim como podemos nos preocupar somente com a força na direção do deslocamento para calcular o trabalho, podemos ---~alternativamente~--- nos preocupar somente com o deslocamento na direção da força. Tal resultado, no entanto, não passa de uma consequência da própria definição do trabalho de uma força constante, uma vez que\footnote{O sinal positivo ou negativo que aparece explicitamente nas Expressões~\ref{eq:TrabalhoDecompForca} e~\ref{eq:TrabalhoDecompDesl}, nas equações abaixo esta ligado ao fato de $\theta$ ser maior ou menor que \degree{90}.}
\begin{align}
    W &= \vec{F}\cdot\vec{d} \\
    &= Fd\cos\theta \\
    &= F(d\cos\theta) \\
    &= (F\cos\theta)d,
\end{align}
%
onde
\begin{align}
    d_{\parallel} = d\cos\theta \\
    F_{\parallel} = F\cos\theta,
\end{align}
%
o que implica em
\begin{equation}
    F_{\parallel}d = Fd_{\parallel}.
\end{equation}

%%%%%%%%%%%%%%%%%%%%%%%%%%%%%
\paragraph{Sinal do trabalho}
%%%%%%%%%%%%%%%%%%%%%%%%%%%%%

Finalmente, devemos destacar ainda que uma maneira simples de verificar se o trabalho tem o sinal adequado, é analisar o efeito da força considerada na energia cinética. Como verificamos na definição do teorema Trabalho--Energia-cinética, os possíveis efeitos de uma força que atua sobre um corpo são aumentar, diminuir, ou manter a energia cinética constante. O último caso ocorre se a força é perpendicular ao deslocamento, o que já sabemos que implica um trabalho nulo.

Os dois primeiros casos ocorrem quando a força atua no sentido de acelerar, ou de desacelerar o corpo. Sempre que a força atua no sentido de acelerar o corpo, isso significa que ela deve causar um \emph{aumento da velocidade}, com um consequente \emph{aumento na energia cinética}. Logo, o trabalho realizado pela força é \emph{positivo}. Caso a força atue no sentido de \emph{diminuir a velocidade}, isto é, a força tende a causar uma desaceleração do corpo, deve haver uma \emph{diminuição da energia cinética} e o trabalho deve ser \emph{negativo}. Verificar tais propriedades é, em geral, muito simples, podendo poupar algum trabalho ao evitar revisar contas devido a erros de sinal.

%%%%%%%%%%%%%%%%%%%%%%%%%%%%%%%%%%%%%%%%%%%%%%%%%%%%%%%%
\subsection{Trabalho realizado pela força peso}
%%%%%%%%%%%%%%%%%%%%%%%%%%%%%%%%%%%%%%%%%%%%%%%%%%%%%%%%

Quando um objeto se move por um caminho qualquer sujeito à força peso, temos que o trabalho realizado por tal força pode ser positivo, negativo ou nulo, dependendo da orientação do deslocamento. Podemos determinar uma expressão para o trabalho realizado pela força peso analisando a Figura~\ref{Fig:DeslocamentoCorpoDiagonal}. O trabalho realizado pela força peso em tal deslocamento será dado por
\begin{equation}
  W_g = \vec{P}\cdot\vec{d}.
\end{equation}

\begin{marginfigure}
\centering
\begin{tikzpicture}[>=Stealth]
    
    \draw[pattern = north west lines] (1,-1) coordinate (O) circle (2mm);
    \draw[dotted] (3,-3) circle (2mm);
    
    \draw[fill] (1,-1) circle (1pt);
    \draw[->, thick] (1,-1) -- +(0,-1) coordinate (P) node[above left]{$\vec{P}$};
    
    \draw[->] (1,-1) -- node[above right]{$\vec{d}$} (3,-3) coordinate (D);
    \draw[fill] (3,-3) circle (1pt);
    
    \draw[dashed,->] (0.5,-1) -- (4,-1) node[above left]{$x$};
    \draw[dashed,->] (1, -0.5) -- (1, -3.5) node[above left]{$y$};
    
    \pic [draw, "$\theta$", angle eccentricity = 1.5] {angle = P--O--D};
\end{tikzpicture}
\caption{Corpo que se desloca diagonalmente para baixo. Veja que a distância percorrida verticalmente é dada pela linha pontilhada e corresponde a $d\cos\theta$. \label{Fig:DeslocamentoCorpoDiagonal}}
\end{marginfigure}

Sabemos que a força peso atua na direção vertical, apontando para baixo. Podemos escrever o vetor deslocamento como a soma de um vetor na direção paralela ao peso e de um vetor na direção perpendicular a ele (veja a Figura~\ref{Fig:DeslocamentoCorpoDiagonalDecomposto}):
\begin{equation}
    \vec{d} = \vec{d}_{\parallel} + \vec{d}_{\perp}.
\end{equation}

\begin{marginfigure}
\centering
\begin{tikzpicture}[>=Stealth]
    
    \draw[pattern = north west lines] (1,-1) coordinate (O) circle (2mm);
    \draw[dotted] (3,-3) circle (2mm);
    
    \draw[fill] (1,-1) circle (1pt);
    \draw[->, thick] (1,-1) -- +(0,-1) coordinate (P) node[above left]{$\vec{P}$};
    
    \draw[->] (1,-1) -- node[above right]{$\vec{d}$} (3,-3) coordinate (D);
    \draw[fill] (3,-3) circle (1pt);
    
    \draw[dashed,->] (0.5,-1) -- (4,-1) node[above left]{$x$};
    \draw[dashed,->] (1, -0.5) -- (1, -3.5) node[above left]{$y$};
    
    \draw[dotted] (3,-1) -- (3,-3) -- (1,-3);
    
    \draw[->] (1,-1) -- node[above, near end]{$\vec{d}_{\perp}$}(3,-1);
    \draw[->] (1,-1) -- node[left, near end]{$\vec{d}_{\parallel}$}(1,-3);
    
    \pic [draw, "$\theta$", angle eccentricity = 1.5] {angle = P--O--D};
\end{tikzpicture}
\caption{O vetor $\vec{d}$ pode ser escrito como a soma dos vetores $\vec{d}_{\parallel}$ e $\vec{d}_{\perp}$. \label{Fig:DeslocamentoCorpoDiagonalDecomposto}}
\end{marginfigure}

\noindent{}Calculando o trabalho, obtemos
\begin{align}
    W &= \vec{F}\cdot\vec{d} \\
    &= \vec{P}\cdot(\vec{d}_{\parallel} + \vec{d}_{\perp}) \\
    &= \vec{P}\cdot\vec{d}_{\parallel} + \vec{P}\cdot\vec{d}_{\perp} \\
    &= \vec{P}\cdot\vec{d}_{\parallel},
\end{align}
%
onde usamos $\vec{P}\cdot\vec{d}_{\perp} = 0$. Note que o módulo de $\vec{d}_{\perp}$ equivale à variação da posição no eixo vertical mostrado na figura. Logo
\begin{align}
    W &= \vec{P}\cdot\vec{d}_{\parallel} \\
    &= P |\Delta y| \cos\degree{0},
\end{align}
%
ou, substituindo $P = mg$ e notando que $\Delta y > 0$
\begin{equation}\label{Eq:TrabalhoPeso}
    W = mg\Delta y. \mathnote{Trabalho realizado pela força peso (eixo $y$ vertical para baixo)}
\end{equation}
%
Temos, portanto, uma expressão que evidencia o fato de que o trabalho realizado pela força peso só depende de deslocamentos \emph{verticais}.

Caso o deslocamento seja tal que o corpo sobe, ou seja, se desloca no sentido negativo de $y$, obtemos
\begin{align}
    W &= \vec{P}\cdot\vec{d}_{\parallel} \\
    &= P |\Delta y| \cos\degree{0} \\
    &= mg |\Delta y| \cos\degree{180},
\end{align}
%
ou, e notando que $\Delta y < 0$ e que $\cos\degree{180} = -1$
\begin{equation}
    W = mg\Delta y.
\end{equation}
%
Verificamos, portanto, que a expressão acima pode ser utilizada para determinar o trabalho para qualquer deslocamento. 

Devemos notar ainda que a expressão acima foi determinada para o caso de um eixo $y$ direcionado verticalmente para baixo. Se adotarmos um eixo orientado verticalmente, porém apontando para cima, verificamos que os sinais de $\Delta y$ se alteram, portanto devemos inserir um sinal que corrige tal diferença, o que resulta em
\begin{equation}
    W = -mg\Delta y. \mathnote{Trabalho realizado pela força peso (eixo $y$ vertical para cima)}
\end{equation}
  
%%%%%%%%%%%%%%%%%%%%%%%%%%%%%%%%%%%%%%%%%%%%%%%%%%%%%%%%%%%%
\subsection{Trabalho realizado por outras forças constantes}
%%%%%%%%%%%%%%%%%%%%%%%%%%%%%%%%%%%%%%%%%%%%%%%%%%%%%%%%%%%%

O cálculo do trabalho de cada uma das forças que atuam em um fenômeno qualquer deve estar sempre pautado pela Expressão~\eqref{Eq:TrabalhoForcaConstante}. Na Figura~\ref{Fig:TrabalhoAtrito}, por exemplo, temos um diagrama de forças para o deslizamento de um bloco sobre um plano inclinado, onde também indicamos o vetor deslocamento. Sabemos que a variação da energia cinética do bloco está associada ao trabalho total efetuado pelas diversas forças que atuam no sistema:
\begin{description}
    \item[Força normal:] No caso da força normal temos que
        \begin{align}
            W_N &= \vec{F}\cdot\vec{d} \\
            &= \vec{N}\cdot\vec{d} \\
            &= Nd\cos\degree{90} \\
            &= 0,
        \end{align}
        onde utilizamos o fato de que a força e o deslocamento são perpendiculares entre si.
    
    \item[Força de atrito:] Podemos verificar que o atrito realiza um trabalho dado por
\begin{align}
  W_{\fat} &= \vecfat\cdot\vec{d} \\
  &= \fat d\cos\theta.
\end{align}

\begin{marginfigure}
\centering
\begin{tikzpicture}[>=Stealth, rotate=-35,
     interface/.style={
        % superfície
        postaction={draw,decorate,decoration={border,angle=-45,
                    amplitude=0.2cm,segment length=2mm}}},
    ]
      
    \draw[interface] (0,0) -- (4,0);
    \draw[pattern = dots] (0.5,0) rectangle (1.5,1);
        
    \draw[fill] (1,0.5) circle (1pt);
    \draw[->, thick] (1,0.5) -- +(-55:1) node[left]{$\vec{P}$};
    \draw[->, thick] (1,1) -- +(0,0.81915) node[below right]{$\vec{N}$};
    \draw[->, thick] (0.5,0) -- +(-0.573576,0) node[above]{$\vec{f}_{at}$};
    
    \draw[dashed] (2.5,0) rectangle (3.5,1);

    \draw[->, thick](1,0.5) -- node[above]{$\vec{d}$} +(2,0);
    \draw[dashdotted, ->](0,0.5) -- (4,0.5) node[below left]{$x$};     
        
    \draw[dashed] (4,0) -- +(-145:1) coordinate (A);
    
    \coordinate (B) at (4,0);
    \coordinate (C) at (0,0);
    
    \pic [draw, "$\alpha$", angle eccentricity=1.5] {angle = C--B--A};

\end{tikzpicture}
\caption{Bloco que se desloca sujeito à força de atrito sobre um plano inclinado. Note que o atrito e o deslocamento são em sentidos opostos do mesmo eixo $x$. \label{Fig:TrabalhoAtrito}}
\end{marginfigure}

\noindent{}Como $\theta = \degree{180}$, obtemos
\begin{equation}
  W_{\fat} = -\fat d.
\end{equation}
%
Os módulos dos vetores $\vecfat$ e $\vec{d}$ são positivos, logo, temos que $W_{\fat}$ é negativo.

Em geral, quando se pensa em situações envolvendo o atrito, nos vêm à mente situações como a discutida acima e ficamos com a impressão de que o trabalho efetuado por essa força é sempre negativo. Isso não é verdade: se temos um corpo sobre uma esteira, sendo acelerado por ela, verificamos através de um diagrama de corpo livre ---~Figura~\ref{Fig:TrabalhoCorpoAceleradoPeloAtrito}~--- que a força de atrito, que tem a mesma direção da aceleração, será no mesmo sentido do deslocamento. Nesse caso, temos que o trabalho realizado pela força de atrito deve ser \emph{positivo}, pois tal força é responsável por \emph{aumentar} a energia cinética.
\begin{marginfigure}
\centering
\begin{tikzpicture}[>=Stealth,
     interface/.style={
        % superfície
        postaction={draw,decorate,decoration={border,angle=-45,
                    amplitude=0.2cm,segment length=2mm}}},
    ]
    
    \draw[interface] (-2,0) -- (2,0);
    
    \draw[pattern= north west lines] (-0.5,0) rectangle (0.5,1);
    \draw[fill] (0,0.5) circle (1pt);
    
    \draw[->, thick] (0,0.5) -- node[above]{$\vec{d}$} +(1.5,0);
    
    \draw[->, thick] (0, 0.5) -- +(0,-1) node[left]{$\vec{P}$};
    \draw[->, thick] (0,1) -- +(0,1) node[left]{$\vec{N}$};
    
    \draw[->, thick] (0.5,0) -- node[below]{$\vec{f}_{at}$} +(0.75,0);
    
    \draw[->, dashed] (-2,0.5) -- (2,0.5) node[above]{$x$};
    \draw[->, dashed] (0, -1) -- (0, 2.5) node[right]{$y$};
    
    \draw[->] (0.25, -1) -- node[below]{$\vec{a}$} +(1,0);

\end{tikzpicture}
\caption{Bloco apoiado sobre uma superfície que se desloca para a direita com aceleração $\vec{a}$. Note que a força de atrito é na mesma direção que o deslocamento e por isso o trabalho realizado pelo atrito é positivo. \label{Fig:TrabalhoCorpoAceleradoPeloAtrito}}
\end{marginfigure}

    \item[Força peso:] Para determinarmos o trabalho realizado pela força peso, basta utilizarmos a Expressão~\eqref{Eq:TrabalhoPeso}:\footnote{Estamos assumindo que o eixo $y$ aponta para baixo.}
    \begin{align}
        W_P &= mg\Delta y \\
        &= mgd\cos\alpha.
    \end{align}
\end{description}

Note que em geral não temos uma expressão fechada que nos dá o trabalho, como no caso da força peso\footnote{Veremos adiante que as forças que possuem expressões como a~\eqref{Eq:TrabalhoPeso} são especiais.}. Se não temos expressões fechadas, devemos então utilizar a Expressão~\eqref{Eq:TrabalhoForcaConstante}, como fizemos acima\footnote{Assumindo que elas são constantes, se não forem verificaremos adiante o que fazer.}. Outras forças ---~como, por exemplo, a tensão e o arrasto~--- também não têm expressões fechadas, e deveremos proceder da mesma maneira que fizemos para a normal e o atrico.

Quando tratamos a força de arrasto, em particular, também temos a impressão de que o trabalho é sempre negativo, o que é incorreto. Quando um paraquedista cai, chegando à sua velocidade terminal, claramente temos um trabalho negativo, pois a força de arrasto tem direção contrária ao deslocamento. Porém se soprarmos uma bola de ping-pong, ela ganhará velocidade. Temos então que a força de arrasto, que tem a mesma direção e sentido que o deslocamento, realiza um trabalho positivo.

%%%%%%%%%%%%%%%%%%%%%%%%%%%%%%%%%%%%
\paragraph{Velocidade de um pêndulo}
%%%%%%%%%%%%%%%%%%%%%%%%%%%%%%%%%%%%

Podemos voltar agora ao problema de determinar a velocidade do pêndulo discutida no início do capítulo. Sabemos que a variação da energia cinética está associada ao trabalho total realizado pelas diversas forças:
\begin{equation*}
    \Delta K = W_t.
\end{equation*}
%
No caso do pêndulo, temos duas forças: a força peso e a tensão no fio.

\begin{marginfigure}
\centering
\begin{tikzpicture}[>=Stealth, scale = 1.2]

    \draw[|-|] (0,0.5) -- node[above]{$L$} +(180:20mm);
    
    \draw[fill] (0,0) circle (1pt);
    \draw[dotted] ([shift={(0,0)}]180:2) arc[radius=2, start angle=180, end angle= 290];
    
    \draw[densely dotted] (0,0) -- (180:18mm);
    \draw[pattern = north west lines, dotted] (180:2) circle (2mm);
    \draw[->, thick, dotted] (180:2) +(0,-0.2) -- +(0,-0.7) node[left]{$\vec{P}$};
    
    \draw[densely dotted] (0,0) -- (-150:18mm);
    \draw[pattern = north west lines, dotted] (-150:2) circle (2mm);
    \draw[->, thick, dotted] (-150:2) +(0,-0.2) -- +(0,-0.7) node[left]{$\vec{P}$};
    \draw[->, thick, dotted] (-150:18mm) -- (-150:14mm) node[above left]{$\vec{T}$};
    
    \draw[densely dotted] (0,0) -- (-120:18mm);
    \draw[pattern = north west lines, dotted] (-120:2) circle (2mm);
    \draw[->, thick, dotted] (-120:2) +(0,-0.2) -- +(0,-0.7) node[left]{$\vec{P}$};
    \draw[->, thick, dotted] (-120:18mm) --  node[left]{$\vec{T}$} (-120:12mm);
    
    \draw (0,0) -- (-90:18mm);
    \draw[pattern = north west lines] (-90:2) circle (2mm);
    \draw[->] (-90:2) +(0.2,0) -- node[below]{$\vec{v}$} +(1,0);
    \draw[->, thick] (-90:2) +(0,-0.2) -- +(0,-0.7) node[left]{$\vec{P}$};
    \draw[->, thick] (-90:18mm) -- (-90:11mm) node[below right]{$\vec{T}$};
    
    \draw[<-] (-2.5,0) node[below left]{$y$} -- +(0,-2); 
    
\end{tikzpicture}
\caption{Note que a tensão, apesar de variar em módulo e direção a cada instante, é sempre perpendicular ao deslocamento instantâneo (que se dá ao longo da curva pontilhada).}
\end{marginfigure}


Já vimos que o trabalho da força peso está ligado à distância percorrida verticalmente:
\begin{equation*}
    W_g = -mg\Delta y,
\end{equation*}
%
onde o eixo $y$ cresce verticalmente para cima. Já no caso da tensão, apesar de ela variar em módulo e direção a cada instante, \emph{sua direção é sempre perpendicular ao deslocamento}, o que implica em um trabalho nulo devido a essa força em todo o movimento. Assim, temos
\begin{equation}
    \frac{1}{2}m v_f^2 - \frac{1}{2}m v_i^2 = -mg\Delta y,
\end{equation}
%
ou, se considerarmos que a velocidade inicial é nula e que o deslocamento no eixo vertical é $\Delta y = - L$,
\begin{align}
    \frac{1}{2} m v_f^2 &= mgL \\
    v_f^2 &= 2gL \\
    v_f &= \sqrt{2gL}.
\end{align}

%%%%%%%%%%%%%%%%%%%%%%%%%%%%%%%%%%%%%%%%%%%%%%%%%%%%%
\paragraph{Exemplo: Trabalho efetuado sobre um bloco}
%%%%%%%%%%%%%%%%%%%%%%%%%%%%%%%%%%%%%%%%%%%%%%%%%%%%%

\begin{quote}
    Na Figura~\ref{Fig:Ex:TrabalhoBlocoArrastadoRampa}, uma força $\vec{F}$ cujo módulo é de \np[N]{100} é aplicada horizontalmente sobre um bloco de massa $m = \np[kg]{2,0}$ que está apoiado em uma rampa cujo ângulo $\alpha$ em relação à horizontal é de \degree{35}. O coeficiente de atrito cinético entre a superfície inferior do bloco e a rampa é $\mu_c = \np{0,6}$. Determine o trabalho realizado pelas forças que atuam no sistema, assim como a velocidade final do bloco, assumindo que ele partiu do repouso e que a distância $d$ percorrida é de \np[m]{5,0}.
\end{quote}

\begin{marginfigure}
\centering
\begin{tikzpicture}[>=Stealth, rotate = -35,
     interface/.style={
        % superfície
        postaction={draw,decorate,decoration={border,angle=-45,
                    amplitude=0.2cm,segment length=2mm}}},
    ]
      
    \draw[interface, gray] (0,0) -- (4,0);
    \draw[dashed] (0.5,0) rectangle (1.5,1);
           
    \draw[pattern = dots] (2.5,0) rectangle (3.5,1);
    \draw[<-, thick] (3.5,1) -- node[above]{$\vec{F}$} +(35:1);
    \draw[fill] (3,0.5) circle (1pt);

    \draw[|-|](1,2) -- node[above]{$d$} +(2,0);

\end{tikzpicture}
\caption{Bloco que se desloca rampa a cima devido a ação de uma força $\vec{F}$.\label{Fig:Ex:TrabalhoBlocoArrastadoRampa}}
\end{marginfigure}

\begin{marginfigure}
\centering
\begin{tikzpicture}[>=Stealth, rotate = -35,
     interface/.style={
        % superfície
        postaction={draw,decorate,decoration={border,angle=-45,
                    amplitude=0.2cm,segment length=2mm}}},
    ]
      
    \draw[interface, gray] (0,0) -- (4,0);
    \draw[dashed] (0.5,0) rectangle (1.5,1);
           
    \draw[pattern = dots] (2.5,0) rectangle (3.5,1);
    \draw[<-, thick] (3.5,1) -- node[above]{$\vec{F}$} +(35:1);
    \draw[fill] (3,0.5) circle (1pt);
    \draw[->, thick] (3,0.5) -- +(-55:1) node[left]{$\vec{P}$};
    \draw[->, thick] (3,1) -- +(0,0.81915) node[below right]{$\vec{N}$};
    \draw[->, thick] (3.5,0) -- +(0.573576,0) node[above right]{$\vec{f}_{at}$};

    \draw[|-|](1,2) -- node[above]{$d$} +(2,0);

\end{tikzpicture}
\caption{Forças que atuam no problema.\label{Fig:Ex:TrabalhoBlocoArrastadoRampaForcas}}
\end{marginfigure}

As forças que atuam no problema, além da força $\vec{F}$, são a força peso $\vec{P}$, a força normal $\vec{N}$, e a força de atrito $\vec{f}_{\rm{at}}$. Na Figura~\ref{Fig:Ex:TrabalhoBlocoArrastadoRampaForcas} temos as direções e sentidos em que cada uma delas atua. Para determinarmos o trabalho, basta utilizarmos a expressão para o trabalho de uma força constante, Equação~\ref{Eq:TrabalhoForcaConstante}:
\begin{description}

    \item[Força $\vec{F}$:] O trabalho da força $\vec{F}$ pode ser calculado através de
    \begin{equation}
        W_F = \vec{F}\cdot\vec{d}.
    \end{equation}
    %
    Considerando que a força é na direção horizontal, e que o deslocamento é paralelo à superfície do plano inclinado, o ângulo entre os vetores força e deslocamento é o mesmo que entre o a horizontal e o plano. Logo,
    \begin{align}
        W &= Fd\cos\alpha \\
        &= (\np[N]{100})\cdot(\np[m]{5})\cdot(\cos\degree{35}) \\
        &= \np[J]{409,6}.
    \end{align}
    
    \item[Força normal:] Sabemos que o deslocamento do bloco é paralelo à superfície do plano inclinado. Como a força normal é perpendicular a superfície, ela também é perpendicular ao deslocamento. Logo,
    \begin{align}
        W_N &= \vec{F}\cdot\vec{d} \\
        &= \vec{N}\cdot\vec{d} \\
        &=0.
    \end{align}
    
\begin{marginfigure}
\centering
\begin{tikzpicture}[>=Stealth, rotate = -35,
     interface/.style={
        % superfície
        postaction={draw,decorate,decoration={border,angle=-45,
                    amplitude=0.2cm,segment length=2mm}}},
    ]
      
    \draw[interface, gray] (2,0) -- (4,0);
           
    \draw[pattern = dots] (2.5,0) rectangle (3.5,1);
    \draw[<-, thick] (3.5,1) -- node[above]{$\vec{F}$} +(35:1);
    \draw[fill] (3,0.5) circle (1pt);
    \draw[->, thick] (3,0.5) -- +(-55:1) node[left]{$\vec{P}$};
    \draw[->, thick] (3,1) -- +(0,0.81915) node[below right]{$\vec{N}$};
    \draw[->, thick] (3.5,0) -- +(0.573576,0) node[below right]{$\vec{f}_{at}$};
    
    \draw[dashdotted, ->] (2,0.5) -- (4,0.5) node[below] {$x$};
    \draw[dashdotted, ->] (3,-1.25) -- (3,2.5) node[right] {$y$};

\end{tikzpicture}
\caption{Referencial para a determinação da força normal.\label{Fig:Ex:TrabalhoBlocoArrastadoRampaForcasRef}}
\end{marginfigure}

    \item[Atrito:] O trabalho do atrito pode ser calculado através de
    \begin{align}
        W_{f_{\rm{at}}} &= \vec{F}\cdot\vec{d} \\
        &= \vec{f}_{\rm{at}}\cdot\vec{d}.
    \end{align}
    %
    A força de atrito cinético é dada por $f_{\rm{at}} = \mu_c N$, e a normal pode ser determinada aplicando a Segunda Lei de Newton ao eixo $y$ (Figura~\ref{Fig:Ex:TrabalhoBlocoArrastadoRampaForcasRef}):
    \begin{description}
        \item[Eixo $y$:]
            \begin{align}
                F_R^y &= m a_y \\
                N + P_y + F_y &= 0 \\
                N &= mg\cos\alpha + F\sen\alpha.
            \end{align}
    \end{description}
    %
    Além disso, sabemos que podemos determinar o trabalho considerando somente a componente da força que aponta na direção do deslocamento, logo
    \begin{align}
        W_{f_{\rm{at}}} &= -[f_{\rm{at}}]_{\parallel} d \\
        &= - [\mu(mg\cos\alpha + F\sen\alpha)\cos\alpha]d \\
        &= -\big[(\np{0,6})\cdot\big((\np[km]{2,0})\cdot(\np[m/s^2]{9,8})\cdot(\cos\degree{35}) \nonumber\\
        &\phantom{=-\big[(\np{0,6})\cdot\big[}+ (\np[N]{100})\cdot(\sen\degree{35})\big)\cdot(\cos\degree{35})\big] \nonumber\\
        &\phantom{=-\big[}\cdot(\cos\degree{35}) \\
        &= -\np[J]{180,4}.
    \end{align}
    
\end{description}

\begin{marginfigure}
\centering
\begin{tikzpicture}[>=Stealth, rotate = -35,
     interface/.style={
        % superfície
        postaction={draw,decorate,decoration={border,angle=-45,
                    amplitude=0.2cm,segment length=2mm}}},
    ]
      
    \draw[interface, gray] (0,0) -- (4,0);
    \draw[dashed] (0.5,0) rectangle (1.5,1);
           
    \draw[pattern = dots] (2.5,0) rectangle (3.5,1);
    \draw[<-, thick] (3.5,1) -- node[above]{$\vec{F}$} +(35:1);
    \draw[fill] (3,0.5) circle (1pt);
    \draw[->, thick] (3,0.5) -- +(-55:1) node[left]{$\vec{P}$};
    \draw[->, thick] (3,1) -- +(0,0.81915) node[below right]{$\vec{N}$};
    \draw[->, thick] (3.5,0) -- +(0.573576,0) node[below right]{$\vec{f}_{at}$};

    \draw[|-|](1,2) -- node[above]{$d$} +(2,0);
    
    \draw[dashdotted, <-] (5.5,0.5) node[above left]{$y'$} -- +(125:3);

\end{tikzpicture}
\caption{Referencial vertical para a determinação do trabalho da força peso.\label{Fig:Ex:TrabalhoBlocoArrastadoRampaForcasRefVert}}
\end{marginfigure}

Para determinarmos o trabalho da força peso, podemos utilizar a Expressão~\ref{Eq:TrabalhoPeso}, bastando para isso adotar um eixo $y'$ \emph{vertical}:
\begin{description}
    \item[Força peso:] 
    \begin{align}
        W_P &= mg\Delta y' \\
        &= mg (-d\cos\alpha) \\
        &= (\np[kg]{2,0})\cdot(\np[m/s^2]{9,8})\cdot(-(\np[m]{5,0})\cdot(\cos\degree{35})) \\
        &= -\np[J]{80,3}.
    \end{align}
\end{description}
%
Note que $\Delta y' = - d\cos\degree{35}$, pois o deslocamento é no sentido negativo do eixo $y'$.

Finalmente, podemos determinar o trabalho total:
\begin{align}
    W &= W_F + W_N + W_P + W_{f_{\rm{at}}} \\
    &= \np[J]{148,9},
\end{align}
%
o que implica em uma velocidade final dada por
\begin{align}
    \Delta K &= W \\
    \frac{1}{2} mv_f^2 - \frac{1}{2} m v_i^2 &= W \\
    v_f &= \sqrt{\frac{2W}{m}},
\end{align}
%
onde usamos o fato de que a velocidade inicial é nula. Substituindo os valores para o trabalho e para a massa, obtemos
\begin{align}
    v_f &= \sqrt{\frac{2W}{m}} \\
    &= \sqrt{\frac{2\cdot(\np[J]{148,9})}{(\np[kg]{2,0})}} \\
    &= \np[m/s]{12,2}.
\end{align}

%%%%%%%%%%%%%%%%%%%%%%%%%%%%%%%%%%%%%%%%%%%%%%%%%%%%%%%%%%
\paragraph{Exemplo: Força necessária para parar um objeto}
%%%%%%%%%%%%%%%%%%%%%%%%%%%%%%%%%%%%%%%%%%%%%%%%%%%%%%%%%%

\begin{quote}
    Suponha que um corpo com massa de \np[kg]{1.0} se desloque em queda livre. Em um certo instante, quando sua velocidade é de \np[m/s]{3,0}, uma força $\vec{F}$ constante passa a atuar verticalmente para cima. Qual é a intensidade da força, se ao final de um metro o corpo tem velocidade nula?
\end{quote}

Analisando o momento a partir do qual o corpo passa a sofrer a influência da força que o para, sabemos que o corpo tem uma energia cinética inicial dada por
\begin{align}
    K_i &= \frac{1}{2}mv_i^2 \\
    &= \frac{1}{2} \cdot(\np[kg]{1,0})\cdot(\np[m/s]{3,0})^2 \\
    &= \np[J]{4,5}.
\end{align}
%
Sua energia cinética final será nula, uma vez que o corpo pára ao final de um metro. Portanto,
\begin{align}
    \Delta K &= K_f - K_i \\
    &= -\np[J]{4,5}.
\end{align}
%
O trabalho total durante o deslocamento do corpo será dado por
\begin{align}
    W &= W_F + W_P \\
    &= \vec{F}\cdot\vec{d} + mg\Delta y,
\end{align}
%
onde o eixo $y$ aponta verticalmente para baixo. Sabemos que o ângulo entre a força $\vec{F}$ e o deslocamento é de \degree{180}, logo
\begin{equation}
    W_F = -Fd.
\end{equation}
%
Consequentemente, temos que
\begin{equation}
    \Delta K = -Fd + mg d,
\end{equation}
%
ou seja,
\begin{align}
    Fd &= -\Delta K + mgd \\
    F &= \frac{\Delta K}{d} + mg \\
    &= \frac{-(-\np[J]{4,5})}{(\np[m]{1,0}} + (\np[kg]{1,0})\cdot(\np[m/s^2]{9,8}) \\
    &= \np[N]{14,3}.
\end{align}

%%%%%%%%%%%%%%%%%%%%%%%%%%%%%%%%%%%%%%%%%%%%%%%%%%%%%%%%%
\section{Trabalho como a área de um gráfico $F \times x$}
%%%%%%%%%%%%%%%%%%%%%%%%%%%%%%%%%%%%%%%%%%%%%%%%%%%%%%%%%

Em um movimento unidimensional, se elaborarmos um gráfico da força $F_x$ que atua sobre um corpo em função de sua posição $x$ em tal eixo, obtemos um gráfico como o da Figura~\ref{Fig:Graf_area_graf_F_vs_x}. O trabalho efetuado pela força no deslocamento entre duas posições $x_i$ e $x_f$ pode então ser calculado como a ``área virtual'' do gráfico compreendida entre as linhas verticais que passam por $x_i$ e $x_f$ e as linhas horizontais do eixo $x$ e da força $F_x$, pois tal área é dada por

\begin{marginfigure}[4cm]
\centering
\begin{tikzpicture}[>=Stealth, extended line/.style={shorten >=-#1,shorten <=-#1},
 extended line/.default=3mm]] % talvez fosse melhor amplicar com scale=1.5
    % Draw axes: acho que o |- é pra desenhar um "canto", um L
    \draw [<->] (0,3)
        |- (4.3,0) node (xaxis) [below left] {$x$};
    % Desenhar função:
    \draw[smooth,name path=plota,samples=1000,domain=0:3]
    plot(\x,{2});
    
     \fill [pattern=north west lines, domain=0.5:2.5, variable=\x]
      (0.5, 0) node[below]{$x_i$}
      -- plot ({\x}, {2})
      -- (2.5, 0) node[below]{$x_f$}
      -- cycle;
      
      \draw[dashed] (0.5, 0) -- (0.5, 2);
      \draw[dashed] (2.5, 0) -- (2.5, 2);
      \path (0, 2) node[left]{$F_x$};
      
      \draw[|-|] (3.2, 0) -- node[right]{$F_x$} (3.2, 2);
      \draw[|-|] (0.5, -0.6) -- node[below]{$\Delta x$} (2.5, -0.6);
     
\end{tikzpicture}
\caption{A área hachurada está relacionada ao trabalho em um movimento sujeito a uma força $\vec{F}$. Note que o gráfico expressa somente o valor da componente da força na direção do movimento.\label{Fig:Graf_area_graf_F_vs_x}}
\end{marginfigure}

\begin{align}
  A &= \textrm{base} \times \textrm{altura} \\
  &= \Delta x \times F_x \\
  &= W_{F_x}.
\end{align}
%
Este artifício é útil para calcularmos o trabalho realizado por forças que não são constantes, bastando que tenhamos uma maneira de calcular a área do gráfico.

%%%%%%%%%%%%%%%%%%%%%%%%%%%%%%%%%%%%%%%%%%%%%%%%%%%%%%
\subsection{Trabalho realizado por uma força elástica}
%%%%%%%%%%%%%%%%%%%%%%%%%%%%%%%%%%%%%%%%%%%%%%%%%%%%%%

Um exemplo de força que não é constante e cujo trabalho estamos interessados em calcular é a força elástica. A força exercida por uma mola varia conforme ela é distendida segundo a expressão
\begin{equation}
  F = -k x,
\end{equation}
%
o que resulta em um gráfico como o da Figura~\ref{Fig:TrabalhoForcaElastica}. Se um corpo submetido a essa força sofre um deslocamento entre as posições $x_i$ e $x_f$, temos que a área do gráfico, será dada pela diferença entre o triângulo maior ($OAx_i$) e o triângulo menor ($OBx_f$). Portanto, o trabalho será 
\begin{equation}
  W_{F_e} = \frac{x_i F(x_i)}{2} - \frac{x_f F(x_f)}{2}
\end{equation}

\begin{marginfigure}[-5cm]
\centering
\begin{tikzpicture}[>=Stealth,
     interface/.style={
        % superfície
        postaction={draw,decorate,decoration={border,angle=-45,
                    amplitude=0.2cm,segment length=2mm}}},
    ]
    
    \draw[interface] (0,-2.5) -- (0,-1);
    \draw[interface] (-4.8,-2.5) -- (0, -2.5);
    
    \draw (0,-2) -- (-0.2,-2);
    \draw[decoration={aspect=0.3, segment length=2.5625mm, amplitude=2mm,coil},decorate] (-0.2,-2) -- (-3.3,-2);
    \draw (-3.3, -2) -- (-3.5,-2);
    
    \draw[pattern = north west lines] (-3.5,-2.5) rectangle (-4.5,-1.5);
    \draw[dotted, pattern = north west lines] (-1.5,-2.5) rectangle (-2.5,-1.5);
    
    \draw[fill] (-4,-2) circle (1pt);
    \draw[->, thick] (-4,-2) -- +(0,-1) node[right]{$\vec{P}$};
    \draw[->, thick] (-4,-1.5) -- node [right]{$\vec{N}$} +(0,1);
    \draw[->, thick] (-3.5, -2) -- node[above left]{$\vec{F}_e$} +(1,0);
    
    %%
    
    \draw[<-] (0,0.5) node[below left]{$x$}-- (-4.8,0.5);
    \draw[->] (-4.5,0) -- (-4.5,3) node[below left]{$F$};
    
    \draw (-4.25, 2.5) -- (-0.5,0.2);
    \node[above right] (O) at (-0.9892,0.5) {$O$};
    \path[pattern = north west lines] (-4,2.3465) -- (-4,0.5) -- (-2,0.5) -- (-2,1.1199) -- cycle;
    \draw[dashed, thick, fill] (-4,2.3465) circle (1pt) node[above right]{$A$} -- (-4,0.5) circle (1pt) node[below]{$x_i$};
    \draw[dashed, thick, fill] (-2,1.1199) circle (1pt) node[above right]{$B$} -- (-2,0.5) circle (1pt) node[below]{$x_f$};
        
\end{tikzpicture}
\caption{O trabalho realizado pela mola no deslocamento entre os pontos $x_i$ e $x_f$ é dado pela área hachurada no gráfico. \label{Fig:TrabalhoForcaElastica}}
\end{marginfigure}

\noindent{}onde usamos o fato de que a altura dos triângulos é dada por $y = F(x)$. Para o cálculo da área, estamos interessados nas distâncias verticais e horizontais, por isso vamos utilizar todos os valores em módulo. Isso implica que utilizaremos a força em módulo: $|F(x)| = kx$. Assim, obtemos
\begin{equation}
  W_{F_e} = \frac{1}{2} k x_i^2 - \left(\frac{1}{2} k x_f^2 \right)
\end{equation}
%
ou
\begin{equation}\label{Eq:TrabalhoForcaElastica}
  W_{F_e} = - \frac{1}{2} k (x_f^2 - x_i^2). \mathnote{Trabalho realizado por uma força elástica}
\end{equation}

Note que se o deslocamento no problema acima fosse no sentido oposto, o trabalho seria negativo, já que ele tende a diminuir a energia cinética. No entanto, a expressão acima para o trabalho continua sendo válida, pois trocamos os pontos inicial e final. Quando fizemos a interpretação da área de um gráfico na cinemática, sempre tinhamos um deslocamento no sentido positivo do eixo horizontal --~pois ele representa o tempo, que sempre cresce~--, e as áreas acima do eixo horizontal representavam quantidades positivas, enquanto áreas abaixo do eixo representavam quantidades negativas\footnote{No caso de um gráfico da velocidade em função do tempo, por exemplo, a área corresponde ao deslocamento. Se a curva $v(t)$ passa para a parte negativa do eixo vertical, ou seja, abaixo do eixo horizontal, temos um deslocamento no sentido negativo do eixo.}.

No caso do trabalho, no entanto, podemos tanto ter deslocamentos no sentido positivo do eixo, quanto no sentido negativo. No caso de termos um deslocamento no sentido positivo do eixo, o trabalho terá um valor positivo se a área estiver acima do eixo e um valor negativo se estiver abaixo do eixo horizontal. Em um deslocamento no sentido negativo do eixo horizontal, no entanto, essa relação se inverte: se a área estiver acima do eixo horizontal, ela representa um trabalho negativo, enquanto se estiver abaixo, ela representa um trabalho positivo. Para todos os casos, todavia, a Expressão~\eqref{Eq:TrabalhoForcaElastica} acima é válida.

%%%%%%%%%%%%%%%%%%%%%%%%%%%%%%%%%%%%%%%%%%%%%%%%%%%%%%%%%%%%%%%%%%%%
\paragraph{Exemplo: Queda de um bloco sobre uma mola}
%%%%%%%%%%%%%%%%%%%%%%%%%%%%%%%%%%%%%%%%%%%%%%%%%%%%%%%%%%%%%%%%%%%%

\begin{quote}
    Um bloco de massa $m = \np[kg]{0,5}$ cai sobre uma mola, conforme mostrado na Figura~\ref{Fig:ex:QuedaBlocoSobreMola}. Se a constante elástica é de \np[N/m]{50} e o bloco tem uma velocidade de \np[m/s]{5,0} imediatamente antes de tocar a mola, determine a distensão máxima em relação à posição em que ela se encontrava relaxada.
\end{quote}

\begin{marginfigure}[-5cm]
\centering
\begin{tikzpicture}[>=Stealth,
     interface/.style={
        % superfície
        postaction={draw,decorate,decoration={border,angle=-45,
                    amplitude=0.2cm,segment length=2mm}}},
    ]
    
    \draw[interface] (-2,0) -- (2,0);

    \draw (0,0) -- (0,0.1);
    \draw[decoration={aspect=0.3, segment length=2mm, amplitude=2mm,coil},decorate] (0,0.1) -- (0,2.5);
    
    \draw[pattern = north east lines] (-0.75,2.5) rectangle (0.75,2.7);
    \draw[pattern = north west lines] (-0.5, 3) rectangle (0.5,4);
    
    \draw[->] (-0.75, 3.75) -- node[left]{$\vec{v}$} (-0.75, 3.25);
    \draw[->] (1.5, 4.5) -- (1.5,2) node[above right]{$y$};

        
\end{tikzpicture}
\caption{Queda de um bloco sobre uma mola disposta verticalmente.\label{Fig:ex:QuedaBlocoSobreMola}}
\end{marginfigure}

Sabemos que
\begin{align}
    \Delta K &= W \\
    &= W_P + W_{F_e} \\
    &= mg\Delta y - \frac{k}{2}(y_f^2 - y_i^2),
\end{align}
%
onde estamos descrevendo a distensão da mola através do próprio eixo $y$ vertical utilizado para determinar o trabalho realizado pela força peso. Na posição em que temos a distensão máxima, o bloco se encontra parado momentaneamente. Logo, podemos assumir que a velocidade final é nula. Assim
\begin{align}
    \Delta K &= mg\Delta y - \frac{k}{2}(y_f^2 - y_i^2) \\
    K_f - K_i &= mg\Delta y - \frac{k}{2}(y_f^2 - y_i^2) \\
    -\frac{1}{2} mv_i^2 &= mg\Delta y - \frac{k}{2}(y_f^2 - y_i^2).
\end{align}
%
Podemos simplificar a expressão acima se escolhermos $y_i = 0$, obtendo
\begin{align}
    -\frac{1}{2} mv_i^2 &= mg y_f - \frac{k}{2}y_f^2 \\
    -mv_i^2 &= 2 mg y_f - k y_f^2 \\
    k y_f^2 - 2 mg y_f - mv_i^2 &= 0.  
\end{align}
%
Resolvendo a equação de segundo grau acima, obtemos
\begin{align}
    y_f &= \frac{-(-2mg) \pm \sqrt{(-2mg)^2 - 4\cdot(k)\cdot(-mv_i^2)}}{2k}\\
    &= \np[m]{0,61}.
\end{align}

%%%%%%%%%%%%%%%%%%%%%%%%%%%%%%%%%%%%%%%%
\section{Trabalho de uma força variável}
%%%%%%%%%%%%%%%%%%%%%%%%%%%%%%%%%%%%%%%%

Verificamos ao discutir o cálculo da área sob uma curva ao ver o movimento unidimensional que podemos determinar a área sob uma curva através de um método aproximativo, que consiste em dividir a área em uma série de retângulos. Esse tipo de aproximação é bastante simples de se fazer utilizando um computador, e também bastante precisa, já que podemos utilizar um número muito grande de retângulos. Esse tipo de procedimento é denominado como \emph{integração por quadratura}. Existem outros métodos numéricos que, com base em uma quadratura, conseguem eliminar alguns erros inerentes a esse tipo de aproximação e determinam valores relativamente bons e com poucos pontos de avaliação da função. No entanto, podemos determinar \emph{exatamente} o valor da área sob uma curva $f(x)$ entre dois limites $x_i$ e $x_f$ ---~o que denominamos como \emph{integral definida}~--- se utilizarmos o \emph{Teorema Fundamental do Cálculo}.

\begin{marginfigure}[-3cm]
\centering
\begin{tikzpicture}[>=Stealth, extended line/.style={shorten >=-#1,shorten <=-#1},
 extended line/.default=3mm]] % talvez fosse melhor amplicar com scale=1.5
    % Draw axes: acho que o |- é pra desenhar um "canto", um L
    \draw [<->] (0,3) node (yaxis) [below left] {$F_x$}
        |- (4.3,0) node (xaxis) [below left] {$x$};
    % Desenhar função:
    \draw[smooth,name path=plota,samples=1000,domain=0:3.5]
    plot(\x,{1.440476 - 1.25*\x + 1.47619*\x^2 - 0.3333333*\x^3});

    \coordinate (a) at (0.25,0);
    \coordinate (b) at (2.75,0);
    \path[name path=froma](a)--+(0,3);
    \path[name path=fromb](b)--+(0,3);
    \draw[dashed, thick, name intersections={of=froma and plota}](a) node[below]{$x_i$} -- (intersection-1);
	\draw[dashed, thick, name intersections={of=fromb and plota}](b) node[below]{$x_f$} -- (intersection-1);

    \fill [pattern=north west lines, domain=0.25:2.75, variable=\x]
     	  (0.25, 0)
    	  -- plot ({\x}, {1.440476 - 1.25*\x + 1.47619*\x^2 - 0.3333333*\x^3})
          -- (2.75, 0)
          -- cycle;
          
    \node (f) at (3.5,2) {$F_x(x)$};
\end{tikzpicture}
\caption{No caso de uma força cuja componente na direção do movimento $F_x(x)$ varie de uma forma complexa, podemos determinar o trabalho utilizando uma integral.}
\end{marginfigure}

%%%%%%%%%%%%%%%%%%%%%%%%%%%%%%%%%%%%%%%%%%%
\subsection{Teorema fundamental do cálculo}
%%%%%%%%%%%%%%%%%%%%%%%%%%%%%%%%%%%%%%%%%%%

\begin{marginfigure}[2cm]
\centering
\begin{tikzpicture}[>=Stealth, extended line/.style={shorten >=-#1,shorten <=-#1},
 extended line/.default=3mm]] % talvez fosse melhor amplicar com scale=1.5
    % Draw axes: acho que o |- é pra desenhar um "canto", um L
    \draw [<->] (0,3)
        |- (4.3,0) node (xaxis) [below left] {$x$};
    % Desenhar função:
    \draw[smooth,name path=plota,samples=1000,domain=0:3.5]
    plot(\x,{1.440476 - 1.25*\x + 1.47619*\x^2 - 0.3333333*\x^3});

    \coordinate (a) at (0.25,0);
    \coordinate (b) at (2.75,0);
    \coordinate (c) at (2.95,0);
    \path[name path=fromc](c)--+(0,3);
    \path[name path=froma](a)--+(0,3);
    \path[name path=fromb](b)--+(0,3);
    \draw[dashed, thick, name intersections={of=froma and plota}](a) node[below]{$a$} -- (intersection-1);
	\draw[dashed, thick, name intersections={of=fromb and plota}](b) node[below]{$x$} -- (intersection-1);
	
    \fill [dotted, pattern color = gray, pattern=north west lines, domain=0.25:2.75, variable=\x]
     	  (0.25, 0)
    	  -- plot ({\x}, {1.440476 - 1.25*\x + 1.47619*\x^2 - 0.3333333*\x^3})
          -- (2.75, 0)
          -- cycle;
          
    \node (f) at (3.5,2) {$f(x)$};
    \node (A) at (1.5,0.65) {$A_{f(x)}(x)$};
\end{tikzpicture}
\caption{Definimos a função $A_{f(x)}(x)$ como sendo a função que dá a área delimitada pela curva $f(x)$, o eixo $x$, e os limites verticais em $a$ e $x$.\label{Fig:TeoremaFundamentalDoCalculoArea}}
\end{marginfigure}

Vamos considerar uma função
\begin{equation}
  A_{f(x)}(x),
\end{equation}
%
cuja interpretação é \emph{a área contida entre a curva $f(x)$, o eixo $x$ e os limites verticais que passam em $a$ e $x$}. Na Figura~\ref{Fig:TeoremaFundamentalDoCalculo}, por exemplo, a área pontilhada é dada por $A_{f(x)}(x_p)$. Se calcularmos a derivada de $A_{f(x)}(x)$ no ponto $x = x_p$ através da definição, temos
\begin{equation}
  \left.A'_{f(x)}(x)\right|_{x = x_p} = \lim_{\ell \to 0} \frac{A_{f(x)}(x_p+\ell) - A_{f(x)}(x_p)}{\ell}.
\end{equation}

\noindent{}A diferença $A_{f(x)}(x_p+\ell) - A_{f(x)}(x_p)$ é simplesmente a área hachurada mostrada na Figura~\ref{Fig:TeoremaFundamentalDoCalculo} e podemos aproximá-la por um retângulo cuja área é dada por $\ell f(x_p)$, o que resulta em:
\begin{align}
  \left.A'_{f(x)}(x)\right|_{x = x_p} &= \lim_{\ell \to 0} \frac{\ell f(x_p)}{\ell} \\
  &= \lim_{\ell \to 0} f(x_p) \\
  &= f(x_p).
\end{align}
Logo, \emph{o valor da derivada de $A_{f(x)}(x)$ no ponto $x_p$ é o próprio valor da função $f(x)$ no ponto $x_p$}. Este resultado é muito importante, pois ele nos dá uma forma de determinar a função $A_{f(x)}(x)$: basta determinarmos a função $\mathcal{F}(x)$ tal que
\begin{equation}
    \mathcal{F}'(x) = f(x).
\end{equation}

\begin{marginfigure}[-6cm]
\centering
\begin{tikzpicture}[>=Stealth, extended line/.style={shorten >=-#1,shorten <=-#1},
 extended line/.default=3mm]] % talvez fosse melhor amplicar com scale=1.5
    % Draw axes: acho que o |- é pra desenhar um "canto", um L
    \draw [<->] (0,3)
        |- (4.3,0) node (xaxis) [below left] {$x$};
    % Desenhar função:
    \draw[smooth,name path=plota,samples=1000,domain=0:3.5]
    plot(\x,{1.440476 - 1.25*\x + 1.47619*\x^2 - 0.3333333*\x^3});

    \coordinate (a) at (0.25,0);
    \coordinate (b) at (2.75,0);
    \coordinate (c) at (2.95,0);
    \path[name path=fromc](c)--+(0,3);
    \path[name path=froma](a)--+(0,3);
    \path[name path=fromb](b)--+(0,3);
    \draw[dashed, thick, name intersections={of=froma and plota}](a) node[below]{$a$} -- (intersection-1);
	\draw[dashed, thick, name intersections={of=fromb and plota}](b) node[below]{$x_p$} -- (intersection-1);
	\draw[dashed, thick, name intersections={of=fromc and plota}](c) -- (intersection-1);
	
	\draw[|-|] (2.75,-0.5) -- node[below]{$\ell$} +(0.2,0);

    \fill [dotted, pattern color = gray, pattern=north west lines, domain=0.25:2.75, variable=\x]
     	  (0.25, 0)
    	  -- plot ({\x}, {1.440476 - 1.25*\x + 1.47619*\x^2 - 0.3333333*\x^3})
          -- (2.75, 0)
          -- cycle;
          
    \fill [pattern=north west lines, domain=2.75:2.95, variable=\x]
     	  (2.75, 0)
    	  -- plot ({\x}, {1.440476 - 1.25*\x + 1.47619*\x^2 - 0.3333333*\x^3})
          -- (2.95, 0)
          -- cycle;
          
    \node (f) at (3.5,2) {$f(x)$};
    \node[circle] (A) at (1.5,0.65) {$A_{f(x)}(x_p)$};
\end{tikzpicture}
\caption{Ao calcularmos a diferença $A_{f(x)}(x = x_p+\ell) - A_{f(x)}(x = x_p)$, obtemos a área da faixa de largura $\ell$ destacada.\label{Fig:TeoremaFundamentalDoCalculo}}
\end{marginfigure}

\noindent{}Note, no entanto, que existe uma \emph{família} de funções cuja derivada é igual a $f(x)$, sendo que cada membro de tal família difere por uma constante. Assim, a área contida entre o eixo vertical que passa por $a$, o eixo horizontal $x$, o eixo vertical que passa por $x_p$, e a própria função $f(x)$ é dada por\footnote[][-2cm]{Isto é, a área pontilhada na Figura~\ref{Fig:TeoremaFundamentalDoCalculo}.}
\begin{equation}
    A_{f(x)}(x_p) = \mathcal{F}(x_p) + C.
\end{equation}
 
A constante $C$ não nos permite\footnote[][-2cm]{Na verdade podemos calcular a área entre $a$ e um limite superior qualquer simplesmente escolhendo um ponto de referência mais a esquerda. Logo, o resultado  não depende da posição de $a$ e é geral.} saber exatamente o valor da área entre $a$ e $x_p$, mas tal constante não nos impede de calcular a área entre dois limites $x_i$ e $x_f$ quaisquer (Figura~\ref{Fig:TeoremaFundamentalDoCalculoIntegralDefinida}): basta determinarmos a diferença entre as áreas contidas nos intervalos $[a,x_f]$ e $[a,x_i]$. Assim,
\begin{align}
    A_{[x_i,x_f]} &= A_{f(x)}(x_f) - A_{f(x)}(x_i) \\
    &= (\mathcal{F}(x_f) + C) - (\mathcal{F}(x_i) + C) \\
    &= \mathcal{F}(x_f) - \mathcal{F}(x_i).
\end{align}

\begin{marginfigure}[-2cm]
\centering
\begin{tikzpicture}[>=Stealth, extended line/.style={shorten >=-#1,shorten <=-#1},
 extended line/.default=3mm]] % talvez fosse melhor amplicar com scale=1.5
    % Draw axes: acho que o |- é pra desenhar um "canto", um L
    \draw [<->] (0,3)
        |- (4.3,0) node (xaxis) [below left] {$x$};
    % Desenhar função:
    \draw[smooth,name path=plota,samples=1000,domain=0:3.5]
    plot(\x,{1.440476 - 1.25*\x + 1.47619*\x^2 - 0.3333333*\x^3});

    \coordinate (a) at (0.25,0);
    \coordinate (b) at (2.75,0);
    \coordinate (c) at (2.25,0);
    \coordinate (d) at (0.75,0);
    \path[name path=fromd](d)--+(0,3);
    \path[name path=fromc](c)--+(0,3);
    \path[name path=froma](a)--+(0,3);
    \path[name path=fromb](b)--+(0,3);
    \draw[dashed, name intersections={of=froma and plota}](a) node[below]{$a$} -- (intersection-1);
	\draw[dashed, name intersections={of=fromb and plota}](b) node[below]{$x_p$} -- (intersection-1);
	\draw[dashed, name intersections={of=fromd and plota}](d) node[below]{$x_f$} -- (intersection-1);
	\draw[dashed, name intersections={of=fromc and plota}](c) node[below]{$x_i$} -- (intersection-1);

    \fill [dotted, pattern color = gray, pattern=north west lines, domain=0.75:2.25, variable=\x]
     	  (0.75, 0)
    	  -- plot ({\x}, {1.440476 - 1.25*\x + 1.47619*\x^2 - 0.3333333*\x^3})
          -- (2.25, 0)
          -- cycle;
          
    \node (f) at (3.5,2) {$f(x)$};
    \node[circle] (A) at (1.55,0.65) {$A_{x_i\to x_f}$};
\end{tikzpicture}
\caption{Podemos determinar a área abaixo da curva $f(x)$ entre dois limites $x_i$ e $x_f$ quaisquer através da diferença $\mathcal{F}(x_f) - \mathcal{F}(x_i)$, onde $\mathcal{F}(x)$ é a função cuja derivada é $f(x)$.\label{Fig:TeoremaFundamentalDoCalculoIntegralDefinida}}
\end{marginfigure}

\noindent{}Portanto, \emph{para determinarmos a área abaixo da curva $f(x)$ entre dois limites $x_i$ e $x_f$, basta determinarmos a função $\mathcal{F}(x)$ cuja derivada é $f(x)$.}

A função $\mathcal{F}(x)$ é denominada como \emph{antiderivada}, \emph{função primitiva}, ou \emph{integral indefinida} de $f(x)$ e é representada por
\begin{equation}
    \mathcal{F}(x) = \int f(x) \;dx.
\end{equation}
%
Já a área sob a curva $f(x)$ entre dois limites de integração $x_i$ e $x_f$ é descrita pelo que denominamos como \emph{integral definida}
\begin{equation}
    A_{x_i\to x_f} = \int_{x_i}^{x_f} f(x) \; dx.
\end{equation}

Os resultados mostrados acima compõe o chamado \emph{Teorema Fundamental do Cálculo}, que pode ser dividido em duas partes:
\begin{description}
  \item[Teorema Fundamental do Cálculo, primeira parte:] Se $f(x)$ é uma função contínua no intervalo $[a,b]$, então a função $g(x)$ definida como
  \begin{align}
    g(x) &= \int_a^x f(\xi) \;d\xi & a&\leq x \leq b
  \end{align}
  é contínua no intervalo $[a,b]$ e diferenciável em $(a,b)$, e
  \begin{equation}
    g'(x) = f(x).
  \end{equation}
  \item[Teorema Fundamental do Cálculo, segunda parte:] Se $f(x)$ é contínua no intervalo $[a,b]$, então
  \begin{equation}
    \int_a^b f(x) \; dx = F(b) - F(a)
  \end{equation}
  onde $F(x)$ é a antiderivada de $f(x)$, isto é, a função cuja derivada é $f(x)$.
\end{description}

A determinação da integral indefinida de uma função é uma tarefa bastante complexa. Para algumas funções simples, tais resultados são tabelados, ou simples de se inferir uma vez que sejam conhecidas as derivadas. Para funções mais complexas, no entanto, são necessárias diversas técnicas que auxiliam nesse processo, mas que não são gerais, se aplicando somente a conjuntos específicos de problemas.

%%%%%%%%%%%%%%%%%%%%%%%%%%%%%%%%%%%%%%%%%%%%%%%%%%%%%%%
\paragraph{Discussão: Função primitiva de um polinômio}
%%%%%%%%%%%%%%%%%%%%%%%%%%%%%%%%%%%%%%%%%%%%%%%%%%%%%%%

Sabemos que a derivada de um polinômio $x^n$ é dada por
\begin{equation}
    \frac{d}{dx} x^n = n x^{n-1}.
\end{equation}
%
Assim, se desejamos determinar a integral indefinida de $x^m$ basta determinarmos a função que quando derivamos resulta em $x^m$. Sabemos através da regra de derivação que o expoente diminui ao derivarmos, logo, podemos tentar escrever
\begin{equation}
    \int x^{m} \;dx = x^{m+1},
\end{equation}
%
porém sabemos que ao derivarmos $x^{m+1}$ obtemos
\begin{equation}
    \frac{d}{dx} x^{m+1} = (m+1) x^{m}.
\end{equation}
%
Portanto, temos que  
\begin{equation}
    \int x^{m} \;dx = \frac{x^{m+1}}{m+1} + C.
\end{equation}

%%%%%%%%%%%%%%%%%%%%%%%%%%%%%%%%%%%%%%%%%%%%%%%%%
\paragraph{Discussão: Constantes multiplicativas}
%%%%%%%%%%%%%%%%%%%%%%%%%%%%%%%%%%%%%%%%%%%%%%%%%

Note que ao efetuarmos uma derivada, qualquer constante multiplicativa pode ser ``retirada'' da derivada:
\begin{equation}
    \frac{d}{dx} af(x) = a \frac{d}{dx} f(x).
\end{equation}
%
Consequentemente, podemos retirar constantes do processo de integração:
\begin{equation}
    \int af(x) \;dx = a \int f(x)\;dx.
\end{equation}

%%%%%%%%%%%%%%%%%%%%%%%%%%%%%%%%%%%%%%%%%%%%%%%%%%%%%%%%%%%%%%%%%%%%%
\paragraph{Exemplo: Variação da velocidade para aceleração constante}
%%%%%%%%%%%%%%%%%%%%%%%%%%%%%%%%%%%%%%%%%%%%%%%%%%%%%%%%%%%%%%%%%%%%%

\begin{quote}
    Determine a variação da velocidade entre os instantes $t_i$ e $t_f$ para o caso de aceleração constante em direção e sentido.
\end{quote}

\begin{marginfigure}[6cm]
\centering
\begin{tikzpicture}[>=Stealth, extended line/.style={shorten >=-#1,shorten <=-#1},
 extended line/.default=3mm]] % talvez fosse melhor amplicar com scale=1.5
    % Draw axes: acho que o |- é pra desenhar um "canto", um L
    \draw [<->] (0,3) node (yaxis)[below left] {$a(t)$} |- (4,0) node (xaxis) [below] {$x$};

    \draw (0,2) node[left]{$a$} -- (3.75,2);
    \draw[dashed] (0.5, 0) node[below]{$t_i$} -- (0.5,2);
    \draw[dashed] (3.5, 0) node[below]{$t_f$} -- (3.5,2);
    
    \fill[pattern = north west lines] (0.5,0) -- (0.5,2) -- (3.5,2) -- (3.5,0) -- cycle;
    
\end{tikzpicture}
\caption{A variação de velocidade é dada pela área do gráfico $a \times t$.}
\end{marginfigure}

Sabemos que a variação da velocidade está ligada à área em um gráfico $a \times t$. Logo, ela pode ser descrita através da integral
\begin{equation}
    \Delta v = \int_{t_i}^{t_f} a \; dt.
\end{equation}
%
Como $a$ é constante, podemos retirá-la da integral:
\begin{align}
    \Delta v &= \int_{t_i}^{t_f} a \; dt \\
    &= a\int_{t_i}^{t_f} dt \\
    &= a\int_{t_i}^{t_f} 1 \;dt \\
    &= a\big[t\big]_{t_i}^{t_f}.
\end{align}
%
Se utilizarmos $t_i = 0$ e $t_f = t$, podemos reescrever a expressão acima como
\begin{equation}
    \Delta v = at,
\end{equation}
%
ou seja,
\begin{equation}
    v_f = v_i + at.
\end{equation}

%%%%%%%%%%%%%%%%%%%%%%%%%%%%%%%%%%%%%%%%%%%%%%
\paragraph{Discussão: Aditividade da integral}
%%%%%%%%%%%%%%%%%%%%%%%%%%%%%%%%%%%%%%%%%%%%%%

\begin{marginfigure}[6cm]
\centering
\begin{tikzpicture}[>=Stealth, extended line/.style={shorten >=-#1,shorten <=-#1},
 extended line/.default=3mm]] % talvez fosse melhor amplicar com scale=1.5
    % Draw axes: acho que o |- é pra desenhar um "canto", um L
    \draw [<->] (0,3) node (yaxis) [below left] {$f(x)$}
        |- (4,0) node (xaxis) [below] {$x$};

    \draw (0,0.75) node[left]{$A$} -- (3.75,0.75);
    \draw[dashed] (0.5, 0) node[below]{$a$} -- (0.5,0.75);
    \draw[dashed] (3.5, 0) node[below]{$b$} -- (3.5,0.75);
    
    \fill[pattern = north west lines] (0.5,0) -- (0.5,0.75) -- (3.5,0.75) -- (3.5,0) -- cycle;
    
    \begin{scope}[shift={(0,-4)}]
    
        % Draw axes: acho que o |- é pra desenhar um "canto", um L
        \draw [<->] (0,3) node (yaxis) [below left] {$g(x)$}
            |- (4,0) node (xaxis) [below] {$x$};

        \draw (0,1.5) node[left]{$B$} -- (3.75,1.5);
        \draw[dashed] (0.5, 0) node[below]{$a$} -- (0.5,1.5);
        \draw[dashed] (3.5, 0) node[below]{$b$} -- (3.5,1.5);
        
        \fill[pattern = north west lines] (0.5,0) -- (0.5,1.5) -- (3.5,1.5) -- (3.5,0) -- cycle;
    \end{scope}
    
    \begin{scope}[shift={(0,-8)}]
    
        % Draw axes: acho que o |- é pra desenhar um "canto", um L
        \draw [<->] (0,3) node (yaxis) [below left] {$h(x)$}
            |- (4,0) node (xaxis) [below] {$x$};

        \draw (0,2.25) node[left]{$C$} -- (3.75,2.25);
        \draw[dashed] (0.5, 0) node[below]{$a$} -- (0.5,2.25);
        \draw[dashed] (3.5, 0) node[below]{$b$} -- (3.5,2.25);
        
        \fill[pattern = north west lines] (0.5,0) -- (0.5,2.25) -- (3.5,2.25) -- (3.5,0) -- cycle;
    \end{scope}
    
\end{tikzpicture}
\caption{A área abaixo de $h(x)$ é igual à soma das áreas abaixo de $f(x)$ e de $g(x)$.\label{Fig:Ex:IntegSomaIgualSomaInteg}}
\end{marginfigure}

Uma propriedade importante da integral é o fato de que ela é aditiva, isto é
\begin{equation}
    \int_{a}^{b} f(x) + g(x)\; dx = \int_a^b f(x) \;dx + \int_a^b g(x) \;dx.
\end{equation}
%
Isso pode ser entendido facilmente se considerarmos funções constantes:
\begin{align}
    f(x) &= A \\
    g(x) &= B,
\end{align}
%
o que implica em áreas dadas por
\begin{align}
    A_{f(x)} = A (b-a) \\
    A_{g(x)} = B (b-a).
\end{align}
%
Considerando ainda
\begin{align}
    h(x) &= f(x) + g(x) \\
    &= A + B,
\end{align}
%
temos uma área dada por
\begin{align}
    A_{h(x)} &= (A + B) (b-a) \\
    &= A(b-a) + B(b-a),
\end{align}
%
isto é, a área abaixo de $h(x)$ é dada pela soma das áreas abaixo de $f(x)$ e abaixo de $g(x)$. Como a área é dada pela integral, concluímos que a integral da soma de duas funçõesem um dado intervalo é igual à soma das integrais de tais funções em tal intervalo.

%%%%%%%%%%%%%%%%%%%%%%%%%%%%%%%%%
\paragraph{Exemplo: Deslocamento}
%%%%%%%%%%%%%%%%%%%%%%%%%%%%%%%%%

\begin{quote}
    Determine o deslocamento de um corpo cuja velocidade é descrita por
    \begin{equation}
        v(t) = v_i + a t
    \end{equation}
    %
    entre os instantes $t_i = 0$ e $t_f = \np[s]{10,0}$, com $v_i = \np[m/s]{2,0}$ e $a = \np[m/s^2]{10,0}$.
\end{quote}

\begin{marginfigure}[3cm]
\centering
\begin{tikzpicture}[>=Stealth, extended line/.style={shorten >=-#1,shorten <=-#1},
 extended line/.default=3mm]] % talvez fosse melhor amplicar com scale=1.5
    % Draw axes: acho que o |- é pra desenhar um "canto", um L
    \draw [<->] (0,2) node (yaxis) [below left] {$v(t)$}
        |- (4,0) node (xaxis) [below left] {$t$};

    \draw (-0.2,0.5) -- (4,1.75);
    \draw[dashed] (0.5, 0) node[below]{$t_i$} -- (0.5,0.708);
    \draw[dashed] (3.5, 0) node[below]{$t_f$} -- (3.5,1.6);
    
    \fill[pattern = north west lines] (0.5,0) -- (0.5,0.708) -- (3.5,1.6) -- (3.5,0) -- cycle;
    
\end{tikzpicture}
\caption{Área delimitada pela função $v(t) = v_i + at$.\label{Fig:Ex:IntegDefinidaPolinômio}}
\end{marginfigure}

Sabemos que o deslocamento é dado pela área abaixo da curva $v(t)$, o que podemos calcular através de uma integral definida. Podemos determinar a integral definida de um polinômio através de:
\begin{align}
    \int_{t_i}^{t_f} v_i + at \;dt &= \left[v_i t + \frac{at^2}{2} + C\right]_{t_i}^{t_f} \\
    &= \left[v_i t_f + \frac{at_f^2}{2} + C\right] - \left[v_i t_i + \frac{at_i^2}{2} + C\right] \\
    &= v_i t_f + \frac{at_f^2}{2}.
\end{align}
%
Substituindo os valores, obtemos
\begin{align}
    \Delta x &= v_i t_f + \frac{at_f^2}{2} \\
    &= (\np[m/s]{2,0})\cdot(\np[t]{10,0}) + \frac{(\np[m/s^2]{10,0})\cdot(\np[s]{10,0})^2}{2} \\
    &= \np[m]{520,0}.
\end{align}

%%%%%%%%%%%%%%%%%%%%%%%%%%%%%%%%%%%%%%%%%%%%%%%%%%%%%%%%%%%%%%%%%%%
\paragraph{Exemplo: Área determinada por uma função trigonométrica}
%%%%%%%%%%%%%%%%%%%%%%%%%%%%%%%%%%%%%%%%%%%%%%%%%%%%%%%%%%%%%%%%%%%

\begin{quote}
Considerando a Figura~\eqref{Fig:AreaIntegSeno}, determine a área delimitada pela função $\sen \theta$ e pelo eixo horizontal.
\end{quote}

\begin{marginfigure}
\centering
\begin{tikzpicture}[>=Stealth, extended line/.style={shorten >=-#1,shorten <=-#1},
 extended line/.default=3mm]] % talvez fosse melhor amplicar com scale=1.5
    % Draw axes: acho que o |- é pra desenhar um "canto", um L
    \draw[->] (0,-1.5) -- (0,1.5) node (yaxis) [above, rotate = 90] {$\sen(\theta)$};
    \draw[->] (0,0) -- (3.7,0) node (xaxis) [below left] {$\theta$};
    
    % Desenhar função:
    \draw[smooth,name path=plota,samples=1000,domain=0:3.4]
    plot(\x,{sin((2 * \x) r)});
    
    \draw[fill] (1.571,0) circle (1pt) node[below left] {$\pi$};
    \draw[fill] (0,0) circle (1pt) node[below left]{$0$};
    \draw[fill] (3.1415,0) circle (1pt);

    \path [pattern=north west lines, domain=0:1.571, variable=\x]
     	  (0, 0)
    	  -- plot ({\x}, {sin((2 * \x) r)})
          -- (3.14, 0)
          -- cycle;
          
\end{tikzpicture}
\caption{Área delimitada pela função $\sen \theta$  no intervalo $\theta = [0;\pi]~\rm{rad}$.\label{Fig:AreaIntegSeno}}
\end{marginfigure}

Podemos determinar o valor de tal área através de uma integral:
\begin{equation}
    A = \int_0^{\pi} \sen(\theta) \; d\theta.
\end{equation}
%
Primeiramente, devemos determinar qual é a função cuja derivada é $\sen(\theta)$, isto é, precisamos determinar $\mathcal{F}(\theta)$ tal que
\begin{align}
    \mathcal{F}'(\theta) &= f(\theta) \\ 
    &= \sen(\theta).
\end{align}
%
As funções trigonométricas são conhecidas por apresentarem uma ciclicidade, a menos de um sinal: sabemos que
\begin{align}
    \frac{d}{d\theta} \sen(\theta) &= \cos(\theta) \\
    \frac{d}{d\theta} \cos(\theta) &= -\sen(\theta),
\end{align}
%
de onde vem que
\begin{equation}
    \mathcal{F}(\theta) = -\cos(\theta).
\end{equation}
%
Logo,
\begin{align}
    A &= \int_0^{\pi} \sen(\theta) \; d\theta \\
    &= \big[-\cos(\theta)\big]_{0}^{\pi} \\
    &= [-\cos(\pi)] - [-\cos(0)] \\
    &= [-(-1)] - [-(1)] \\
    &= [1 + 1] \\
    &= 2.
\end{align}

%%%%%%%%%%%%%%%%%%%%%%%%%%%%%%%%%%%%%%
\paragraph{Discussão: Áreas negativas}
%%%%%%%%%%%%%%%%%%%%%%%%%%%%%%%%%%%%%%

\begin{marginfigure}
\centering
\begin{tikzpicture}[>=Stealth, extended line/.style={shorten >=-#1,shorten <=-#1},
 extended line/.default=3mm]] % talvez fosse melhor amplicar com scale=1.5
    % Draw axes: acho que o |- é pra desenhar um "canto", um L
    \draw[->] (0,-1.5) -- (0,1.5) node (yaxis) [above, rotate = 90] {$\sen(\theta)$};
    \draw[->] (0,0) -- (4.2,0) node (xaxis) [below left] {$\theta$};
    
    % Desenhar função:
    \draw[smooth,name path=plota,samples=1000,domain=0:3.4]
    plot(\x,{sin((2 * \x) r)});
    
    \draw[fill] (1.571,0) circle (1pt);
    \draw[fill] (0,0) circle (1pt) node[below left]{$0$};
    \draw[fill] (3.1415,0) circle (1pt) node[below right] {$2\pi$};

    \path [pattern=north west lines, domain=0:3.1415, variable=\x]
     	  (0, 0)
    	  -- plot ({\x}, {sin((2 * \x) r)})
          -- (3.14, 0)
          -- cycle;
          
\end{tikzpicture}
\caption{Área delimitada pela função $\sen \theta$  no intervalo $\theta = [0;2\pi]~\rm{rad}$.\label{Fig:AreaIntegSenoPeriodo}}
\end{marginfigure}

Verificamos no exemplo anterior a integral da função seno no intervalo 0 a $pi$ radianos, obtendo \emph{exatamente} 2. Se calcularmos a integral no intervalo 0 a $2\pi$ radianos, obtemos:
\begin{align}
    A &= \int_0^{2\pi} \sen(\theta) \; d\theta \\
    &= \big[-\cos(\theta)\big]_{0}^{2\pi} \\
    &= [-\cos(2\pi)] - [-\cos(0)] \\
    &= [-(1)] - [-(1)] \\
    &= [-1 + 1] \\
    &= 0.
\end{align}
%
Nesse caso verificamos que a integral é \emph{nula}. De fato, a área entre a curva e o eixo horizontal conta como negativa se tal área fica \emph{abaixo} do eixo. Note que essa interpretação já foi utilizada ao interpretarmos a área de gráficos em cinemática: em um gráfico de velocidade em função do tempo, por exemplo, uma região onde a curva $v(t)$ fica abaixo do eixo significa que a velocidade é \emph{negativa}, o que implica em um deslocamento no sentido negativo do eixo.


%%%%%%%%%%%%%%%%%%%%%%%%%%%%%%%%%%%%%%%%%%%%%%
\subsection{Trabalho como a integral da força}
%%%%%%%%%%%%%%%%%%%%%%%%%%%%%%%%%%%%%%%%%%%%%%

De acordo com os resultados das seções anteriores, se conhecemos a função $F_x(x)$ que nos dá a força, podemos calcular o trabalho $W_{F_x}$ no deslocamento de $x_i$ a $x_f$ através de
\begin{equation}\label{Eq:TrabalhoIntegral}
    W_{F_x} = \int_{x_i}^{x_f} F_x(x) \;dx,
\end{equation}
%
o que equivale a realizar o processo inverso à diferenciação, encontrando uma função $\mathcal{F}(x)$ e então calcular a diferença $\mathcal{F}(x_f) - \mathcal{F}(x_i)$. 

%%%%%%%%%%%%%%%%%%%%%%%%%%%%%%%%%%%%%%%%%%%%%%%%%%%%%%%%%%%%%%
\paragraph{Exemplo: Trabalho realizado por uma força elástica}
%%%%%%%%%%%%%%%%%%%%%%%%%%%%%%%%%%%%%%%%%%%%%%%%%%%%%%%%%%%%%%

Podemos obter o resultado para o trabalho realizado por uma força elástica, dado pela Equação~\eqref{Eq:TrabalhoForcaElastica}, através da definição de trabalho de uma força variável, isto é, através da Equação~\eqref{Eq:TrabalhoIntegral}. Sabemos que
\begin{equation}
    F_e = - k x,
\end{equation}
%
portanto, temos que
\begin{equation}
    W_{F_e} = \int_{x_i}^{x_f} -kx\;dx.
\end{equation}
%
Uma das propriedades de uma integral é a de que constantes multiplicativas podem ser ``passadas para fora'' da integral. Essa propriedade é análoga àquela para derivadas, onde uma constante multiplicativa também pode ser ``passada para fora'' da derivada. Logo,
\begin{equation}
    W_{F_e} = -k \int_{x_i}^{x_f} x \;dx.
\end{equation}

Precisamos agora determinar a função $\mathcal{F}(x)$ tal que
\begin{align}
    \mathcal{F}'(x) &= f(x) \\
    & = x.
\end{align}
%
A determinação de tal função pode ser feita por tentativa e erro, sendo que obtemos
\begin{equation}
    \mathcal{F}(x) = \frac{x^2}{2} + C.
\end{equation}
%
Logo,
\begin{align}
    W_{F_e} &= -k \left[\frac{x^2}{2} + C\right] \\
    &= -k \left[\left(\frac{x_f^2}{2} + C\right) - \left(\frac{x_i^2}{2} + C\right)\right] \\
    &= -k \left[\frac{x_f^2}{2} - \frac{x_i^2}{2}\right] \\
    &= -\frac{k}{2} (x_f^2 - x_i^2).
\end{align}
%
Verificamos, portanto, que recuperamos a Expressão~\eqref{Eq:TrabalhoForcaElastica} para o trabalho realizado por uma força elástica.

%%%%%%%%%%%%%%%%%%
\section{Potência}
%%%%%%%%%%%%%%%%%%

Muitas vezes estamos mais interessados na quantidade de trabalho realizado por unidade de tempo do que no trabalho total realizado. Esse é o caso de motores, por exemplo. Definimos então uma grandeza, denominada \emph{potência}, cujo valor médio é dado por
\begin{equation}
  \mean{P} = \frac{W}{\Delta t}.
\end{equation}
%
No caso de termos valores diferentes de trabalho realizados em intervalos de tempo diferentes, mas de mesma duração, podemos definir a \emph{potência instantânea} como
\begin{equation}\label{Eq:DefPotenciaInstantanea}
  P = \frac{dW}{dt}.
\end{equation}

Analisando a dimensão da potência temos
\begin{align}
  [P] &= \frac{[W]}{[t]} \\
  &= \nicefrac{\rm{J}}{\rm{s}}.
\end{align}
%
Como a potência é uma grandeza muito comum em áreas técnicas, científicas e mesmo no cotidiano, suas unidades ganham uma denominação especial --~o \emph{watt}~--, da mesma forma que a unidade de energia. O watt é representado\footnote{Tome cuidado para não confundir o símbolo em itálico para o trabalho $W$ com o símbolo W da unidade para a potência.} por W:
\begin{equation}
  \rm{W} \equiv \nicefrac{\rm{J}}{\rm{s}}.
\end{equation}

Finalmente, vale notar que podemos relacionar a potência instantânea exercida por uma força constante à velocidade desenvolvida pelo corpo sobre o qual a força atua. Para isso, basta substituirmos a expressão para o trabalho
\begin{equation}
  W = F r \cos \theta,
\end{equation}
%
onde utilizamos $x$ para denotar a distância percorrida durante a aplicação da força $F$, na definição de potência instantânea dada pela Equação~\ref{Eq:DefPotenciaInstantanea}. Obtemos então
\begin{align}
  P &= \frac{dW}{dt} \\
  &= \frac{d}{dt}(Fr\cos\theta) \\
  &= F\frac{dr}{dt} \cos\theta\\
  &= F v \cos\theta \\
  &= \vec{F}\cdot\vec{v}.
\end{align}

%\textbf{Discuutir aqui a ideia de que se um corpo é submetido a uma aceleração constante, a potência aumenta linearmente com o tempo, uma vez que $v = v_0 + at$. Usar essa ideia para discutir por que a aceleração decresce com o aumento da velocidade, uma vez que se considere $P$ constante. (Veja que não é necessário considerar a força de arrasto, afinal conforme $v$ aumenta, o custo energético de cada $\delta v$ aumenta.)}

%%%%%%%%%%%%%%%%%%%
\section{Potencial}
%%%%%%%%%%%%%%%%%%%

%\textbf{Aqui listar (itemize) as características, fica mais fácil.}
Ao estudar a energia cinética e o trabalho, verificamos que tais conceitos são úteis para se calcular algumas quantidades físicas sem nos preocupar com o caráter vetorial das grandezas. Veremos agora que existem outras formas de energia que estão relacionadas às forças que atuam em um sistema de partículas e \emph{à própria configuração -- isto é, à disposição} -- das partículas que compõe o sistema. Tais formas de energia são denominadas como \emph{energias potenciais}. (Portanto, as expressões para o potencial dependem só de características da força de interação, e da disposição das partículas.)

Uma das propriedades dessa forma de energia é a de que ela independe do histórico de configurações do sistema, dependendo somente de seu estado em um dado momento. Verificaremos também que a propriedade de independer do histórico do sistema faz com que nem todas as forças têm potenciais associados a elas, porém teremos uma maneira simples de verificar quais forças os têm. Finalmente, é importante notar que assim como cada força tem uma expressão diferente, o mesmo pode ser dito sobre os potenciais, já que eles são determinados diretamente a partir da expressão para a força.

%\textbf{Discutir que nem toda força tem um potencial associado. Discutir que cada tem uma expressão diferente de potencial.}

Quando associamos a energia potencial à energia cinética, verificamos que podemos definir a \emph{energia mecânica} de um sistema. Tal grandeza é constante sob certas condições, e podemos utilizá-la para obter informações sobre sistemas físicos de maneira relativamente simples.

%%%%%%%%%%%%%%%%%%%%%%%%%%%%%%%%%%%%%%%%%%%%
\subsection{Energia potencial gravitacional}
%%%%%%%%%%%%%%%%%%%%%%%%%%%%%%%%%%%%%%%%%%%%

Se considerarmos a Terra e um objeto qualquer, próximo à superfície do planeta, temos um sistema constituido pelo objeto e pela Terra. Vamos desconsiderar momentaneamente a força de arrasto do ar e analisar o trabalho realizado pela força peso. Se o objeto é lançado verticalmente para cima, com velocidade inicial $v_i$, à medida que ele se desloca, sua velocidade diminui, eventualmente chegando a zero. Sabemos que há um trabalho exercido pelo peso, de forma que -- utilizando o Teorema Trabalho--Energia Cinética e a Equação~\ref{Eq:TrabalhoPeso} --, podemos escrever
\begin{align}
  \Delta K &= W_g \\
  K_f - K_i &=  -mg\Delta y \\
  K_f - K_i &=  -(mgy_f - mgy_i) \\
  K_i &= mgh,
\end{align}
%
onde $y_i = 0$ devido à escolha do sistema de referência.

\begin{marginfigure}[-2cm]
\centering
\begin{tikzpicture}[>=Stealth]
    \draw[pattern = north west lines] (0,0) circle (2mm);
    \draw[fill] (0,0) circle (1pt);
    \draw[->, thick] (0,0) -- +(0,-1) node[right]{$\vec{P}$};
    \draw[->] (0,0.2) -- +(0, 0.75) node[right]{$\vec{v}_i$};
    
    \draw[draw = gray, pattern = north west lines, pattern color = gray] (0,2.75) circle (2mm);
    \draw[gray, fill] (0,2.75) circle (1pt);
    \draw[->, thick, gray] (0,2.75) -- +(0,-1) node[right]{$\vec{P}$};
    
    \draw[->] (-1, -1) -- (-1, 3.5) node[below left]{$y$};
    \draw[|-|] (-1,0) node[left, shift={(-1mm,0)}]{$y_i = 0$} -- (-1,2.75) node[left, shift={(-1mm,0)}]{$y_f = h$};
    
\end{tikzpicture}
\caption{Quando um objeto sobe verticalmente sujeito à força peso, sua velocidade diminui devido ao trabalho realizado por tal força.}
\end{marginfigure}

Analisando essa expressão, vemos que a quantidade inicial de energia cinética é igual a $mgh$. Podemos interpretar o processo acima como a \emph{transferência} da energia cinética para outra forma de energia, que denominamos como \emph{energia potencial gravitacional} e que definimos como
\begin{equation}
  U_g = mgy. \mathnote{Energia potencial gravitacional}
\end{equation}

É importante notar que utilizamos a força peso para descrever a atração exercida pelo planeta sobre o corpo, o que limita a utilização desse potencial às imediações da superfície da Terra. No caso de estarmos interessados em calcular o potencial gravitacional a grandes distâncias, precisamos utilizar a Lei da Gravitação Universal (Equação~\eqref{Eq:LeiGravitacaoUniversal}) para deduzir outra expressão para o potencial gravitacional.

Finalmente, note que utilizamos na discussão acima a expressão para o trabalho realizado pela força gravitacional. Portanto, assim como para o trabalho, devemos utilizar um eixo $y$ vertical orientado de forma crescente para cima. 

%%%%%%%%%%%%%%%%%%%%%%%%%%%%%%%%%%%%%%%
\subsection{Energia potencial elástica}
%%%%%%%%%%%%%%%%%%%%%%%%%%%%%%%%%%%%%%%

Outro caso em que podemos identificar a existência de um potencial é quando atua sobre o sistema uma força elástica. Considere um bloco disposto sobre uma mesa sem atrito e sujeito a uma força elástica exercida ao longo de um eixo $x$ por uma mola presa ao bloco e a uma parede. Em um dado instante, o bloco se encontra na posição $x_i$ e se afasta da parede com velocidade $v$. Sabemos que nesse caso o trabalho realizado é dado pela Equação~\eqref{Eq:TrabalhoForcaElastica}. Utilizando essa expressão e o Teorema Trabalho -- Energia Cinética, obtemos
\begin{align}
  \Delta K &= W_e \\
  K_f - K_i &= -\frac{k}{2}(x_f^2 - x_i^2) \\
  K_i &= \frac{k}{2}d^2,
\end{align}
%
onde $x_i = 0$ pois a origem coincide com a posição na qual a mola está relaxada.

\begin{marginfigure}[-2cm]
\centering
\begin{tikzpicture}[>=Stealth, scale = 0.9,
     interface/.style={
        % superfície
        postaction={draw,decorate,decoration={border,angle=-45,
                    amplitude=0.2cm,segment length=2mm}}},
    ]
    
    %%
    
    \draw[interface] (0,-1) -- (0,-2.5);
    \draw[interface] (0,-2.5) -- (4.8, -2.5);
    
    \draw (0,-2) -- (0.2,-2);
    \draw[decoration={aspect=0.3, segment length=2.5625mm, amplitude=2mm,coil},decorate] (0.2,-2) -- (3.3,-2);
    \draw (3.3, -2) -- (3.5,-2);
    
    \draw[pattern = north west lines, pattern color = gray] (3.5,-2.5) rectangle (4.5,-1.5);
    \draw[dotted, pattern = north west lines, pattern color = gray] (2,-2.5) rectangle (3,-1.5);
    
    \draw[fill] (4,-2) circle (1pt);
    \draw[->, thick] (4,-2) -- +(0,-1) node[right]{$\vec{P}$};
    \draw[->, thick] (4,-1.5) -- node [right]{$\vec{N}$} +(0,1);
    \draw[->, thick] (3.5, -2) -- node[below right]{$\vec{F}_e$} +(-1,0);
    
    \draw[->] (0,-3.5) -- (4.8,-3.5) node[below]{$x$};
    \draw[|-|] (2.5,-3.5) node[below]{$x_i = 0$} -- (4,-3.5) node[below]{$x_f = d$};
    \draw[dotted,->] (2.25, -1.25) -- node[above]{$\vec{v}$} +(0.5,0);
        
\end{tikzpicture}
\caption{Podemos associar uma energia potencial à força elástica exercida por uma mola.}
\end{marginfigure}

Da forma análoga ao caso do potencial gravitacional, podemos interpretar o processo descrito acima como a transferência de uma quantidade $kd^2/2$ de energia que estava armazenada na mola para a forma cinética. Assim, definimos um \emph{potencial elástico} $U_e$:
\begin{equation}
  U_e = \frac{k}{2}x^2. \mathnote{Energia potencial elástica}
\end{equation}

%%%%%%%%%%%%%%%%%%%%%%%%%%%%%%%%%
\subsection{Potencial e trabalho}
%%%%%%%%%%%%%%%%%%%%%%%%%%%%%%%%%

Em ambos os casos vistos acima, temos que a variação na energia potencial associada a uma força pode ser escrita através do trabalho por tal força como
\begin{equation}
  \Delta U = - W.
\end{equation}
%
Essa relação tem como implicação direta o fato de que podemos escrever a variação da energia cinética como
\begin{equation}
    \Delta K = - \Delta U.
\end{equation}

%%%%%%%%%%%%%%%%%%%%%%%%%%%%%%%%%%%%%%%%%%%%%%%%%%%%%%%%%%%%%%%%%%%%
\paragraph{Exemplo: Queda de um bloco sobre uma mola}
%%%%%%%%%%%%%%%%%%%%%%%%%%%%%%%%%%%%%%%%%%%%%%%%%%%%%%%%%%%%%%%%%%%%

\begin{quote}
    Um bloco de massa $m = \np[kg]{0,5}$ cai sobre uma mola, conforme mostrado na Figura~\ref{Fig:ex:QuedaBlocoSobreMola}. Se a constante elástica é de \np[N/m]{50}, a altura $h$ é de \np[m]{2,0}, e a velocidade inicial do bloco é nula, determine a distensão máxima $\ell_{\rm{max}}$ da mola.
\end{quote}

\begin{marginfigure}[-5cm]
\centering
\begin{tikzpicture}[>=Stealth,
     interface/.style={
        % superfície
        postaction={draw,decorate,decoration={border,angle=-45,
                    amplitude=0.2cm,segment length=2mm}}},
    ]
    
    \draw[interface] (-2,0) -- (2,0);

    \draw (0,0) -- (0,0.1);
    \draw[decoration={aspect=0.3, segment length=2mm, amplitude=2mm,coil},decorate] (0,0.1) -- (0,2.5);
    
    \draw[pattern = north east lines] (-0.75,2.5) rectangle (0.75,2.7);
    \draw[pattern = north west lines] (-0.5, 4) rectangle (0.5,5);
    \draw[fill] (0,4.5) circle (1pt);
    
    \draw[dashed] (-0.5, 2.7) rectangle (0.5,3.7);
    \draw[fill] (0,3.2) circle (1pt);
    
    \draw[|-|] (-0.75, 3.2) -- node[left]{$h$} (-0.75, 4.5);
    \draw[<-] (1.5, 4.5) node[below right]{$y$} -- (1.5,2);

    \draw[|-|] (-0.95, 2.6) -- node[left]{$\ell_{\rm{max}}$} (-0.95, 1.0);
    
\end{tikzpicture}
\caption{Queda de um bloco sobre uma mola disposta verticalmente.\label{Fig:ex:QuedaBlocoSobreMolaEnPot}}
\end{marginfigure}

Sabemos que
\begin{align}
    \Delta K &= W \\
    &= W_P + W_{F_e} \\
    &= -\Delta U_g - \Delta U_e,
\end{align}
%
onde utilizamos o fato de que $W = -\Delta U$.

Na posição em que temos a distensão máxima, o bloco se encontra parado momentaneamente, temos portanto uma variação nula da energia cinética, uma vez que o bloco parte do repouso. Também sabemos que a mola está inicialmente relaxada, o que resulta em uma energia potencial elástica inicial nula. Assim
\begin{align}
    \Delta K &= -\Delta U_g - \Delta U_e \\
    0 &= -[mg y_f - mg y_i] - \left[\frac{k}{2} y_f^2\right] \\
\end{align}
%
onde estamos descrevendo a distensão da mola através do próprio eixo $y$ vertical e que aponta para cima utilizado para determinar o potencial gravitacional. Como $y_i = h$ e\footnote{O sinal negativo é importante pois $\ell_{\rm{max}}$ representa um comprimento, ou seja, um valor positivo, porém $y_f$ representa uma posição e está na região negativa do eixo.} $y_f = -\ell_{\rm{max}}$, temos:
%
Podemos simplificar a expressão acima se escolhermos $y_i = 0$, obtendo
\begin{align}
    0 &= -[mg y_f - mg y_i] - \left[\frac{k}{2} y_f^2\right] \\
    0 &= -[mg \ell_{\rm{max}} - mg h] - \left[\frac{k}{2} \ell_{\rm{max}}^2\right] \\
    0 &= \frac{k}{2} \ell_{\rm{max}}^2 + mg \ell_{\rm{max}} - mgh.
\end{align}
%
Resolvendo a equação de segundo grau acima, obtemos
\begin{align}
    y_f &= \frac{-(mg) \pm \sqrt{(mg)^2 - 4\cdot\left(\frac{k}{2}\right)\cdot(-mgh)}}{2\cdot\left(\frac{k}{2}\right)}\\
    &= \np[m]{0,536}.
\end{align}

%%%%%%%%%%%%%%%%%%%%%%%%%%%%%%%%%%%%%%%%%%%%%%%%%%%%%%%%%%%%%%
\subsection{Determinação do potencial para uma força qualquer}
%%%%%%%%%%%%%%%%%%%%%%%%%%%%%%%%%%%%%%%%%%%%%%%%%%%%%%%%%%%%%%

No caso de termos uma força que varia com a posição, temos que o trabalho é dado pela integral da componente da força na direção do movimento (Equação~\eqref{Eq:TrabalhoIntegral}), logo
\begin{equation}\label{Eq:CalculoDoPotencial}
  \Delta U = - \int_{x_i}^{x_f} F(x) dx.
\end{equation}

Portanto, se o potencial existe\footnote{Nem todas as forças dão origem a potenciais. Verificaremos as condições para a existência de um potencial adiante.}, temos uma maneira definida de o encontrar uma vez que se conheça a expressão para a força: sabemos -- através da segunda parte do Teorema Fundamental do Cálculo -- que 
\begin{align}
    \Delta U &= - \int_{x_i}^{x_f} F(x) dx \\
    &= -\big[\mathcal{F}(x) + C\big]_{x_i}^{x_f} \\
    U_f - U_i &= -[\mathcal{F}(x_f) - \mathcal{F}(x_i)],
\end{align}
%
de onde podemos identificar que o \emph{potencial $U(x)$ é a própria função $\mathcal{F}(x)$}, porém com um sinal negativo:
\begin{equation}\label{Eq:DefPotAntiderivada}
    U(x) = - \mathcal{F}(x) + C.
\end{equation}
%
A função $\mathcal{F}(x)$, é a integral indefinida da função $F(x)$ que descreve a força. Note que que o potencial é uma função somente da posição atual $x$, não dependendo do histórico de posições ocupadas pelo corpo em questão.

Note que uma consequência da expressão acima é a de que
\begin{align}
    \mathcal{F}'(x) &= F(x) \\
    -U(x) &= F(x),
\end{align}
%
ou seja,
\begin{equation}\label{Eq:ForcaGradPot}
  F =  -\frac{dU}{dx},
\end{equation}
%
onde o sinal reflete o sinal da definição do potencial. Esse resultado será muito útil ao analisarmos curvas de potencial mais adiante.

%%%%%%%%%%%%%%%%%%%%%%%%%%%%%%%%%%%%%%%%%%%%%%%%%%%%%%%%%%%%%
\subsection{Dependência da energia na escolha do referencial}
%%%%%%%%%%%%%%%%%%%%%%%%%%%%%%%%%%%%%%%%%%%%%%%%%%%%%%%%%%%%%


Na Equação~\eqref{Eq:DefPotAntiderivada} temos um detalhe importante: a função $\mathcal{F}(x) + C$ representa uma \emph{família} de funções que diferem por uma constante. Isso significa que o potencial pode sofrer uma alteração de seu valor por um valor constante, sem que isso altere suas propriedades. De fato, ao calcularmos a \emph{variação} do potencial entre dois estados quaisquer $x_i$ e $x_f$, temos
\begin{align}
    \Delta U &= U(x_f) - U(x_i) \\
    &= [\mathcal{F}(x_f) + C] - [\mathcal{F}(x_i) + C] \\
    &= \mathcal{F}(x_f) - \mathcal{F}(x_i)
\end{align}
%
A existência da constante $C$, não representa um problema: podemos a utilizar para que possamos definir um valor arbitrário de potencial a um ponto qualquer. Em geral, escolhermos como nulo o potencial na origem do sistema de referência adotado, porém em alguns casos pode ser útil escolher outro valor, ou que o zero esteja em outro ponto.\footnote{Em potenciais proporcionais a $1/r^2$, onde $r$ representa a distância até a origem, por exemplo, é mais conveniente adotar que o zero do potencial ocorre quando $r \to \infty$.} Note que a expressão acima equivale a
\begin{equation}
  \Delta U = - \int_{x_i}^{x_f} F(x) dx,
\end{equation}
%
e que em uma integral definida a constante não interfere no resultado obtido.

Uma outra maneira de verificar a independência do potencial no valor de $C$ é verificar que a força obtida através da Equação~\eqref{Eq:ForcaGradPot} não varia:
\begin{align}
  F &= - \frac{d(U+C)}{dx} \\
  &= -\left(\frac{dU}{dx} + \frac{dC}{dx}\right) \\
  &= -\frac{dU}{dx}.
\end{align}

A constante $C$ deve sua existência ao fato de que a origem de um sistema de coordenadas é uma decisão arbitrária ---~não existe uma origem preferencial para descrever um fenômeno\footnote{Porém, em geral, existe uma origem mais \emph{conveniente}.}~---. Os resultados obtidos através de cálculos envolvendo potencial e energia cinética não serão influenciados diretamente pelo \emph{valor} do potencial, mas sim pela sua \emph{variação}, que independe de $C$.

%%%%%%%%%%%%%%%%%%%%%%%%%%%%%%%%%%%%%%%%%%%%%%%%%%%%%%%%%%%%%%%%%%%%%%%%%%%%%%%
\paragraph{Exemplo: Independência do resultado em relação à escolha da origem}
%%%%%%%%%%%%%%%%%%%%%%%%%%%%%%%%%%%%%%%%%%%%%%%%%%%%%%%%%%%%%%%%%%%%%%%%%%%%%%%

\begin{quote}
    Um pêndulo é liberado a partir do repouso a partir de uma posição que faz um ângulo de \degree{90} em relação à vertical. Determine a velocidade do pêndulo quando ele passa no ponto mais baixo de sua trajetória.
\end{quote}

Podemos determinar a velocidade utilizando o Teorema Trabalho--Energia-Cinética:
\begin{equation}
    \Delta K = W.
\end{equation}
%
Substituindo na expressão acima a relação
\begin{align}
    W &= - \Delta U \\
    &= - [mgy_f - mgy_i]
\end{align}
%
obtemos
\begin{align}
    \Delta K &= -[mgy_f - mgy_i] \\
    K_f - K_i &= -mgy_f + mgy_i.
\end{align}
%
Note que a energia cinética inicial $K_i$ é nula, uma vez que o pêndulo parte do respouso. Logo,
\begin{equation}
    \frac{mv_f^2}{2} = -mgy_f + mgy_i,
\end{equation}
%
onde utilizamos $K_f = mv_f^2/2$.

Isolando a velocidade, obtemos
\begin{align}
    v^2 &= -2g(y_f - y_i) \\
    v &= \sqrt{-2g(y_f - y_i)}.
\end{align}
%
Note que na expressão acima o resultado depende da \emph{diferença} entre as posições final e inicial. Logo, a escolha do zero do sistema de referência não altera o resultado obtido.

\begin{marginfigure}
\centering
\begin{tikzpicture}[>=Stealth, scale = 1.2]

    \draw[|-|] (0,0.5) -- node[above]{$L$} +(180:20mm);
    
    \draw[fill] (0,0) circle (1pt);
    \draw[dotted] ([shift={(0,0)}]180:2) arc[radius=2, start angle=180, end angle= 290];
    
    \draw[densely dotted] (0,0) -- (180:18mm);
    \draw[pattern = north west lines, dotted] (180:2) circle (2mm);
    \draw[->, thick, dotted] (180:2) +(0,-0.2) -- +(0,-0.7) node[left]{$\vec{P}$};
    
    \draw[densely dotted] (0,0) -- (-150:18mm);
    \draw[pattern = north west lines, dotted] (-150:2) circle (2mm);
    \draw[->, thick, dotted] (-150:2) +(0,-0.2) -- +(0,-0.7) node[left]{$\vec{P}$};
    \draw[->, thick, dotted] (-150:18mm) -- (-150:14mm) node[above left]{$\vec{T}$};
    
    \draw[densely dotted] (0,0) -- (-120:18mm);
    \draw[pattern = north west lines, dotted] (-120:2) circle (2mm);
    \draw[->, thick, dotted] (-120:2) +(0,-0.2) -- +(0,-0.7) node[left]{$\vec{P}$};
    \draw[->, thick, dotted] (-120:18mm) --  node[left]{$\vec{T}$} (-120:12mm);
    
    \draw (0,0) -- (-90:18mm);
    \draw[pattern = north west lines] (-90:2) circle (2mm);
    \draw[->] (-90:2) +(0.2,0) -- node[below]{$\vec{v}$} +(1,0);
    \draw[->, thick] (-90:2) +(0,-0.2) -- +(0,-0.7) node[left]{$\vec{P}$};
    \draw[->, thick] (-90:18mm) -- (-90:11mm) node[below right]{$\vec{T}$};
    
    \draw[<-] (-2.5,0.5) node[below left]{$y$} -- +(0,-2.5);
    \draw[|-] (-2.5,0) node[left]{$0$}  -- +(0,-0.5);
    
\end{tikzpicture}
\caption{Sistema de referência em que o zero do eixo vertical coincide com a posição inicial.\label{Fig:PotencialPosZeroRef}}
\end{marginfigure}

\begin{marginfigure}
\centering
\begin{tikzpicture}[>=Stealth, scale = 1.2]

    \draw[|-|] (0,0.5) -- node[above]{$L$} +(180:20mm);
    
    \draw[fill] (0,0) circle (1pt);
    \draw[dotted] ([shift={(0,0)}]180:2) arc[radius=2, start angle=180, end angle= 290];
    
    \draw[densely dotted] (0,0) -- (180:18mm);
    \draw[pattern = north west lines, dotted] (180:2) circle (2mm);
    \draw[->, thick, dotted] (180:2) +(0,-0.2) -- +(0,-0.7) node[left]{$\vec{P}$};
    
    \draw[densely dotted] (0,0) -- (-150:18mm);
    \draw[pattern = north west lines, dotted] (-150:2) circle (2mm);
    \draw[->, thick, dotted] (-150:2) +(0,-0.2) -- +(0,-0.7) node[left]{$\vec{P}$};
    \draw[->, thick, dotted] (-150:18mm) -- (-150:14mm) node[above left]{$\vec{T}$};
    
    \draw[densely dotted] (0,0) -- (-120:18mm);
    \draw[pattern = north west lines, dotted] (-120:2) circle (2mm);
    \draw[->, thick, dotted] (-120:2) +(0,-0.2) -- +(0,-0.7) node[left]{$\vec{P}$};
    \draw[->, thick, dotted] (-120:18mm) --  node[left]{$\vec{T}$} (-120:12mm);
    
    \draw (0,0) -- (-90:18mm);
    \draw[pattern = north west lines] (-90:2) circle (2mm);
    \draw[->] (-90:2) +(0.2,0) -- node[below]{$\vec{v}$} +(1,0);
    \draw[->, thick] (-90:2) +(0,-0.2) -- +(0,-0.7) node[left]{$\vec{P}$};
    \draw[->, thick] (-90:18mm) -- (-90:11mm) node[below right]{$\vec{T}$};
    
    \draw[<-] (-2.5,0.5) node[below left]{$y$} -- +(0,-2.5);
    \draw[|-] (-2.5,-2) node[left]{$0$}  -- +(0,-0.5);
    
\end{tikzpicture}
\caption{Sistema de referência em que o zero do eixo vertical coincide com o ponto mais baixo da trajetória.\label{Fig:PotencialPosZeroRef2}}
\end{marginfigure}

Se, por exemplo, adotarmos um sistema de referência como o da Figura~\ref{Fig:PotencialPosZeroRef}, verificamos que
\begin{align}
    y_i &= 0 \\
    y_f &= -L \\
    \Delta y &= -L.
\end{align}
%
Se, por outro lado, escolhermo o sistema de referência mostrado na Figura~\ref{Fig:PotencialPosZeroRef2}, obtemos
\begin{align}
    y_i &= L \\
    y_f &= 0 \\
    \Delta y &= -L.
\end{align}
%
Consequentemente, não só os resultados são iguais em ambos os casos, como serão os mesmos para qualquer posição da origem dos eixos de referência.

%%%%%%%%%%%%%%%%%%%%%%%%%%%%%%%%%%%%%%%%%%%%%%%%%%%%%%%%
\subsection{Condições para a existência de um potencial}
%%%%%%%%%%%%%%%%%%%%%%%%%%%%%%%%%%%%%%%%%%%%%%%%%%%%%%%%

Identificamos anteriormente que o potencial pode ser definido através da expressão para o cálculo do trabalho através da integral da força que dá origem ao potencial, Equação~\eqref{Eq:CalculoDoPotencial}. Tal expressão respeita uma propriedade fundamental do potencial que é sua dependência exclusiva na configuração atual do sistema, o que implica em -- no caso de o estado inicial e o final serem o mesmo --
\begin{align}
  \Delta U &= - W \\
  &= -\int_{x_i}^{x_f=x_i} F(x) dx \\
  &= 0.
\end{align}

Entretanto, nem todas as forças respeitam a condição acima. A força de atrito, por exemplo, realiza um trabalho diferente de zero em um trajeto cujos pontos inicial e final são o mesmo: Se tomarmos um bloco que se desloca sobre uma mesa -- preso a um eixo por um fio de forma a descrever um movimento circular--, quando o bloco completa uma volta completa, sua energia cinética certamente é menor. Consequentemente, o trabalho é negativo:
\begin{equation}
  W_{\rm{at}} = \int_{x_i}^{x_i} \fat dx < 0.
\end{equation}

\begin{marginfigure}[-5cm]
\centering
\begin{tikzpicture}[>=Stealth, scale = 2.2]
    
    \def\r{0.5}% raio

    \coordinate (origin) at (\r,0,0);

    \draw[gray] (-0.75,0,-0.75) -- (0.75,0,-0.75) -- (0.75,0,0.75) -- (-0.75,0,0.75) -- cycle;
    \draw[dotted] (origin) \foreach \t in {5,10,...,360} {--({\r*cos(\t)},0,{\r*sin(\t)})}--cycle;
    \draw[dashdotted] (0,0,0) -- node[above]{$R$} +(0.5,0,0);
    \draw[fill] (0,0.0) circle (0.5pt);
    
    \draw[fill, gray, draw = black] (0.45,0,-0.05) rectangle (0.55,0.1,-0.05);
    \draw[fill, gray, draw = black] (0.45,0,0.05) rectangle (0.55,0.1,0.05);
    \draw[fill, gray, draw = black] (0.55,0,0.05) -- (0.55,0,-0.05) -- (0.55,0.1,-0.05) -- (0.55,0.1,0.05) -- cycle;
    \draw[fill, gray, draw = black] (0.45,0.1,0.05) -- (0.55,0.1,0.05) -- (0.55,0.1,-0.05) -- (0.45,0.1,-0.05) -- cycle;
    
    \draw[->] (0.5,0.05, 0.05) -- node[right]{$\vec{v}$} +(0,0,0.65);
\end{tikzpicture}
\caption{Num movimento circular sujeito ao atrito, ao executarmos uma volta completa, não temos um trabalho nulo. Consequentemente, não existe um potencial associado à força de atrito.}
\end{marginfigure}

\noindent{}Devido a isso, não podemos escrever um potencial para tal força. Como a integral acima é proveniente da expressão para o trabalho, podemos dizer que
\begin{quote}
\emph{se o trabalho realizado por uma força em um caminho fechado é diferente de zero, não podemos escrever um potencial para tal força.}
\end{quote}

\begin{marginfigure}
\centering
\begin{tikzpicture}[>=Stealth]
    \draw[dashed] (0,0) rectangle (4,3);
    
    \coordinate (A) at (0.5,1);
    \coordinate (B) at (3.5, 2);
    
    \draw[fill] (A) node[left]{$A$} circle (1pt);
    \draw[fill] (B) node[right]{$B$} circle (1pt);
    
    \draw[->] (A) .. controls (1,2.5) and (3,3) .. (B);
    \draw[->] (A) .. controls (2,-1) and (2,1.2) .. (B);
    
    \node (C1) at (2.5,0.5) {$C_1$};
    \node (C2) at (1,2.2) {$C_2$};
    
\end{tikzpicture}
\caption{Se o trabalho $W_{A\to B}^{C_1}$ efetuado por uma $F$ no deslocamento de $A$ até $B$ pelo caminho $C_1$ é diferente do trabalho $W_{A\to B}^{C_2}$ efetuado no mesmo deslocamento, porém pelo caminho $C_2$, então a força $F$ não é conservativa.}
\end{marginfigure}

Podemos obter uma forma alternativa, porém equivalente, a tal afirmação ao analisarmos o deslocamento entre duas configurações distintas $A$ e $B$ para um sistema, porém considerando deslocamentos por caminhos diferentes. Se o potencial é função somente da configuração atual do sistema, os valores de potencial $U_A$ e $U_B$ são os mesmo para qualquer dois caminhos tomados, logo, podemos afirmar que
\begin{quote}
\emph{se o trabalho realizado por uma força no deslocamento entre dois pontos depende do caminho, não podemos escrever um potencial para tal força.}
\end{quote}

As forças podem ser então classificadas\footnote{A denominação \emph{conservativa} e \emph{não-conservativa} será justificada posteriormente.} em dois tipos, as forças \emph{conservativas} e as \emph{não-conservativas}, ou \emph{dissipativas}. Aquelas que se encaixam no primeiro tipo são as forças para as quais podemos definir um potencial -- ou seja, são as forças para as quais o trabalho em um caminho fechado $A\to B\to A$ é nulo, ou, equivalentemente, para as quais o trabalho independe do caminho --. Já as forças dissipativas são aquelas que não respeitam tal condição. Como exemplos de forças conservativas, podemos citar o peso, a força gravitacional, a força elástica, e forças elétricas entre cargas. Por outro lado, as forças de atrito, arrasto, normal e de tensão ---~por exemplo~--- são não-conservativas.

%%%%%%%%%%%%%%%%%%%%%%%%%%%%%%%%
\section{Energia mecânica}
\label{Sec:EnergiaMecanica}
%%%%%%%%%%%%%%%%%%%%%%%%%%%%%%%%

%\textbf{É inútil querer separar isso em duas partes, é melhor já mostar que $\Delta E = W_{\rm{NC}}$}

Verificamos nas seções anteriores a existência de meios diferentes de ``armazenar'' energia e que ela pode ser transferida de um tipo para outro. Veremos a seguir, que em algumas situações podemos dizer que a soma de todos os tipos de energia em um sistema é uma \emph{constante}. Isso será de grande utilidade ao analisarmos fenômenos físicos diversos, facilitando a determinação de grandezas que temos interesse em calcular.

Se analisarmos um caso em que existam $n_f$ forças conservativas atuando sobre o sistema, podemos escrever
\begin{align}
  \Delta K &= W_{\textrm{Total}} \\
  &= \sum_{j = 1}^{n_f} W_{F_j}.
\end{align}
%
%\textbf{Claro que podem, estamos supondo que as forças são conservativas:} Se cada um dos trabalhos associados às forças $F_j$ puder ser escrito como\footnote{Note que admitimos a possibilidade de que hajam forças não conservativas, mas a soma dos trabalhos devidos a essas forças é \emph{nula}.}
\begin{equation}
  W_j = \Delta U_j,
\end{equation}
%
então temos
\begin{align}
  \Delta K &= \sum_{j = 1}^{n_f} \Delta U_j \label{Eq:TeoremTrabEnergiaSohConservativas}\\
  &= \sum_{j = 1}^{n_f} (U_j^f - U_j^i).
\end{align}
%
Podemos escrever o somatório da diferença como a diferença dos somatórios, o que resulta em
\begin{equation}
  \Delta K = \left(\sum_{j = 1}^{n_f} U_j^f\right) - \left(\sum_{j = 1}^{n_f} U_j^i\right),
\end{equation}
%
de onde obtemos
\begin{equation}
  K_f + \sum_{j = 1}^{n_f} U_j^f = K_i + \sum_{j = 1}^{n_f} U_j^i.
\end{equation}

A equação acima é válida quaisquer sejam as configurações inicial e final do sistema. Logo, a soma da energia cinética e das potenciais deve ser uma constante:
\begin{equation}\label{Eq:DefEnergiaMecanicaParticula}
  K + \sum_{j = 1}^{n_f} U_j = E, \mathnote{Definição de Energia Mecânica}
\end{equation}
%
onde $E$ representa o que denominamos como \emph{energia mecânica}. Assim, podemos escrever a relação
\begin{equation}
    E_i^{\textrm{mec}} = E_f^{\textrm{mec}}, \mathnote{Conservação da energia mecânica}
\end{equation}
%
ou, equivalentemente,
\begin{equation}
    \Delta E^{\textrm{mec}} = 0,
\end{equation}
%
onde $E_i$ representa a energia em uma configuração inicial qualquer do sistema e $E_f$ a energia em uma configuração final qualquer do sistema.

Verificamos, portanto, que se as forças que atuam em um sistema dão origem a potenciais, a energia mecânica de tal sistema é uma constante. Isso é extremamente útil não só do ponto de vista prático, pois facilita os cálculos envolvidos na determinação de grandezas físicas, mas também do ponto de vista teórico: podemos agora imaginar que existe uma grandeza ---~a energia~--- que é passada de uma forma a outra dentro de um sistema, de maneira que seu valor total, somando todas as formas, permanece constante. Anteriormente classificamos as forças que dão origem a potenciais como \emph{forças conservativas}. Nos referíamos justamente ao fato de que quando somente forças desse tipo atuam em um sistema, temos uma energia mecânica \emph{constante}, ou seja, tais forças \emph{conservam} a energia mecânica.

Por outro lado, em muitos casos existem \emph{forças não-conservativas} que atuam no sistema. Caso tais forças não realizem trabalho, não precisamos nos preocupar com elas, pois a relação dada pela Equação~\eqref{Eq:TeoremTrabEnergiaSohConservativas} continua sendo válida. Já no caso de as forças não-conservativas efetuarem trabalho, teremos uma situação um pouco mais complexa, que será discutida nas Seções~\ref{Sec:TrabalhoForcasNaoConservativas} e~\ref{Sec:PrincipioDaConsDaEnergia}.\footnote{Veremos que no caso em que a energia mecânica não é constante poderemos associá-la a outras formas de energia e teremos um princípio geral de conservação da energia.}

%%%%%%%%%%%%%%%%%%%%%%%%%%%%%%%%%%%%%%%%%%%%%%%
\paragraph{Exemplo: Bloco lançado por uma mola}
%%%%%%%%%%%%%%%%%%%%%%%%%%%%%%%%%%%%%%%%%%%%%%%

\begin{quote}
    A Figura~\ref{Fig:BlocoArremessadoEmLoop} mostra um bloco que será lançado, empregando uma mola, em uma pista composta de uma seção plana e de uma seção circular vertical. Se a constante elástica da mola é $k = \np[N/m]{1200}$, a massa do bloco é $m = \np[kg]{0.750}$, e o raio da pista é $\np[cm]{130}$, qual deve ser a compressão $\ell$ da mola para que o bloco chegue ao ponto mais alto da pista sem que perca contato com ela, assumindo que ele partiu do repouso?
\end{quote}

\begin{marginfigure}
\centering
\begin{tikzpicture}[>=Stealth, scale = 0.4,
     interface/.style={
        % superfície
        postaction={draw,decorate,decoration={border,angle=-45,
                    amplitude=0.2cm,segment length=2mm}}},
    ]
    
    %%
    
    \draw[interface] (0,-0.5) -- (0,-2.5);
    \draw[interface] (0,-2.5) -- (7.05, -2.5);
    \draw[interface] (7.05,-2.5) arc[start angle = -90, end angle = 123, radius = 3 cm];
    
    \draw[->] (0,-3.2) -- (6.5,-3.2) node[below left]{$x$};
    \draw[|-|] (2.2, -3.2) -- (4.2,-3.2);
    \path (2.2,-3.3) node[below]{$\ell$} -- (4.2,-3.3) node[below]{0};
    
    \draw (0,-2) -- (0.2,-2);
    \draw[decoration={aspect=0.4, segment length=1.5625mm, amplitude=1.6mm,coil},decorate] (0.2,-2) -- (3.3,-2);
    \draw (3.3, -2) -- (3.5,-2);
    
    \draw[pattern = north west lines] (3.7,-2.5) rectangle (4.7,-1.5);
    \draw[pattern = north east lines] (3.5,-2.5) rectangle (3.7,-1.5);

        
\end{tikzpicture}
\caption{Lançamento de um bloco em uma pista circular vertical. \label{Fig:BlocoArremessadoEmLoop}}
\end{marginfigure}

No sistema formado pelo bloco, a mola, e a pista, temos a atuação de três forças: o peso do bloco, a força normal exercida pelas superfícies, e a força exercida pela mola. Dessas, somente a força normal não é conservativa, porém seu trabalho é nulo pois ela atua sempre perpendicularmente ao deslocamento instantâneo do bloco. Portanto, podemos utilizar a conservação da energia mecânica:
\begin{align}
    E_i^{\textrm{mec}} &= E_f^{\textrm{mec}} \\
    K_i + U_e^i + U_g^i &= K_f + U_g^f + U_e^f
\end{align}
%
Como o bloco parte do repouso, sua energia cinética inicial é zero. Além disso, podemos escolher a posição inicial do bloco como a origem do eixo vertical, assim a energia potencial gravitacional inicial é nula. Por fim, sabemos que quando o bloco deixa a mola, a energia potencial elástica é zero. Assim, a expressão acima simplifica-se a:
\begin{equation}\label{Eq:EnergiasParaOBlocoEmLoop}
    U_e^i = K_f + U_g^f.
\end{equation}
%
Note que a energia cinética final não é nula, já que o bloco não pode ter velocidade nula ao chegar ao topo da seção circular da pista. Sua velocidade será dada por\footnote{Esse resultado foi obtido na subseção \emph{Condição de perda de contato e velocidade mínima}, na Seção~\ref{Sec:ForcasNoMovCircular}.}
\begin{equation}
    v = \sqrt{Rg}.
\end{equation}
%
Se as dimensões do bloco são desprezíveis\footnote{Lembre-se que estamos tratando do movimento de partículas, trataremos corpos extensos somente a partir do próximo capítulo.}, sua posição vertical final será igual ao diâmetro da pista circular ---~ou seja, $y_f = 2R$ ~---.

Substituindo os resultados acima na Equação~\ref{Eq:EnergiasParaOBlocoEmLoop} obtemos
\begin{align}
    \frac{kx_i^2}{2} &= \frac{mv_f^2}{2} + mgy_f \\
    \frac{k\ell^2}{2} &= \frac{m(\sqrt{Rg})^2}{2} + mg2R \\
    \frac{k\ell^2}{2} &= \frac{mgR}{2} + 2mgR \\
    k\ell^2 &= 5mgR \\
    \ell &= \sqrt{\frac{5mgR}{k}}.
\end{align}
%
Substituindo os valores das constantes, temos finalmente
\begin{align}
    \ell &= \sqrt{\frac{5\cdot(\np[kg]{0.750}) \cdot (\np[m/s^2]{9.8}) \cdot (\np[m]{1.30}) }{(\np[N/m]{1200})}} \\
    &\approx \np[m]{0.20}.
\end{align}


%%%%%%%%%%%%%%%%%%%%%%
%\paragraph{Exemplo: ?}
%%%%%%%%%%%%%%%%%%%%%%

%\textbf{Algo parecido com o 25 da lista.}

%%%%%%%%%%%%%%%%%%%%%%%%%%%%%%%%%%%%%%%%%%%
\paragraph{Discussão: Oscilador harmônico}
%%%%%%%%%%%%%%%%%%%%%%%%%%%%%%%%%%%%%%%%%%%

Um sistema formado por um corpo de massa $m$ ligado a uma mola de massa desprezível e de constante elástica $k$, de forma que o primeiro pode se mover livremente em um eixo, constitui o que denominamos como um \emph{oscilador harmônico}. A principal característica de um oscilador harmônico é o fato de que as funções temporais da posição, velocidade, e aceleração são dadas por funções trigonométricas: Se escrevermos a Segunda Lei de Newton para o eixo $x$ mostrado na Figura~\ref{Fig:OsciladorHarmDeducao}, temos
\begin{align}
    F_R^x &= m a_x \\
    -k x &= m a_x,
\end{align}
%
o que pode ser escrito como
\begin{equation}
    a + \frac{k}{m} x = 0.
\end{equation}

\begin{marginfigure}
\centering
\begin{tikzpicture}[>=Stealth,
     interface/.style={
        % superfície
        postaction={draw,decorate,decoration={border,angle=-45,
                    amplitude=0.2cm,segment length=2mm}}},
    ]
    
    %%
    
    \draw[interface] (0,-1) -- (0,-2.5);
    \draw[interface] (0,-2.5) -- (4.8, -2.5);
    
    \draw (0,-2) -- (0.2,-2);
    \draw[decoration={aspect=0.3, segment length=2.5625mm, amplitude=2mm,coil},decorate] (0.2,-2) -- (3.3,-2);
    \draw (3.3, -2) -- (3.5,-2);
    
    \draw[pattern = north west lines] (3.5,-2.5) rectangle (4.5,-1.5);
    
    \draw[fill] (4,-2) circle (1pt);
    \draw[->, thick] (4,-2) -- +(0,-1) node[right]{$\vec{P}$};
    \draw[->, thick] (4,-1.5) -- node [right]{$\vec{N}$} +(0,1);
    \draw[->, thick] (3.5, -2) -- node[below right]{$\vec{F}_e$} +(-1,0);
    
    \draw[->] (0,-3.5) -- (4.8,-3.5) node[below left]{$x$};
    \draw[|-|] (2.5,-3.5) -- (4,-3.5);
    \path (2.5,-3.6) node[below]{$0$} -- (4,-3.6) node[below]{$A$};
    \draw[-|] (2.5,-3.5) -- (1,-3.5);
    \path (2.5,-3.6) -- (1,-3.6) node[below]{\llap{$-$}$A$};
        
\end{tikzpicture}
\caption{Um oscilador harmônico exibe duas formas de energia: cinética e potencial elástica. \label{Fig:OsciladorHarmDeducao}}
\end{marginfigure}

Sabemos que a aceleração é a derivada segunda em relação ao tempo, logo,
\begin{equation}\label{Eq:EqDifOscilador}
    \frac{d^2x}{dt^2} + \frac{k}{m} x = 0,
\end{equation}
%
de onde percebemos que a solução para a equação acima é uma \emph{função do tempo}:
\begin{equation}
    t \mapsto x = x(t).
\end{equation}
%
Como sabemos que o movimento do sistema é oscilatório, podemos assumir uma solução tentativa na forma
\begin{equation}\label{Eq:SolPosOscilador}
    x(t) = A \sen(\omega t + \phi),
\end{equation}
%
cujas derivadas são
\begin{align}
    v(t) &\equiv \frac{d}{dt} x(t) = A \omega \cos(\omega t + \phi) \\
    a(t) &\equiv \frac{d^2}{dt^2} x(t) = -A \omega^2 \sen(\omega t + \phi), \label{Eq:SolAcelOscilador}
\end{align}
%
onde $\omega$ e $\phi$ são constantes e $A$ representa o máximo deslocamento em relação à posição de equilíbrio, e é conhecida como \emph{amplitude} da oscilação. Substituindo as Expressões~\eqref{Eq:SolPosOscilador} e~\eqref{Eq:SolAcelOscilador} na Equação~\eqref{Eq:EqDifOscilador}, obtemos
\begin{equation}
    -A \omega^2 \sen(\omega t + \phi) + \frac{k}{m} A \sen(\omega t + \phi) = 0,
\end{equation}
%
de onde obtemos que as soluções para a posição, velocidade, e aceleração são válidas se
\begin{equation}
    \omega^2 = \frac{k}{m}. \label{Eq:FrequenciaAngular}
\end{equation}
%
A grandeza $\omega$ é denominada \emph{frequência angular} do oscilador e está relacionada à frequência de oscilação $\nu$, sendo que
\begin{equation}
    \omega = 2\pi\nu.
\end{equation}
%
O \emph{ângulo de fase} $\phi$ é uma constante que ajusta os valores de posição, velocidade, e aceleração a valores particulares de uma situação qualquer. Se, por exemplo, temos uma posição $x(t) = 0$ em $t = 0$, então temos que $\phi$ é nulo. A amplitude $A$ nos dá o valor máximo de deslocamento do oscilador em relação à posição de equilíbrio $x = 0$.

Se calcularmos as energias potencial elástica e cinética de um oscilador harmônico como funções do tempo, obtemos
\begin{align}
    K &= \frac{m}{2} v^2(t) \\
    &= \frac{1}{2}m \left(A \omega \cos(\omega t + \phi)\right)^2 \\
    &= \frac{1}{2}m A^2 \omega^2 \cos^2(\omega t + \phi) \\
    U_e &= \frac{k}{2} x^2(t) \\
    &= \frac{1}{2} k \left(A \sen(\omega t + \phi)\right)^2 \\
    &= \frac{1}{2} k A^2 \sen^2(\omega t + \phi).
\end{align}
%
Sabemos que nesse problema não existem forças dissipativas, por isso a energia mecânica é constante. Podemos utilizar as expressões acima para verificar que isso é verdade:
\begin{align}
    E &= K + U \\
    &= \frac{1}{2}m A^2 \omega^2 \cos^2(\omega t + \phi) + \frac{1}{2} k A^2 \sen^2(\omega t + \phi) \\
    &= \frac{1}{2} A^2 \left(m \omega^2 \cos^2(\omega t + \phi) + k \sen^2(\omega t + \phi)\right).
\end{align}
%
Usando a expressão~\eqref{Eq:FrequenciaAngular} para a frequência angular, obtemos
\begin{align}
    E &= \frac{1}{2} A^2 \left(\frac{k}{m} m \cos^2(\omega t + \phi) + k \sen^2(\omega t + \phi)\right) \\
    &= \frac{1}{2} A^2 k \; \big(\cos^2(\omega t + \phi) + \sen^2(\omega t + \phi)\big) \\
    &= \frac{1}{2} k A^2,
\end{align}
%
onde usamos $\sen^2\alpha + \cos^2\alpha = 1$. Observe que o resultado obtido corresponde à energia potencial elástica na posição de máximo afastamento da posição de equilíbrio, onde temos que o corpo pára momentaneamente e sua energia cinética é nula.
\begin{marginfigure}[-8cm]
\centering
\begin{tikzpicture}[>=Stealth, scale = 1.45, extended line/.style={shorten >=-#1,shorten <=-#1},
 extended line/.default=3mm]] % talvez fosse melhor amplicar com scale=1.5
    % Draw axes: acho que o |- é pra desenhar um "canto", um L
    \draw[->] (0,0) -- (0,1.5) node[below left] {$E$};
	\draw[->] (0,0) -- (3,0) node[below left] {$t$};

    % Desenhar função:
    \draw[smooth, thick, name path=plot,samples=1000,domain=0:2.8]
    plot(\x,{(sin((3 * \x) r))^2}) node[right]{$U_e$};
    \draw[smooth, thick, dashed, name path=plot,samples=1000,domain=0:2.8]
    plot(\x,{(cos((3 * \x) r))^2}) node[right]{$K$};
    
    \draw[dotted, thick] (0,1) -- (3, 1) node[above]{$E$};

	\end{tikzpicture}
\caption{Energias cinética $K$, potencial elástica $U_e$, e mecânica $E$ para um oscilador harmônico. Note que a energia mecânica não varia com o tempo.\label{Fig:GraphEnergiasOsciladorHarmonico}}
\end{marginfigure}

Na Figura~\ref{Fig:GraphEnergiasOsciladorHarmonico} mostramos a dependência temporal de cada uma das energias. Note que, como esperado, a energia mecânica é constante. Podemos afirmar que a energia do oscilador harmônico é convertida ciclicamente entre as duas formas disponíveis no sistema.

%%%%%%%%%%%%%%%%%%%%%%
%\paragraph{Exemplo: ?}
%%%%%%%%%%%%%%%%%%%%%%

%\textbf{Alguns exemplos que possam ser resolvidos utilizando conservação da energia mecânica.}

%%%%%%%%%%%%%%%%%%%%%%%%%%%%%%%%%%%%%%%
\section{Sistemas com múltiplos corpos}
\label{Sec:Sistemas}
%%%%%%%%%%%%%%%%%%%%%%%%%%%%%%%%%%%%%%%

Um \emph{sistema} pode ser definido como um \emph{conjunto de partículas que interagem através de uma ou mais forças}. Ao tomarmos tal definição, automaticamente definimos uma \emph{fronteira} que separa o sistema do resto do universo. A utilidade de tal definição reside no fato de que em muitos casos podemos efetuar uma escolha de maneira que nenhum trabalho externo é realizado sobre qualquer partícula pertencente ao sistema. Isso nos permite dizer que a energia mecânica do sistema ---~isto é, a soma das energia mecânicas das partículas que compõe o sistema~--- é constante.

Devido às forças entre as partículas, pode ocorrer a transferência de energia cinética ou potencial entre elas, porém a energia mecânica do sistema como um todo se mantém constante. Esse tipo de sistema é denominado um \emph{sistema fechado}. Temos então que para $n$ partículas, a energia do sistema e dada por
\begin{align}
    E_S &= \sum_{i = 1}^{n} E\\
    &= \sum_{i = 1}^{n} K_i + \sum_{i = 1}^n \sum_{j = 1}^{n_f} U_j,
\end{align}
%
onde a energia $E$ de cada partícula é dada pela Equação~\ref{Eq:DefEnergiaMecanicaParticula}.

%%%%%%%%%%%%%%%%%%%%%%%%%%%%%
\paragraph{Máquina de Atwood}
%%%%%%%%%%%%%%%%%%%%%%%%%%%%%

Podemos utilizar a expressão acima para determinar a velocidade dos blocos em uma máquina de Atwood após eles terem percorrido uma determinada distância. Se definirmos um sistema como o mostrado na Figura~\ref{Fig:MaquinaDeAtwood}, temos que não existe nenhuma força que realize trabalho sobre qualquer dos corpos pertencentes a tal sistema. Além disso, vamos considerar que as massas da polia e da corda são desprezíveis, e também que a corda é inextensível. Logo, a energia mecânica no sistema deve ser constante. Podemos então escrever
\begin{align}
    E_S^i &= E_S^f \\
    E_1^i + E_2^1 &= E_1^f + E_2^f \\
    K_1^i + U_{g, 1}^i + K_2^i + U_{g, 2}^i &= K_1^f + U_{g, 1}^f + K_2^f + U_{g, 2}^f.
\end{align}
%
Se os blocos partem do repouso, então as energias cinéticas iniciais são nulas. Além disso, devido à escolha do sistema de coordenadas, a energias potenciais inicial do primeiro bloco  e final do segundo são nulas. Assim,
%
\begin{marginfigure}
\centering
\begin{tikzpicture}[>=Stealth,  interface/.style={
        % superfície
        postaction={draw,decorate,decoration={border,angle=-45,
                    amplitude=0.2cm,segment length=2mm}}}
    ]
    
    \draw[dashed] (-2.3, 0) rectangle (2.3, -5.2);
    
    \draw[interface] (2,-0.3) -- (-2,-0.3);
    
    \draw[pattern = north west lines] (0,-1) circle (0.5);
    \draw[fill] (0,-1) circle (1pt);
    \draw[fill] (0,-1) -- (0.2,-0.3) -- (-0.2,-0.3) -- cycle;
    
    \draw[pattern = north west lines] (-0.8,-4) rectangle (-0.2, -4.6);
    \draw[dotted, pattern = north west lines] (-0.2,-2) rectangle (-0.8,-2.6);
    \draw (-0.5,-1) -- (-0.5,-4);
    \draw[fill] (-0.5, -4.3) circle (1pt);
    \draw[thick, ->] (-0.5,-4.3) -- +(0,-0.6) node[right]{$\vec{P}_1$};
    \draw[thick, ->] (-0.5,-4) -- +(0,0.75) node[right]{$\vec{T}$};
    
    \draw[pattern = north west lines] (0.2,-2) rectangle (0.8,-2.6);
    \draw[dotted, pattern = north west lines] (0.8,-4) rectangle (0.2, -4.6);
    \draw (0.5,-1) -- (0.5,-2);
    \draw[dotted] (0.5, -2.6) -- (0.5,-4);
    \draw[fill] (0.5, -2.3) circle (1pt);
    \draw[thick, ->] (0.5,-2.3) -- +(0,-1) node[right]{$\vec{P}_2$};
    \draw[thick, ->] (0.5,-2) -- +(0,0.75) node[right]{$\vec{T}$};
    
    \draw[|-|] (1,-2.3) -- node[right]{$d$} (1,-4.3);
    \draw[->] (-1,-5) -- (-1, -2) node[left]{$y$};
    \draw (-0.95, -4.3) -- (-1.05,-4.3) node[left]{$y=0$};
\end{tikzpicture}
\caption{Se definirmos um sistema como sendo o mostrado na figura acima, incluindo a Terra, temos que nenhuma força externa ao sistema realiza trabalho sobre ele. Consequentemente, a energia mecânica é constante. \label{Fig:MaquinaDeAtwood}}
\end{marginfigure}
%
\begin{align}
    U_{g, 2}^i &= K_1^f + U_{g, 1}^f + K_2^f \\
    m_2gy_2^i &= \frac{1}{2} m_1 (v_1^f)^2 + m_1gy_1^f + \frac{1}{2} m_2 (v_2^f)^2,
\end{align}
%
ou, como $v_1^f = v_2^f \equiv v_f$,
\begin{align}
    m_2gy_2^i &= \frac{1}{2} (m_1 + m_2) v_f^2 + m_1gy_1^f \\
    2gd(m_2-m_1) &= (m_1+m_2) v_f^2,
\end{align}
%
onde usamos o fato de que $y_2^i = y_1^f = d$. Finalmente, temos que a velocidade final é dada por
\begin{equation}
    v_f = \sqrt{2gd\frac{m_2-m_1}{m_1+m_2}}.
\end{equation}

%%%%%%%%%%%%%%%%%%%%%%%%%%%%%%%%%%%%%%%%%%
\section{Análise de gráficos de potencial}
%%%%%%%%%%%%%%%%%%%%%%%%%%%%%%%%%%%%%%%%%%

O fato de que a energia é um escalar faz com que em algumas situações seja muito mais simples se obter informações analisando diretamente o potencial associado a uma força\footnote{Isso é notavel ao se tratar de potenciais elétricos: a tensão elétrica em um condutor em um circuito elétrico nada mais é do que uma medida da diferença de potencial elétrico entre tal ponto e o ponto de menor potencial.} Veremos a seguir que um simples gráfico do potencial em função da posição encerra várias informações interessantes.

%%%%%%%%%%%%%%%%%%%%%%%%%%%%%%%%%%%%%%%%%%%%%%
\subsection{Forças em um gráfico de potencial}
%%%%%%%%%%%%%%%%%%%%%%%%%%%%%%%%%%%%%%%%%%%%%%

Verificamos anteriormente que a força pode ser obtida através da derivada do potencial:\footnote{Vamos fazer uma discussão considerando forças unidimensionais. No entanto, as conclusões aqui obtidas podem ser extendidas às três dimensões.}
\begin{equation*}
    F = -\frac{dU}{dx}.
\end{equation*}
%
A derivada nos dá a \emph{taxa de variação} de uma grandeza. Em um gráfico, isso equivale à inclinação da reta tangente. Isso significa que em gráfico $U \times x$, \emph{a inclinação da reta tangente à curva num dado ponto $(x_p, U(x_p))$ nos dá o módulo da força exercida sobre um corpo quando ele ocupa a posição $x_p$}.

Para determinar o sentido, basta notarmos que:
\begin{itemize}
    \item Se a derivada é \emph{positiva}\footnote{Isto é, a inclinação da reta tangente faz um ângulo $\theta$ maior que zero em relação ao eixo $x$ ---~ângulo $\alpha$ na Figura~\ref{Fig:SinalInclinacao}~---.}, temos que, devido ao sinal presente na expressão acima, \emph{a força é no sentido negativo do eixo $x$};
    \item Se ---~por outro lado~--- a inclinação é \emph{negativa}\footnote{Inclinação da reta tangente fazendo ângulo $\theta$ menor que zero em relação ao eixo $x$ ---~ângulo $\beta$ na Figura~\ref{Fig:SinalInclinacao}~---.}, temos que \emph{a força é no sentido positivo do eixo $x$.}
\end{itemize}

\begin{marginfigure}[1cm]
\begin{tikzpicture}[>=Stealth]
    \draw[->] (0,0) -- (4,0);
    \draw[->] (0,0) -- (0,3);
    
    \draw[thick] (0.5,0.5) -- (3, 3) coordinate (i1);
    \draw[dotted] (2,2) coordinate (O1) -- +(1,0) coordinate (h1);
    \pic[draw, "$\alpha$", angle eccentricity = 1.5] {angle = h1--O1--i1};
    
    \draw[thick, dashed] (0.5,2) -- (3.5,0.5) coordinate (i2);
    \draw[dotted] (2,1.25) coordinate (O2) -- +(1,0) coordinate (h2);
    \pic[draw, "$\beta$", angle eccentricity = 1.5] {angle = i2--O2--h2}; 
\end{tikzpicture}
\caption{O ângulo de uma reta é positivo se ele está acima do eixo horizontal, como no caso do ângulo $\alpha$ na figura acima. Se ele estiver abaixo da reta horizontal, caso do ângulo $\beta$, ele é negativo. \label{Fig:SinalInclinacao}}
\end{marginfigure}

%%%%%%%%%%%%%%%%%%%%%%%%%%%%%%%%%
\paragraph{Forças eletrostáticas}
%%%%%%%%%%%%%%%%%%%%%%%%%%%%%%%%%

Como um exemplo onde podemos extrair informações sobre a força a partir de um gráfico de potencial, verificaremos agora as características do potencial entre duas cargas elétricas pontuais. Tal potencial também será essencial para que possamos analisar o potencial inter-atômico que veremos na seção seguinte.

A força eletrostática entre duas cargas pontuais quaisquer é dada por
\begin{equation}
    F_{1\to 2}^e(r) = k \frac{q_1 q_2}{r^2}.
\end{equation}
%
Nas equações acima, $q_1$ e $q_2$ representam os valores de duas cargas elétricas que podem ser positivas ou negativas e estão separadas por uma distância $r$. A constante $k$ é caraterística da interação eletromagnética e seu valor é \np[N\cdot m^2 / C^2]{8.99E9}.

A expressão para a força $F_{1\to 2}^e$ nos dá a força exercida sobre a carga $q_2$ pela carga $q_1$. A reação $F_{2\to 1}^e$ exercida pela carga $q_2$ sobre $q_1$ tem o mesmo módulo que $F_{1\to 2}^e$. A força se dá sempre na direção da reta que liga as duas cargas e, consequentemente, temos um problema unidimensional. O sentido, porém, depende das cargas envolvidas: se as cargas possuem o mesmo sinal, a força é repulsiva; se possuem sinais opostos, a força é atrativa. A Figura~\ref{Fig:ForcasEntreCargas} ilustra tais possibilidades.

\begin{marginfigure}[-1cm]
\centering
\begin{tikzpicture}[>=Stealth]
    \draw[pattern = north west lines, pattern color = gray] (0,0) node {$+$} circle (2mm);
    \draw[pattern = north west lines, pattern color = gray] (2.5,0) node {$-$} circle (2mm);
    
    \draw[dashed] (-0.7,0) -- (-0.2,0);
    \draw[dashed] (0.2,0) -- (2.3,0);
    \draw[dashed, ->] (2.7,0) -- (3.8,0) node[below left]{$r$};
    \draw[dashed] (0,-0.7) -- (0,-0.2);
    \draw[dashed] (0,0.2) -- (0,0.7);
    
    \draw[->, thick] (0.2,0) -- node[above]{$F'$} +(0.5,0);
    \draw[->, thick] (2.3,0) -- node[above]{$F$} +(-0.5,0);
    
    \draw[pattern = north west lines, pattern color = gray] (0,-2) node{$+$} circle (2mm);
    \draw[pattern = north west lines, pattern color = gray] (2.5,-2) node{$+$} circle (2mm);
    
    \draw[dashed] (-0.7,-2) -- (-0.2,-2);
    \draw[dashed] (0.2,-2) -- (2.3,-2);
    \draw[dashed, ->] (2.7,-2) -- (3.8,-2) node[below left]{$r$};
    \draw[dashed] (0,-2.7) -- (0,-2.2);
    \draw[dashed] (0,-1.8) -- (0,-1.3);
    
    \draw[->, thick] (-0.2,-2) -- node[above]{$F'$} +(-0.5,0);
    \draw[->, thick] (2.7,-2) -- node[above]{$F$} +(0.5,0);
\end{tikzpicture}
\caption{Forças entre cargas elétricas de sinais opostos são atrativas (figura superior); entre cargas elétricas de mesmo sinal são repulsivas (figura inferior). \label{Fig:ForcasEntreCargas}}
\end{marginfigure}

O potencial eletrostático  pode ser obtido através da Equação~\ref{Eq:CalculoDoPotencial}. Escolhendo o ponto inicial como sendo no \emph{infinito}, onde atribuimos energia potencial \emph{nula}, obtemos\footnote{Essa escolha do referencial leva a uma classificação do sistema em termos da energia mecânica: se $E < 0$, as partículas estão \emph{ligadas} ---~isto é, elas não tem energia para se afastar indefinidamente~---; se $E > 0$, existe energia suficiente para que elas estão \emph{livres} ---~ou seja, se afastem indefinidamente após uma aproximação máxima~---.} 
\begin{align}
    V(r) &= - \int_\infty^r k\frac{q_1q_2}{r^2} dr \\
    &= -k q_1 q_2 \int_\infty^r k\frac{1}{r^2} dr \\
    &= -k q_1 q_2 \left[-\frac{1}{r} + C\right]_\infty^r \\
    &= -k q_1 q_2 \left[\left(\frac{-1}{r} + C \right) - \left(\frac{-1}{\infty} + C\right)\right] \\
    &= k\frac{q_1 q_2}{r}.
\end{align}
%
Note que o potencial elétrico é denotado por $V(r)$ ao invés de $U(r)$.

Podemos analisar o potencial acima através da inclinação da reta tangente ao potencial. Na Figura~\ref{Fig:PotencialEletromagRepulsivoAtrativo} à esquerda temos um potencial repulsivo, cuja derivada nos dá a força exercida sobre a partícula da direita. Note que para qualquer valor de $r$ temos uma inclinação \emph{negativa} da reta tangente. Portanto, para todos os pontos, temos uma força no sentido positivo do eixo $r$ (força repulsiva).

No caso de termos cargas com sinais opostos, temos que a força é atrativa. Na Figura~\ref{Fig:PotencialEletromagRepulsivoAtrativo} à direita temos o potencial que corresponde a essa situação. Veja que para todos os valores de $r$ temos uma reta com inclinação positiva, o que corresponde a uma força no sentido negativo do eixo $r$ (força atrativa).

\begin{figure}\forcerectofloat
\centering
\begin{minipage}[c]{0.45\linewidth}
\centering
\begin{tikzpicture}[>=Stealth,
     interface/.style={
        % superfície
        postaction={draw,decorate,decoration={border,angle=-45,
                    amplitude=0.2cm,segment length=2mm}}},
    ]
    
    % Desenhar função:
    \draw[smooth, thick, name path=plot,samples=1000,domain=0.25:3.5]
    plot(\x,{1/(2*\x)}) node[above left]{$r^{-1}$};
    
    \draw[fill] (1, 0.5) circle (1.3pt);
    \draw[dashed] (1, 0.5)+(-26.57:0.75) -- +(153.43:0.75);
       
    \draw[->] (-0.2,0) -- (4,0) node[above left]{$r$};
    \draw[->] (0,-0.2) -- (0,2.2) node[below left]{$V$};
    
    \draw[pattern = north west lines, pattern color = gray] (0,-1.5) node{$+$} circle (2mm);
    \draw[pattern = north west lines, pattern color = gray] (1,-1.5) node{$+$} circle (2mm);
    
    \draw[dashed] (-0.7,-1.5) -- (-0.2,-1.5);
    \draw[dashed] (0.2,-1.5) -- (0.8,-1.5);
    \draw[dashed, ->] (1.2,-1.5) -- (4,-1.5) node[below left]{$r$};
    \draw[dashed] (0,-2.2) -- (0,-1.7);
    \draw[dashed] (0,-1.3) -- (0,-0.8);
    
    \draw[->, thick] (-0.2,-1.5) -- node[above]{$F'$} +(-0.5,0);
    \draw[->, thick] (1.2,-1.5) -- node[above]{$F$} +(0.5,0);
    
    \draw[dotted] (1,0.5) -- (1,-1.3);
        
\end{tikzpicture}
\end{minipage}
%
\begin{minipage}[c]{0.45\linewidth}
\centering
\begin{tikzpicture}[>=Stealth,
     interface/.style={
        % superfície
        postaction={draw,decorate,decoration={border,angle=-45,
                    amplitude=0.2cm,segment length=2mm}}},
    ]
    
    % Desenhar função:
    \draw[smooth, thick, name path=plot,samples=1000,domain=0.25:3.5]
    plot(\x,{-1/(2*\x)}) node[below left]{$-r^{-1}$};
    
    \draw[fill] (1.25, -0.4) circle (1.3pt);
    \draw[dashed] (1.25, -0.4)+(17.74:0.75) -- +(-162.26:0.75);
       
    \draw[->] (-0.2,0) -- (4,0) node[above left]{$r$};
    \draw[->] (0,-2.2) -- (0,0.5) node[below left]{$V$};
    
    %
    
    \draw[pattern = north west lines, pattern color = gray] (0,1.5) node {$+$} circle (2mm);
    \draw[pattern = north west lines, pattern color = gray] (1.25,1.5) node {$-$} circle (2mm);
    
    \draw[dashed] (-0.7,1.5) -- (-0.2,1.5);
    \draw[dashed] (0.2,1.5) -- (1.05,1.5);
    \draw[dashed, ->] (1.45,1.5) -- (4,1.5) node[below left]{$r$};
    \draw[dashed] (0,0.8) -- (0,1.3);
    \draw[dashed] (0,1.7) -- (0,2.2);
    
    \draw[->, thick] (0.2,1.5) -- node[above]{$F'$} +(0.4,0);
    \draw[->, thick] (1.05,1.5) -- node[above]{$F$} +(-0.4,0);
    
    \draw[dotted] (1.25,-0.4) -- (1.25,1.3);
        
\end{tikzpicture}
\end{minipage}
\caption{Potenciais eletrostáticos repulsivo (esquerda) e atrativo (direita). \label{Fig:PotencialEletromagRepulsivoAtrativo}}
\end{figure}

%%%%%%%%%%%%%%%%%%%%%%%%%%%%%%%%%%%%%%%%%%%
\subsection{Pontos e regiões de equilíbrio}
%%%%%%%%%%%%%%%%%%%%%%%%%%%%%%%%%%%%%%%%%%%

Se um potencial tem um ponto, ou uma região, onde a inclinação da reta tangente é zero --~isto é, a reta tangente é \emph{horizontal}~--, temos que a derivada do potencial é nula. Logo,
\begin{align}
    F &= - \frac{dU}{dx} \\
    &= 0.
\end{align}
%
Portanto, nos pontos onde a inclinação da reta tangente ao potencial é zero, temos \emph{pontos de equilíbrio}.

%%%%%%%%%%%%%%%%%%%%%%%%%%%%%%%%%%%
\paragraph{Potencial inter-atômico}
%%%%%%%%%%%%%%%%%%%%%%%%%%%%%%%%%%%

Se tomarmos dois átomos quaisquer, temos diversas cargas interagindo. Quando os átomos ficam muito próximos, existe uma força repulsiva muito intensa, devido à proximidade de suas eletrosferas. Por outro lado, a distâncias grandes existe uma força atrativa. Podemos entender esse processo considerando a interação de potenciais como os mostrados na Figura~\ref{Fig:PotencialEletromagRepulsivoAtrativo}. A soma de potenciais desse tipo dão origem a um potencial como o mostrado na Figura~\ref{Fig:PotLennardJones}.

De uma maneira aproximada, o potencial entre dois átomos pode ser descrito através da expressão
\begin{equation}
    V(r) = \frac{A}{r^{12}} - \frac{B}{r^{6}}, \mathnote{Potencial de Lennard-Jones}
\end{equation}
%
onde $A$ e $B$ são constantes. Verificamos que esse potencial apresenta um ponto de equilíbrio que podemos determinar igualando a derivada do potencial a zero:
\begin{align}
    \frac{d}{dr}V(r) &= -\frac{12A}{r^{13}} + \frac{6B}{r^7} \\
    &= 0
\end{align}
%
o que nos leva à distância
\begin{equation}
    r_0 = \sqrt[6]{\frac{2A}{B}}.
\end{equation}
%
de equilíbrio entre os dois átomos.

\begin{marginfigure}[-2cm]
\centering
\begin{tikzpicture}[>=Stealth,
     interface/.style={
        % superfície
        postaction={draw,decorate,decoration={border,angle=-45,
                    amplitude=0.2cm,segment length=2mm}}},
    ]
    
    % Desenhar função:
    \draw[smooth, thick, name path=plot,samples=1000,domain=1.92:5]
    plot(\x,{5*((\x/2)^(-12) - (\x/2)^(-6))});
    
    \node (U) at (2.2,1.7) {$U$};
    
    \draw[->] (1.3,0) -- (5.5,0) node[above left]{$r$};
    \draw[->] (1.5,-1.5) -- (1.5,2) node[below left]{$U$};
    
    \draw[fill] (2.25,-1.23) circle (1.3pt);
    
    %
    
    \draw[<->] (1.5,-2) -- node[above]{$r_0$} (2.25,-2);
    
    \draw[pattern = north west lines, pattern color = gray] (1.5,-2.5) circle (2mm);
    \draw[pattern = north west lines, pattern color = gray] (2.25,-2.5) circle (2mm);
    
    \draw[dashed] (0.8,-2.5) -- (1.3,-2.5);
    \draw[dashed] (1.7,-2.5) -- (2.05,-2.5);
    \draw[dashed, ->] (2.45,-2.5) -- (5.5,-2.5) node[below left]{$r$};
    \draw[dashed] (1.5,-3.2) -- (1.5,-2.7);
    \draw[dashed] (1.5,-2.3) -- (1.5,-1.8);
    
    \draw[dotted] (2.25,-1.23) -- (2.25,-2.3);
        
\end{tikzpicture}
\caption{A combinação dos diversos potenciais atrativos e repulsivos entre dois átomos dá origem a um potencial efetivo que possui uma região fortemente repulvisa (quando a distância de separação é pequena), uma região moderadamente atrativa, uma região fracamente atrativa, e um \emph{ponto de equilíbrio}. \label{Fig:PotLennardJones}}
\end{marginfigure}

%%%%%%%%%%%%%%%%%%%%%%%%%%%%%%%%%%%%
\paragraph{Potencial de Woods-Saxon}
%%%%%%%%%%%%%%%%%%%%%%%%%%%%%%%%%%%%

Outro potencial cuja origem é a interação complexa entre diversas partículas é o potencial à que estão sujeitas as partículas em um núcleo atômico. O potencial a que uma partícula está sujeita, devido às demais, pode ser aproximado pela expressão
\begin{equation}
    U_{WS}(r) = - \frac{U_0}{1 + e^{(r - R) / a}}, \mathnote{Potencial de Woods-Saxon}
\end{equation}
%
onde as constantes $U_0$ e $R$ determinam a profundidade e a largura do ``poço'' de potencial, enquanto $a$ determina largura da região atrativa do potencial. A largura do poço é próxima do raio do núcleo.

\begin{marginfigure}[2cm]
\centering
\begin{tikzpicture}[>=Stealth]
    
    % Desenhar função:
    \draw[smooth, thick, name path=plot,samples=1000,domain=0:3.4]
    plot(\x,{-2 / (1 + exp((\x - 2) / 0.15))});
    
    \node (U) at (2.4,-1) {$U(r)$};
    \node (U0) at (-0.3, -2) {$U_0$};
    
    \draw[->] (-0.2,0) -- (3.5,0) node[below left]{$r$};
    \draw[->] (0,-2.3) -- (0,0.5) node[below left]{$U$};
        
\end{tikzpicture}
\caption{O potencial nuclear é marcado por uma força atrativa muito intensa, pelo curto alcance, e por uma região central de equilíbrio.}
\end{marginfigure}

As características notáveis desse potencial são uma região externa onde a força é nula, uma região onde a existe uma força atrativa muito intensa e uma região central de equilíbrio. Tais propriedades descrevem as propriedades qualitativas da interação de prótons e nêutrons com um núcleo: a distâncias pouco maiores do que o raio do núcleo, não existe nenhuma atração entre elas e o núcleo; próximo da superfície existe uma força atrativa que faz com que a partícula seja absorvida pelo núcleo; dentro do núcleo, a força média sobre uma partícula é nula (a região plana está associada\cite{Matta2017} a uma melhor descrição da \emph{saturação nuclear}, isto é, ao fato de que a densidade nuclear é aproximadamente constante).

%%%%%%%%%%%%%%%%%%%%%%%%%%%%%%%%%%%%%%
\subsection{Equilíbrio e estabilidade}
%%%%%%%%%%%%%%%%%%%%%%%%%%%%%%%%%%%%%%

Os pontos de equilíbrio podem ser classificados com base em sua estabilidade: se o ponto de equilíbrio for um máximo local, ele é um ponto de equilíbrio \emph{instável}; se for um ponto de mínimo local, é um ponto de equilíbrio \emph{estável}.

\begin{marginfigure}[2cm]
\centering
\begin{tikzpicture}[>=Stealth,
     interface/.style={
        % superfície
        postaction={draw,decorate,decoration={border,angle=-45,
                    amplitude=0.2cm,segment length=2mm}}},
    ]
    
    % Desenhar função:
    \draw[smooth, thick, name path=plot,samples=1000,domain=0:3.14]
    plot(\x,{-sin(2*\x r)}) node[right]{$U(x)$};
    
    \draw[->] (-0.5,-1.5) -- (4,-1.5) node[below left]{$x$};
    \draw[->] (-0.5,-1.5) -- (-0.5,1.5) node[below left]{$U$};
    
    \draw[fill] (0.785,-1) circle (1pt) node[above]{$A$};
    \draw[fill] (2.356,1) circle (1pt) node[below]{$B$};
    
    \draw[dashed] (0.785,-1)+(-0.75,0) -- +(0.75,0);
    \draw[dashed] (2.356,1)+(-0.75,0) -- +(0.75,0);
    

\end{tikzpicture}
\caption{Potencial com dois pontos de equilíbrio, um estável em $A$ e outro instável em $B$. \label{Fig:PotEquilEstavelEInstavel}}
\end{marginfigure}

Na Figura~\ref{Fig:PotEquilEstavelEInstavel} temos dois pontos de equilíbrio em um potencial $U(x)$, um estável em $A$ e outro instável em $B$. Essa classificação é simples de se entender através da figura:
\begin{description}
    \item[Equilíbrio estável:] Se tomarmos o ponto $A$ temos que um deslocamento ao longo do eixo $x$ em qualquer direção dá origem a uma força que sempre aponta em direção ao ponto de equilíbrio: um deslocamento no sentido positivo de $x$ dá origem a uma força no sentido negativo, enquanto um deslocamento no sentido negativo de $x$ dá origem a uma força no sentido positivo.

    \item[Equilíbrio instável:] No caso do $B$, no entanto, temos uma situação diferente. Um deslocamento em relação ao ponto de equilíbrio faz surgir sempre uma força que aponta para longe do ponto de equilíbrio: para um deslocamento no sentido positivo de $x$, surge uma força no sentido positivo; se o deslocamento é no sentido negativo, surge uma força também no sentido negativo.
\end{description}

Podemos determinar se o ponto é estável ou instável do ponto de vista de cálculo ao verificar o valor da derivada segunda no ponto de equilíbrio. Podemos determinar se um ponto crítico ---~isto é, o ponto onde a derivada é zero~--- de uma função qualquer é ponto de máximo ou ponto de mínimo através da derivada segunda:
\begin{itemize}
    \item Se $\frac{d^2}{dx^2}U(x)|_{x = x_p} > 0$, onde o ponto $x_p$ é o ponto crítico, então o ponto é um \emph{mínimo} ---~ou seja, um ponto de equilíbrio \emph{estável}---;
    \item Se $\frac{d^2}{dx^2}U(x)|_{x = x_p} < 0$, onde o ponto $x_p$ é o ponto crítico, então o ponto é um \emph{máximo} ---~ou seja, um ponto de equilíbrio \emph{instável}---;
\end{itemize}
%
Essa interpretação pode ser entendida da seguinte maneira: se nos deslocamos no sentido positivo de $x$, ao atingirmos um ponto de equilíbrio temos uma derivada nula; nesse ponto, se a derivada segunda for maior que zero, isso implica que se avançarmos além do ponto de equilíbrio, encontraremos uma região onde a derivada primeira é maior do que zero, indicando que o ponto era de equilíbrio estável. Se a derivada segunda no ponto de equilíbrio for negativa, isso implica que ao avançarmos além desse ponto encontraremos uma região em que o potencial é menor do que aquele registrado no próprio ponto de equilíbrio, indicando um ponto de equilíbrio instável.

\begin{marginfigure}[-2cm]
\centering
\begin{tikzpicture}[>=Stealth,
     interface/.style={
        % superfície
        postaction={draw,decorate,decoration={border,angle=-45,
                    amplitude=0.2cm,segment length=2mm}}},
    ]
    
    % Desenhar função:
    \draw[smooth, thick, name path=plot,samples=1000,domain=-2:2]
    plot(\x,{(\x)^3/8}) node[right]{$U(x)$};
    
    \draw[->] (-2.5,-1.5) -- (2.5,-1.5) node[below left]{$x$};
    \draw[->] (-2.5,-1.5) -- (-2.5,1.5) node[below left]{$U$};
    
    \draw[fill] (0,0) circle (1pt) node[above]{$A$};
    
    \draw[dashed] (-1.5,0) -- (1.5,0);
    
\end{tikzpicture}
\caption{Potencial com um ponto de inflexão. \label{Fig:PotPontoDeSela}}
\end{marginfigure}

Vale notar ainda que existe um outro tipo de ponto de crítico, denominado \emph{ponto de inflexão}, ou \emph{ponto de sela}\footnote{Em um gráfico de uma função bidimensional o ponto de inflexão está ligado a uma região cujo formato é o de uma sela de cavalo}. Na Figura~\ref{Fig:PotPontoDeSela} mostramos um potencial com esse tipo de ponto crítico. Note que para deslocamentos à esquerda do ponto de equilíbrio, temos características de instabilidade, enquanto que para deslocamentos à direta temos características de estabilidade.

%%%%%%%%%%%%%%%%%%%%%%%%%%%%%%%%%%%%%%%
\paragraph{Exemplo: Molécula de amônia}
%%%%%%%%%%%%%%%%%%%%%%%%%%%%%%%%%%%%%%%

\begin{figure}[tb]
\centering
\begin{tikzpicture}[>=Stealth, scale = 1, extended line/.style={shorten >=#1,shorten <=#1},
 extended line/.default=2mm]]
 
 \begin{scope}[ rotate around y = -15, rotate around z = -20]
    % Origem e eixos
    \draw[fill] (0,0,0) circle (1pt);

    \draw[densely dotted, ->] (0,0,-2) -- (0,0,2) node[below]{$x$};
    \draw[densely dotted, ->] (-2,0,0) -- (2,0,0) node[below]{$y$};
    \draw[densely dotted, ->] (0,-2,0) -- (0,2,0) node[below right]{$z$};

    % Cubo
    \draw[dotted] (1.25,1.25,1.25) coordinate (A1) -- (1.25,-1.25,1.25) coordinate (A2) -- (-1.25,-1.25,1.25) coordinate (A3) -- (-1.25, 1.25,1.25) coordinate (A4)  -- cycle;
    \draw[dotted] (1.25,1.25,-1.25) coordinate (B1) -- (1.25,-1.25,-1.25) coordinate (B2) -- (-1.25,-1.25,-1.25) coordinate (B3) -- (-1.25, 1.25,-1.25) coordinate (B4)  -- cycle;
    \draw[dotted] (A1) -- (B1) (A2) -- (B2) (A3) -- (B3) (A4) -- (B4);
    
    % Inserseções dos eixos e das faces
    \draw[fill] (1.25, 0, 0) circle (0.5pt);
    \draw[fill] (-1.25, 0, 0) circle (0.5pt);
    \draw[fill] (0, 1.25, 0) circle (0.5pt);
    \draw[fill] (0, -1.25, 0) circle (0.5pt);
    \draw[fill] (0, 0, 1.25) circle (0.5pt);
    \draw[fill] (0, 0, -1.25) circle (0.5pt);
    
    % Molécula
    \draw[pattern = north west lines] (0,1,0) coordinate (H1) circle (1.3mm);
    \draw[pattern = north west lines] (0.866,-0.5,0) coordinate (H2) circle (1.3mm);
    \draw[pattern = north west lines] (-0.866,-0.5,0) coordinate (H3) circle (1.3mm);
    
    \draw[dashed, extended line] (H1) -- (H2);
    \draw[dashed, extended line] (H2) -- (H3);
    \draw[dashed, extended line] (H3) -- (H1);
    
    \draw[pattern = north east lines] (0,0,0.35) coordinate (N) circle (1.5mm);
    \draw[pattern = north east lines, pattern color = gray, draw = gray] (0,0,-0.35) coordinate (N2) circle (1.5mm);
    \draw[dashed, extended line] (N) -- (H1);
    \draw[dashed, extended line] (N) -- (H2);
    \draw[dashed, extended line] (N) -- (H3);
    
    \draw[dotted, extended line] (N2) -- (H1);
    \draw[dotted, extended line] (N2) -- (H2);
    \draw[dotted, extended line] (N2) -- (H3);
\end{scope}

\begin{scope}[shift={(5,0)}, rotate around y = -11, rotate around z = -20]
    % Origem e eixos
    \draw[fill] (0,0,0) circle (1pt);

    \draw[densely dotted, ->] (0,0,-2) -- (0,0,2) node[below]{$x$};
    \draw[densely dotted, ->] (-2,0,0) -- (2,0,0) node[below]{$y$};
    \draw[densely dotted, ->] (0,-2,0) -- (0,2,0) node[below right]{$z$};

    % Cubo
    \draw[dotted] (1.25,1.25,1.25) coordinate (A1) -- (1.25,-1.25,1.25) coordinate (A2) -- (-1.25,-1.25,1.25) coordinate (A3) -- (-1.25, 1.25,1.25) coordinate (A4)  -- cycle;
    \draw[dotted] (1.25,1.25,-1.25) coordinate (B1) -- (1.25,-1.25,-1.25) coordinate (B2) -- (-1.25,-1.25,-1.25) coordinate (B3) -- (-1.25, 1.25,-1.25) coordinate (B4)  -- cycle;
    \draw[dotted] (A1) -- (B1) (A2) -- (B2) (A3) -- (B3) (A4) -- (B4);
    
    % Inserseções dos eixos e das faces
    \draw[fill] (1.25, 0, 0) circle (0.5pt);
    \draw[fill] (-1.25, 0, 0) circle (0.5pt);
    \draw[fill] (0, 1.25, 0) circle (0.5pt);
    \draw[fill] (0, -1.25, 0) circle (0.5pt);
    \draw[fill] (0, 0, 1.25) circle (0.5pt);
    \draw[fill] (0, 0, -1.25) circle (0.5pt);
    
    % Molécula
    \draw[pattern = north west lines] (0,1,0) coordinate (H1) circle (1.3mm);
    \draw[pattern = north west lines] (0.866,-0.5,0) coordinate (H2) circle (1.3mm);
    \draw[pattern = north west lines] (-0.866,-0.5,0) coordinate (H3) circle (1.3mm);
    
    \draw[dashed, extended line] (H1) -- (H2);
    \draw[dashed, extended line] (H2) -- (H3);
    \draw[dashed, extended line] (H3) -- (H1);
    
    \draw[pattern = north east lines] (0,0,0.35) coordinate (N) circle (1.5mm);
    \draw[pattern = north east lines, pattern color = gray, draw = gray] (0,0,-0.35) coordinate (N2) circle (1.5mm);
    \draw[dashed, extended line] (N) -- (H1);
    \draw[dashed, extended line] (N) -- (H2);
    \draw[dashed, extended line] (N) -- (H3);
    
    \draw[dotted, extended line] (N2) -- (H1);
    \draw[dotted, extended line] (N2) -- (H2);
    \draw[dotted, extended line] (N2) -- (H3);
\end{scope}
\end{tikzpicture}
\caption{\emph{Figura estereoscópica:} A molécula de amônia é composta por três átomos de hidrogênio e um átomo de nitrogênio, formando um tetraedro. Os circulos nos eixos denotam o ponto onde atravessam os planos das faces do cubo. \label{Fig:MoleculaAmonia}}
\end{figure}

Um exemplo de potencial que exibe pontos de equilíbrio estáveis e instáveis é o potencial efetivo sobre o átomo de nitrogênio em uma molécula de amônia. Essa molécula é formada por três átomos de hidrogênio e um de nitrogênio, de forma que sua estrutura tem um aspecto tetraédrico, com os átomos de hidrogênio formando um triângulo (veja a Figura~\ref{Fig:MoleculaAmonia}).

Considerando que a interação entre os átomos tenha um potencial com as mesmas propriedades qualitativas que o potencial inter-atômico discutido anteriormente, verificamos que existem duas posições de equilíbrio para o átomo de nitrogênio, ambas fora do plano dos átomos de hidrogênio, simétricas a ele. Em tais posições, o nitrogênio fica equidistante dos três hidrogênios. Se considerarmos o caso em que os átomos são pontuais, verificamos que além dessas duas posições de equilíbrio, deve haver uma terceira, contida no plano dos hidrogênios e equidistante a eles.

Na Figura~\ref{Fig:PotencialMoleculaAmonia}, temos um esboço do potencial efetivo a que o átomo de nitrogênio está sujeito. Nela verificamos a existência dos três pontos de equilíbrio, sendo dois de equilíbrio estável, e um de equilíbrio instável.

\begin{marginfigure}
\centering
\begin{tikzpicture}[>=Stealth,
     interface/.style={
        % superfície
        postaction={draw,decorate,decoration={border,angle=-45,
                    amplitude=0.2cm,segment length=2mm}}},
    ]
    
    % Desenhar função:
    \draw[smooth, thick, name path=plot,samples=1000,domain=-1.75:1.75]
    plot(\x,{0.25*(0.1*(2*\x)^4 - (2*\x)^2)}) node[right]{$U(x)$};
    
    \draw[->] (-2,1) -- (2,1) node[above left]{$x$};
    \draw[->] (0,-0.2) -- (0,2.5) node[below left]{$U$};
    
    \draw[fill] (-1.118, -0.625) circle (1.3pt);
    \draw[fill] (0,0) circle (1.3pt);
    \draw[fill] (1.118,-0.625) circle (1.3pt);

\end{tikzpicture}
\caption{O potencial ao qual o nitrogênio está sujeito tem a forma mostrada à direita. Note que existem dois pontos de equilíbrio \emph{estáveis} e um ponto de equilíbrio \emph{instável}. \label{Fig:PotencialMoleculaAmonia}}
\end{marginfigure}

%%%%%%%%%%%%%%%%%%%%%%%%%%%%%%%%%%%%%%%%%%%%%%%%%%%%%%%%%%%
\paragraph{Exemplo: Classificação de pontos de equilíbrio}
%%%%%%%%%%%%%%%%%%%%%%%%%%%%%%%%%%%%%%%%%%%%%%%%%%%%%%%%%%%

\begin{quote}
A Figura~\ref{Fig:ClassificarPontos} mostra o potencial associado a uma força conservativa unidimensional. Determine os pontos de equilíbrio e os classifique como estáveis e instáveis. O que podemos afirmar sobre a intensidade da força nos pontos onde $U(x) = 0$?
\end{quote}

\begin{figure}[b]
\centering
\begin{tikzpicture}[>=Stealth, scale = 1.35,
     interface/.style={
        % superfície
        postaction={draw,decorate,decoration={border,angle=-45,
                    amplitude=0.2cm,segment length=2mm}}},
    ]
    
    % Desenhar função:
    \draw[smooth, thick, name path=plot,samples=1000,domain=-0.05:3.8]
    plot(\x,{2*sin(6*\x r) * exp(-\x)});
    
    \draw[->] (0,0) -- (4.5,0) node[above left]{$x$};
    \draw[->] (0,-1.25) -- (0,2) node[below left]{$U$};
    
\end{tikzpicture}
\caption{Um potencial unidimensional $U(x)$. \label{Fig:ClassificarPontos}}
\end{figure}

Sabemos que os pontos de equilíbrio em um gráfico de potencial em função da posição são aqueles que são máximos ou mínimos locais, marcados na Figura~\ref{Fig:ClassificarPontos2}. Além disso, sabemos que os pontos de máximo são pontos de equilíbrio instável e os pontos de mínimo são pontos de equilíbrio estável. Assim, temos que
\begin{description}
    \item[Equilíbrio instável:] Pontos $A$, $C$, $E$, $G$.
    \item[Equilíbrio estável:] Pontos $B$, $D$, $F$.
\end{description}

\begin{marginfigure}
\centering
\begin{tikzpicture}[>=Stealth,
     interface/.style={
        % superfície
        postaction={draw,decorate,decoration={border,angle=-45,
                    amplitude=0.2cm,segment length=2mm}}},
    ]
    
    % Desenhar função:
    \draw[smooth, thick, name path=plot,samples=1000,domain=-0.05:3.8]
    plot(\x,{2*sin(6*\x r) * exp(-\x)});
    
    \draw[->] (0,0) -- (4.5,0) node[above left]{$x$};
    \draw[->] (0,-1.25) -- (0,2) node[below left]{$U$};
    
    % Pontos de equilíbrio 
    \fill (0.234274608230045, 1.56076031702256)  circle (1pt) node[above] {$A$};
    \fill (0.757873383828344, -0.924570761897113)  circle (1pt) node[below] {$B$};
    \fill (1.28147215942664, 0.547701709501274)  circle (1pt) node[above] {$C$};
    \fill (1.80507093502494, -0.324450193487731)  circle (1pt) node[below] {$D$};
    \fill (2.32866971062324, 0.192199378289473)  circle (1pt) node[above] {$E$};
    \fill (2.85226848622154, -0.113855999337713)  circle (1pt) node[below] {$F$};
    \fill (3.37586726181984, 0.067446568769152)  circle (1pt) node[above] {$G$};
    
    % Zeros do potencial
    \draw[fill = white, draw = black] (0,0) circle (1pt);
    \draw[fill = white, draw = black] (0.523583333,0) circle (1pt);
    \draw[fill = white, draw = black] (1.047166667,0) circle (1pt);
    \draw[fill = white, draw = black] (1.57075,0) circle (1pt);
    \draw[fill = white, draw = black] (2.094333333,0) circle (1pt);
    \draw[fill = white, draw = black] (2.617916667,0) circle (1pt);
    \draw[fill = white, draw = black] (3.1415,0) circle (1pt);
    \draw[fill = white, draw = black] (3.665083333,0) circle (1pt);

\end{tikzpicture}
\caption{Pontos de equilíbrio do potencial $U(x)$. \label{Fig:ClassificarPontos2}}
\end{marginfigure}

Já sobre os pontos em que $U(x) = 0$, marcados com circulos vazados sobre o eixo $x$, podemos destacar que, apesar de o potencial ser nulo, a sua derivada é diferente de zero. Percebemos que no sentido crescente de $x$ se alternam pontos onde a força aponta para o sentido negativo do eixo $x$ e pontos onde a força aponta no sentido positivo de $x$. Além disso, percebemos que a intensidade da força diminui progressivamente, pois a inclinação da curva do potencial diminui em cada passagem sobre o eixo $x$.

\begin{quote}
    Se o potencial $U(x)$ é dado pela expressão
    \begin{equation}
        U(x) = 2\cdot\sen(6x)\cdot e^{-x},
    \end{equation}
    %
    determine os pontos de equilíbrio através da derivada do potencial e os classifique quanto à estabilidade através da derivada segunda.
\end{quote}

Para determinarmos os pontos de equilíbrio, basta determinarmos a derivada primeira e a igualar a zero:
\begin{align}
    \frac{d}{dx} U(x) &= 0 \\
    \frac{d}{dx} [2\cdot\sen(6x)\cdot e^{-x}] &= 0 \\
    2\cdot[(6\cos(6x)e^{-x}) + (\sen(6x)\cdot (-1) \cdot e^{-x})] &= 0 \\
    2\cdot(6\cos(6x) - \sen(6x))\cdot e^{-x} &= 0.
\end{align}
%
A expressão acima só é zero se o termo entre parêntesis é zero, uma vez que $e^{-x}$ só é zero no limite $x \to \infty$. Assim, temos que
\begin{align}
    6\cos(6x) - \sen(6x) &= 0 \\
    6\cos(6x) &= \sen(6x) \\
    6 &= \tan(6x),
\end{align}
%
de onde obtemos
\begin{align}
    x &= \frac{\arctan(6)}{6} \\
    &\approx \np{0.2342746}.
\end{align}
%
Para encontrar as demais soluções, devemos notar que $\tan(\theta) = \tan(\theta + n\pi)$ ---~onde $n = 0, 1, 2, \dots$~---. Portanto,
\begin{align}
    \tan(6x + n\pi) &= 6 \\
    \tan\left(6\left(x + \frac{n\pi}{6}\right)\right) &= 6,
\end{align}
%
ou seja, as soluções seguintes à primeira são dadaos por
\begin{equation}
    x_n = x + \frac{n\pi}{6},
\end{equation}
%
onde $x = \arctan(6)/ 6 = \np{0.2342746}$, conforme a solução encontrada acima. Isso resulta nas soluções
\begin{align*}
x_0 &= \np{0.2342746} & x_4 &= \np{2.3286697} \\
x_1 &= \np{0.7578734} & x_5 &= \np{2.8522685} \\
x_2 &= \np{1.2814722} & x_6 &= \np{3.3758676} \\
x_3 &= \np{1.8050709}.
\end{align*}

Para classificarmos quanto à estabilidade, basta calcularmos a derivada segunda,
\begin{align}
    \frac{d^2}{dx^2} U(x) &= \frac{d^2}{dx^2} [2\cdot\sen(6x)\cdot e^{-x}] \\
    &= \frac{d}{dx} [2\cdot(6\cos(6x) - \sen(6x))\cdot e^{-x}] \\
    &= 2\cdot[(-36\sen(6x) - 6\cos(6x)) e^{-x} \\
    &\phantom{=2\cdot[} + (6\cos(6x) - \sen(6x))\cdot(-1)\cdot e^{-x}] \nonumber \\
    &= -2\cdot[35\sen(6x) + 12\cos(6x)]e^{-x},
\end{align}
%
e determinarmos o valor de tal derivada nos pontos de equilíbrio:
\begin{align*}
    \frac{d^2}{dx^2} U(x)|_{x = x_0} &= -57.748 < 0 & \frac{d^2}{dx^2} U(x)|_{x = x_4} &= -7.111 < 0 \\
    \frac{d^2}{dx^2} U(x)|_{x = x_1} &= 34.209 > 0  & \frac{d^2}{dx^2} U(x)|_{x = x_5} &= 4.213 > 0 \\
    \frac{d^2}{dx^2} U(x)|_{x = x_2} &= -20.265 < 0 & \frac{d^2}{dx^2} U(x)|_{x = x_6} &= -2.496 < 0 \\
    \frac{d^2}{dx^2} U(x)|_{x = x_3} &= 12.005 > 0.
\end{align*}
%
Logo, são pontos de equilíbrio instável os pontos $x_0$, $x_2$, $x_4$, e $x_6$ ---~que equivalem aos pontos $A$, $C$, $E$, e $F$ na Figura~\ref{Fig:ClassificarPontos2}~---, enquanto os pontos de equilíbrio estável são os pontos $x_1$, $x_3$, e $x_5$ ---~que equivalem aos pontos $B$, $D$, e $F$~---.

Finalmente, devemos notar que através da função do potencial podemos determinar diversos outros pontos de equilíbrio. Acima determinamos somente os pontos que estão presentes no gráfico da Figura~\ref{Fig:ClassificarPontos2}.

%%%%%%%%%%%%%%%%%%%%%%%%%%%%%%%%%%%%%%%%%%%%%%%%%%%%%%%%%%%%%%%%%%%%%%%%%
\subsection{Energia mecânica em gráficos de potencial: Pontos de retorno}
%%%%%%%%%%%%%%%%%%%%%%%%%%%%%%%%%%%%%%%%%%%%%%%%%%%%%%%%%%%%%%%%%%%%%%%%%

Se tomarmos um sistema como o da Figura~\ref{Fig:PotElasticoComMassaMola}, onde o atrito entre a mesa e o bloco é desprezível, e o bloco pode se deslocar em torno de sua posição de equilíbrio $x = 0$, temos que o potencial que sofre variação\footnote{Temos também um potencial gravitacional, porém se a superfície de apoio é perfeitamente horizontal, esse potencial não varia.} é o potencial elástico. Ao deslocarmos o sistema até uma posição $x_A$, temos que o potencial será dado por $U_e^A = kx_A^2 /2$. Se liberarmos a movimentação do bloco a partir desse ponto, com velocidade inicial nula, temos que a energia mecânica será dada por
\begin{align}
  E_A &= U_A + K_A \\
  &= U_A.
\end{align}

\begin{marginfigure}[5cm]
\centering
\begin{tikzpicture}[>=Stealth,
     interface/.style={
        % superfície
        postaction={draw,decorate,decoration={border,angle=-45,
                    amplitude=0.2cm,segment length=2mm}}},
    ]
    
    % Desenhar função:
    \draw[smooth, thick, name path=plot,samples=1000,domain=0.5:4.5]
    plot(\x,{0.5*(\x - 2.5)^2}) node[right]{$U_e$};
    
    %%
    
    \draw[interface] (0,-1.5) -- (0,-3);
    \draw[interface] (0,-3) -- (4.8, -3);
    
    \draw (0,-2.5) -- (0.2,-2.5);
    \draw[decoration={aspect=0.3, segment length=2.5625mm, amplitude=2mm,coil},decorate] (0.2,-2.5) -- (3.3,-2.5);
    \draw (3.3, -2.5) -- (3.5,-2.5);
    
    \draw[pattern = north west lines, pattern color = gray] (3.5,-3) rectangle (4.5,-2);
    \draw[dotted, pattern = north west lines, pattern color = gray] (2,-3) rectangle (3,-2);
    
    \draw[fill] (4,-2.5) circle (1pt);
    \draw[->, thick] (4,-2.5) -- +(0,-1) node[right]{$\vec{P}$};
    \draw[->, thick] (4,-2) -- node [right]{$\vec{N}$} +(0,1);
    \draw[->, thick] (3.5, -2.5) -- node[below right]{$\vec{F}_e$} +(-1,0);
    
    \draw[->] (0,0) -- (4.8,0) node[below left]{$x$};
    \draw[<->] (2.5,-0.5) -- node[below]{$d$} (4,-0.5);
    \draw[->] (2.5,-0.2) -- (2.5,2.5) node[below left]{$U$};
    \draw[fill] (4,0) node[below left]{$A$} circle (1pt);
    
    \draw[dotted] (2.5,-0.2) -- (2.5,-2);
    \draw[dashed] (4,-1) -- (4,1.125);
    
    \draw[dashdotted] (0.5, 1.125) node[left]{$E$} -- (4.5, 1.125);
    \draw[fill] (4, 1.125) circle (1.2pt);
        
\end{tikzpicture}
\caption{Através de um gráfico do potencial em função da posição, podemos verificar em cada ponto qual é a distribuição da energia mecânica $E$. \label{Fig:PotElasticoComMassaMola} }
\end{marginfigure}

Como a energia mecânica é constante, podemos traçar uma reta horizontal no gráfico do potencial. Verificamos que para cada valor da posição $x$, a soma entre a energia potencial $U$ e a energia cinética $K$ deve ser igual a $E$ --afinal, não temos nenhuma força não-conservativa atuando no sistema~--. Logo, através do gráfico, podemos identificar rapidamente quais os pontos onde temos maior o menor energia cinética ao verificarmos onde a distância vertical entre a curva do potencial e a reta da energia mecânica é maior.

Além disso verificamos que em qualquer movimento, a energia está limitada ao valor numérico de $E$, portanto qualquer forma de energia no sistema tem no máximo tal valor. Isso implica em um valor máximo para a energia potencial que será no instante em que $K = 0$. Isso corresponde aos pontos de interseção entre a reta $E$ e a curva $U$ nos gráficos (afinal,nesses pontos a distância entre a curva e a reta é nula, o que implica em $K=0$).

Os pontos de interseção da curva do potencial pela reta da energia mecânica são denominados \emph{pontos de retorno}. No exemplo do oscilador massa-mola se o bloco se desloca em direção a um ponto de retorno, ele sofre uma força dada por
\begin{equation*}
  F = -\frac{dU}{dx},
\end{equation*}
%
o que fará com que ele sofra uma aceleração no sentido contrário ao deslocamento e eventualmente pare. Como ele continua sob efeito da força, ele passará a \emph{retornar} após atingir tal ponto. Devido a esse comportamento, em potenciais mais complexos, podem ocorrer regiões em que o movimento está confinado a um \emph{poço de potencial} limitado por dois pontos de retorno, mesmo que existam outras regiões em que o movimento do sistema seria possível (veja a Figura~\ref{Fig:PocoMoleculaAmonia}). Um aumento da energia mecânica possibilitaria, se o ganho energético for suficiente, que o sistema ultrapassasse as \emph{barreiras de potencial} que delimitam o poço e ampliasse o tamanho da região de confinamento.

\begin{marginfigure}
\centering
\begin{tikzpicture}[>=Stealth,
     interface/.style={
        % superfície
        postaction={draw,decorate,decoration={border,angle=-45,
                    amplitude=0.2cm,segment length=2mm}}},
    ]
    
    % Desenhar função:
    \draw[smooth, thick, name path=plot,samples=1000,domain=-2:2]
    plot(\x,{0.25*(0.1*(2*\x)^4 - (2*\x)^2)}) node[right]{$U$};
    
    \draw[->] (-2,0) -- (2,0) node[above left]{$x$};
    \draw[->] (0,-0.2) -- (0,2.5) node[below left]{$U$};
    
%    \draw[fill] (-1.118, -0.625) circle (1.3pt);
%    \draw[fill] (0,0) circle (1.3pt);
%    \draw[fill] (1.118,-0.625) circle (1.3pt);
    
    \draw[dashed] (-1.75, -0.3) node[below]{$E$} -- (1.75, -0.3);
\end{tikzpicture}
\caption{Dependendo do valor de energia mecânica, o movimento pode ficar restrito a um ``poço de potencial''. Se aumentássemos a energia mecânica, poderíamos ultrapassar a barreira de potencial central e ter uma oscilação entre as barreiras esquerda e direita. \label{Fig:PocoMoleculaAmonia}}
\end{marginfigure}

%%%%%%%%%%%%%%%%%%%%%%%%%%%%%%%%%%%%%%%%%%%%%%%%%%%%%%%%%%%%%%%%%%%%%%%%%%%%%%%%%%%%%%%
\paragraph{Exemplo: Aproximação máxima entre duas partículas com cargas de mesmo sinal}
%%%%%%%%%%%%%%%%%%%%%%%%%%%%%%%%%%%%%%%%%%%%%%%%%%%%%%%%%%%%%%%%%%%%%%%%%%%%%%%%%%%%%%%

\begin{quote}
    Uma partícula com carga positiva $q_1$ é lançada em direção a outra partícula, fixa, também de carga positiva $q_2$. Se sua velocidade inicial é $v$, sua massa é $m$, as cargas são $q_1 = q_2 = q$, e o ponto de lançamento é muito distante da carga $q_2$, qual é a máxima aproximação entre as partículas?
\end{quote}

No sistema formado pelas duas cargas, podemos considerar que só atua a força eletrostática, uma vez que as demais forças têm intensidade muito pequena em comparação a ela. Como essa força é conservativa, podemos considerar que a energia mecânica é constante. Verificando o gráfico do potencial eletrostático, mostrado na Figura~\ref{Fig:PotencialAproximacaoCargas} identificamos um ponto de retorno, que corresponde justamente ao ponto de máxima aproximação. Portanto, basta igualarmos o potencial à energia mecânica:
\begin{align}
    V(r) &= E \\
    \frac{kq_1q_2}{r} &= E.
\end{align}

\begin{marginfigure}
\centering
\begin{tikzpicture}[>=Stealth,
     interface/.style={
        % superfície
        postaction={draw,decorate,decoration={border,angle=-45,
                    amplitude=0.2cm,segment length=2mm}}},
    ]
    
    % Desenhar função:
    \draw[smooth, thick, name path=plot,samples=1000,domain=0.25:3.5]
    plot(\x,{1/(2*\x)});
    
       
    \draw[->] (-0.2,0) -- (4,0) node[below left]{$r$};
    \draw[->] (0,-0.2) -- (0,2.2) node[below left]{$V$};
    
    \draw[pattern = north west lines, pattern color = gray] (0,-1.5) node{$+$} circle (2mm);
    \draw[pattern = north west lines, pattern color = gray] (1,-1.5) node{$+$} circle (2mm);
    
    \draw[dashed] (-0.7,-1.5) -- (-0.2,-1.5);
    \draw[dashed] (0.2,-1.5) -- (0.8,-1.5);
    \draw[dashed, ->] (1.2,-1.5) -- (4,-1.5) node[below left]{$r$};
    \draw[dashed] (0,-2.2) -- (0,-1.7);
    \draw[dashed] (0,-1.3) -- (0,-0.8);
    
    \draw[->, thick] (-0.2,-1.5) -- node[above]{$F'$} +(-0.5,0);
    \draw[->, thick] (1.2,-1.5) -- node[above]{$F$} +(0.5,0);
    
    \draw[dotted] (1,0.5) -- (1,-1.3);
    \draw[dashdotted] (1,0.5)+(-1,0) node[left]{$E$} -- +(3,0);
    \fill (1,0.5) circle (1.5pt);
    
\end{tikzpicture}
\caption{Gráfico do potencial eletrostático $V(r)$ entre as partículas, juntamente com a energia mecânica $E$. Note a existência de um ponto de retorno. \label{Fig:PotencialAproximacaoCargas}}
\end{marginfigure}

Como a partícula está muito distante, podemos considerar que no momento de seu lançamento sua energia potencial é nula. Assim, temos que
\begin{align}
    \frac{kq_1q_2}{r} &= E \\
    &= K_i + V_i \\
    &= K_i \\
    &= \frac{mv^2}{2},
\end{align}
%
de onde obtemos
\begin{equation}
    r = \frac{2kq^2}{mv^2}.
\end{equation}


%%%%%%%%%%%%%%%%%%%%%%
%\paragraph{Exemplo: ?}
%%%%%%%%%%%%%%%%%%%%%%

%\textbf{Exemplos de determinação de pontos de retorno.}

%%%%%%%%%%%%%%%%%%%%%%%%%%%
%\subsection{Montanha russa}
%%%%%%%%%%%%%%%%%%%%%%%%%%%

%Questão do potencial no caso da montanha russa da montanha russa (problema bidimensional, deslocamento em $x$, potencial dependente de $y$, sendo que $dy = g(x) dx$ e $g(x)$ é a derivada da função $y(x)$, isto é, a derivada do perfil vertical da montanha russa).

%%%%%%%%%%%%%%%%%%%%%%%%%%%%%%%%%%%%%%%%%%%%%%
\section{Trabalho de forças não conservativas}
\label{Sec:TrabalhoForcasNaoConservativas}
%%%%%%%%%%%%%%%%%%%%%%%%%%%%%%%%%%%%%%%%%%%%%%

Notamos na Seção~\ref{Sec:EnergiaMecanica} que se o trabalho de forças não-conservativas é nulo, então a energia mecânica do sistema é constante. Nos preocuparemos agora com os efeitos da existência de forças não-conservativas que realizem trabalho em um sistema.

Vamos considerar um sistema de $N$ partículas que interagem através de $n_f$ forças conservativas e também um conjunto de forças não-conservativas. A partir do teorema trabalho-energia, temos que
\begin{align}
    \sum_{n=1}^N \Delta K &= W_{\textrm{Total}} \\
    \sum_{n=1}^N K_n^f - \sum_{n=1}^N K_n^f &= W_{\textrm{NC}} - \sum_{n = 1}^N \sum_{j = 1}^{n_f} \Delta U_{j,n}
\end{align}
%
onde $W_{\textrm{NC}}$ representa o trabalho efetuado pelo conjunto de forças não-conservativas e também usamos o fato de que para uma força conservativa
\begin{align}
    W &= - \Delta U \\
    &= -(U_f - U_i).
\end{align}
%
Com isso podemos escrever\footnote{Podemos juntar os vários somatórios em $n$, pois $\sum_i a_i + \sum_i b_i = \sum_i(a_i + b_i)$.}
\begin{equation}
    \sum_{n=1}^N[ K_n^f + \sum_{j = 1}^{n_f} U_{j,n}] - \sum_{n=1}^N[K_n^i + \sum_{j = 1}^{n_f} U_{j,n}^i] = W_{\textrm{NC}}.
\end{equation}
%
Os termos entre colchetes são a energia mecânica final e inicial de cada partícula, logo:
\begin{align}
    \sum_{n=1}^N E_n^f - \sum_{n=1}^N E_n^i &= W_{\textrm{NC}} \\
    E_{S}^f - E_S^i &= W_{\textrm{NC}}
\end{align}
%
e, finalmente,
\begin{equation}\label{Eq:TrabalhoNCIgualVarEnergiaMec}
    \Delta E_S = W_{\textrm{NC}}. \mathnote{Variação da energia mecânica de um sistema devido a forças não-conservativas}
\end{equation}
%
Portanto, através da expressão acima, verificamos que no caso de uma ou mais forças não-conservativas realizarem trabalho sobre um sistema, \emph{a variação da energia mecânica do sistema é igual ao trabalho realizado por tais forças.} Note que não impusemos nenhum sinal para o trabalho, portanto ele pode aumentar ou diminuir a energia do sistema. Finalmente, esse resultado também é válido para um sistema que possui só um corpo. Nesse caso a energia mecânica total do sistema é a própria energia mecânica de tal corpo. 

Devemos notar que a Equação~\eqref{Eq:TrabalhoNCIgualVarEnergiaMec} relaciona o trabalho \emph{total} devido a forças não conservativas com a variação da energia mecânica de um sistema. Em muitos sistemas, existem forças não conservativas que realizam trabalho, porém de forma que o trabalho total devido a esse tipo de força seja nulo. Um exemplo disso são pares ação-reação de forças como a tensão, a normal e a força de atrito: devido à própria Terceira Lei de Newton, sabemos que ambas as forças do par ação-reação têm a mesma intensidade, mesma direção, e têm sentidos opostos. Logo, se o deslocamento dos corpos que interagem é o mesmo, temos que
\begin{equation}\label{Eq:RelTrabParAcaoReacao}
    W_{F} = -W_{F'}.
\end{equation}
%
Na Figura~\ref{Fig:TrabalhoNCAcaoReacao}, por exemplo, temos um sistema composto por dois blocos, um ligado a uma mola comprimida e outro que toca o primeiro, do lado oposto àquele ligado à mola. Se liberarmos o sistema para se mover, verificamos que as forças normais entre os blocos e a superfície de apoio realizam trabalhos nulos, por serem perpendiculares ao deslocamento. Verificamos que esse também é o caso para as forças peso de cada bloco. No entanto, na interface entre os blocos deve haver um par ação-reação devido à interação entre eles, sendo que o trabalho de cada uma das forças é diferente de zero. Todavia, de acordo com a equação acima, o trabalho total devido a essas duas forças é nulo. Logo, a energia mecânica do sistema não sofre alteração durante a descompressão da mola.
\begin{marginfigure}[-5cm]
\centering
\begin{tikzpicture}[>=Stealth, scale = 0.88,
     interface/.style={
        % superfície
        postaction={draw,decorate,decoration={border,angle=-45,
                    amplitude=0.2cm,segment length=2mm}}},
    ]
    
    \draw[interface] (0,-1.5) -- (0,-3);
    \draw[interface] (0,-3) -- (4.5, -3);
    
    \draw[dashed] (-0.5, -1) rectangle (5, -3.5);
    
    \draw (0,-2.5) -- (0.2,-2.5);
    \draw[decoration={aspect=0.3, segment length=0.8mm, amplitude=2mm,coil},decorate] (0.2,-2.5) -- (1.5,-2.5);
    \draw (1.5, -2.5) -- (1.7,-2.5);
    
    \draw[pattern = north west lines, pattern color = gray] (1.7,-3) rectangle (2.7,-2);
    \draw[pattern = north east lines, pattern color = gray] (2.7,-3) rectangle (3.7,-2);
    
%    \draw[fill] (2.2,-2.5) circle (1pt);
%    \draw[->, thick] (1.7, -2.5) -- node[below right]{$\vec{F}_e$} +(-1,0);
 
\end{tikzpicture}
\caption{No sistema acima a energia mecânica se conserva, pois o trabalho total devido às várias forças não conservativas é nulo.\label{Fig:TrabalhoNCAcaoReacao}}
\end{marginfigure}

\begin{marginfigure}[1cm]
\centering
\begin{tikzpicture}[>=Stealth, scale = 0.88,
     interface/.style={
        % superfície
        postaction={draw,decorate,decoration={border,angle=-45,
                    amplitude=0.2cm,segment length=2mm}}},
    ]
    
    
    \draw[interface] (0,-1.5) -- (0,-3);
    \draw[interface] (0,-3) -- (4.5, -3);
    
    \draw[dashed] (-0.5, -1) rectangle (5, -3.5);
    
    \draw (0,-2.5) -- (0.2,-2.5);
    \draw[decoration={aspect=0.3, segment length=0.8mm, amplitude=2mm,coil},decorate] (0.2,-2.5) -- (1.5,-2.5);
    \draw (1.5, -2.5) -- (1.7,-2.5);
    
    \draw[pattern = north west lines, pattern color = gray] (1.7,-3) rectangle (3.7,-2);
    \draw[pattern = north east lines, pattern color = gray] (3,-2) rectangle (3.5,-1.5);
    
\end{tikzpicture}
\caption{Se o bloco superior deslisa quando liberarmos a mola para se expandir, verificamos que o deslocamento de tal bloco é menor do que o do bloco inferior. Como as forças de atrito a que ambos estão sujeitos formam um par ação-reação, concluímos que ocorre uma variação da energia mecânica do sistema. \label{Fig:TrabalhoNCAcaoReacaoNaoNulo}}
\end{marginfigure}

Note que em algumas situações é possível que o deslocamento dos corpos não seja o mesmo: dois corpos que interagem através de uma força de atrito cinético, por exemplo, podem ter deslocamentos distintos. Isso resulta em uma variação de energia mecânica, uma vez que os trabalho das forças do par ação-reação não respeitam a relação dada pela Equação~\eqref{Eq:RelTrabParAcaoReacao} acima. Na Figura~\ref{Fig:TrabalhoNCAcaoReacaoNaoNulo} temos um sistema composto por um bloco que está apoiado sobre um bloco maior. O segundo bloco está ligado a uma mola comprimida. Se ao liberarmos o movimento do sistema o bloco superior deslisar, verificamos que seu deslocamento será menor do que o do bloco inferior, porém ambos estarão sujeitos a forças de atrito cinético que compõe um mesmo par ação-reação. Nesse caso, concluímos que o trabalho total devido ao par ação-reação relativo ao atrito entre os blocos é não nulo, ou seja, ocorre uma alteração na energia mecânica do sistema.

%%%%%%%%%%%%%%%%%%%%%%%%%%%%%%%%%%%%%%%%%%%%%%%%%%%%%
\paragraph{Discussão: Oscilador harmônico com atrito}
%%%%%%%%%%%%%%%%%%%%%%%%%%%%%%%%%%%%%%%%%%%%%%%%%%%%%

Se voltarmos ao problema do oscilador massa-mola, mas desta vez considerarmos a possibilidade da existência de uma força de atrito, teremos que o sistema oscilará durante um tempo, porém eventualmente parará. Verificamos que a energia mecânica sofrerá uma variação dada por
\begin{align}
    \Delta E_S &= W_{F_{\textrm{NC}}^{\textrm{int}}} \\
    &= W_{\textrm{at}} \\
    &= \vec{f}_{\textrm{at}} \cdot \vec{d}.
\end{align}
%
No sistema em questão, a força de atrito é cinético; além disso, a normal é igual ao peso. Logo, temos que
\begin{equation}
    W_{\textrm{at}} = \mu_c m g d \cos \theta,
\end{equation}
%
onde $\theta$ é o ângulo formado entre a força e o deslocamento. Esse ângulo será sempre de \degree{180}, e --~portanto~--,
\begin{equation}
    W_{\textrm{at}} = - \mu_c m g d.
\end{equation}

\begin{marginfigure}
\centering
\begin{tikzpicture}[>=Stealth,
     interface/.style={
        % superfície
        postaction={draw,decorate,decoration={border,angle=-45,
                    amplitude=0.2cm,segment length=2mm}}},
    ]
    
    % Desenhar função:
    \draw[smooth, thick, name path=plot,samples=1000,domain=0.5:4.5]
    plot(\x,{0.5*(\x - 2.5)^2}) node[right]{$U_e$};
    
    %%
    
    \draw[interface] (0,-1.5) -- (0,-3);
    \draw[interface] (0,-3) -- (4.8, -3);
    
    \draw (0,-2.5) -- (0.2,-2.5);
    \draw[decoration={aspect=0.3, segment length=2.5625mm, amplitude=2mm,coil},decorate] (0.2,-2.5) -- (3.3,-2.5);
    \draw (3.3, -2.5) -- (3.5,-2.5);
    
    \draw[pattern = north west lines, pattern color = gray] (3.5,-3) rectangle (4.5,-2);
    \draw[dotted, pattern = north west lines, pattern color = gray] (2,-3) rectangle (3,-2);
    
    \draw[fill] (4,-2.5) circle (1pt);
    \draw[->, thick] (4,-2.5) -- +(0,-1) node[right]{$\vec{P}$};
    \draw[->, thick] (4,-2) -- node [right]{$\vec{N}$} +(0,1);
    \draw[->, thick] (3.5, -2.5) -- node[below right]{$\vec{F}_e$} +(-1,0);
    
    \draw[->] (0,0) -- (4.8,0) node[below left]{$x$};
    \draw[<->] (2.5,-0.5) -- node[below]{$d$} (4,-0.5);
    \draw[->] (2.5,-0.2) -- (2.5,2.5) node[below left]{$U$};
    \draw[fill] (4,0) node[below left]{$A$} circle (1pt);
    
    \draw[dotted] (2.5,-0.2) -- (2.5,-2);
    \draw[dashed] (4,-1) -- (4,1.125);
    
    \draw[dashdotted] (0.5, 1.125) node[left]{$E_i$} -- (4.5, 1.125);
    \draw[fill] (4, 1.125) circle (1.2pt);
    
    \draw[dashed] (1.1771,0.875) coordinate (r1)-- (4,1.125);
    \draw[dashed] (1.1771,0.875) -- (3.618,0.625) coordinate (r2);
    \draw[dashed] (1.634,0.375) coordinate (r3) -- (3.618,0.625);
    
    \draw[fill] (r3) node[left]{$E_f$} circle (1.3pt);
    \draw[dotted] (r1) -- +(0,-0.875);
    \draw[dotted] (r2) -- +(0,-0.625);
    \draw[dotted] (r3) -- +(0,-0.375);
        
\end{tikzpicture}
\caption{Devido ao atrito, a energia mecânica do sistema sofre uma diminuição progressiva, fazendo com que os pontos de retorno fiquem mais próximos à posição de equilíbrio em cada oscilação. Eventualmente o sistema parará devido ao atrito estático, restando uma energia residual na forma de energia potencial elástica.\label{Fig:PotElasticoComMassaMolaComAtrito}}
\end{marginfigure}

Na Figura~\ref{Fig:PotElasticoComMassaMolaComAtrito} temos um gráfico da energia em função do deslocamento. Note que a cada oscilação, devido à diminuição da energia mecânica, a amplitude da oscilação diminui. Note também que nos pontos de retorno a velocidade é nula, e por isso a condição passa a ser de atrito estático. Nas primeiras oscilações, a força exercida pela mola é suficiente para fazer com que o corpo volte a oscilar, porém, eventualmente, a amplitude será tal que a força exercida pela mola é \emph{menor que a força de atrito estático máxima}. Nesse momento, o sistema para de oscilar, armazenando uma energia residual na forma de energia potencial elástica na mola.

%%%%%%%%%%%%%%%%%%%%%%%%%%%%%%%%%%%%%%%%%%%%%%%
\paragraph{Exemplo: Bloco lançado por uma mola}
%%%%%%%%%%%%%%%%%%%%%%%%%%%%%%%%%%%%%%%%%%%%%%%

\begin{marginfigure}[2cm]
\centering
\begin{tikzpicture}[>=Stealth, scale = 0.4,
     interface/.style={
        % superfície
        postaction={draw,decorate,decoration={border,angle=-45,
                    amplitude=0.2cm,segment length=2mm}}},
    ]
    
    %%
    
    \draw[interface] (0,-0.5) -- (0,-2.5);
    \draw[interface] (0,-2.5) -- (7.05, -2.5);
    \draw[interface] (7.05,-2.5) arc[start angle = -90, end angle = 123, radius = 3 cm];
    
    \draw[->] (0,-3.2) -- (9,-3.2) node[below left]{$x$};
    \draw[|-|] (2.2, -3.2) -- (4.2,-3.2);
    \path (2.2,-3.3) node[below]{$\ell$} -- (4.2,-3.3) node[below]{0};
    \draw[-|] (4.2,-3.2) -- (7.05, -3.2);
    \path (4.2,-3.3) -- (7.05, -3.3) node[below]{$d$};
    
    \draw (0,-2) -- (0.2,-2);
    \draw[decoration={aspect=0.4, segment length=1.5625mm, amplitude=1.6mm,coil},decorate] (0.2,-2) -- (3.3,-2);
    \draw (3.3, -2) -- (3.5,-2);
    
    \draw[pattern = north west lines] (3.7,-2.5) rectangle (4.7,-1.5);
    \draw[pattern = north east lines] (3.5,-2.5) rectangle (3.7,-1.5);

\end{tikzpicture}
\caption{Lançamento de um bloco em uma pista circular vertical. \label{Fig:BlocoArremessadoEmLoopComAtrito}}
\end{marginfigure}

\begin{quote}
    A Figura~\ref{Fig:BlocoArremessadoEmLoopComAtrito} mostra um bloco que será lançado, empregando uma mola, em uma pista composta de uma seção plana e de uma seção circular vertical. Na seção plana da pista, o coeficiente de atrito cinético é $\mu_c = \np{0.75}$, enquanto na parte circular o atrito é nulo. Se a constante elástica da mola é $k = \np[N/m]{1200}$, a massa do bloco é $m = \np[kg]{0.750}$, o tamanho da seção reta é $d = \np[cm]{60}$, e o raio da pista é $\np[cm]{130}$, qual deve ser a compressão $\ell$ da mola para que o bloco chegue ao ponto mais alto da pista sem que perca contato com ela, assumindo que ele partiu do repouso?\footnote[][1cm]{Esse problema foi visto na Seção~\ref{Sec:EnergiaMecanica}, porém sem atrito.}
\end{quote}

Sabemos que
\begin{equation}
    \Delta E^{\textrm{mec}} = W_{\textrm{NC}}.
\end{equation}
%
No sistema em questão, temos a presença de uma força não-conservativa que realiza trabalho ---~a força de atrito~---. Temos também uma força normal, porém ela não realiza trabalho por estar sempre perpendicular ao deslocamento instantâneo. Vamos desprezar os efeitos da força de arrasto do ar. Além dessas forças não-conservativas, existem duas forças conservativas: a força elástica e a força peso. Considerando tais forças, temos então que
\begin{equation}
    (K_f + U_e^f + U_g^f) - (K_i + U_e^i + U_g^i) = W_{\textrm{at}}.
\end{equation}

Como o bloco parte do repouso, sua energia cinética inicial é zero. Além disso, podemos escolher a posição inicial do bloco como a origem do eixo vertical, assim a energia potencial gravitacional inicial é nula. Por fim, sabemos que quando o bloco deixa a mola, a energia potencial elástica é zero. Assim, a expressão acima simplifica-se a:
\begin{equation}\label{Eq:EnergiasParaOBlocoEmLoopComAtrito}
    (K_f + U_g^f) - (U_e^i) = f_{\textrm{at}} d \cos(\degree{180}).
\end{equation}
%
Note que a energia cinética final não é nula, já que o bloco não pode ter velocidade nula ao chegar ao topo da seção circular da pista. Sua velocidade será dada por\footnote{Esse resultado foi obtido na subseção \emph{Condição de perda de contato e velocidade mínima}, na Seção~\ref{Sec:ForcasNoMovCircular}.}
\begin{equation}
    v = \sqrt{Rg}.
\end{equation}
%
Se as dimensões do bloco são desprezíveis\footnote{Lembre-se que estamos tratando do movimento de partículas, trataremos corpos extensos somente a partir do próximo capítulo.}, sua posição vertical final será igual ao diâmetro da pista circular ---~ou seja, $y_f = 2R$ ~---. Finalmente, sabemos que o atrito é dado por $f_{\textrm{at}} = \mu_c N$, sendo que ---~devido ao fato de que não há aceleração vertical na seção reta da pista~--- a normal é igual ao peso.

Substituindo os resultados acima na Equação~\ref{Eq:EnergiasParaOBlocoEmLoopComAtrito} obtemos
\begin{equation}
    \left(\frac{mv_f^2}{2} + mgy_f\right) - \left(\frac{kx_i^2}{2}\right) = - \mu_c mg d.
\end{equation}
%
de onde podemos escrever
\begin{align}
    \frac{k\ell^2}{2} &= \frac{m(\sqrt{Rg})^2}{2} + mg2R + \mu_c mg d \\
    &= \frac{5mgR}{2} + \mu_c mg d,
\end{align}
%
o que resulta em
\begin{align}
    k\ell^2 &= 5mgR + 2 \mu_c mg d \\
    \ell &= \sqrt{\frac{mg(5R + 2 \mu_c d)}{k}}
\end{align}
%
Substituindo os valores das constantes, temos finalmente
\begin{align}
    \ell &= \sqrt{\frac{(\np[kg]{0.750}) \cdot (\np[m/s^2]{9.8}) \cdot [5 \cdot (\np[m]{1.30}) + 2 \cdot (\np{0.75}) \cdot (\np[m]{0,60})]}{(\np[N/m]{1200})}} \\
    &\approx \np[m]{0.21}.
\end{align}

%%%%%%%%%%%%%%%%%%%%%%%%%%%%%%%%%%%%%%%%%%%%%
\section{Princípio da conservação da energia}
\label{Sec:PrincipioDaConsDaEnergia}
%%%%%%%%%%%%%%%%%%%%%%%%%%%%%%%%%%%%%%%%%%%%%

%\textbf{Enfatizar que as forças normal, tensão, etc. que não podem ser escritas como derivada de um potencial sempre entram como não-conservativas, estejam dentro do sistema ou não, façam trabalho ou não. O que vai mudar é pra onde elas levam a energia: ou para energia interna, ou para fora do sistema.}

%\textbf{Discutir um massa-mola com arrasto e usar como exemplo em que a energia é movida para fora do sistema.}

Na seção anterior verificamos que o trabalho de uma força não conservativa implica em uma alteração da energia mecânica do sistema. Vamos considerar agora o que acontece com a energia em duas situações diferentes. 

\begin{marginfigure}
\centering
\begin{tikzpicture}[>=Stealth,
     interface/.style={
        % superfície
        postaction={draw,decorate,decoration={border,angle=-45,
                    amplitude=0.2cm,segment length=2mm}}},
    ]
    
    \draw[interface] (-2,0) -- (2,0);
    
    \draw[pattern = north west lines] (-0.1, 1.4) -- (-0.1,1.2) -- (-0.5, 1.2) -- (-0.5, 0.2) -- (0.5, 0.2) -- (0.5, 1.2) -- (0.1, 1.2) -- (0.1,1.4) -- (0.7,1.4) -- (0.7,0) -- (-0.7,0) -- (-0.7, 1.4) -- cycle;
    
    \draw[pattern = north east lines] (-0.4, 0.55) -- (-0.4, 0.45) -- (0.4, 0.45) -- (0.4, 0.55) -- (0.05, 0.55) -- (0.05, 2) -- (-0.05, 2) -- (-0.05, 0.55) -- cycle;
    
    \fill[pattern = dots] (-0.5, 0.2) rectangle (0.5, 1.2);
    
    \draw[pattern = north west lines] (-0.5, 2) rectangle (0.5, 3);
       
    \draw[decoration={aspect=0.5, segment length=1mm, amplitude=1mm,coil},decorate] (0,3) -- (0,4);
    
    \draw[interface] (2,4) -- (-2,4);
    
    \draw[dashed] (-2.2, 4.5) rectangle (2.2, 1.7); 

    
\end{tikzpicture}
\caption{Bloco suspenso por uma mola, livre para oscilar, porém conectado a um sistema de amortecimento.\label{Fig:OsciladorAmortecido}}
\end{marginfigure}

Primeiramente, vamos considerar um bloco suspenso por uma mola e livre para oscilar. Na parte inferior do bloco, existe uma haste que está fixada a ele e cuja extremidade oposta está ligada a uma placa plana, submersa em água ---~veja a Figura~\ref{Fig:OsciladorAmortecido}~---. Sabemos que o deslocamento da placa dentro do fluido está sujeito a uma força de arrasto, sempre na direção contrária ao seu deslocamento. Isso implica em uma força que atua sobre o bloco através da haste, realizando trabalho negativo, fazendo com que sua energia mecânica seja diminuida a cada oscilação:
\begin{align}
    \Delta E_S &= W_{\textrm{NC}} \\
    &= W_{F_{\textrm{haste}}}.
\end{align}
%
Eventualmente, o movimento oscilatório cessará devido ao fato de que a energia mecânica se tornará nula. Se considerarmos o sistema formado somente pelo bloco e pela mola, podemos dizer que há uma \emph{força externa} ---~a força exercida pela haste~--- que diminui a energia mecânica do sistema. Por outro lado, verificamos também que tal força exerce um trabalho \emph{positivo} sobre o fluido, forçando-o a se movimentar. Dessa forma, podemos dizer que o trabalho de uma força externa \emph{transfere} a energia através da fronteira do sistema (representada na figura pelo retângulo tracejado).\footnote{Sabemos que pelo menos uma parte da energia é transferida para o fluido, manifestando-se como energia cinética.}

Vamos supor agora que temos um sistema onde atuam forças conservativas e uma força não-conservativa \emph{interna}, ou seja, uma força cujo par ação-reação esteja completamente contido dentro da fronteira do sistema. Um exemplo desse caso pode ser uma força de atrito entre um bloco e uma superfície em um sistema como o mostrado na Figura~\ref{Fig:MassaMolaSistemaFechado}. Vamos assumir que nenhuma força externa \emph{que realize trabalho} atue sobre tal sistema, atravessando sua fronteira ---~vamos desconsiderar efeitos da força de arrasto; a força normal que atua sobre o sistema é externa, porém seu trabalho é nulo~---. Nessa situação, verificamos que se aplica a expressão
\begin{align}
    \Delta E_S &= W_{\textrm{NC}} \\
    &= W_{\textrm{at}}.
\end{align}

\begin{marginfigure}
\centering
\begin{tikzpicture}[>=Stealth, scale = 0.75,
     interface/.style={
        % superfície
        postaction={draw,decorate,decoration={border,angle=-45,
                    amplitude=0.2cm,segment length=2mm}}},
    ]
      
    \draw[pattern = north east lines] (0,-1.5) -- (0,-3) -- (4.8, -3) -- (4.8, -3.4) -- (-0.5,-3.4) -- (-0.5,-1.5) -- cycle;
    
    \draw (0,-2.5) -- (0.2,-2.5);
    \draw[decoration={aspect=0.3, segment length=2.5625mm, amplitude=2mm,coil},decorate] (0.2,-2.5) -- (3.3,-2.5);
    \draw (3.3, -2.5) -- (3.5,-2.5);
    
    \draw[interface] (-0.7, -3.4) -- (5, -3.4);
    
    \draw[pattern = north west lines] (3.5,-3) rectangle (4.5,-2);
    
    \draw[dashed] (-1, -1) rectangle (5.3, -3.9);
        
\end{tikzpicture}
\caption{Sistema formado por um oscilador massa-mola sujeito a uma força de atrito. Se nenhuma força que atravessa a fronteira do sistema realiza trabalho, então o sistema é \emph{fechado}.\label{Fig:MassaMolaSistemaFechado}}
\end{marginfigure}

Segundo a equação acima, a energia mecânica deve diminuir, porém estamos considerando um \emph{sistema fechado}, ou seja, não pode haver fluxo de energia através da fronteira. Nesse momento, aparentemente deveríamos argumentar que a energia se perdeu, afinal, estamos tratando de uma situação em que atua uma força não conservativa. Essa ideia, porém, não descreve o que acontece quando uma força não conservativa atua de maneira a realizar trabalho através da fronteira, como no caso da força exercida pela haste no exemplo anterior. Além disso, é comum em situações como a da Figura~\ref{Fig:MassaMolaSistemaFechado} que se note um \emph{aumento da temperatura do sistema}, isto é, um aumento da temperatura do bloco e da superfície onde ele se apoia.\footnote{\textrm{Falar sobre a observação do Rumford ao perfurar canhões: enquanto houvesse trabalho sendo efetuado, havia aumento de temperatura (na real colocavam água para que isso não acontecesse). Diz que ele até usou uma broca intencionalmente cega, para poder ver essa relação melhor. Falar sobre a exp. do Joule para determinar o equivalente mecânico do calor.}}

Concluímos, portanto, que a energia mecânica deve ter sido transformada em algum outro tipo de energia, que denominamos como \emph{energia interna}, e que está relacionada à temperatura do corpo\footnote{Microscopicamente, a energia interna associada à temperatura de um corpo está relacionada à velocidade média das partículas que o compõe.}. Devido à força de arrasto presente no sistema, verifica-se que ocorre um aumento da temperatura da água ---~ou seja, a energia mecânica é transformada em energia interna~---. Logo, podemos tornar a Expressão~\eqref{Eq:TrabalhoNCIgualVarEnergiaMec} para a variação da energia mecânica de um sistema em algo mais completo se escrevermos
\begin{equation}\label{Eq:ConservacaoDaEnergia1}
    \Delta E_S^{\textrm{mec}} + \Delta E_S^{\textrm{int}} = W_{F_{\textrm{ext}}}.
\end{equation}
%
Note que a expressão acima assume o caráter de uma \emph{lei de conservação} para a energia do sistema, uma vez que só é possível alterá-la através do trabalho de uma força externa ---~ isto é, de uma força que cruza a fronteira do sistema e que realize trabalho~---.

Apesar de a expressão acima ser mais geral que a Expressão~\eqref{Eq:TrabalhoNCIgualVarEnergiaMec}, ainda podemos imaginar uma série de possibilidades que não são bem descritas por ela:
\begin{description}
    \item[Energia química:] Se, por exemplo, utilizarmos um motor elétrico ligado a uma bateria, sabemos que é possível o utilizar para alterar a energia mecânica do sistema, mesmo sem a ação de uma força externa. De fato, existe uma energia química armazenada na bateria que nos permite utilizar o trabalho da força não-conservativa exercida pelo motor para fazer variar a energia mecânica do sistema, ou a energia interna do sistema, ou mesmo ambas. Outra possibilidade é a de utilizar um combustível e substituir o motor elétrico por uma máquina térmica: assim como no caso da bateria, os combustíveis são uma maneira de armazenar energia química. Como esse tipo de energia tem um caráter similar à energia interna associada à temperatura, não há necessidade de alterar a expressão~\eqref{Eq:ConservacaoDaEnergia1}.\footnote{Microscopicamente, a energia química é uma energia potencial elétrica armazenada nas configurações moleculares dos átomos.}
    
    \item[Condução térmica:] Se a energia interna está associada à temperatura, então existe uma maneira de transferir energia para fora de um sistema que não envolve a realização de trabalho por uma força externa: se tormarmos dois sistemas distintos ---~ como, por exemplo, dois blocos metálicos~--- a temperaturas diferentes, é fácil perceber que ao deixarmos que ambos fiquem em contato direto, a temperatura do bloco frio aumenta, enquanto a temperatura do bloco quente diminui. A tal fluxo de energia interna denominamos \emph{calor}\footnote{Microscopicamente, como a temperatura dos blocos é diferente, a energia cinética média das partículas que compõe os blocos também é diferente. No contato direto entre os blocos, as partículas do corpo quente interagem com as do corpo frio, forçando-as a oscilar. Ocorre uma transferência de energia cinética entre as partículas de cada um dos corpos através do trabalho das forças entre as cargas elétricas dos átomos.}. Devemos então adicionar um termo $Q$ à direita da Expressão~\eqref{Eq:ConservacaoDaEnergia1}, relativo ao calor transmitido.

    \item[Radiação eletromagnética:] Sabemos que o Sol é capaz de causar variações de temperatura na Terra, mesmo na ausência de um \emph{meio de contato} entre ambos. Devido à ausência de tal meio, temos que não é possível que haja transmissão de energia interna através de um fluxo de calor. Também não existe uma força que atravesse ambas as fronteiras, realizando trabalho\footnote{Na verdade, a força gravitacional entre o Sol e a Terra pode realizar trabalho, essa é a origem das marés.}. Resta então a transmissão de energia através de \emph{radiação eletromagnética}. A radiação eletromagnética, é uma perturbação de caráter ondulatório nos campos elétrico e magnético que preenchem o espaço, e também é conhecida como \emph{ondas eletromagnéticas}. Como tais campos podem existir mesmo no vácuo, as ondas eletromagnéticas podem se propagar na ausência de matéria. Representamos a radiação eletromagnética através de um termo $R$ à direita da Expressão~\eqref{Eq:ConservacaoDaEnergia1}.
\end{description}
 
Considerando as observações acima, podemos \emph{postular} um \emph{princípio de conservação da energia}:
\begin{equation}\label{Eq:ConservacaoDaEnergia2}
    \Delta E_S^{\textrm{mec}} + \Delta E_S^{\textrm{int}} = W_{F_{\textrm{ext}}} + Q + R. \mathnote{Princípio da conservação da energia}
\end{equation}
%
Tal expressão é um postulado pois não é possível prová-la: verificamos nos exemplos discutidos acima que existem indícios de que a energia é transmitida entre sistemas diferentes, através de diferentes meios, porém não é possível verificar/provar que nenhuma parte da energia é perdida. No entanto, todas as evidências experimentais levam a crer que o princípio acima é válido.\footnote{Ao se realizar um experimento, é muito comum que parte da energia seja \emph{dissipada} em outras formas. Se no entanto, mensurarmos a quantidade de energia dissipada, ou alterarmos o aparato experimental de maneira a minimizar a perda energética, verificamos que a Equação~\eqref{Eq:ConservacaoDaEnergia2} é sempre satisfeita. Isso, no entanto, é extremamente difícil na prática.}

Através do princípio da conservação da energia e da flexibilidade que temos ao definir um sistema, podemos nos preocupar simplesmente com o \emph{fluxo} da energia. Isso tende a facilitar muito a interpretação de diversos problemas pois podemos nos preocupar com a origem e o destino da energia, sem se ater a cada detalhe dos processos físicos que ocorrem durante a transferência da energia entre uma forma e outra.

%%%%%%%%%%%%%%%%%%%%%%%%%%
\section{Seções opcionais}
%%%%%%%%%%%%%%%%%%%%%%%%%%

%%%%%%%%%%%%%%%%%%%%%%%%%%%%%%%%
\section{Referenciais inerciais}
%%%%%%%%%%%%%%%%%%%%%%%%%%%%%%%%

Discutir o fato de que a energia depende do referencial inercial adotado, porém que sua variação não. De alguma maneira, mostrar que é a variação que importa.

%%%%%%%%%%%%%%%%%%%%%%%%%%%%
%\subsection{Pseudo trabalho}
%%%%%%%%%%%%%%%%%%%%%%%%%%%%

%\textbf{pseudo trabalho (esqueci completamente o que é isso) [acho que é aquela história o patinador que empurra a parede do Fiolhais. Ver, se for, explicar com uma caneta de mola que pula (modelar com dois blocos ligados por uma mola comprimida, ao liberar a mola, o sistema pula)]}

%%%%%%%%%%%%%%%%%%%%%%%%%%%%%%%%%%%%%%%%%%%%%%%%%%%
%\subsection{Energia cinética e referenciais inerciais}
%%%%%%%%%%%%%%%%%%%%%%%%%%%%%%%%%%%%%%%%%%%%%%%%%%%

%%%%%%%%%%%%%%%%%%%%%%%%%%%%%%%%%%%%%%%%%%%%%%%%%%%%%%%%%%%%%%%%%%%%%%%%%%%
\subsection{Cálculo da velocidade de um pêndulo através das Leis de Newton}
\label{Sec:SolPenduloLeisDeNewton}
%%%%%%%%%%%%%%%%%%%%%%%%%%%%%%%%%%%%%%%%%%%%%%%%%%%%%%%%%%%%%%%%%%%%%%%%%%%

Em um referencial tangente à trajetória, temos que
\begin{description}
    \item[Eixo $x$:] No eixo tangente à trajetória temos
        \begin{align}
            F_R^x &= m a_x \\
            P_x &= m a_x \\
            mg \sen\theta &= m a_x \\
            a_x &= g \sen\theta
        \end{align}
    \item[Eixo $y$:]
        \begin{align}
            F_R^y &= m a_y \\
            T - P_y &= m \frac{v^2}{r}
        \end{align}
\end{description}

\begin{marginfigure}
\centering
\begin{tikzpicture}[>=Stealth, scale = 1.2]
    \draw[fill] (0,0) circle (1pt);
    \draw[densely dotted] ([shift={(0,0)}]180:2) arc[radius=2, start angle=180, end angle= 290];
    
    \draw[gray] (0,0) coordinate (fix) -- (-140:18mm);
    \draw[gray, dashed] (0,0) -- +(0,-1) coordinate (down);
    
    \draw[pattern = north west lines, draw = gray, pattern color = gray] (-140:2) circle (2mm);
    \draw[fill, gray] (-140:2) coordinate (bob) circle (1pt);
    \draw[->, thick, gray] (-140:2) -- +(0,-0.7) coordinate (peso) node[left]{$\vec{P}$};
    \draw[->, thick, gray] (-140:18mm) -- (-140:14mm) node[above left]{$\vec{T}$};
    
    \draw[dashed,->] (-140:2) +(130:1) -- +(-50:1.25) coordinate (x) node[below]{$x$};
    \draw[dashed,->] (-140:3) -- (-140:1) node[below]{$y$};
    
    \draw[dotted] (-140:2) ++(0,-0.7) -- +(40:0.45);
    \draw[thick, ->] (-140:2) -- +(-50:0.536) node[above right]{$P_x$};
    
    \pic [draw, "$\theta$", angle eccentricity = 1.5] {angle = bob--fix--down};
    \pic [draw, "$\theta$", angle eccentricity = 1.4, angle radius = 4mm] {angle = peso--bob--x};
        
\end{tikzpicture}
\caption{Em um pêndulo, a aceleração está ligada à componente da força peso na direção tangente à trajetória. Para que possamos determinar a velocidade precisamos realizar uma integração.}
\end{marginfigure}

\noindent{}Sabemos que a aceleração centrípeta é responsável por alterar somente a direção da velocidade, por isso não precisamos nos preocupar com o eixo $y$. Podemos reescrever a expressão para a aceleração como
\begin{equation}
    \frac{dv}{dt} = g \sen\theta,
\end{equation}
%
onde $\theta$ varia entre 0 e $\pi/2$. Podemos então escrever
\begin{equation}\label{Eq:DifVVelPendulo}
    dv = g\sen\theta \;dt.
\end{equation}

A distância percorrida pelo pêndulo ao longo da trajetória circular é dada por
\begin{equation}
    s = \theta r.
\end{equation}
%
A velocidade que estamos interessados em calcular é dada pela derivada da posição ao longo do arco em relação ao tempo:
\begin{equation}
    v = \frac{ds}{dt},
\end{equation}
%
de onde podemos escrever
\begin{equation}
    dt = \frac{ds}{v}.
\end{equation}
%
Substituindo essa expressão na Equação~\ref{Eq:DifVVelPendulo}, temos
\begin{align}
    dv &= \frac{g \sen\theta \;ds}{v} \\
    v\;dv &=  g \sen \theta \; ds
\end{align}

Integrando a expressão acima entre os instantes inicial e final, cujas velocidades e posições são $v_i$, $s_i$, $v_f$, e $s_f$, respectivamente, temos
\begin{align}
    \int_{v_i}^{v_f} v \;dv &= \int_{s_i}^{s_f} g \sen \theta \;ds \\
    \frac{v}{2}\Big|_{v_i}^{v_f} &= \int_{s_i}^{s_f} g \sen \theta \;ds.
\end{align}
%
Fazendo uma mudança de variáveis, temos
\begin{equation}
    s = \theta r,
\end{equation}
%
o que resulta na relação
\begin{equation}
    ds = r d\theta.
\end{equation}
%
Os novos limites são dados por
\begin{align}
    s_i &= \theta_i = 0\\
    s_f &= \theta_f = \pi/2.
\end{align}
%
Assim, se assumirmos que
\begin{align}
    v_i &= 0 \\
    v_f &= v,
\end{align}
%
temos,
\begin{align}
    \frac{v^2}{2} &= \int_{0}^{\pi/2} rg \sen\theta \;d\theta \\
    &= -rg [\cos\theta]_{0}^{\pi/2} \\
    &= -rg [(0) - (1)] \\
    &= rg.
\end{align}
%
Finalmente,
\begin{equation}
    v = \sqrt{2rg}.
\end{equation} 

%%%%%%%%%%%%%%%%%%%%%%%%%%%%%%%%%%%%%%%%%%%%%%%%%%%%%%%%%%%%%%%%%%%%%%%%%%%%%%%%%
\subsection{Determinação do teorema trabalho--energia através das leis de Newton}
\label{Sec:DetTeoremaTrabEnergiaCalculoVetorial}
%%%%%%%%%%%%%%%%%%%%%%%%%%%%%%%%%%%%%%%%%%%%%%%%%%%%%%%%%%%%%%%%%%%%%%%%%%%%%%%%%

A solução acima está muito próxima de ser uma dedução do teorema trabalho-energia. Se partirmos da relação
\begin{equation}
    a = \frac{dv}{dt}
\end{equation}
%
e de
\begin{equation}
    v = \frac{ds}{dt}
\end{equation}
%
podemos escrever
\begin{equation}
    dv = \frac{a}{v} ds.
\end{equation}
%
Através da Segunda Lei de Newton, podemos escrever, considerando um eixo tangencial à trajetória
\begin{align}
    dv &= \frac{F_t}{m} \frac{ds}{v} \\
    m \; dv &= \frac{F_t}{v} ds \\
    mv \; dv &= F_t \; ds.
\end{align}
%
Integrando entre os instantes inicial e final, aos quais correspondem as velocidades inicial $v_i$ e final $v_f$ e as posições $s_i$ e $s_f$ ao longo do arco descrito pelo pêndulo, temos
\begin{align}
    \frac{v^2}{2}\Big|_{v_i}^{v_f} &= \int_{s_i}^{s_f} F_t \; ds \\
    \frac{mv_f^2}{2} - \frac{mv_i^2}{2} &= \int_{s_i}^{s_f} F_t \; ds.
\end{align}
%
A integral acima ``soma'' a contribuição da componente da força projetada na direção do deslocamento infinitesimal $d\vec{r}$ ao longo da trajetória circular. Podemos expressar isso de uma maneira geral como
\begin{equation}
    \int_C \vec{F} \cdot d\vec{r}.
\end{equation}
%
A expressão acima é uma \emph{integral de linha da força $\vec{F}$ sobre o caminho $C$}. A equação acima é a expressão mais geral para o trabalho. Consequentemente, rededuzimos o teorema trabalho-energia:
\begin{equation}
    \frac{mv_f^2}{2} - \frac{mv_i^2}{2} = W,
\end{equation}
%
onde
\begin{equation}
    W = \int_C \vec{F} \cdot d\vec{r}.
\end{equation}

%%%%%%%%%%%%%%%%%%%%%%
\section{Exercícios}
%%%%%%%%%%%%%%%%%%%%%%

%%%
%%%

\begin{question}[type={exam}]\label{Q:PtoEquilDeriv}
Uma força conservativa unidimensional $F(x)$ atua sobre uma partícula de \np[kg]{1,0} que pode se mover ao longo do eixo $x$. A energia potencial $U(x)$ associada à força é dada por (veja a Figura~\ref{Fig:PtoEquilDeriv})
\begin{equation}
	U(x) = -4x e^{-x/4}~\rm{J}.
\end{equation}
onde $x$ é dado em metros. Na posição $x = \np[m]{5,0}$, a partícula possui uma energia cinética de \np[J]{2,0}.
\begin{itemize}
	\item Determine a posição do ponto de equilíbrio através da derivada do potencial.
	\item Considerando que a energia mecânica do sistema seja constante, determine seu valor e trace uma reta no gráfico representando tal quantidade. Determine os pontos de retorno de maneira aproximada.
\end{itemize}

\begin{marginfigure}
\centering
\begin{tikzpicture}[>=Stealth, extended line/.style={shorten >=-#1,shorten <=-#1},
 extended line/.default=3mm]] % talvez fosse melhor amplicar com scale=1.5
    % Draw axes: acho que o |- é pra desenhar um "canto", um L
    \draw [->] (0,-3) -- (0,1) node [below left]{$U$};
    \draw[->] (0,0) -- (4,0)  node [above left] {$x$};
    
    \draw[|-|] (0,-1) node[left]{-2} -- (0,-2) node[left]{-4};
    \draw[-|] (0,-2) -- (0,-3) node[left]{-6};
    \draw[|-|] (1, 0) node[above]{5} -- (2,0) node[above]{10};
    \draw[|-] (3, 0) node[above]{15} -- (4, 0);
    
    % Desenhar função:
    \draw[thick, smooth,name path=plota,samples=1000,domain=0:3.8]
    plot(\x,{-(1/2)*4*(\x * 5)*exp(-(\x * 5)/4)});
    
\end{tikzpicture}
\caption{Questão~\ref{Q:PtoEquilDeriv}.\label{Fig:PtoEquilDeriv}}
\end{marginfigure}
\end{question}

%%%
%%%

% Considerando uma força de arrasto específica, qual é a velocidade máxima dada uma certa potência.

%%%
%%%

\begin{question}[type={exam}]
Considerando que a energia cinética de um projétil que emerge de um canhão é sempre a mesma, que a energia potencial gravitacional só depende da posição no eixo vertical, e desprezando a força de arrasto, não deveríamos observar sempre a mesma altura máxima para qualquer ângulo de lançamento do projétil? Explique.
\end{question}
