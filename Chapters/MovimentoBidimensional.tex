%%%%%%%%%%%%%%%%%%%%%%%%%%%%%%%%%%%%%%%%%
\chapter{Movimentos bi e tridimensionais}
\label{Chap:MovimentoBidimensional}
%%%%%%%%%%%%%%%%%%%%%%%%%%%%%%%%%%%%%%%%%
%\minitoc

%\clearpage

\begin{fullwidth}
{\it
Neste capítulo vamos redefinir as variáveis cinemáticas em termos de vetores, utilizando as propriedades descritas no Capítulo~\ref{Chap:Vetores}. Obteremos assim relações vetoriais entre as variáveis cinemáticas que nos darão uma descrição completa do movimento em três dimensões. Para simplificar a interpretação dos movimentos, nos valeremos do fato de que as equações da cinemática podem ser escritas como um conjunto de três equações --~uma para cada eixo~-- sendo que em diversos movimentos um ou dois eixos apresentarão equações triviais.
}
\end{fullwidth}

%%%%%%%%%%%%%%%%%%%%
\section{Introdução}
%%%%%%%%%%%%%%%%%%%%

Para que possamos descrever o movimento da maneira mais geral possível, devemos considerar um espaço tridimensional: ao --~por exemplo~-- descrever o movimento de um veículo em uma estrada, se tomarmos um eixo de referência ao longo de um segmento da pista, temos ainda possíveis movimentos laterais devidos a curvas e verticais devidos a subidas ou descidas.

Apesar de a descrição completa do movimento exigir três dimensões, é comum que possamos tratar o movimento em duas, ou mesmo uma dimensão. Isso se deve ao fato de que ao dividirmos as equações nos eixos do sistema de referência, poderemos ignorar uma ou duas dessas equações simplesmente por não haver movimento no eixo a que elas correspondem. Para o caso do movimento em linha reta, por exemplo, ao alinharmos um dos eixos do sistema de referência tridimensional ao longo do movimento, temos uma situação em que não há movimento nos outros dois eixos. Isso corresponde ao que denominamos como \emph{movimento unidimensional} no Capítulo~\ref{Chap:MovimentoUnidimensional}, ou seja, o movimento unidimensional é só um caso especial do movimento tridimensional. No caso do movimento bidimensional temos algo semelhante, porém só conseguimos eliminar um dois eixos.

Verificaremos adiante como fazer uma descrição do movimento em três dimensões e como simplificar o tratamento ao eliminar um ou dois eixos. Verificaremos também que podemos separar o movimento em cada eixo e tratá-los de maneira independente mesmo quando há movimento em mais que um deles. Isso facilitará a interpretação do movimento. 

%%%%%%%%%%%%%%%%%%%%%%%%%%%%%%%%%%%%%%%%
\section{Vetores posição e deslocamento}
%%%%%%%%%%%%%%%%%%%%%%%%%%%%%%%%%%%%%%%%

Quando tratamos do movimento unidimensional, utilizamos a distância até a origem (isto é, um ponto de referência) para descrever o movimento através da posição, velocidade e aceleração. Verificamos também que essas grandezas variavam no tempo e pudemos as definir como funções do tempo.

%%%%%%%%%%%%%%%%%%%%
\subsection{Posição}
%%%%%%%%%%%%%%%%%%%%

\begin{marginfigure}
\centering
\begin{tikzpicture}[>=Stealth, scale=1.3]

    \draw[->] (0,0,0) -- (2,0,0) node[below left]{$x$};
    \draw[->] (0,0,0) -- (0,2,0) node[below left]{$y$};
    \draw[->] (0,0,0) -- (0,0,2) node[below]{$z$};

    \draw[->, thick] (0,0,0) -- node[above]{$\vec{r}$}(1,0.75,0.5);
    \draw[dotted] (1,0,0.5) -- (1,0.75,0.5);
    \draw[dotted] (1,0,0.5) -- (1,0,0);
    \draw[dotted] (1,0,0.5) -- (0,0,0.5);
    
\end{tikzpicture}
\caption{Um vetor posição em três dimensões.}
\end{marginfigure}

Utilizando vetores, podemos fazer o mesmo para um movimento bidimensional ou tridimensional. Vamos escolher um ponto como origem de um sistema de coordenadas e descrever a posição por um vetor que parte da origem e termina no ponto onde o objeto se encontra. Como escolhemos a origem coincidindo com o início do vetor, a extremidade está no ponto $(x, y, z)$ do sistema de coordenadas. Além disso, o tamanho das componentes é igual aos valores de $x$, $y$, e $z$, logo, temos que o vetor posição será dado por
\begin{equation}
  \vec{r} = x \versi + y \versj + z \versk.
\end{equation}
%
O conjunto de posições ocupadas por um corpo ao longo do tempo é representado pela evolução temporal do vetor posição $\vec{r}$. Como a posição pode variar no tempo, temos que tal vetor é uma função do tempo $\vec{r}(t)$:
\begin{figure}
\centering
\begin{tikzpicture}
\draw (0,0.5) node[above] {$t$};
\draw (3.1,0.5) node[above] {$\vec{r}$};
\draw (0,-1) ellipse [x radius=12pt, y radius=40pt];
\draw (3.1,-1) ellipse [x radius=12pt, y radius=40pt];
\node [circle,draw,fill,scale=0.3] (A){};
\node [circle,draw,fill,scale=0.3] (B) [right=3cm of A] {};
\node [circle,draw,fill,scale=0.3] (C) [below=of A] {};
\node [circle,draw,fill,scale=0.3] (D) [right=3cm of C] {};
\node [circle,draw,fill,scale=0.3] (E) [below=of C] {};
\node [circle,draw,fill,scale=0.3] (F) [right=3cm of E] {};
\draw [thick, arrows={ - Stealth}]
(A) edge [bend left=45] node[above]{$\vec{r} = \vec{r}(t)$}(B)
(C) edge [bend left=45] (D)
(E) edge [bend left=45] (F);
\end{tikzpicture}
\caption{A cada valor de tempo $t$ temos um vetor posição $\vec{r}$ associado. A função $\vec{r}(t)$ descreve a relação entre essas duas variáveis.}
\end{figure}

% Adaptado de http://pgfplots.net/tikz/examples/spiral-cone/
\def\Point{36.9}
\begin{figure}\forceversofloat
\centering
\begin{tikzpicture}[>=Stealth]
  \begin{axis}[
    view       = {-25}{-25},
    axis lines = center,
    zmax       = 60,
    height     = 8cm,
    xtick      = \empty,
    ytick      = \empty,
    ztick      = \empty
 ]
  \addplot3+ [
    ytick      = \empty,
    yticklabel = \empty,
    domain     = 8*pi:12*pi,
    samples    = 400,
    samples y  = 0,
    mark       = none,
    thick,
    black
  ]
  ( {x*sin(0.28*pi*deg(x))},{x*cos(0.28*pi*deg(x)},{x}) coordinate [pos=0.18] (A)
            coordinate [pos=0.28] (B) coordinate [pos=0.35] (C);
  \addplot3+ [
    mark options = {color=black},
    mark         = none
  ] 
  coordinates {(0,0,0)};
%  \addplot3+ [
%    domain    = 0:12*pi,
%    samples   = 100,
%    samples y = 0,
%    mark      = none,
%    dashed,
%  ]  
%  ( {\Point*sin(0.28*pi*deg(\Point))}, {\Point*cos(0.28*pi*deg(\Point)}, {x} );
%  \addplot3[
%    mark=none,
%    dashed
%  ]
%  coordinates {(0,0,0) ({\Point*sin(0.28*pi*deg(\Point))},
%    {\Point*cos(0.28*pi*deg(\Point)}, {0})};

%  \node (P1) at (axis cs:20,0,30) {$P$};
%  \node (origin) at (0,0,0){};
  
%  \draw[->, thick] (axis cs:0,0,0) -- (axis cs:20,0,30);
  \draw[->, thick] (axis cs:0,0,0) -- node[left]{$\vec{r}(t_1)$} (A);
  \draw[->, thick] (axis cs:0,0,0) -- node[above left]{$\vec{r}(t_2)$} (B);
  \draw[->, thick] (axis cs:0,0,0) -- node[below right]{$\vec{r}(t_3)$} (C);
   
  \end{axis}
    
\end{tikzpicture}
\caption{A trajetória de um corpo pode ser descrita através do conjunto de posições $\vec{r}(t)$ ocupadas nos diferentes valores de tempo $t$. Na figura, destacamos três posições correspondentes a três valores diferentes de tempo.}
\end{figure}

%%%%%%%%%%%%%%%%%%%%%%%%%
\subsection{Deslocamento}
%%%%%%%%%%%%%%%%%%%%%%%%%

O deslocamento $\Delta \vec{r}$ de uma partícula entre dois instantes quaisquer é dado por
\begin{equation}
  \Delta\vec{r} = \vec{r}_f - \vec{r}_i,
\end{equation}
%
onde os vetores $\vec{r}_i$ e $\vec{r}_f$ são vetores que indicam as posições inicial e final da partícula entre dois instantes quaisquer. Utilizando a notação de versores temos para $\vec{r}_i$ e $\vec{r}_f$
\begin{align}
  \vec{r}_i &= x_i \versi + y_i \versj + z_i \versk \\
  \vec{r}_f &= x_f \versi + y_f \versj + z_f \versk.
\end{align}
%
Consequentemente, o vetor deslocamento será dado por
\begin{align}
  \Delta\vec{r} &= (x_f \versi + y_f \versj + z_f \versk) - (x_i \versi + y_i \versj + z_i \versk) \\
  &= (x_f - x_i)\versi + (y_f - y_i)\versj + (z_f - z_i)\versk \\
  &= \Delta x\versi + \Delta y\versj + \Delta z\versk.
\end{align}
%
Vemos que é possível separar o movimento descrito pelo vetor posição em três componentes distintas, uma para cada eixo coordenado. Isso facilita a análise do movimento, permitindo que tratemos cada uma das componentes de acordo com suas particularidades. Em muitos casos não precisamos nem mesmo tratar todas as três componentes, devido ao fato de que o movimento ocorre em um plano.
 
\begin{marginfigure}[-10cm]
\centering
\begin{tikzpicture}[>=Stealth]
    \path[name path=traj, draw] (0,2) .. controls (1,0.5) and (2,2.5) .. (3,1.5);
    
    \coordinate (a) at (0.1,0);
    \coordinate (b) at (1.2,0);
    
    \path[name path=verta] (a)--+(0,3);
    \path[name path=vertb] (b)--+(0,3);
       
    \path[draw, ->, name intersections={of=traj and verta}](0,0) -- node[left]{$\vec{r}_i$} (intersection-1) coordinate (intersec1);
    \path[draw, ->, name intersections={of=traj and vertb}](0,0) -- node[right]{$\vec{r}_f$}(intersection-1) coordinate (intersec2);
    
    \draw[->] (intersec1) -- node[above]{$\Delta \vec{r}$} (intersec2);
    
    \draw[fill] (0,0) node[below left]{$O$} circle (1pt);
    
\end{tikzpicture}
\caption{O vetor deslocamento $\Delta \vec{r}$ pode ser calculado a partir da diferença entre os vetores $\vec{r}_f$ e $\vec{r}_i$.}
\end{marginfigure}

%%%%%%%%%%%%%%%%%%%%
\section{Velocidade}
%%%%%%%%%%%%%%%%%%%%

%%%%%%%%%%%%%%%%%%%%%%%%%%%%%
\subsection{Velocidade média}
%%%%%%%%%%%%%%%%%%%%%%%%%%%%%

Em três dimensões, caso um corpo sofra um deslocamento, ele o faz com uma velocidade média dada por
\begin{equation}\label{Eq:VelMediaVetorial}
  \vec{\mean{v}} = \frac{\Delta\vec{r}}{\Delta t},
\end{equation}
%
onde estendemos a definição de velocidade ao caso bi e tridimensional ao substituir o deslocamento ao longo de um eixo retilíneo $x$ pelo vetor deslocamento $\Delta \vec{r}$. Como, nesse caso, temos a divisão de um vetor por um escalar, \emph{a direção do vetor velocidade média é a mesma do deslocamento}. O módulo, no entanto, é diferente, assim como a dimensão: temos que $[v] = L/T$ e o módulo é dado pelo valor numérico obtido pela divisão do módulo do vetor deslocamento pelo valor do intervalo de tempo em que o movimento ocorre. Decompondo o vetor, temos
\begin{align}
  \vec{\mean{v}} &= \frac{\Delta x \versi + \Delta y\versj + \Delta z\versk}{\Delta t} \\
  &= \frac{\Delta x}{\Delta t} \versi + \frac{\Delta y\versj}{\Delta t} + \frac{\Delta z\versk}{\Delta t} \\
  &= \mean{v}_x \versi + \mean{v}_y \versj + \mean{v}_z \versk.
\end{align}

%%%%%%%%%%%%%%%%%%%%%%%%%%%%%%%%%%%
\subsection{Velocidade instantânea}
%%%%%%%%%%%%%%%%%%%%%%%%%%%%%%%%%%%

Podemos definir a velocidade instantânea a partir da Equação~\ref{Eq:VelMediaVetorial}, bastando tomar o limite $\Delta t \to 0$:
\begin{align}
    \vec{v} &= \lim_{\Delta t \to 0} \vec{\mean{v}} \\
    &= \lim_{\Delta t \to 0} \frac{\Delta\vec{r}}{\Delta t} \\
    &= \lim_{\Delta t \to 0} \frac{\Delta x}{\Delta t}\versi + \frac{\Delta y}{\Delta t}\versj + \frac{\Delta z}{\Delta t}\versk.
\end{align}
%
Podemos utilizar a propriedade de que o limite da soma é a soma dos limites para escrever
\begin{align}
  \vec{v} &= \lim_{\Delta t \to 0} \frac{\Delta x}{\Delta t}\versi + \lim_{\Delta t \to 0} \frac{\Delta y}{\Delta t}\versj + \lim_{\Delta t \to 0} \frac{\Delta z}{\Delta t}\versk \\
  &= v_x \versi + v_y \versj + v_z \versk.
\end{align}
%
Portanto, podemos simplesmente definir o vetor velocidade através da velocidade nos eixos $x$, $y$ e $z$.

Se analisarmos a trajetória de uma partícula em um plano $xy$, mostrada na Figura~\ref{Fig:Dir_vel}, verificamos que para $\Delta t \to 0$, o vetor $\delta\vec{r}$, dado por\footnote[][-2cm]{Utilizaremos a notação $\delta \xi$ para todas as variáveis do tipo $\Delta \xi$ quando tomamos o limite $\Delta t \to 0$. Podemos interpretar isso como \emph{uma variação infinitamente pequena}.}
\begin{equation}
    \delta\vec{r} = \lim_{\Delta t \to 0}\Delta \vec{r},
\end{equation}
%
é tangente à trajetória. Como a direção da velocidade é a mesma de $\delta\vec{r}$, temos que o vetor velocidade instantânea é tangente à trajetória.

% adaptado de https://tex.stackexchange.com/questions/25928/how-to-draw-tangent-line-of-an-arbitrary-point-on-a-path-in-tikz
\begin{marginfigure}[-1cm]
\centering
\begin{tikzpicture}[>=Stealth,
    tangent/.style={
        decoration={
            markings,% switch on markings
            mark=
                at position #1
                with
                {
                    \coordinate (tangent point-\pgfkeysvalueof{/pgf/decoration/mark info/sequence number}) at (0pt,0pt);
                    \coordinate (tangent unit vector-\pgfkeysvalueof{/pgf/decoration/mark info/sequence number}) at (1,0pt);
                    \coordinate (tangent orthogonal unit vector-\pgfkeysvalueof{/pgf/decoration/mark info/sequence number}) at (0pt,1);
                }
        },
        postaction=decorate
    },
    use tangent/.style={
        shift=(tangent point-#1),
        x=(tangent unit vector-#1),
        y=(tangent orthogonal unit vector-#1)
    },
    use tangent/.default=1
]

%%%

    \path[name path=traj, draw, tangent=0.30] (0,2) .. controls (1,0.3) and (2,2.2) .. (3,1.5) coordinate [pos=0.24] (point);
    \draw [dashed, thin, use tangent] (-1.5,0) -- (1.5,0);
    \draw [->, use tangent] (0,0) -- node[below]{$\delta\vec{r}$}(0.5,0);
    \draw [->, use tangent] (0,0) -- node[below right]{$\vec{v}$}(1,0);
    
    \coordinate (a) at (0.1,0);
    \coordinate (b) at (1.9,0);
    
    \path[name path=verta] (a)--+(0,3);
    \path[name path=vertb] (b)--+(0,3);
       
    \path[draw, gray, ->, name intersections={of=traj and verta}](0,0) --  (intersection-1) coordinate (intersec1);
    \path[draw, gray, ->, shorten >=1pt, name intersections={of=traj and vertb}](0,0) -- (intersection-1) coordinate (intersec2);
    
    \draw[->, gray, shorten >=1pt] (intersec1) -- (intersec2);
    
    \coordinate (c) at (0.3,0);
    \coordinate (d) at (1.7,0);
    
    \path[name path=vertc] (c)--+(0,3);
    \path[name path=vertd] (d)--+(0,3);
       
    \path[draw, gray, ->, name intersections={of=traj and vertc}](0,0) --  (intersection-1) coordinate (intersec3);
    \path[draw, gray, ->, shorten >=1pt, name intersections={of=traj and vertd}](0,0) -- (intersection-1) coordinate (intersec4);
    
    \draw[->, gray, shorten >=1pt] (intersec3) -- (intersec4);
    
    
    \draw[fill] (0,0) node[below left]{$O$} circle (1pt);
    \draw[fill, use tangent] (0,0) circle (1pt);
    \draw[->] (0,0) -- node[above]{$\vec{r}$} (point);
    
\end{tikzpicture}
\caption{No limite $\Delta t \to 0$, temos que a direção do vetor deslocamento instantâneo $\delta\vec{r}$ no ponto denotado por $\vec{r}$ é a mesma direção que a da reta que tange a curva no ponto.\label{Fig:Dir_vel}}
\end{marginfigure}

%%%%%%%%%%%%%%%%%%%%
\section{Aceleração}
%%%%%%%%%%%%%%%%%%%%

%%%%%%%%%%%%%%%%%%%%%%%%%%%%%
\subsection{Aceleração média}
%%%%%%%%%%%%%%%%%%%%%%%%%%%%%

No caso do cálculo da velocidade média, bastou redefinirmos a velocidade em termos do vetor deslocamento $\Delta\vec{r}$ para verificarmos que a velocidade é uma grandeza vetorial. Devido a essa conclusão, temos que uma variação de velocidade pode ser uma variação tanto de módulo, quanto de direção ou sentido. Portanto, precisamos redefinir a aceleração em termos de um vetor $\Delta\vec{v}$:
\begin{align}
  \vec{\mean{a}} &= \frac{\Delta\vec{v}}{\Delta t} \\
  &= \frac{\Delta v_x}{\Delta t}\versi + \frac{\Delta v_y}{\Delta t}\versj + \frac{\Delta v_z}{\Delta t}\versk \\
  &= \mean{a}_x\versi + \mean{a}_y\versj + \mean{a}_z\versk.
\end{align} 
%
A direção do vetor aceleração média é a própria direção do vetor $\Delta{\vec{v}}$. 

\begin{marginfigure}[-3cm]
   \begin{tikzpicture}[>=Stealth]
        \draw[dotted] ([shift={(0,0)}]120:2) arc[radius=2, start angle=120, end angle= 0];
        \draw[->] (0,0) -- node[left]{$\vec{r}_i$} (60:2);
        \draw[->] (60:2) -- node[above]{$\vec{v}_i$} +(-30:1);
        \draw[->] (0,0) -- node[below]{$\vec{r}_f$} (30:2);
        \draw[->] (30:2) -- node[right]{$\vec{v}_f$}+(-60:0.7);
        
        \draw[->] (1, -1) -- node[above]{$\vec{v}_i$} +(-30:1) coordinate (A);
        \draw[->] (1, -1) -- node[left]{$\vec{v}_f$} +(-60:0.7) coordinate (B);
        \draw[->] (A) -- node[below]{$\Delta \vec{v}$} (B);
        
   \end{tikzpicture}
   \caption{Velocidades em diferentes instantes e a correspondente variação da velocidade $\Delta\vec{v}$ determinada através da diferença entre os vetores.}
\end{marginfigure}

%%%%%%%%%%%%%%%%%%%%%%%%%%%%%%%%%%%
\subsection{Velocidade instantânea}
%%%%%%%%%%%%%%%%%%%%%%%%%%%%%%%%%%%

Assim como no caso do cálculo da velocidade instantânea, podemos calcular a aceleração instantânea vetorial através do limite $\Delta t \to 0$:
\begin{align}
  \vec{a} &= \lim_{\Delta t \to 0} \vec{\mean{a}} \\
  &= \lim_{\Delta t \to 0} \frac{\Delta \vec{v}}{\Delta t} \\
  &= \lim_{\Delta t \to 0} \frac{\Delta {v_x}}{\Delta t}\versi + \lim_{\Delta t \to 0} \frac{\Delta {v_y}}{\Delta t}\versj + \lim_{\Delta t \to 0} \frac{\Delta {v_z}}{\Delta t}\versk,
\end{align}
%
onde utilizamos novamente a propriedade de que o limite da soma é a soma dos limites. Obtemos então
\begin{equation}
  \vec{a} = a_x \versi + a_y \versj + a_z \versk,
\end{equation}
%
e observamos que para o caso da aceleração, também temos que as componentes do vetor são dadas pelos valores de -- neste caso -- aceleração dos eixos $x$, $y$ e $z$.

%%%%%%%%%%%%%%%%%%%%%%%%%%%%%%%%%%%%%%%%%%%%%%%%%%%%%%%%%%%%%
\section{Independência do movimento em eixos perpendiculares}
%%%%%%%%%%%%%%%%%%%%%%%%%%%%%%%%%%%%%%%%%%%%%%%%%%%%%%%%%%%%%

Verificamos no Capítulo~\ref{Chap:Vetores} que se temos
\begin{equation}
    \vec{c} = \vec{a} + \vec{b},
\end{equation}
%
então
\begin{align}
    c_x &= a_x + b_x \\
    c_y &= a_y + b_y.
\end{align}
%
Em três dimensões, adicionamos mais uma relação desse tipo para o eixo $z$:
\begin{equation}
    c_z = a_z + b_z.
\end{equation}
%
Verificamos através dessas relações que os três eixos são independentes no que se refere à soma de vetores.

Observamos ainda que se tormarmos a definição de velocidade média
\begin{equation}
    \mean{\vec{v}} = \frac{\Delta \vec{r}}{\Delta t},
\end{equation}
%
podemos escrever
\begin{align}
    \vec{r}_f - \vec{r}_i &= \mean{\vec{v}} \Delta t \\
    \vec{r}_f &= \vec{r}_i + \mean{\vec{v}} \Delta t.
\end{align}
%
Através da definição de aceleração média, obtemos uma relação semelhante:
\begin{equation}
    \vec{v}_f = \vec{v}_i + \mean{\vec{a}} \Delta t.
\end{equation}

\pagebreak
\noindent{}Note que ambas as expressões acima são dadas por \emph{somas de vetores}. Portanto, podemos separar as expressões acima em três eixos:\footnote{Usamos \begin{align*} x \equiv r_x \\ y \equiv r_y \\ z \equiv r_z. \end{align*}}
\begin{align}
    x_f &= x_i + \mean{v}_x \Delta t \\
    y_f &= y_i + \mean{v}_y \Delta t \\
    z_f &= z_i + \mean{v}_y \Delta t
\end{align}
%
e
\begin{align}
    v_x^f &= v_x^i + \mean{a}_x \Delta t \\
    v_y^f &= v_y^i + \mean{a}_y \Delta t \\
    v_z^f &= v_z^i + \mean{a}_z \Delta t. \\
\end{align}
%
se tomarmos o limite $\Delta t \to 0 \equiv \delta t$, podemos escrever o equivalente das equações acima para o caso de velocidade e aceleração instantânea:
\begin{align}
    x_f &= x_i + v_x \delta t \\
    y_f &= y_i + v_y \delta t \\
    z_f &= z_i + v_y \delta t
\end{align}
%
e
\begin{align}
    v_x^f &= v_x^i + a_x \delta t \\
    v_y^f &= v_y^i + a_y \delta t \\
    v_z^f &= v_z^i + a_z \delta t. \\
\end{align}

Podemos considerar que todo e qualquer movimento pode ser descrito através da soma de \emph{infinitos} deslocamentos infinitezimais, como descrito pelas equações acima. Isso implica no fato de que \emph{todo movimento pode ser separado em três eixos independentes}. Podemos, portanto, utilizar as equações obtidas para a cinemática unidimensional em cada eixo de maneira independente.

\begin{marginfigure}
\centering
\begin{tikzpicture}[>=Stealth, scale=1.8]
    \draw[->] (0:2) -- (10:2);
    \draw[->] (10:2) -- (20:2);
    \draw[->] (20:2) -- (30:2);
    \draw[->] (30:2) -- (40:2);
    \draw[->] (40:2) -- (50:2);
    \draw[->] (50:2) -- (60:2);
    \draw[->] (60:2) -- (70:2);
    \draw[->] (70:2) -- (80:2);
    \draw[->] (80:2) -- (90:2);
    
\end{tikzpicture}
\caption{Todo movimento pode ser considerado como uma série de deslocamentos infinitamente pequenos. Na figura mostramos uma série de pequenos deslocamentos que constituem de maneira aproximada uma curva. Se tomarmos deslocamentos menores, a curva passará a ser mais suave. A direção de cada deslocamento muda pois estamos considerando que a direção da velocidade mude devido a uma aceleração.}
\end{marginfigure}

%%%%%%%%%%%%%%%%%%%%%%%%%%%%%%%%
\section{Movimento de projéteis}
%%%%%%%%%%%%%%%%%%%%%%%%%%%%%%%%
% TODO Descrição

O movimento de um projétil ao ser lançado com velocidade que faz um ângulo com a horizontal é conhecido como movimento balístico. Neste movimento, podemos usar a propriedade da decomposição da velocidade para analisar o movimento em cada eixo separadamente.

Podemos decompor a velocidade em dois eixos, um horizontal (eixo $x$) e um vertical (eixo $y$), obtendo
\begin{equation}
  \vec{v}_i = v_{ix} \versi + v_{iy} \versj,
\end{equation}
%
onde
\begin{align}
  v_{ix} &= v_i\cos\theta \\
  v_{iy} &= v_i\sen\theta.
\end{align}
%
Como a aceleração gravitacional está na direção vertical e é dirigida para baixo, a denotamos como
\begin{equation}
  \vec{g} = -g\versj,
\end{equation}
%
onde $g$ denota o módulo da aceleração da gravidade, cujo valor é de aproximadamente \np[m/s^2]{9,8}.

%%%%%%%%%%%%%%%%%%%%%%%%%%%%%%%%%%%%%%%%%%%%%%%%%%%%%%%%%
\subsection{Eixo $x$: Movimento com velocidade constante}
%%%%%%%%%%%%%%%%%%%%%%%%%%%%%%%%%%%%%%%%%%%%%%%%%%%%%%%%%

Analisando o movimento no eixo $x$, temos uma velocidade inicial -- dada por $v_{ix} = v_i\cos\theta$ -- e \emph{não temos nenhuma aceleração}. Como podemos analisar o movimento em cada eixo de maneira completamente independente dos demais, concluímos que
\begin{align}
  v_{ix} &= \text{constante} \\
  x_{f} &= x_{i} + v_{ix}t.\label{Eq:PosXProj}
\end{align}

%%%%%%%%%%%%%%%%%%%%%%%%%%%%%%%%%%%%%%%%%%%%%%%%%%%%%%%%%
\subsection{Eixo $y$: Movimento com aceleração constante}
%%%%%%%%%%%%%%%%%%%%%%%%%%%%%%%%%%%%%%%%%%%%%%%%%%%%%%%%%

Verticalmente, temos um movimento com aceleração constante, dirigida para baixo. Se adotarmos o eixo $y$ crescendo para cima, a partir das Equações~\eqref{Eq:VV0AT} e~\eqref{Eq:XX0V0TAT22}, temos
\begin{align}
  v_{fy} &= v_{iy} - gt \\
  y_f &= y_i + v_{iy}t - \frac{g}{2}t^2. \label{Eq:PosYProj}
\end{align}

%%%%%%%%%%%%%%%%%%%%%%%%%%
\subsection{Altura máxima}
%%%%%%%%%%%%%%%%%%%%%%%%%%

\begin{marginfigure}[-7cm]
\centering
\begin{tikzpicture}[>=Stealth]
    \draw[->] (0,0) -- (4,0) node[below left]{$x$};
    \draw[->] (0,0) -- (0,2) node[below left]{$y$};
    
    \draw[dashed, smooth, samples=1000, domain=0:3.33] plot (\x, {2*\x - 0.6*\x*\x});
    
    \coordinate (O) at (0,0);
    \coordinate (A) at (1,0);
    \coordinate (B) at (63.434948823:1);
       
    \pic [draw, "$\theta$", angle eccentricity=1.5] {angle = A--O--B};
        
    \draw[dotted] (1.666, 1.666) -- +(2,0);
    \draw[<->] (3.5, 0) -- node[right]{$H$} (3.5, 1.666);
    
\end{tikzpicture}
\caption{Altura máxima em relação ao ponto de lançamento.}
\end{marginfigure}

A partir da equação acima, podemos determinar qual é o valor de altura máxima que o projétil alcança ao ser lançado com velocidade $\vec{v}_i$. Sabemos que no ponto onde o projétil atinge a altura máxima, sua velocidade no eixo vertical deve ser nula, afinal ocorre uma inversão no sentido do movimento. Utilizando a equação de Torricelli, obtemos\footnote[][-2cm]{Na dedução das equações para altura máxima, alcance horizontal e para a trajetória, escolheremos um sistema de coordenadas onde o eixo $y$ cresce verticalmente para cima. Nesse caso, ao utilizar as fórmulas é importante que tal escolha também seja efetuada ao se resolver exercícios e problemas, respeitando a convenção que originou as fórmulas. Se isso não acontecer, ocorrerão problemas com os sinais de algumas variáveis cinemáticas.}
\begin{equation}
  v_{fy}^2 = v_{iy}^2 - 2 g \Delta y.
\end{equation}
%
Substituindo $v_{fy} = 0$ e $v_{iy} = v_i\sen\theta$, obtemos
\begin{equation}
  v_i\sen\theta = 2g\Delta y,
\end{equation}
%
e, finalmente, denotando a altura máxima por $H$ e sabendo que $H = \Delta y$,
\begin{equation}
  H = \frac{v_i^2\sen^2\theta}{2g}. \mathnote{Altura máxima.}
\end{equation}

%%%%%%%%%%%%%%%%%%%%%%%%%%%%%%%
\subsection{Alcance horizontal}
%%%%%%%%%%%%%%%%%%%%%%%%%%%%%%%

O alcance horizontal de um projétil pode ser calculado se soubermos qual é o tempo decorrido entre o objeto ser lançado e voltar à mesma posição no eixo $y$ que ocupava no momento do lançamento. Temos então que o deslocamento $\Delta y$ será nulo, logo, à partir da Equação~\eqref{Eq:XX0V0TAT22}, temos
\begin{equation}
  y_f-y_i = v_{iy}t - \frac{g}{2}t^2,
\end{equation}
%
ou, devido à nossa observação de que $\Delta y = 0$
\begin{equation}
  v_{iy}t = \frac{g}{2} t^2.
\end{equation}
%
Esta equação admite a solução $t = 0$, que corresponde ao momento do lançamento (o que não é particularmente útil), ou -- dividindo ambos os membros da equação por $t$ e isolando a variável $t$ restante --
\begin{align}
  t &= \frac{2v_{iy}}{g} \\
  &= 2\frac{v_i\sen\theta}{g}.
\end{align}

Para calcularmos a distância percorrida pelo projétil, basta utilizarmos a Equação~\eqref{Eq:PosXProj}, obtendo
\begin{align}
  R &\equiv \Delta x = v_{ix} t \\
  &= \left(v_i\cos\theta\right) \left(2\frac{v_i\sen\theta}{g}\right) \\
  &= \frac{2v_i^2}{g}\sen\theta\cos\theta.
\end{align}
%
Utilizando a relação trigonométrica $\sen 2\theta = 2 \sen\theta\cos\theta$, podemos reescrever a expressão acima de uma maneira mais amigável:
\begin{equation}
  R = \frac{v_i^2}{g}\sen2\theta. \mathnote{Alcance horizontal.}
\end{equation}

\begin{marginfigure}[-7cm]
\centering
\begin{tikzpicture}[>=Stealth]
    \draw[->] (0,0) -- (4,0) node[below left]{$x$};
    \draw[->] (0,0) -- (0,2) node[below left]{$y$};
    
    \draw[dashed, smooth, samples=1000, domain=0:3.33] plot (\x, {2*\x - 0.6*\x*\x});
    
    \coordinate (O) at (0,0);
    \coordinate (A) at (1,0);
    \coordinate (B) at (63.434948823:1);
       
    \pic [draw, "$\theta$", angle eccentricity=1.5] {angle = A--O--B};
        
    \draw[|<->|] (0, -0.3) -- node[below]{$R$} (3.33, -0.3);
    
\end{tikzpicture}
\caption{Alcance em relação ao ponto de lançamento.}
\end{marginfigure}

%%%%%%%%%%%%%%%%%%%%%%%%%%%%%%%%%%%%%%
\subsection{Equação para a trajetória}
%%%%%%%%%%%%%%%%%%%%%%%%%%%%%%%%%%%%%%

Podemos determinar a forma da trajetória do projétil a escrevendo como uma função $y(x)$. Para isso, podemos isolar o tempo na Equação~\eqref{Eq:PosXProj}, obtendo
\begin{equation}
  t = \frac{x_f - x_i}{v_{ix}}.
\end{equation}
%
Substituindo essa expressão na Equação~\eqref{Eq:PosYProj}, obtemos
\begin{equation}
  y_f = y_i + v_i\sen\theta \frac{x_f-x_i}{v_i\cos\theta} - \frac{g}{2}\frac{(x_f-x_i)^2}{(v_i\cos\theta)^2},
\end{equation}
%
onde utilizamos $v_{ix} = v_i\cos\theta$ e $v_{iy} = v_i\sen\theta$. Para simplificar a expressão acima, vamos escolher $y_i = x_i = 0$, $y_f = y$ e $x_f = x$. Obtemos assim
\begin{equation}
  y = (\tan\theta) \; x - \left(\frac{g}{2v_i^2\cos^2\theta}\right) x^2. \mathnote{Equação da trajetória.}
\end{equation}

Se compararmos a equação acima a um polinômio de segundo grau, cuja forma característica é a de uma parábola, 
\begin{equation}
  y = A + B x + C x^2,
\end{equation}
%
verificamos que a equação da trajetória segue o mesmo formato, porém com $A = 0$. Concluímos então que a trajetória seguida pelo projétil é tem a forma de uma parábola, com concavidade voltada para baixo\footnote{A concavidade de um polinômio de segundo grau é determinada através do sinal do coeficiente $C$: se o coeficiente é positivo, a concavidade é voltada para cima; se for negativo, é voltada para baixo.}.

%%%%%%%%%%%%%%%%%%%%%%%%%%%%
\section{Movimento circular}
%%%%%%%%%%%%%%%%%%%%%%%%%%%%
%%%%%%%%%%%%%%%%%%%%%%%%%%%%%%%%%%
\subsection{Aceleração centrípeta}
%%%%%%%%%%%%%%%%%%%%%%%%%%%%%%%%%%

%TiKZ 3.0 introduces angles and quotes libraries which simplifies this task.
%A command like
%\draw pic[draw] {angle=A--O--C};
%will draw a sector line between line A--O and O--C with center at O and anticlockwise. Sector radius is 5mm by default but can be changed with angle radius parameter.
%This sector line can be labelled with label or with new quotes syntax. The label is placed with certain angle eccentricity (default = 0.6) which is a factor to be applied to angle radius.
%Next example is similar to student's code but using angle pic. 

\begin{marginfigure}
\centering
   \begin{tikzpicture}[>=Stealth, scale=1.4]
        \draw[dotted] ([shift={(0,0)}]120:2) arc[radius=2, start angle=120, end angle= 0];
        \draw[->, gray] (0,0) -- node[left]{$\vec{r}_i$} (60:2);
        \draw[->] (60:2) -- node[above right]{$\vec{v}_i$} +(-30:0.51763809) coordinate (A);
        \draw[->, gray] (0,0) -- node[below]{$\vec{r}_f$} (30:2);
        \draw[->, dashed] (30:2) -- node[right]{$\vec{v}_f$}+(-60:0.51763809);
        \draw[dashdotted] (0,0) -- (45:2.6);
        
        \draw[->] (60:2) -- node[below left]{$\vec{v}_f$} +(-60:0.51763809) coordinate (B);
        \draw[->] (A) -- node[right]{$\Delta \vec{v}$} (B);
        
   \end{tikzpicture}
   \caption{Em um movimento circular com velocidade constante, o vetor $\Delta\vec{v}$ aponta para o centro da trajetória quando disposto exatamente no ponto intermediário entre as posições inicial e final. Essa é a mesma direção que a aceleração média, consequentemente, quando tomamos o limite $\Delta t \to 0$ e aproximamos os pontos, verificamos que a aceleração instantânea aponta para o centro da trajetória.\label{Fig:Acel_mov_cir_unif}}
\end{marginfigure}

Analisando o movimento circular -- restrito ao caso de velocidade constante em módulo --, verificamos que temos uma alteração constante da direção do vetor velocidade. Na Figura~\ref{Fig:Acel_mov_cir_unif} vemos uma parte da trajetória seguida por uma partícula. Em dois instantes diferentes, temos dois vetores velocidade que têm o mesmo módulo, porém direções diferentes. Se calculamos geometricamente a diferença entre esses vetores, vemos que $\Delta \vec{v}$ aponta perpendicularmente à trajetória quando disposto na região central entre as posições inicial e final, isto é, ele aponta para o centro da trajetória circular.

Sabemos, que o vetor aceleração média é dado por
\begin{equation}
  \vec{\mean{a}} = \frac{\Delta v}{\Delta t}.
\end{equation}
%
Portanto, mesmo no caso de $v$ constante, temos uma aceleração caso ocorram mudanças na direção do vetor velocidade. 

Podemos calcular o módulo desta aceleração se considerarmos a Figura~\ref{Fig:Acel_mov_cir_unif_mod}. Inicialmente uma partícula ocupa a posição $\vec{r}_i$, com velocidade $\vec{v}_i$ no instante $t_i$. Após um intervalo de tempo, ela passa a ocupar a posição $\vec{r}_f$, com velocidade $\vec{v}_f$ no instante $t_f$. Vamos assumir que $v_i = v_f = v$ e que $r_i = r_f = r$. 


\begin{marginfigure}
\centering
   \begin{tikzpicture}[>=Stealth, scale=1.4]
        \draw[dotted] ([shift={(0,0)}]120:2) arc[radius=2, start angle=120, end angle= 0];
        \coordinate (O) at (0,0);
        
        \draw[->] (O) -- node[left]{$\vec{r}_i$} (60:2) coordinate (A);
        \draw[->] (O) -- node[below]{$\vec{r}_f$} (30:2) coordinate (B);
        \draw[->] (A) -- node[above right]{$\Delta \vec{r}$} (B);
        
        \pic [draw, "$\theta$", angle eccentricity=1.5] {angle = B--O--A};
        \pic [draw, "$\alpha$", angle eccentricity=1.5, angle radius = 3mm] {angle = O--A--B};
        \pic [draw, "$\alpha$", angle eccentricity=1.5, angle radius = 3mm] {angle = A--B--O};
        
        \coordinate (O) at (0, -1);    
        
        \draw[->] (O) -- node[above]{$\vec{v}_i$} +(-30:1.5) coordinate (A);
        \draw[->] (O) -- node[left]{$\vec{v}_f$} +(-60:1.5) coordinate (B);
        \draw[->] (A) -- node[right]{$\Delta \vec{v}$} (B);
        
        \pic [draw, "$\theta$", angle eccentricity=1.5] {angle = B--O--A};
        \pic [draw, "$\alpha$", angle eccentricity=1.5, angle radius = 3mm] {angle = O--A--B};
        \pic [draw, "$\alpha$", angle eccentricity=1.5, angle radius = 3mm] {angle = A--B--O};
    
   \end{tikzpicture}
   \caption{Triângulos formados pelos vetores $\vec{r}_i$, $\vec{r}_f$, e $\Delta \vec{r}$ e pelos vetores $\vec{v}_i$, $\vec{v}_f$, e $\Delta\vec{v}$. Note que este último foi ampliado em relação à Figura~\ref{Fig:Acel_mov_cir_unif} unicamente para facilitar a visualização.\label{Fig:Acel_mov_cir_unif_mod}}
\end{marginfigure}

Verificamos que existe um ângulo $\theta$ entre os vetores $\vec{r}_i$ e $\vec{r}_f$. Além disso, como $r_i = r_f = r$, temos que os outros dois ângulos do triângulo são $\alpha$. Podemos utilizar a lei dos senos para estabelecer a seguinte relação:
\begin{equation}
    \frac{\Delta r}{\sen \theta} = \frac{r}{\sen \alpha},
\end{equation}
%
ou, equivalentemente,
\begin{equation}
    \frac{\Delta r}{r} = \frac{\sen \theta}{\sen \alpha}.
\end{equation}

Os vetores velocidade inicial e final são perpendiculares aos vetores posição inicial e final, respectivamente. Portanto, o ângulo formado pelos vetores velocidade é o mesmo ângulo formado pelos vetores posição, isto é, o ângulo $\theta$ (imagine o seguinte: o vetor $\vec{r}_i$ é girado por um ângulo $\theta$ para se tornar o vetor $\vec{r}_f$. Essa rotação também afeta o vetor velocidade $\vec{v}_i$ o transformando no vetor $\vec{v}_f$, pois a relação de perpendicularidade entre o vetor velocidade e o vetor posição se mantém para todos os pontos em um movimento circular). Além disso, como $v_i = v_f = v$, os demais ângulos são iguais entre si e são iguais ao mesmo ângulo $\alpha$ que aparece no triângulo formado pelos vetores posição e deslocamento. Aplicando novamente a lei dos senos, obtemos
\begin{equation}
    \frac{\Delta v}{\sen \theta} = \frac{v}{\sen \alpha},
\end{equation}
%
ou
\begin{equation}
    \frac{\Delta v}{v} = \frac{\sen\alpha}{\sen\theta}.
\end{equation}

A partir desses resultados, temos que
\begin{equation}
    \frac{\Delta v}{v} = \frac{\Delta r}{r}.
\end{equation}
%
Isolando $\Delta v$ e substituindo na expressão para a aceleração média, obtemos para o módulo
\begin{equation}
  \mean{a} = \frac{v}{r} \frac{\Delta r}{\Delta t}.
\end{equation}
%
Tomando o limite $\Delta t \to 0$, obtemos a aceleração instantânea:
\begin{equation}
  a = \frac{v}{r} \lim_{\Delta t \to 0} \frac{\Delta r}{\Delta t},
\end{equation}
%
onde usamos a propriedade
\begin{equation}
    \lim_{\epsilon \to \xi} \lambda f(\epsilon) = \lambda \lim_{\epsilon\to \xi} f(\epsilon),
\end{equation}
%
$\lambda$ e $\xi$ representandos constantes quaisquer. Notamos que o limite que resta é a razão entre a distância percorrida pela partícula e o tempo necessário para efetuar tal deslocamento, ou seja, é a velocidade $v$. Logo,
\begin{equation}
  a = \frac{v^2}{r}.
\end{equation}

Verificamos no início desta seção que a aceleração média aponta para o centro da trajetória quando a dispomos exatamente na região central entre os pontos inicial e final (para o cálculo da aceleração média em questão). Quando tomamos o limite $\Delta t \to 0$, o que fazemos é mover tais pontos de forma que eles se tornam infinitamente próximos e se tornem o mesmo ponto, o que faz com que \emph{a aceleração instantânea aponte para o centro da trajetória circular}. Denominamos essa aceleração como \emph{aceleração centrípeta} $a_c$, cujo módulo é dado por
\begin{equation}\label{Eq:AceleracaoCentripeta}
  a_c = \frac{v^2}{r}. \mathnote{Módulo da aceleração centrípeta}
\end{equation}
%
Essa aceleração é responsável por alterar constantemente a direção do vetor velocidade, possibilitando que a partícula execute um movimento curvilíneo. Veremos adiante que ela está relacionada à força resultante que deve ser exercida por um agente externo para que uma partícula execute um movimento circular com velocidade $v$ e raio $r$.

%%%%%%%%%%%%%%%%%%%%%%%%%%%%%%%%%%%%%%%%%%%%%%%%%%%%%%%%%%%%%%%%%%%%%%%%%%%%%%
\subsection{Decomposição da aceleração em componentes tangencial e centrípeta}
%%%%%%%%%%%%%%%%%%%%%%%%%%%%%%%%%%%%%%%%%%%%%%%%%%%%%%%%%%%%%%%%%%%%%%%%%%%%%%

Verificamos que a aceleração é dada por
\begin{displaymath}
  \vec{a} = \lim_{\Delta t \to 0} \frac{\Delta \vec{v}}{\Delta t},
\end{displaymath}
%
e no caso específico de um movimento circular com velocidade constante, ela aponta para o centro da trajetória circular, com módulo dado pela Equação~\eqref{Eq:AceleracaoCentripeta}. No entanto, o caso de velocidade constante não é geral: podemos ter um movimento circular com velocidade que muda constantemente, ou mesmo ter um movimento curvilíneo que não tem um formato circular.

No caso de termos um movimento curvilíneo com velocidade constante, porém não circular, temos que a aceleração centrípeta não é constante. Se temos uma curvatura que se ``fecha'' paulatinamente, isto é, se o raio de curvatura diminui progressivamente, a aceleração centrípeta aumenta. Se ocorre o contrário, a curvatura se ``abre'' e o raio aumenta, temos que a aceleração centrípeta diminui. De qualquer forma, podemos calcular \emph{instantaneamente} a aceleração através da Expressão~\ref{Eq:AceleracaoCentripeta}.

Quando ocorre uma alteração do módulo da velocidade, podemos determinar a aceleração simplesmente calculando a razão $\Delta v/\Delta t$. Temos então dois efeitos possíveis da aceleração: um deles é alterar o módulo da velocidade, o outra é alterar a direção do vetor velocidade. Tais efeitos podem ocorrer isoladamente, como num movimento retilíneo com velocidade variável -- onde ocorre somente o primeiro --, ou num movimento circular com velocidade constante -- onde ocorre somente o segundo --, ou podem ocorrer \emph{ao mesmo tempo}.

O caso de ambos os efeitos ocorrerem conjuntamente é, na prática, o mais comum: um carro que trafega por uma rodovia, por exemplo, executa curvas e altera sua velocidade a todo momento, e em inúmeras vezes, essas mudandas de direção e de módulo da velocidade acontecem concomitantemente. Apesar de termos dois papéis distintos para a aceleração, temos \emph{somente um vetor aceleração} -- dado pela equação para $\vec{a}$ definida acima --. No entanto, \emph{podemos decompor o vetor aceleração em duas componentes} cujos efeitos correspondem aos discutidos acima, isto é, mudar a direção e o módulo da velocidade. 

% adaptado de https://tex.stackexchange.com/questions/25928/how-to-draw-tangent-line-of-an-arbitrary-point-on-a-path-in-tikz
\begin{marginfigure}
\centering
\begin{tikzpicture}[>=Stealth,scale=1.4,
    tangent/.style={
        decoration={
            markings,% switch on markings
            mark=
                at position #1
                with
                {
                    \coordinate (tangent point-\pgfkeysvalueof{/pgf/decoration/mark info/sequence number}) at (0pt,0pt);
                    \coordinate (tangent unit vector-\pgfkeysvalueof{/pgf/decoration/mark info/sequence number}) at (1,0pt);
                    \coordinate (tangent orthogonal unit vector-\pgfkeysvalueof{/pgf/decoration/mark info/sequence number}) at (0pt,1);
                }
        },
        postaction=decorate
    },
    use tangent/.style={
        shift=(tangent point-#1),
        x=(tangent unit vector-#1),
        y=(tangent orthogonal unit vector-#1)
    },
    use tangent/.default=1
]

%%%

    \path[name path=traj, draw, dotted, tangent=0.30] (0,2) .. controls (1,0.3) and (2,2.2) .. (3,1.5) coordinate [pos=0.24] (point);
    \draw [dashed, thin, use tangent] (-1.5,0) -- (1.5,0);
    \draw [->, use tangent] (0,0) -- node[below]{$a_t$}(0.5,0);
    \draw [->, use tangent] (0,0) -- node[above right]{$a_c$}(0,0.5);
    \draw [use tangent, thin, dashed] (0,-1) -- (0,1);
    \draw[thin, use tangent] ([shift={(0,0.75)}]210:0.75) arc[radius=0.75, start angle=210, end angle= 330];
    \draw[use tangent, fill] (0,0.75) node[right]{$C$} circle (1pt);
   
\end{tikzpicture}
\caption{Podemos aproximar uma região qualquer de uma curva por um círculo. Assim, é possível se decompor o vetor aceleração em uma componente \emph{centrípeta} (que aponto para o centro $C$ do círculo), cujo papel é o de alterar a direção, e uma componente tangencial (que aponta tangencialmente à trajetória, na mesma direção da velocidade instantânea), cujo papel é o de alterar o módulo da velocidade.\label{Fig:Decomp_acel_tan_centrip}}
\end{marginfigure}

Qualquer ponto de uma trajetória curvilínea pode ser aproximado por uma trajetória circular, com um centro e um raio bem definido\footnote{Até mesmo uma reta pode ser interpretada dessa maneira, nesse caso temos um círculo de raio infinito!}. Levando isso em conta, de forma geral, podemos assumir que o vetor aceleração não aponta para o centro da trajetória, porém podemos o decompor em duas partes:
\begin{description}
  \item[Componente radial] Uma das componentes é a projeção do vetor aceleração na direção do eixo radial que liga a partícula ao centro da trajetória circular. Temos então uma componente radial $a_r$ cujo módulo é dado pela aceleração centrípeta:
  \begin{equation}
    a_r = a_c = \frac{v^2}{r}.
  \end{equation}
  Podemos atribuir a esta componente o papel exclusivo de alterar a \emph{direção} do vetor velocidade. Podemos entender isso se considerarmos que se só existisse essa componente, então teríamos um movimento circular, onde somente a direção do vetor velocidade é alterada.
  \item[Componente tangencial] A outra componente da aceleração é a projeção do vetor na direção tangencial à trajetória -- ou seja, na direção perpendicular ao eixo radial --. Esta componente tem o papel exclusivo de alterar o \emph{módulo} da velocidade. Isso pode ser entendido se imaginarmos que houvesse somente essa componente da aceleração. Nesse caso, a aceleração seria na própria direção da velocidade, portanto ao calcularmos a velocidade final após um intervalo $\delta t$,
  \begin{equation}
    \vec{v}_f = \vec{v}_i + \vec{a} \delta t,
  \end{equation}
%
percebemos que estamos somando dois vetores colineares. Logo, o resultado $\vec{v}_f$ de tal soma se mantém na mesma direção que os termos do lado direito na equação acima.
\end{description}

\begin{marginfigure}
\centering

\begin{tikzpicture}[>=Stealth]
    \draw[->] (0,0) -- node[above]{$\vec{v}_0$} +(1,0);
    \draw[->] (1,0) -- node[above]{$\vec{a} \delta t$} +(0.5,0);
    \draw[->] (0,-0.3) -- node[below]{$\vec{v}_f$} +(1.5,0);
\end{tikzpicture}
\caption{Caso o vetor aceleração seja colinear à velocidade instantânea, o vetor velocidade final também o é. Isto é, a aceleração só poderá mudar o módulo da velocidade. Esse raciocínio também vale quando estamos tratando o eixo tangencial à trajetória, que é justamente o eixo da velocidade instantânea; nesse caso utilizamos a aceleração tangencial para determinar a variação da velocidade (em módulo).}
\end{marginfigure}

Caso conheçamos ambas as componentes da aceleração -- o que é bastante comum, uma vez que é mais fácil determinar o módulo da velocidade e eventuais alterações desse valor, o que nos dará os valores de $a_c$ e de $a_t$ -- podemos determinar o vetor $\vec{a}$ utilizando as relações
\begin{align}
  \mod{\vec{a}} &= \sqrt{a_t^2 + a_c^2} \\
  \theta &= \arctan \frac{a_c}{a_t},
\end{align}
%
onde $\theta$ é o ângulo que o vetor $\vec{a}$ faz com a direção tangencial. Essas relações são oriundas das propriedades vetoriais discutidas no capítulo anterior. Da mesma forma, podemos conhecer as componentes a partir do vetor $\vec{a}$ através de
\begin{align}
  a_t &= a\cos\theta \\
  a_c &= a\sen\theta.
\end{align}

% adaptado de https://tex.stackexchange.com/questions/25928/how-to-draw-tangent-line-of-an-arbitrary-point-on-a-path-in-tikz
\begin{marginfigure}
\centering
\begin{tikzpicture}[>=Stealth, scale=1.4,
    tangent/.style={
        decoration={
            markings,% switch on markings
            mark=
                at position #1
                with
                {
                    \coordinate (tangent point-\pgfkeysvalueof{/pgf/decoration/mark info/sequence number}) at (0pt,0pt);
                    \coordinate (tangent unit vector-\pgfkeysvalueof{/pgf/decoration/mark info/sequence number}) at (1,0pt);
                    \coordinate (tangent orthogonal unit vector-\pgfkeysvalueof{/pgf/decoration/mark info/sequence number}) at (0pt,1);
                }
        },
        postaction=decorate
    },
    use tangent/.style={
        shift=(tangent point-#1),
        x=(tangent unit vector-#1),
        y=(tangent orthogonal unit vector-#1)
    },
    use tangent/.default=1
]

%%%

    \path[name path=traj, draw, dotted, tangent=0.30] (0,2) .. controls (1,0.3) and (2,2.2) .. (3,1.5) coordinate [pos=0.24] (point);
    \draw [dashed, thin, use tangent] (-1.5,0) -- (1.5,0);
    \draw [->, use tangent] (0,0) -- node[below]{$a_t$}(0.5,0);
    \draw [->, use tangent] (0,0) -- node[above left]{$a_c$}(0,0.5);
    \draw [use tangent, thin, dashed] (0,-1) -- (0,1);
    \draw[thin, use tangent, gray] ([shift={(0,0.75)}]210:0.75) arc[radius=0.75, start angle=210, end angle= 330];
    \draw[use tangent, fill] (0,0.75) node[right]{$C$} circle (1pt);
    \draw[use tangent, ->] (0,0) -- node[above]{$\vec{a}$}(0.5,0.5);
    \draw[thin, use tangent] ([shift={(0,0)}]45:0.3) arc[radius=0.3, start angle=45, end angle= 0] node[above right]{$\theta$};
   
    
\end{tikzpicture}
\caption{Se conhecermos as componentes, podemos determinar o vetor aceleração através das propriedades dos vetores.\label{Fig:Decomp_acel_tan_centrip_mod_vet_acel}}
\end{marginfigure}

Através das componentes radial e tangencial, se -- por exemplo -- necessitarmos descrever o movimento de um carro que ganha velocidade com aceleração tangencial constante em uma pista circular, podemos utilizar as equações da cinemática para calcular a evolução temporal da velocidade. Assim, temos que
\begin{align}
  v_f = v_i + a_t t \\
  s_f = s_i + v_i t + a_t t^2/2,
\end{align}
%
dentre outras relações. Veja que nesse caso utilizamos somente a aceleração tangencial $a_t$, sem nos preocupar com a componente radial. Além disso, utilizamos $s$ ao invés de $x$ para lembrar que estamos descrevendo a distância percorrida, não um deslocamento. Nesse movimento, a velocidade não é constante e, portanto, temos que a componente radial $a_r$ da aceleração aumenta ou diminui à medida que o tempo passa.

%%%%%%%%%%%%%%%%%%%%%%%%%%%%
\section{Movimento Relativo}
%%%%%%%%%%%%%%%%%%%%%%%%%%%%

Geralmente escolhemos o referencial para descrever um fenômeno como sendo fixo no solo. Assim, quando um ônibus passa, a velocidade que atribuímos a seus passageiros é a mesma do próprio ônibus, caso eles estejam sentados. Se fixarmos o referencial no piso do ônibus, veremos que a velocidade dos passageiros é nula. Portanto, a velocidade que medimos depende do referencial adotado. Isso se deve ao fato de que a própria posição depende do referencial.

\begin{figure}[!h]
\centering
\begin{tikzpicture}[>=Stealth]

    \coordinate (OS) at (0,0) node[below left]{$S$};
    
    \draw[->] (OS) -- +(4,0) node[below left]{$x$};
    \draw[->] (OS) -- +(0,3) node[below left]{$y$};
    
    \coordinate (OSP) at (2,1.1);
    
    \draw[->] (OSP) node[below]{$S'$} -- +(4,0) node[below left]{$x'$};
    \draw[->] (OSP) -- +(0,3) node[below left]{$y'$};
    
    \coordinate (P) at (3,2.5);
    
    \draw[fill] (P) circle (1pt) node[above right]{$P$};

    \draw[->] (OS) -- node[above]{$\vec{r}_S$} (P);
    \draw[->] (OSP) -- node[right]{$\vec{r}_{S'}$} (P);
    \draw[->] (OS) -- node[below right]{$\vec{r}_{S'S}$}(OSP);
    
    \draw[->] (OSP) ++(0,2.5) -- node[above]{$\vec{v}_{S'S}$} +(1,0);
\end{tikzpicture}
\caption{Posião de uma partícula $P$ em dois referenciais diferentes. \label{Fig:Ref_mov_relativo}}
\end{figure}

Vamos analisar a situação mostrada na Figura~\ref{Fig:Ref_mov_relativo}. A posição da partícula $P$ no referencial $S'$ é dada pelo vetor $\vec{r}_{S'}$. Se estivermos interessados em calcular a posição desta partícula em um referencial $S$, sendo que a posição do referencial $S'$ é dada por $\vec{r}_{S'S}$ em relação a $S$, temos
\begin{equation}
  \vec{r}_S = \vec{r}_S' + \vec{r}_{S'S},\mathnote{Transformação galileana de posição.}
\end{equation}
%
ou seja, temos uma simples adição de dois vetores, como pode ser visto na própria figura. Esta transformação, juntamente com as equivalentes para velocidade e aceleração, são denominadas \emph{transformações galileanas}. Dentro de Mecânica Clássica essas são as transformações que devem ser utilizadas, já para o caso da mecânica relativística, devemos utilizar as \emph{transformações de Lorentz}, que levam em conta o fato de que a velocidade da luz tem um valor absoluto (igual em todos os referenciais) e que não pode ser ultrapassado.

De maneira semelhante, se a partícula tem velocidade $v_{S'}$ em relação ao referencial $S'$ e temos que esse referencial se move com velocidade $\vec{v}_{S'S}$ em relação a $S$, temos
\begin{equation}
  \vec{v}_S = \vec{v}_{S'} + \vec{v}_{S'S}.
\end{equation}
Para obter esta relação, basta utilizarmos a definição da velocidade instantânea $\vec{v} = \lim_{\Delta t \to 0} \Delta \vec{r} / \Delta t$, escrevendo
\begin{equation}
  \vec{v}_S = \lim_{\Delta t \to 0} \frac{\Delta \vec{r}_s}{\Delta t}.
\end{equation}
%
Através da relação para a posição, podemos calcular o deslocamento no referencial $S$ como
\begin{align}
  \Delta \vec{r}_S &= \vec{r}_S^f - \vec{r}_S^i \\
  &= (\vec{r}_{S'}^f + \vec{r}_{S'S}^f) - (\vec{r}_{S'}^i + \vec{r}_{S'S}^i) \\
  &= (\vec{r}_{S'}^f - \vec{r}_{S'}^i) + (\vec{r}_{S'S}^f - \vec{r}_{S'S}^i) \\
  &= \Delta \vec{r}_{S'} + \Delta \vec{r}_{S'S}.
\end{align}
%
Substituindo esse resultado na equação anterior para a velocidade, temos
\begin{equation}
  \vec{v}_S = \lim_{\Delta t \to 0} \frac{\Delta \vec{r}_{S'} + \Delta \vec{r}_{S'S}}{\Delta t}.
\end{equation}
%
Separando os termos do numerador em duas frações e sabendo que $\lim_{\epsilon \to \xi} f(\epsilon) + g(\epsilon) = \lim_{\epsilon \to \xi} f(\epsilon) + \lim_{\epsilon \to \xi} g(\epsilon)$, temos
\begin{equation}
  \vec{v}_S = \lim_{\Delta t \to 0} \frac{\Delta\vec{r}_{S'}}{\Delta t} + \lim_{\Delta t \to 0} \frac{\Delta \vec{r}_{S'S}}{\Delta t}.
\end{equation}
%
Os limites acima definem as velocidades $\vec{v}_{S'}$ e $\vec{v}_{S'S}$, logo,
\begin{equation}
  \vec{v}_S = \vec{v}_{S'} + \vec{v}_{S'S}. \mathnote{Transformação galileana de velocidade}
\end{equation}

Podemos ainda calcular as transformações para a aceleração através de um cálculo análogo ao utilizado para o caso da velocidade, obtendo
\begin{equation}
  \vec{a}_S = \vec{a}_{S'} + \vec{a}_{S'S}.  \mathnote{Transformação galileana de aceleração}
\end{equation}
%
Temos um interesse particular nessa equação devido ao conceito de \emph{referencial inercial}. Veremos adiante que as Leis de Newton só têm validade dentro de um referencial inercial, que é um referencial que não está submetido a acelerações. Dessa forma, se -- por exemplo -- o referencial $S$ for um referencial inercial, o referencial $S'$ só será inercial se a aceleração $\vec{a}_{S'S}$ for igual a zero. Um referencial não inercial pode ser identificado quando surgem forças ``inexplicáveis''. Um exemplo disso é quando estamos em carro que acelera para a frente e verificamos que um objeto suspenso se desloca para trás, como se tivesse sido puxado através de um fio invisível. Esta força não tem origem dentro do referencial do carro, só podendo ser explicada quando levamos em conta o fato de que o referencial sofre uma aceleração em relação ao solo. Mesmo no caso do solo, como temos uma rotação da Terra em torno do próprio eixo, não temos um referencial verdadeiramente inercial. Este impasse pode ser resolvido através da Primeira Lei de Newton, que veremos no próximo capítulo.

%%%%%%%%%%%%%%%%%%%%%%
\section{Questionário}
%%%%%%%%%%%%%%%%%%%%%%

\begin{question}[type={exam}]
Uma questão de cinemática.
\end{question}
