%%%%%%%%%%%%%%%%%%%%%%%%%%%%%%%%%%%%%%%%%
\chapter{Movimentos bi e tridimensionais}
\label{Chap:MovimentoBidimensional}
%%%%%%%%%%%%%%%%%%%%%%%%%%%%%%%%%%%%%%%%%
%\minitoc

%\clearpage

\begin{fullwidth}
{\it
Neste capítulo vamos redefinir as variáveis cinemáticas em termos de vetores, utilizando as propriedades descritas no Capítulo~\ref{Chap:Vetores}. Obteremos assim relações vetoriais entre as variáveis cinemáticas que nos darão uma descrição completa do movimento em três dimensões. Para simplificar a interpretação dos movimentos, nos valeremos do fato de que as equações da cinemática podem ser escritas como um conjunto de três equações ---~uma para cada eixo~--- sendo que em diversos movimentos um ou dois eixos apresentarão equações triviais.
}
\end{fullwidth}

%%%%%%%%%%%%%%%%%%%%
\section{Introdução}
%%%%%%%%%%%%%%%%%%%%

Para que possamos descrever o movimento da maneira mais geral possível, devemos considerar um espaço tridimensional: ao ---~por exemplo~--- descrever o movimento de um veículo em uma estrada, se tomarmos um eixo de referência ao longo de um segmento da pista, temos ainda possíveis movimentos laterais devidos a curvas e verticais devidos a subidas ou descidas.

Apesar de a descrição completa do movimento exigir três dimensões, é comum que possamos tratar o movimento em duas, ou mesmo uma dimensão. Isso se deve ao fato de que ao dividirmos as equações nos eixos do sistema de referência, poderemos ignorar uma ou duas dessas equações simplesmente por não haver movimento no eixo a que elas correspondem. Para o caso do movimento em linha reta, por exemplo, ao alinharmos um dos eixos do sistema de referência tridimensional ao longo do movimento, temos uma situação em que não há movimento nos outros dois eixos. Isso corresponde ao que denominamos como \emph{movimento unidimensional} no Capítulo~\ref{Chap:MovimentoUnidimensional}, ou seja, o movimento unidimensional é só um caso especial do movimento tridimensional. No caso do movimento bidimensional temos algo semelhante, porém só conseguimos eliminar um dois eixos.

Verificaremos adiante como fazer uma descrição do movimento em três dimensões e como simplificar o tratamento ao eliminar um ou dois eixos. Verificaremos também que podemos separar o movimento em cada eixo e tratá-los de maneira independente mesmo quando há movimento em mais que um deles. Isso facilitará a interpretação do movimento. 

%%%%%%%%%%%%%%%%%%%%%%%%%%%%%%%%%%%%%%%%
\section{Vetores posição e deslocamento}
%%%%%%%%%%%%%%%%%%%%%%%%%%%%%%%%%%%%%%%%

Quando tratamos do movimento unidimensional, utilizamos a distância até a origem (isto é, um ponto de referência) para descrever o movimento através da posição, velocidade e aceleração. Verificamos também que essas grandezas variavam no tempo e pudemos as definir como funções do tempo.

%%%%%%%%%%%%%%%%%%%%
\subsection{Posição}
%%%%%%%%%%%%%%%%%%%%

\begin{marginfigure}
\centering
\begin{tikzpicture}[>=Stealth, scale=2]

    \draw[->] (0,0,0) -- (1.5,0,0) node[below left]{$x$};
    \draw[->] (0,0,0) -- (0,1.5,0) node[below left]{$y$};
    \draw[->] (0,0,0) -- (0,0,1.5) node[below]{$z$};

    \draw[->, thick] (0,0,0) -- node[above]{$\vec{r}$}(1,0.75,0.5);
    \draw[dotted] (1,0,0.5) -- (1,0.75,0.5);
    \draw[dotted] (1,0,0.5) -- (1,0,0);
    \draw[dotted] (1,0,0.5) -- (0,0,0.5);
    \draw[dotted] (1,0.75,0.5) -- (0,0.75,0.5) -- (0,0.75,0) -- (1,0.75,0) -- cycle;
    \draw[dotted] (1,0.75,0) -- (1,0,0);
    \draw[dotted] (0,0.75,0.5) --(0,0,0.5);
    
\end{tikzpicture}
\caption{Um vetor posição em três dimensões.}
\end{marginfigure}

Utilizando vetores, podemos fazer o mesmo para um movimento bidimensional ou tridimensional. Vamos escolher um ponto como origem de um sistema de coordenadas e descrever a posição por um vetor que parte da origem e termina no ponto onde o objeto se encontra. Como escolhemos a origem coincidindo com o início do vetor, a extremidade está no ponto $(x, y, z)$ do sistema de coordenadas. Além disso, o tamanho das componentes é igual aos valores de $x$, $y$, e $z$, logo, temos que o vetor posição será dado por
\begin{equation}
  \vec{r} = x \versi + y \versj + z \versk.
\end{equation}
%
O conjunto de posições ocupadas por um corpo ao longo do tempo é representado pela evolução temporal do vetor posição $\vec{r}$. Como a posição pode variar no tempo, temos que tal vetor é uma função do tempo $\vec{r}(t)$:
\begin{figure}
\centering
\begin{tikzpicture}
\draw (0,0.5) node[above] {$t$};
\draw (3.1,0.5) node[above] {$\vec{r}$};
\draw (0,-1) ellipse [x radius=12pt, y radius=40pt];
\draw (3.1,-1) ellipse [x radius=12pt, y radius=40pt];
\node [circle,draw,fill,scale=0.3] (A){};
\node [circle,draw,fill,scale=0.3] (B) [right=3cm of A] {};
\node [circle,draw,fill,scale=0.3] (C) [below=of A] {};
\node [circle,draw,fill,scale=0.3] (D) [right=3cm of C] {};
\node [circle,draw,fill,scale=0.3] (E) [below=of C] {};
\node [circle,draw,fill,scale=0.3] (F) [right=3cm of E] {};
\draw [thick, arrows={ - Stealth}]
(A) edge [bend left=45] node[above]{$\vec{r} = \vec{r}(t)$}(B)
(C) edge [bend left=45] (D)
(E) edge [bend left=45] (F);
\end{tikzpicture}
\caption{A cada valor de tempo $t$ temos um vetor posição $\vec{r}$ associado. A função $\vec{r}(t)$ descreve a relação entre essas duas variáveis.}
\end{figure}

% Adaptado de http://pgfplots.net/tikz/examples/spiral-cone/
\def\Point{36.9}
\begin{figure}\forceversofloat
\centering
\begin{tikzpicture}[>=Stealth]
  \begin{axis}[
    view       = {-25}{-25},
    axis lines = center,
    zmax       = 60,
    height     = 8cm,
    xtick      = \empty,
    ytick      = \empty,
    ztick      = \empty
 ]
  \addplot3+ [
    ytick      = \empty,
    yticklabel = \empty,
    domain     = 8*pi:12*pi,
    samples    = 400,
    samples y  = 0,
    mark       = none,
    thick,
    black
  ]
  ( {x*sin(0.28*pi*deg(x))},{x*cos(0.28*pi*deg(x)},{x}) coordinate [pos=0.18] (A)
            coordinate [pos=0.28] (B) coordinate [pos=0.35] (C);
  \addplot3+ [
    mark options = {color=black},
    mark         = none
  ] 
  coordinates {(0,0,0)};
%  \addplot3+ [
%    domain    = 0:12*pi,
%    samples   = 100,
%    samples y = 0,
%    mark      = none,
%    dashed,
%  ]  
%  ( {\Point*sin(0.28*pi*deg(\Point))}, {\Point*cos(0.28*pi*deg(\Point)}, {x} );
%  \addplot3[
%    mark=none,
%    dashed
%  ]
%  coordinates {(0,0,0) ({\Point*sin(0.28*pi*deg(\Point))},
%    {\Point*cos(0.28*pi*deg(\Point)}, {0})};

%  \node (P1) at (axis cs:20,0,30) {$P$};
%  \node (origin) at (0,0,0){};
  
%  \draw[->, thick] (axis cs:0,0,0) -- (axis cs:20,0,30);
  \draw[->, thick] (axis cs:0,0,0) -- node[left]{$\vec{r}(t_1)$} (A);
  \draw[->, thick] (axis cs:0,0,0) -- node[above left]{$\vec{r}(t_2)$} (B);
  \draw[->, thick] (axis cs:0,0,0) -- node[below right]{$\vec{r}(t_3)$} (C);
   
  \end{axis}
    
\end{tikzpicture}
\caption{A trajetória de um corpo pode ser descrita através do conjunto de posições $\vec{r}(t)$ ocupadas nos diferentes valores de tempo $t$. Na figura, destacamos três posições correspondentes a três valores diferentes de tempo.}
\end{figure}

%%%%%%%%%%%%%%%%%%%%%%%%%
\subsection{Deslocamento}
%%%%%%%%%%%%%%%%%%%%%%%%%

\begin{marginfigure}
\centering
\begin{tikzpicture}[>=Stealth]
    \path[name path=traj, draw] (0,2) .. controls (1,0.5) and (2,2.5) .. (3,1.5);
    
    \coordinate (a) at (0.1,0);
    \coordinate (b) at (1.2,0);
    
    \path[name path=verta] (a)--+(0,3);
    \path[name path=vertb] (b)--+(0,3);
       
    \path[draw, ->, name intersections={of=traj and verta}](0,0) -- node[left]{$\vec{r}_i$} (intersection-1) coordinate (intersec1);
    \path[draw, ->, name intersections={of=traj and vertb}](0,0) -- node[right]{$\vec{r}_f$}(intersection-1) coordinate (intersec2);
    
    \draw[->] (intersec1) -- node[above]{$\Delta \vec{r}$} (intersec2);
    
    \draw[fill] (0,0) node[below left]{$O$} circle (1pt);
    
\end{tikzpicture}
\caption{O vetor deslocamento $\Delta \vec{r}$ pode ser calculado a partir da diferença entre os vetores $\vec{r}_f$ e $\vec{r}_i$.\label{Fig:DeslocamentoInterpGeometrica}}
\end{marginfigure}

Geometricamente, o vetor deslocamento é aquele que liga o ponto inicial ao ponto final, em linha reta (Figura~\ref{Fig:DeslocamentoInterpGeometrica}). Podemos interpretar tal vetor através da diferença entre dois vetores $\vec{r}_i$ e $\vec{r}_f$ que denotam a posição nos instantes inicial e final. Assim:
\begin{equation}
  \Delta\vec{r} = \vec{r}_f - \vec{r}_i,
\end{equation}
%
Utilizando a notação de versores para escrever os vetores em termos dos eixos de referência, temos para $\vec{r}_i$ e $\vec{r}_f$
\begin{align}
  \vec{r}_i &= x_i \versi + y_i \versj + z_i \versk \\
  \vec{r}_f &= x_f \versi + y_f \versj + z_f \versk.
\end{align}
%
Consequentemente, o vetor deslocamento pode ser escrito como
\begin{align}
  \Delta\vec{r} &= (x_f \versi + y_f \versj + z_f \versk) - (x_i \versi + y_i \versj + z_i \versk) \\
  &= (x_f - x_i)\versi + (y_f - y_i)\versj + (z_f - z_i)\versk \\
  &= \Delta x\versi + \Delta y\versj + \Delta z\versk.
\end{align}
%
Vemos que é possível separar o movimento descrito pelo vetor posição em três componentes distintas, uma para cada eixo coordenado\footnote{Novamente, situações que envolvem o produto vetorial não podem ser separados em eixos distintos. Verificaremos situações desse tipo no Capítulo~\ref{Chap:MomentoAngular}.}. Isso facilita a análise do movimento, permitindo que tratemos cada uma das componentes de acordo com suas particularidades. Em muitos casos não precisamos nem mesmo tratar todas as três componentes, devido ao fato de que o movimento ocorre em um plano.

\begin{marginfigure}
\centering
\begin{tikzpicture}[>=Stealth]
    \draw[->] (0,0) -- (3.3,0) node[below left]{$x$};
    \draw[->] (0,0) -- (0,3.5) node[below left]{$y$};
    
    \draw[->] (0,0) -- node[below]{$\vec{r}_i$} (1.5,1);
    \draw[arrows = {-Stealth[harpoon, swap]}] (1.5,1) -- node[right]{$\Delta \vec{r}$} (3,3);
    \draw[arrows = {-Stealth[harpoon]}] (0,0) -- node[above]{$\vec{r}_f$} (3,3);
    
    \draw[|-] (0,-0.3) -- node[below]{$x_i\equiv r_i^x$} (1.5,-0.3);
    \draw[|-|] (1.5,-0.3) -- node[below]{$\Delta x \equiv \Delta r_x$} (3,-0.3);
    \draw[|-|] (0, -0.9) -- node[below]{$x_f \equiv r_x^f$} (3,-0.9);
    
    \draw[|-] (-0.3,0) -- node[above, sloped]{$y_i \equiv r_y^i$} (-0.3, 1);
    \draw[|-|] (-0.3,1) -- node[above, sloped]{$\Delta y \equiv \Delta r_y$} (-0.3,3);
    \draw[|-|] (-0.9, 0) -- node[above, sloped]{$y_f \equiv r_y^f$}(-0.9,3);
    
    \draw[densely dotted] (0,1) -- (1.5,1) -- (1.5,0);
    \draw[densely dotted] (3,0) -- (3,3) -- (0,3);
    
\end{tikzpicture}
\caption{Com o auxílio de um sistema de referência ortogonal, verificamos os deslocamentos independentes em cada um dos eixos.}
\end{marginfigure}

%%%%%%%%%%%%%%%%%%%%
\section{Velocidade}
%%%%%%%%%%%%%%%%%%%%

%%%%%%%%%%%%%%%%%%%%%%%%%%%%%
\subsection{Velocidade média}
%%%%%%%%%%%%%%%%%%%%%%%%%%%%%

Em três dimensões, caso um corpo sofra um deslocamento, ele o faz com uma velocidade média dada por
\begin{equation}\label{Eq:VelMediaVetorial}
  \vec{\mean{v}} = \frac{\Delta\vec{r}}{\Delta t},
\end{equation}
%
onde estendemos a definição de velocidade ao caso bi e tridimensional ao substituir o deslocamento ao longo de um eixo retilíneo $x$ pelo vetor deslocamento $\Delta \vec{r}$. Como, nesse caso, temos a divisão de um vetor por um escalar, \emph{a direção do vetor velocidade média é a mesma do deslocamento}. O módulo, no entanto, é diferente, assim como a dimensão: temos que $[v] = L/T$ e o módulo é dado pelo valor numérico obtido pela divisão do módulo do vetor deslocamento pelo valor do intervalo de tempo em que o movimento ocorre. Decompondo o vetor, temos
\begin{align}
  \vec{\mean{v}} &= \frac{\Delta x \versi + \Delta y\versj + \Delta z\versk}{\Delta t} \\
  &= \frac{\Delta x}{\Delta t} \versi + \frac{\Delta y\versj}{\Delta t} + \frac{\Delta z\versk}{\Delta t} \\
  &= \mean{v}_x \versi + \mean{v}_y \versj + \mean{v}_z \versk.
\end{align}

Como definimos a velocidade média em termos do vetor deslocamento, podemos escrever
\begin{equation}
	\Delta \vec{r} = \mean{\vec{v}} \Delta t.
\end{equation}
%
No caso especial de termos uma velocidade constante em módulo e sentido, o vetor velocidade média é igual ao vetor velocidade instantânea, logo, podemos escrever
\begin{equation}
	\vec{r}_f = \vec{r}_i + \vec{v} \Delta t.
\end{equation}
%
De acordo com a equação acima, as posições $\vec{r}_f$ ocupadas pela partícula estão todas ao logo da direção determinada pelo vetor $\vec{v}$.

%%%%%%%%%%%%%%%%%%%%%%%%%%%%%%%%%%%
\subsection{Velocidade instantânea}
%%%%%%%%%%%%%%%%%%%%%%%%%%%%%%%%%%%

Podemos definir a velocidade instantânea a partir da Equação~\ref{Eq:VelMediaVetorial}, bastando tomar o limite $\Delta t \to 0$:
\begin{align}
    \vec{v} &= \lim_{\Delta t \to 0} \mean{\vec{v}} \\
    &= \lim_{\Delta t \to 0} \frac{\Delta\vec{r}}{\Delta t} \\
    &= \lim_{\Delta t \to 0} \left(\frac{\Delta x}{\Delta t}\versi + \frac{\Delta y}{\Delta t}\versj + \frac{\Delta z}{\Delta t}\versk\right).
\end{align}
%
Podemos utilizar a propriedade de que o limite da soma é a soma dos limites para escrever
\begin{align}
  \vec{v} &= \lim_{\Delta t \to 0} \frac{\Delta x}{\Delta t}\versi + \lim_{\Delta t \to 0} \frac{\Delta y}{\Delta t}\versj + \lim_{\Delta t \to 0} \frac{\Delta z}{\Delta t}\versk \\
  &= v_x \versi + v_y \versj + v_z \versk.
\end{align}
%
Portanto, podemos simplesmente definir o vetor velocidade através da velocidade nos eixos $x$, $y$ e $z$.

Podemos determinar a direção da velocidade instantânea se analisarmos a trajetória de uma partícula em um plano $xy$, mostrada na Figura~\ref{Fig:Dir_vel}. Verificamos que para $\Delta t \to 0$, o vetor ``deslocamento instantâneo'' $\delta\vec{r}$, dado por\footnote[][-2cm]{Utilizaremos a notação $\delta \xi$ para todas as variáveis do tipo $\Delta \xi$ quando tomamos o limite $\Delta t \to 0$. Podemos interpretar isso como \emph{uma variação infinitamente pequena}.}
\begin{equation}
    \delta\vec{r} = \lim_{\Delta t \to 0}\Delta \vec{r},
\end{equation}
%
é tangente à trajetória. Como a direção da velocidade é a mesma de $\delta\vec{r}$, temos que o vetor velocidade instantânea é tangente à trajetória.

% adaptado de https://tex.stackexchange.com/questions/25928/how-to-draw-tangent-line-of-an-arbitrary-point-on-a-path-in-tikz
\begin{marginfigure}[-1cm]
\centering
\begin{tikzpicture}[>=Stealth,
    tangent/.style={
        decoration={
            markings,% switch on markings
            mark=
                at position #1
                with
                {
                    \coordinate (tangent point-\pgfkeysvalueof{/pgf/decoration/mark info/sequence number}) at (0pt,0pt);
                    \coordinate (tangent unit vector-\pgfkeysvalueof{/pgf/decoration/mark info/sequence number}) at (1,0pt);
                    \coordinate (tangent orthogonal unit vector-\pgfkeysvalueof{/pgf/decoration/mark info/sequence number}) at (0pt,1);
                }
        },
        postaction=decorate
    },
    use tangent/.style={
        shift=(tangent point-#1),
        x=(tangent unit vector-#1),
        y=(tangent orthogonal unit vector-#1)
    },
    use tangent/.default=1
]

%%%

    \path[name path=traj, draw, tangent=0.30] (0,2) .. controls (1,0.3) and (2,2.2) .. (3,1.5) coordinate [pos=0.24] (point);
    \draw [dashed, thin, use tangent] (-1.5,0) -- (1.5,0);
    \draw [->, use tangent] (0,0) -- node[below]{$\delta\vec{r}$}(0.5,0);
    \draw [->, use tangent] (0,0) -- node[below right]{$\vec{v}$}(1,0);
    
    \coordinate (a) at (0.1,0);
    \coordinate (b) at (1.9,0);
    
    \path[name path=verta] (a)--+(0,3);
    \path[name path=vertb] (b)--+(0,3);
       
    \path[draw, densely dotted, ->, name intersections={of=traj and verta}](0,0) --  (intersection-1) coordinate (intersec1);
    \path[draw, densely dotted, ->, shorten >=1pt, name intersections={of=traj and vertb}](0,0) -- (intersection-1) coordinate (intersec2);
    
    \draw[->, densely dotted, shorten >=1pt] (intersec1) -- (intersec2);
    
    \coordinate (c) at (0.3,0);
    \coordinate (d) at (1.7,0);
    
    \path[name path=vertc] (c)--+(0,3);
    \path[name path=vertd] (d)--+(0,3);
       
    \path[draw, densely dotted, ->, name intersections={of=traj and vertc}](0,0) --  (intersection-1) coordinate (intersec3);
    \path[draw, densely dotted, ->, shorten >=1pt, name intersections={of=traj and vertd}](0,0) -- (intersection-1) coordinate (intersec4);
    
    \draw[->, densely dotted, shorten >=1pt] (intersec3) -- (intersec4);
    
    
    \draw[fill] (0,0) node[below left]{$O$} circle (1pt);
    \draw[fill, use tangent] (0,0) circle (1pt);
    \draw[->] (0,0) -- node[above]{$\vec{r}$} (point);
    
\end{tikzpicture}
\caption{No limite $\Delta t \to 0$, temos que a direção do vetor deslocamento instantâneo $\delta\vec{r}$ no ponto denotado por $\vec{r}$ é a mesma direção que a da reta que tange a curva no ponto.\label{Fig:Dir_vel}}
\end{marginfigure}

Finalmente, devemos destacar que é comum descrever situações físicas e/ou problemas e exercícios e se referir a um valor particular de velocidade. Nesses casos subentende-se que estamos nos referindo ao \emph{módulo} do vetor velocidade. Em alguns casos, tal valor pode também ser chamado de \emph{velocidade escalar}.

%%%%%%%%%%%%%%%%%%%%%%%%%%%%%%%%%%%%%%%%%%%%%%%%%%%%%%%%%%%%%
\paragraph{Variação do vetor velocidade com módulo constante}
%%%%%%%%%%%%%%%%%%%%%%%%%%%%%%%%%%%%%%%%%%%%%%%%%%%%%%%%%%%%%

Devido ao fato de que agora estamos trabalhando com vetores, temos a possibilidade de que exista uma variação de velocidade ao considerarmos dois vetores velocidade $\vec{v}_i$ e $\vec{v}_f$ cujos módulos são \emph{iguais}. Na Figura~\ref{Fig:DeltaVelocidadeComModulosIguais} temos uma partícula que efetua um movimento circular com velocidade cujo módulo é constante. Consideramos dois intantes inicial e final em particular. Note que os vetores apontam em direções diferentes. Determinando a diferença
\begin{equation}
	\Delta \vec{v} = \vec{v}_f - \vec{v}_i,
\end{equation}
%
podemos ver que ela não é nula, isto é, \emph{mesmo no caso de termos dois vetores com módulos iguais, a diferença entre eles pode ser não nula}.

\begin{marginfigure}
   \begin{tikzpicture}[>=Stealth]
        \draw[dotted] ([shift={(0,0)}]120:2) arc[radius=2, start angle=120, end angle= 0];
        \draw[->] (0,0) -- node[left]{$\vec{r}_i$} (60:2);
        \draw[->] (60:2) -- node[above]{$\vec{v}_i$} +(-30:1);
        \draw[->] (0,0) -- node[below]{$\vec{r}_f$} (30:2);
        \draw[->] (30:2) -- node[right]{$\vec{v}_f$}+(-60:1);
        
        \draw[->] (1, -1) -- node[above]{$\vec{v}_i$} +(-30:1) coordinate (A);
        \draw[->] (1, -1) -- node[left]{$\vec{v}_f$} +(-60:1) coordinate (B);
        \draw[->] (A) -- node[right]{$\Delta \vec{v}$} (B);
        
   \end{tikzpicture}
   \caption{Velocidades em diferentes instantes e a correspondente variação da velocidade $\Delta\vec{v}$ determinada através da diferença entre os vetores. \label{Fig:DeltaVelocidadeComModulosIguais}}
\end{marginfigure}

Esse aspecto é uma característica dos vetores também se manifesta com o vetor deslocamento, mas nesse caso nos parece mais natural, já que os vetores posição e deslocamento têm uma manifestação mais concreta do que o vetor velocidade. Veremos ao estudar o movimento circular que essa propriedade terá consequências interessantes para a aceleração, que definiremos a seguir.

%%%%%%%%%%%%%%%%%%%%%%%%%%%%%%%%%%%%%%%%%%
\paragraph{Discussão: Rotação de um vetor}
\label{Disc:RotVetor}
%%%%%%%%%%%%%%%%%%%%%%%%%%%%%%%%%%%%%%%%%%

No caso de termos uma trajetória que não seja retilínea, podemos interpretar o deslocamento de acordo com deslocamentos infinitesimais na direção da velocidade instantânea:\footnote[][-1cm]{Veja que nessa situação a velocidade certamente não é constante, já que sua direção varia continuamente, porém podemos tomá-la como constante devido ao fato que estamos considerando um tempo infinitesimal.}
\begin{equation}
	\delta \vec{r} = \vec{v} \delta t.
\end{equation}

\begin{marginfigure}
\centering
\begin{tikzpicture}[>=Stealth, scale=1.2]
    \draw[->] (0:2) -- (10:2);
    \draw[->] (10:2) -- (20:2);
    \draw[->] (20:2) -- (30:2);
    \draw[->] (30:2) -- (40:2);
    \draw[->] (40:2) -- (50:2);
    \draw[->] (50:2) -- (60:2);
    \draw[->] (60:2) -- (70:2);
    \draw[->] (70:2) -- (80:2);
    \draw[->] (80:2) -- (90:2);
    
\end{tikzpicture}
\caption{Todo movimento pode ser considerado como uma série de deslocamentos infinitamente pequenos. Na figura mostramos uma série de pequenos deslocamentos que constituem de maneira aproximada uma curva. Se tomarmos deslocamentos menores, a curva passará a ser mais suave. A direção de cada deslocamento muda pois estamos considerando que a direção da velocidade mude devido a uma aceleração.}
\end{marginfigure}

Através desse conceito de deslocamento infinitesimal, podemos descrever um movimento que acontece com uma velocidade sempre perpendicular ao vetor posição. Na Figura~\ref{Fig:VetorPosicaoRotacaoDevidoVelocidadePerpendicular} temos uma partícula cuja posição é descrita por um vetor posição que parte de uma origem fixada no centro de um circulo. A partícula segue a trajetória circular, sendo que a velocidade é sempre tangente à trajetória, ou seja, \emph{a velocidade é sempre perpendicular ao vetor posição}.

\begin{marginfigure}
\centering
\begin{tikzpicture}[>=Stealth]

	\coordinate (origin) at (0,0) node[below left]{$O$};

	\draw[dashdotted] (2,0)  arc[start angle = 0, end angle = 180, radius = 2];
	\draw[fill] (0,0) circle (0.5pt);

	\draw[->] (0,0) -- node[below, sloped]{$\vec{r}$}(60:1.95cm);
	\draw[fill] (60:2cm) circle (0.05cm);

	\draw[->] (60:2cm) coordinate (part) -- +(150:1cm) coordinate (v) node[above, sloped]{$\vec{v}$};

	\pic[draw, "$\cdot$", angle eccentricity = 0.5, angle radius = 3mm]{angle = v--part--origin};

	\draw[->] (0,0) -- node[below, sloped]{$\vec{r}$}(120:1.95cm);
	\draw[fill] (120:2cm) circle (0.05cm);

	\draw[->] (120:2cm) coordinate (part2) -- +(210:1cm) coordinate (v2) node[above, sloped]{$\vec{v}$};

	\pic[draw, "$\cdot$", angle eccentricity = 0.5, angle radius = 3mm]{angle = v2--part2--origin};

	\draw[->] (60:0.5) arc[start angle = 60, end angle = 120, radius = 0.5] node[above, midway]{$\omega$};

\end{tikzpicture}
\caption{O efeito de um vetor velocidade continuamente perpendicular ao vetor posição é o de causar uma rotação neste último. \label{Fig:VetorPosicaoRotacaoDevidoVelocidadePerpendicular}}
\end{marginfigure}

A situação acima descreve uma propriedade importante de um vetor: quando ele ele é somado a uma variação infinitesimal que é perpendicular ao proprio vetor, o vetor resultante têm o mesmo módulo, porém é \emph{rotacionado} na direção da variação infinitesimal.

%%%%%%%%%%%%%%%%%%%%
\section{Aceleração}
%%%%%%%%%%%%%%%%%%%%

%%%%%%%%%%%%%%%%%%%%%%%%%%%%%
\subsection{Aceleração média}
%%%%%%%%%%%%%%%%%%%%%%%%%%%%%

No caso do cálculo da velocidade média, bastou redefinirmos a velocidade em termos do vetor deslocamento $\Delta\vec{r}$ para verificarmos que a velocidade é uma grandeza vetorial. Devido a essa conclusão, temos que uma variação de velocidade pode ser uma variação tanto de módulo, quanto de direção ou sentido. Portanto, precisamos redefinir a aceleração em termos de um vetor $\Delta\vec{v}$:
\begin{align}
  \vec{\mean{a}} &= \frac{\Delta\vec{v}}{\Delta t} \\
  &= \frac{\Delta v_x}{\Delta t}\versi + \frac{\Delta v_y}{\Delta t}\versj + \frac{\Delta v_z}{\Delta t}\versk \\
  &= \mean{a}_x\versi + \mean{a}_y\versj + \mean{a}_z\versk.
\end{align} 
%
A direção do vetor aceleração média é a própria direção do vetor $\Delta{\vec{v}}$. 

%%%%%%%%%%%%%%%%%%%%%%%%%%%%%%%%%%%
\subsection{Aceleração instantânea}
%%%%%%%%%%%%%%%%%%%%%%%%%%%%%%%%%%%

Assim como no caso do cálculo da velocidade instantânea, podemos calcular a aceleração instantânea vetorial através do limite $\Delta t \to 0$:
\begin{align}
  \vec{a} &= \lim_{\Delta t \to 0} \vec{\mean{a}} \\
  &= \lim_{\Delta t \to 0} \frac{\Delta \vec{v}}{\Delta t} \\
  &= \lim_{\Delta t \to 0} \frac{\Delta {v_x}}{\Delta t}\versi + \lim_{\Delta t \to 0} \frac{\Delta {v_y}}{\Delta t}\versj + \lim_{\Delta t \to 0} \frac{\Delta {v_z}}{\Delta t}\versk,
\end{align}
%
onde utilizamos novamente a propriedade de que o limite da soma é a soma dos limites. Obtemos então
\begin{equation}
  \vec{a} = a_x \versi + a_y \versj + a_z \versk,
\end{equation}
%
e observamos que para o caso da aceleração, também temos que as componentes do vetor são dadas pelos valores de -- neste caso -- aceleração dos eixos $x$, $y$ e $z$.

%%%%%%%%%%%%%%%%%%%%%%%%%%%%%%%%%%%%%%%%%%%%%%%%%%%%%%%%%
\paragraph{Aceleração com módulo da velocidade constante}
%%%%%%%%%%%%%%%%%%%%%%%%%%%%%%%%%%%%%%%%%%%%%%%%%%%%%%%%%

Conforme verificamos na seção anterior, existem casos em que a velocidade se mantém constante, mas que mesmo assim possuem uma variação do \emph{vetor} velocidade. Isso se deve ao fato de que temos um $\Delta \vec{v}$ dado por
\begin{equation}
	\Delta \vec{v} = \vec{v}_f - \vec{v}_i,
\end{equation}
%
que é diferente de zero. Segundo a definição de aceleração, portanto, podemos ter acelerações que mantém o módulo da velocidade constante. Na Seção~\eqref{Sec:MovimentoCircular}, analisaremos mais a fundo os dois aspectos da aceleração --~alterar o módulo e/ou a direção da velocidade~--utilizando como exemplo o movimento circular.

%%%%%%%%%%%%%%%%%%%%%%%%%%%%%%%%
\section{Movimento de projéteis}
%%%%%%%%%%%%%%%%%%%%%%%%%%%%%%%%

O movimento de um projétil ao ser lançado com velocidade que faz um ângulo com a horizontal é conhecido como movimento balístico. Neste movimento, podemos usar a propriedade da decomposição da velocidade para analisar o movimento em cada eixo separadamente: no eixo horizontal, temos um movimento com velocidade constante, enquanto no eixo vertical temos um movimento com aceleração constante. Esses dois tipos de movimento correspondem aos dois mais simples que podemos estudar. Dessa forma, o movimento de um projétil serve como um bom exemplo de aplicação da descrição vetorial das variáveis cinemáticas.

\begin{figure}[!htb]\forceversofloat
\centering
\begin{tikzpicture}[>=Stealth, scale = 2,
     interface/.style={
        % superfície
        postaction={draw,decorate,decoration={border,angle=-45,
                    amplitude=0.2cm,segment length=2mm}}},
    ]
    
    \draw[interface] (-0.25,0) -- (4,0) coordinate (hd);
 
    \draw[smooth, dash dot, name path=plotc,samples=1000,domain=0:3.75]
        plot(\x,{\x*tan(45) - 9.8 * \x^2 / (2 * 6.062^2*(cos(45))^2)});
    
    \draw[->, thick] (0,0) coordinate(origin) -- node[above left]{$\vec{v}_i$} (45:1) coordinate (v);
 
    \pic [draw, "$\theta$", angle eccentricity = 1.5] {angle = hd--origin--v};
\end{tikzpicture}
\caption{Lançamento oblíquo de um projétil.}
\end{figure}

Devemos ter em mente, no entanto, que vamos fazer sérias restrições ao modelo, que fazem com que ele não possa ser aplicado de maneira realista para muitas situações: vamos desconsiderar a força de arrasto do ar, vamos considerar que a aceleração gravitacional é uniforme, e vamos desconsiderar a rotação da Terra\footnote{O arrasto faz com que a trajetória perca o aspecto parabólico e também faz com que o ângulo de maior alcance seja maior que \np[\tcdegree]{45}, já que a força de arrasto é menor em camadas superiores da atmosfera. Se o lançamento atingir alturas e distâncias muito grandes, ele tende a descrever uma elípse, já que a aceleração gravitacional diminui com $r^{-2}$. Finalmente, devido à força de Coriolis, característica de referenciais girantes, temos que a ``parábola'' descrita pelo projétil não está contida num plano.} . Todos esses fatores podem ser muito relevantes para lançamentos a longas distâncias. Apesar disso, tais restrições não prejudicam a principal característica na qual estamos interessados: temos um movimento bidimensional que pode ser descrito de maneira simples.

%%%%%%%%%%%%%%%%%%%%%%%%%%%%%%%%%%
\subsection{Sistema de referência}
%%%%%%%%%%%%%%%%%%%%%%%%%%%%%%%%%%

Para descrever o lançamento, vamos considerar que o projétil tem uma velocidade inicial $\vec{v}_i$ cujo módulo $v_i$ e o ângulo em relação à horizontal $\theta$ são conhecidos. Vamos estabelecer um sistema de referência bidimensional para descrever o movimento, tomando um eixo horizontal (eixo $x$) e um vertical (eixo $y$). Essa escolha é a mais adequada pois permite que \emph{haja aceleração somente no eixo vertical $y$}. Isso significa que o movimento no eixo horizontal $x$ é um movimento com \emph{velocidade constante}, enquanto no eixo $y$ temos que\footnote{Note que o sinal negativo reflete o fato de que a aceleração da gravidade aponta no \emph{sentido negativo} do eixo $y$. Se tomássemos um eixo $y$ cujo sentido fosse para baixo, então a aceleração da gravidade seria positiva.}
\begin{equation}
	a = -g,
\end{equation}
%
o que implica que o movimento pode ser descrito nesse eixo pelas expressões para aceleração constante estudadados no Capítulo~\ref{Chap:MovimentoUnidimensional}

\begin{marginfigure}
\centering
\begin{tikzpicture}[>=Stealth, 
     interface/.style={
        % superfície
        postaction={draw,decorate,decoration={border,angle=-45,
                    amplitude=0.2cm,segment length=2mm}}},
    ]
    
    \draw[interface] (-0.25,0) -- (4,0) coordinate (hd);
 
    \draw[smooth, dash dot, name path=plotc,samples=1000,domain=0:3.75]
        plot(\x,{\x*tan(45) - 9.8 * \x^2 / (2 * 6.062^2*(cos(45))^2)});
    
    \draw[->, thick] (0,0) coordinate(origin) -- (45:1) coordinate (v) node[above left]{$\vec{v}_i$};
 
    \pic [draw, "$\theta$", angle eccentricity = 1.5] {angle = hd--origin--v};

	\draw[->, thick] (0,0) -- (0,1.5) node[below left]{$y$};
	\draw[->, thick] (0,0) -- (4.25,0) node[above left]{$x$};

\end{tikzpicture}
\caption{O sistema de referência mais adequado para o estudo do lançamento oblíquo é aquele em que a aceleração aponta na direção de um dois eixos coordenados, ou seja, temos um eixo vertical e um horizontal.}
\end{marginfigure}

Em termos dos eixos $x$ e $y$, o vetor velocidade inicial pode ser escrito como
obtendo
\begin{equation}
  \vec{v}_i = v_{i}^{x} \versi + v_{i}^{y} \versj,
\end{equation}
%
onde
\begin{align}
  v_{i}^{x} &= v_i\cos\theta \\
  v_{i}^{y} &= v_i\sen\theta.
\end{align}

%%%%%%%%%%%%%%%%%%%%%%%%%%%%%%%%%%%%%%%%%%%%%%%%%%%%%%%%
\paragraph{Eixo $x$: Movimento com velocidade constante}
%%%%%%%%%%%%%%%%%%%%%%%%%%%%%%%%%%%%%%%%%%%%%%%%%%%%%%%%

Analisando o movimento no eixo $x$, temos uma velocidade inicial -- dada por $v_{i}^{x} = v_i\cos\theta$ -- e \emph{não temos nenhuma aceleração}. Como podemos analisar o movimento em cada eixo de maneira completamente independente dos demais, concluímos que
\begin{align}
  v_{i}^x &= \text{constante} \\
  x_{f} &= x_{i} + v_{i}^x t.\label{Eq:PosXProj}
\end{align}

%%%%%%%%%%%%%%%%%%%%%%%%%%%%%%%%%%%%%%%%%%%%%%%%%%%%%%%%
\paragraph{Eixo $y$: Movimento com aceleração constante}
%%%%%%%%%%%%%%%%%%%%%%%%%%%%%%%%%%%%%%%%%%%%%%%%%%%%%%%%

Verticalmente, temos um movimento com aceleração constante, dirigida para baixo. Se adotarmos o eixo $y$ crescendo para cima, a partir das Equações~\eqref{Eq:VV0AT} e~\eqref{Eq:XX0V0TAT22}, temos
\begin{align}
  v_{f}^y &= v_{i}^y - a^y t \\
  &= v_{i}^y - g t \\
  y_f &= y_i + v_{i}^y t - \frac{g}{2}t^2. \label{Eq:PosYProj}
\end{align}
%
Além das expressões acima, podemos utilizar as demais que também se aplicam ao movimento com aceleração constante:
\begin{align}
	y_f &= y_i + v_f^y t - \frac{(-g)t^2}{2} \\
	y_f &= y_i + \frac{v_i^y + v_f^y}{2} t \\
	(v_f^y)^2 &= (v_i^y)^2 - 2 g \Delta x.
\end{align}

%%%%%%%%%%%%%%%%%%%%%%%%%%%%%%%%%%%%%%%%%%%%%%%%%%%
\subsection{Equações para o movimento de projéteis}
%%%%%%%%%%%%%%%%%%%%%%%%%%%%%%%%%%%%%%%%%%%%%%%%%%%

Para explorar melhor as características do movimento de projéteis, vamos a seguir determinar algumas expressões que nos permitem calcular alguns resultados interessantes para o lançamento de projéteis. É importante frizar que elas podem ser aplicadas a vários exercícios/problemas, porém elas não são suficientes para esgotar os resultados interessantes que podem ser retirados do movimento balístico. Muitas vezes seremos obrigados a obter resultados parciais para que possamos resovê-los.

%%%%%%%%%%%%%%%%%%%%%%%%%%
\paragraph{Altura máxima}
%%%%%%%%%%%%%%%%%%%%%%%%%%

\begin{marginfigure}
\centering
\begin{tikzpicture}[>=Stealth]
    \draw[->] (0,0) -- (4,0) node[below left]{$x$};
    \draw[->] (0,0) -- (0,2) node[below left]{$y$};
    
    \draw[dashdotted, smooth, samples=1000, domain=0:3.33] plot (\x, {2*\x - 0.6*\x*\x});
    
    \coordinate (O) at (0,0);
    \coordinate (A) at (1,0);
    \coordinate (B) at (63.434948823:1);
       
	\draw[->] (O) -- (63.434948823:1.3) node[above left]{$\vec{v}$};

    \pic [draw, "$\theta$", angle eccentricity=1.5] {angle = A--O--B};
        
    \draw[dotted] (1.666, 1.666) -- +(2,0);
    \draw[<->] (3.5, 0) -- node[right]{$H$} (3.5, 1.666);
    
\end{tikzpicture}
\caption{Altura máxima em relação ao ponto de lançamento.}
\end{marginfigure}

A partir da equação acima, podemos determinar qual é o valor de altura máxima que o projétil alcança ao ser lançado com velocidade $\vec{v}_i$. Sabemos que no ponto onde o projétil atinge a altura máxima, sua velocidade no eixo vertical deve ser nula, afinal ocorre uma inversão no sentido do movimento. Utilizando a equação de Torricelli, obtemos\footnote[][-1cm]{Na dedução das equações para altura máxima, alcance horizontal e para a trajetória, escolheremos um sistema de coordenadas onde o eixo $y$ cresce verticalmente para cima. Nesse caso, ao utilizar as fórmulas é importante que tal escolha também seja efetuada ao se resolver exercícios e problemas, respeitando a convenção que originou as fórmulas. Se isso não acontecer, ocorrerão problemas com os sinais de algumas variáveis cinemáticas.}
\begin{equation}
  (v_{f}^{y})^2 = (v_{i}^{y})^2 - 2 g \Delta y.
\end{equation}
%
Substituindo $v_{f}^{y} = 0$ e $v_{i}^{y} = v_i\sen\theta$, obtemos
\begin{equation}
  v_i\sen\theta = 2g\Delta y,
\end{equation}
%
e, finalmente, denotando a altura máxima por $H$ e sabendo que $H = \Delta y$,
\begin{equation}
  H = \frac{v_i^2\sen^2\theta}{2g}. \mathnote{Altura máxima.}
\end{equation}

%%%%%%%%%%%%%%%%%%%%%%%%%%%%%%%
\paragraph{Alcance horizontal}
%%%%%%%%%%%%%%%%%%%%%%%%%%%%%%%

O alcance horizontal de um projétil pode ser calculado se soubermos qual é o tempo decorrido entre o objeto ser lançado e voltar à mesma posição no eixo $y$ que ocupava no momento do lançamento. Temos então que o deslocamento $\Delta y$ será nulo, logo, à partir da Equação~\eqref{Eq:XX0V0TAT22}, temos
\begin{equation}
  y_f-y_i = v_{i}^{y}t - \frac{g}{2}t^2,
\end{equation}
%
ou, devido à nossa observação de que $\Delta y = 0$
\begin{equation}
  v_{i}^{y}t = \frac{g}{2} t^2.
\end{equation}
%
Esta equação admite a solução $t = 0$, que corresponde ao momento do lançamento (o que não é particularmente útil), ou -- dividindo ambos os membros da equação por $t$ e isolando a variável $t$ restante --
\begin{align}
  t &= \frac{2v_{i}^{y}}{g} \\
  &= 2\frac{v_i\sen\theta}{g}.
\end{align}

Para calcularmos a distância percorrida pelo projétil, basta utilizarmos a Equação~\eqref{Eq:PosXProj}, obtendo
\begin{align}
  R &\equiv \Delta x = v_{i}^{x} t \\
  &= \left(v_i\cos\theta\right) \left(2\frac{v_i\sen\theta}{g}\right) \\
  &= \frac{2v_i^2}{g}\sen\theta\cos\theta.
\end{align}
%
Utilizando a relação trigonométrica $\sen 2\theta = 2 \sen\theta\cos\theta$, podemos reescrever a expressão acima de uma maneira mais amigável:
\begin{equation}\label{Eq:Alcance}
  R = \frac{v_i^2}{g}\sen2\theta. \mathnote{Alcance horizontal.}
\end{equation}

\begin{marginfigure}[-7cm]
\centering
\begin{tikzpicture}[>=Stealth]
    \draw[->] (0,0) -- (4,0) node[below left]{$x$};
    \draw[->] (0,0) -- (0,2) node[below left]{$y$};
    
    \draw[dashdotted, smooth, samples=1000, domain=0:3.33] plot (\x, {2*\x - 0.6*\x*\x});
    
    \coordinate (O) at (0,0);
    \coordinate (A) at (1,0);
    \coordinate (B) at (63.434948823:1);
       
	\draw[->] (O) -- (63.434948823:1.3) node[above left]{$\vec{v}$};

    \pic [draw, "$\theta$", angle eccentricity=1.5] {angle = A--O--B};
        
    \draw[|<->|] (0, -0.3) -- node[below]{$R$} (3.33, -0.3);
    
\end{tikzpicture}
\caption{Alcance em relação ao ponto de lançamento.}
\end{marginfigure}

%%%%%%%%%%%%%%%%%%%%%%%%%%%%%%%%%%%%%%%%%%%
\paragraph{Distância horizontal percorrida}
%%%%%%%%%%%%%%%%%%%%%%%%%%%%%%%%%%%%%%%%%%%

Em muitos casos estamos interessados em determinar a distância horizontal $\Delta x$ percorrida pelo projétil, mas tal distância difere do alcance $R$. Temos da própria definição do alcance que o projétil deve voltar ao mesmo ponto de que foi lançado, em relação ao eixo $y$. Em um cálculo mais geral da distância horizontal percorrida, podemos ter um lançamento que ocorre de cima de uma elevação, atingindo o solo mais abaixo (como no caso de um lançamento que parte de cima de uma mesa e que atinge o chão), ou mesmo um lançamento que parte de um ponto mais baixo e atinge uma elevação.

\begin{marginfigure}[3cm]
\centering
\begin{tikzpicture}[>=Stealth]
    \draw[->] (0,0) -- (4,0) node[below left]{$x$};
    \draw[->] (0,0) -- (0,2) node[below left]{$y$};
    
    \draw[dashdotted, smooth, samples=1000, domain=0:2.75] plot (\x, {2*\x - 0.6*\x*\x});
    
    \coordinate (O) at (0,0);
    \coordinate (A) at (1,0);
    \coordinate (B) at (63.434948823:1);

	\draw[->] (O) -- (63.434948823:1.3) node[above left]{$\vec{v}$};

    \pic [draw, "$\theta$", angle eccentricity=1.5] {angle = A--O--B};

	\draw[pattern = north west lines] (2.26875,0) rectangle +(0.9625,0.9625);
        
	\draw[dotted] (2.75,0.9625) -- (2.75,2.5) (0,2) -- (0,2.5);

    \draw[|<->|] (0, 2.5) -- node[above]{$\Delta x$} (2.75, 2.5);
	\draw[|<->|] (3.4,0) -- node[right]{$h$} +(0,0.9625);
    
\end{tikzpicture}
\caption{Distância horizontal percorrida no caso de um lançamento que parte de um ponto mais baixo e atinge uma elevação de altura $h$.}
\end{marginfigure}

Para determinar o valor de $\Delta x$, devemos proceder como no caso do alcance, determinando primeiramente o tempo através de
\begin{equation}
	y_f = y_i + v_{i}^{y}t - \frac{gt^2}{2},
\end{equation}
%
que, assumindo que $\Delta y \equiv h$, podemos escrever como
\begin{equation}
	- \frac{gt^2}{2} + v_{i}^{y}t - h = 0.
\end{equation}
%
Note que mesmo que a diferença de altura seja conhecida, não temos como determinar uma expressão para o cálculo do tempo. Nesse caso, devemos o determinar utilizando os dados particulares do problema em questão e então obter $\Delta x$ através de
\begin{equation}
	\Delta x = v_{i}^{x} t.
\end{equation}

%%%%%%%%%%%%%%%%%%%%%%%%%%%%%%%%%%%%%%%
\paragraph{Discussão: Alcance máximimo}
%%%%%%%%%%%%%%%%%%%%%%%%%%%%%%%%%%%%%%%

Através da Expresão~\eqref{Eq:Alcance}, podemos verificar que para um mesmo valor de velocidade $v_i$, obtemos distâncias diferentes de acordo com diferentes valores para o ângulo $\theta$ de lançamento. Através da Figura~\ref{Fig:SenoLimitado}, podemos notar que a função seno é limitada, e seu valor máximo é 1, o que corresponde a um ângulo de \np[\tcdegree]{90}. Essa observação nos permite determinar que quando o argumento $2\theta$ que aparece na expressão para o alcance $R$ for igual a \np[\tcdegree]{90}, teremos o alcance máximo. Logo,
\begin{equation}
	2\theta = \np[\tcdegree]{90}
\end{equation}
%
o que implica em
\begin{equation}
	\theta = \np[\tcdegree]{45}
\end{equation}
%
para o ângulo de máximo alcance.

\begin{marginfigure}[-4cm]
\centering
\begin{tikzpicture}[>=Stealth]
    \draw[->] (0,0) -- (4,0);
    \draw[->] (0,-1.5) -- (0,2);
    
    \draw[dashdotted, smooth, samples=1000, domain=0:2.75]
		plot (\x, {sin(4*\x r)});

	\draw[dotted] (0,1) node[left]{1} -- (3,1);
	\draw[dotted] (0,-1) node[left]{-1} -- (3,-1);
    
\end{tikzpicture}
\caption{A função seno é limitada a valores dentro do intervalo $[-1,1]$.\label{Fig:SenoLimitado}}
\end{marginfigure}

%%%%%%%%%%%%%%%%%%%%%%%%%%%%%%%%%%%%%%
\paragraph{Equação para a trajetória}
%%%%%%%%%%%%%%%%%%%%%%%%%%%%%%%%%%%%%%

Podemos determinar a forma da trajetória do projétil a escrevendo como uma função $y(x)$. Para isso, podemos isolar o tempo na Equação~\eqref{Eq:PosXProj}, obtendo
\begin{equation}
  t = \frac{x_f - x_i}{v_{i}^{x}}.
\end{equation}
%
Substituindo essa expressão na Equação~\eqref{Eq:PosYProj}, obtemos
\begin{equation}
  y_f = y_i + v_i\sen\theta \frac{x_f-x_i}{v_i\cos\theta} - \frac{g}{2}\frac{(x_f-x_i)^2}{(v_i\cos\theta)^2},
\end{equation}
%
onde utilizamos $v_{i}^{x} = v_i\cos\theta$ e $v_{i}^{y} = v_i\sen\theta$. Para simplificar a expressão acima, vamos escolher $y_i = x_i = 0$, $y_f = y$ e $x_f = x$. Obtemos assim
\begin{equation}\label{Eq:EqDaTrajetoria}
  y = (\tan\theta) \; x - \left(\frac{g}{2v_i^2\cos^2\theta}\right) x^2. \mathnote{Equação da trajetória.}
\end{equation}

Se compararmos a equação acima a um polinômio de segundo grau, cuja forma característica é a de uma parábola, 
\begin{equation}
  y = A + B x + C x^2,
\end{equation}
%
verificamos que a equação da trajetória segue o mesmo formato, porém com $A = 0$. Concluímos então que a trajetória seguida pelo projétil é tem a forma de uma parábola, com concavidade voltada para baixo\footnote{A concavidade de um polinômio de segundo grau é determinada através do sinal do coeficiente $C$: se o coeficiente é positivo, a concavidade é voltada para cima; se for negativo, é voltada para baixo.}.

Note que podemos utilizar a expressão para a trajetória obtida acima mesmo para lançamentos horizontais --~bantando utilizar $\theta = \np[\tcdegree]{0}$~--, ou lançamentos para baixo. Nesse último caso, devemos utilizar ângulos negativos, medidos em relação ao eixo horizontal.

%%%%%%%%%%%%%%%%%%%%%%%%%%%%%%%%%%%%%%%%%%%%%%%%%%%%%%%%
\paragraph{Exemplo: Lançamento a partir de uma elevação}
%%%%%%%%%%%%%%%%%%%%%%%%%%%%%%%%%%%%%%%%%%%%%%%%%%%%%%%%

\begin{quote}
Uma bola é lançada a partir de uma elevação com altura $h = \np[m]{2,0}$, com uma velocidade $v_i = \np[m/s]{4,0}$, fazendo um ângulo $\theta = \np[\tcdegree]{60}$ com a horizontal. Instantes depois, a bola atinge a região plana abaixo da elevação com velocidade $\vec{v}_f$.
\begin{itemize}
\item[(a)] Determine a distância horizontal percorrida pela bola.
\item[(b)] Determine o módulo $v_f$ da velocidade final e o ângulo $\phi$ que ela faz com o solo.
\end{itemize}
\end{quote}
\begin{marginfigure}
\centering
\begin{tikzpicture}[>=Stealth, 
     interface/.style={
        % superfície
        postaction={draw,decorate,decoration={border,angle=-45,
                    amplitude=0.2cm,segment length=2mm}}},
    ]

	\draw[interface] (0.5,-1) -- (4.5,-1) coordinate (pd);

	\draw[pattern = north west lines] (0.5,-1) rectangle (2,0);
	\draw[|-|] (0.3, 0) -- node[left]{$h$} (0.3, -1);
	
    \draw[shift={(2, 2pt)},dashdotted, smooth, samples=1000, domain=0:1.86]
		plot (\x, {\x * tan(60) - 9.8*\x^2 / (2*4^2*(cos(60))^2)});

	\draw[->] (2, 2pt) coordinate (origin) -- +(60:0.7) coordinate (v) node[left]{$\vec{v}_i$};
	\draw[fill] (origin) circle (2pt);

	\draw[dotted,->] (origin) -- +(0,1.5) node[below left]{$y$};
	\draw[dotted,->] (origin) -- +(2,0) coordinate (x) node[below left]{$x$};

	\pic [draw, "$\theta$", angle eccentricity = 1.7, angle radius = 3mm]{angle = x--origin--v};

	\draw[->] (3.86, -1) coordinate (ovf) -- node[left]{$\vec{v}_f$} +(-70.4:1.0) coordinate (vf);

	\draw[fill, gray] (ovf)+(-70.4:-2pt) circle (2pt);

	\pic [draw, "$\phi$", angle eccentricity = 1.6, angle radius = 4mm]{angle = vf--ovf--pd};

\end{tikzpicture}
\caption{Lançamento a partir de uma elevação.}
\end{marginfigure}

Para a solução, vamos considerar que o sistema de coordenadas possui um eixo horizontal $x$ e um eixo $y$ que aponta verticalmente para cima. Nesse caso temos que
\begin{equation}
	a_y = -g.
\end{equation}
%
Nos eixos de referência, temos que a velocidade inicial tem componentes
\begin{align}
	v_{i}^{x} &= v_i \cos\theta \label{Eq:VelocidadeXExemploLancamentoBola} \\
	v_{i}^{y} &= v_i \sen\theta.
\end{align}

Podemos determinar a distância horizontal percorrida pela bola através de
\begin{equation}
	\Delta x = v_x t,
\end{equation}
%
bastando que descubramos o tempo necessário para que a bola descreva a trajetória enre o ponto inicial e o final. Se considerarmos o deslocamento vertical, temos
\begin{equation}
	\Delta y = v_{i}^{y}t + \frac{at^2}{2},
\end{equation}
%
ou, se considerarmos que $h = \Delta y$ e $a = -g$,
\begin{equation}
	h = v_{i}^{y} t - \frac{gt^2}{2}.
\end{equation}

Podemos escrever a expressão acima com a forma
\begin{equation}
	-\frac{g}{2}t^2 + v_{i}^{y} t - h = 0,
\end{equation}
%
que podemos identificar com a forma geral de uma equação de segundo grau:
\begin{equation}
	At^2 + Bt + c = 0.
\end{equation}
%
As raízes desse tipo de equação podem ser calculadas através de
\begin{equation}
	t = \frac{-B \pm \sqrt{B^2 - 4 A C}}{2A}.
\end{equation}
%
Portanto, realizando as substituições, temos
\begin{align}
	t_{+} &= -\np[s]{0,38} \\
	t_{-} &= \np[s]{1,08}.
\end{align}
%
O primeiro tempo corresponde ao instante de tempo \emph{anterior} ao lançamento em que o corpo estaria na linha horizontal do solo. Tal solução não tem sentido físico. Logo, através da expressão para o deslocamento horizontal temos
\begin{equation}
	\Delta x = \np[m]{2,17}.
\end{equation}

Para determinar o módulo e o ângulo em relação à horizontal para o vetor velocidade final, devemos notar que a velocidade no eixo horizontal $x$ é constante, e seu valor é dado pela Equação~\eqref{Eq:VelocidadeXExemploLancamentoBola}. Além disso, a componente da velocidade final no eixo vertical pode ser calculada através de
\begin{align}
	v_{f}^{y} &= v_{i}^{y} - at \\
	&= v_{i}^{y} - gt \\
	&= -\np[m/s]{8,58},
\end{align}
%
onde utilizamos o valor de tempo determinado anteriormente. Para derterminarmos o módulo da velocidade, basta calcularmos
\begin{equation}
	v = \sqrt{v_x^2 + v_y^2}
\end{equation}
%
de onde obtemos
\begin{equation}
	v = \np[m/s]{8,81}.
\end{equation}


\begin{marginfigure}
\centering
\begin{tikzpicture}[>=Stealth, scale = 4]

	\draw[dotted] (0,0) coordinate (origin) -- (0.335, 0) coordinate (right) -- (0.335, -0.9421);
	\draw[->] (0,0) -- node[left]{$\vec{v}_f$} (0.335, -0.9421) coordinate (v);

	\pic [draw, "$\phi$", angle eccentricity = 1.5]{angle = v--origin--right};

	\draw[|-|] (0,0.1) -- node[above]{$v_x$} (0.335, 0.1);
	\draw[|-|] (0.435, 0) -- node[right]{$v_y$} (0.435, -0.9421);

\end{tikzpicture}
\caption{Detalhe do ângulo entre o vetor velocidade final $\vec{v}_f$ e a horizontal.\label{Fig:TrianguloRetanguloAnguloVelocidadeFinalEmRelacaoAHorizontal}}
\end{marginfigure}

Para determinarmos o ângulo em relação à horizontal, basta notarmos que o as componentes e o próprio vetor formam um triângulo retângulo (Figura~\ref{Fig:TrianguloRetanguloAnguloVelocidadeFinalEmRelacaoAHorizontal}). Assim,
\begin{equation}
	\tan\phi = \frac{v_y}{v_x},
\end{equation}
%
de onde podemos escrever
\begin{equation}
	\phi = \arctan \frac{v_y}{v_x}
\end{equation}
%
e, finalmente,
\begin{equation}
	\phi = \np[\tcdegree]{-76,88}.
\end{equation}

%%%%%%%%%%%%%%%%%%%%%%%%%%%%%%%%%%%%%%%%%%
\paragraph{Exemplo: Lançamento horizontal}
%%%%%%%%%%%%%%%%%%%%%%%%%%%%%%%%%%%%%%%%%%

\begin{quote}
	Uma esfera rola sobre uma mesa com velocidade $v_r = \np[m/s]{3,5}$ e, ao chegar ao final dela, cai ao chão. Se a distância horizontal entre a borda da mesa e o ponto onde a esfera atinge o chão é de \np[m]{1,3}, qual é a altura da mesa?
\end{quote}

\begin{marginfigure}
\centering
\begin{tikzpicture}[>=Stealth, 
     interface/.style={
        % superfície
        postaction={draw,decorate,decoration={border,angle=-45,
                    amplitude=0.2cm,segment length=2mm}}},
    ]
	\draw (0,0) -- (2,0) -- (2,-0.2) -- (0,-0.2);
	\draw (1.8, -0.2) -- (1.8, -0.6) -- (0,-0.6);
	\draw (1.6, -0.6) -- (1.6, -2) -- (1.2, -2) -- (1.0, -0.6);
	\draw[interface] (0,-2) -- (4,-2);
	
	\draw[fill] (0.5, 2pt) circle (2pt);
	\draw[->] (0.5, 2pt) -- node[above]{$\vec{v}_r$} +(0.5,0);
	\draw[densely dotted] (2, 2pt) circle (2pt);

    \draw[shift={(2, 2pt)},dashdotted, smooth, samples=1000, domain=0:1.3]
		plot (\x, {-9.8*\x^2 / (2*2^2)});

	\draw[dotted] (2,-0.2) -- (2,-2) (3.3, -2) -- +(0,1);
	\draw[|<->|] (2,-1.25) -- node[below]{$\Delta x$} (3.3,-1.25);

	\draw[->] (3.3,0) -- +(0.5,0) node[above left]{$x$};
	\draw[->] (3.3,0) -- +(0,-0.5) node[above left]{$y$};

\end{tikzpicture}
\caption{Laçamento horizontal.}
\end{marginfigure}

Como sistema de referência, vamos escolher um sistema de coordenadas com um eixo vertical $y$ apontando para baixo, e um eixo horizontal $x$ apontando para a direita. Nesse caso temos que
\begin{align}
	a_x &= 0 \\
	a_y &= g.
\end{align}
%
Note também que a velocidade inicial de no momento em que a esfera passa a estar sujeita à gravidade é dada por
\begin{align}
	v_{i}^{x} &= v_r \\
	v_{i}^{y} &= 0.
\end{align}

Podemos então determinar a altura da mesa em termos do tempo de queda através da expressão
\begin{equation}
	y_f = y_i + v_{i}^{y}t + \frac{at^2}{2}
\end{equation}
%
que pode ser escrita como
\begin{equation}
	\Delta y = \frac{gt^2}{2}
\end{equation}
%
ao utilizarmos $a_y = g$ e $v_{i}^{y} = 0$.

Resta agora determinar o valor de $t$, mas ele pode ser obtido através do eixo horizontal $x$, pois
\begin{equation}
	x_f = x_i + v_{i}^{x}t,
\end{equation}
%
o que pode ser reescrito como
\begin{equation}
	t = \frac{\Delta x}{v_r}.
\end{equation}
%
Finalmente, obtemos para a altura da mesa
\begin{align}
	H &= \Delta y \\
	  &= \frac{g}{2} \left(\frac{\Delta x}{v_r}\right)^2 \\
	  &= \frac{g (\Delta x)^2}{2 v_r^2} \\
	  &= \np[m]{0,68}.
\end{align}

\pagebreak
%%%%%%%%%%%%%%%%%%%%%%%%%%%%%%%%%%%%%%%%%%%%%%
\paragraph{Exemplo: Arremesso sobre obstáculo}
%%%%%%%%%%%%%%%%%%%%%%%%%%%%%%%%%%%%%%%%%%%%%%

\begin{quote}
Uma bola é rebatida com uma raquete e adquire uma velocidade $\vec{v}_i$, cujo módulo é de \np[m/s]{6,0} e que faz um ângulo $\theta$ de \np[\tcdegree]{75} com a horizontal. A altura a partir da qual a bola é rebatida é $h = \np[m]{1,2}$. A uma distância $\ell$ de \np[m]{1.5}, medida horizontalmente, há uma cerca de altura $H = \np[m]{2,0}$. A bola passa sobre a cerca?
\end{quote}

\begin{marginfigure}
\centering
\begin{tikzpicture}[>=Stealth, scale = 1.2,
     interface/.style={
        % superfície
        postaction={draw,decorate,decoration={border,angle=-45,
                    amplitude=0.2cm,segment length=2mm}}},
    ]

	\draw[interface] (1.25,-0.5) -- (5,-0.5) coordinate (pd);

	\draw[|<->|] (1.7, 0) -- node[left]{$h$} (1.7, -0.5);
	
    \draw[shift={(2, 2pt)},dashdotted, smooth, samples=1000, domain=0:1.25]
		plot (\x, {\x * tan(75) - 9.8*\x^2 / (2*6^2*(cos(75))^2)});

	\draw[->] (2, 2pt) coordinate (origin) -- coordinate (v) node[left]{$\vec{v}_i$} +(75:0.7);
	\draw[fill] (origin) circle (2pt);

	\draw[dotted,->] (origin) +(0,-0.25) -- +(0,1.5) node[below left]{$y$};
	\draw[dotted,->] (origin) +(-0.25,0) -- +(2.5,0) coordinate (x) node[below left]{$x$};

	\pic [draw, "$\theta$", angle eccentricity = 1.7, angle radius = 3mm]{angle = x--origin--v};

	\draw[pattern = north west lines] (3.5, -0.5) rectangle +(0.15,1.5);
	\draw[|<->|] (3.85, -0.5) -- node[right]{$H$} +(0,1.5); 

	\draw[<->] (2,-0.05) -- node[below]{$\ell$} (3.5,-0.05);

\end{tikzpicture}
\caption{Lançamento a partir de uma elevação com possível colisão com obstáculo.}
\end{marginfigure}

Para determinarmos se haverá colisão, basta sabermos a que altura em relação ao solo a bola estará na posição onde está localizada a cerca. Podemos determinar tal altura em relação ao ponto de lançamento ao utilizar a equação para a trajetória:
\begin{equation}
	y(x) = (\tan\theta) x - \frac{g x^2}{2 v_i^2\cos^2\theta}.
\end{equation}
%
Utilizando as informações dadas no problema,
\begin{align}
	\theta &= \np[\tcdegree]{75} \\
	x &\equiv \ell = \np[m]{1.5} \\
	v_i &= \np[m/s]{6,0},
\end{align}
%
obtemos
\begin{equation}
	y(x) = \np[m]{1.03}.
\end{equation}
%
Essa distância se refere à altura da bola em relação ao ponto de lançamento ao chegar na posição da cerca. Como a bola é rebatida a uma distância de \np[m]{1,2} acima do solo, concluímos que a altura da bola em relação ao solo ao passar sobre a cerca é de \np[m]{2.23}, passando a \np[cm]{23} acima dela.

%%%%%%%%%%%%%%%%%%%%%%%%%%%%
\section{Movimento circular}
\label{Sec:MovimentoCircular}
%%%%%%%%%%%%%%%%%%%%%%%%%%%%

Como um segundo exemplo de movimento bidimensional, vamos analisar o movimento circular. O efeito do vetor aceleração no vetor velocidade é mais difícil de visualizar do que o efeito da velocidade no vetor deslocamento. É mais fácil considerarmos dois efeitos separadamente: a alteração do módulo, e a alteração da direção do vetor $\vec{v}$. Ambos podem estar presentes em um movimento circular.

\begin{marginfigure}
\centering
\begin{tikzpicture}[>=Stealth]
    \draw[->] (0,0) -- node[above]{$\vec{v}_i$} +(1,0);
    \draw[->] (1,0) -- node[above]{$\vec{a} \Delta t$} +(0.5,0);
    \draw[->] (0,-0.3) -- node[below]{$\vec{v}_f$} +(1.5,0);
\end{tikzpicture}
\caption{O efeito de uma aceleração constante e colinear com o vetor velocidade inicial é o de alterar o módulo da velocidade.\label{Fig:EfeitoAceleraaoColinear}}
\end{marginfigure}

Em um movimento retilíneo, se há aceleração, o vetor $\vec{a}$ tem a mesma direção do vetor velocidade inicial, então a soma
\begin{equation}
    \vec{v}_f = \vec{v}_i + \vec{a} t
\end{equation}
%
corresponde à Figura~\ref{Fig:EfeitoAceleraaoColinear}, e temos que os vetores $\vec{v}_i$ e $\vec{v}_f$ têm a mesma direção, porém têm módulos diferentes.

Já no caso de termos um vetor aceleração que seja sempre perpendicular ao vetor velocidade, devemos novamente utilizar a ideia de variações infinitesimais (\emph{Rotação de um vetor}, página~\pageref{Disc:RotVetor}): da mesma forma que o vetor posição $\vec{r}$ de uma partícula que executa um movimento circular é rotacionado por um vetor velocidade continuamente perpendicular a $\vec{r}$, uma velocidade $\vec{v}$ é rotacionada por um vetor aceleração continuamente perpendicular a $\vec{v}$. Nesse caso, percebemos que o papel de acelerações perpendiculares à velocidade é o de alterar a \emph{direção} da velocidade, sem alterar o módulo.

Em situações mais complexas, podemos ter uma composição dos dois efeitos da aceleração. Se temos uma situação como a da Figura~\ref{Fig:DoisEfeitosAceleracaoNaVelocidade}, por exemplo, temos que o vetor velocidade final tem um módulo maior que o vetor velocidade inicial, e que sofreu uma rotação em relação ao vetor inicial. De qualquer forma, podemos utilizar a ideia de componentes vetoriais para decompor a aceleração em uma componente na direção da velocidade inicial, e outra na direção perpendicular ao vetor velocidade inicial. Essa estratégia simplifica bastante a análise de movimentos bi e tridimensionais, e será a base para a análise do movimento circular nas seções seguintes.

\begin{marginfigure}
\centering
\begin{tikzpicture}[>=Stealth, scale = 0.75]

	\draw[->] (0,0) -- node[below]{$\vec{v}_i$} (2,0);

	\draw[arrows={[-Stealth[harpoon,swap]}] (2,0) -- node[right]{$\Delta v = \vec{a} t$} +(1,2) coordinate (at);

	\draw[arrows={[-Stealth[harpoon]}] (0,0) -- node[above, sloped]{$\vec{v}_i$} (at);

\end{tikzpicture}
\caption{O efeito mais geral de uma aceleração é o de alterar o módulo do vetor velocidade e também o de causar uma rotação em tal vetor.\label{Fig:DoisEfeitosAceleracaoNaVelocidade}}
\end{marginfigure}

%%%%%%%%%%%%%%%%%%%%%%%%%%%%%%%%%%
\subsection{Aceleração centrípeta}
%%%%%%%%%%%%%%%%%%%%%%%%%%%%%%%%%%

\begin{marginfigure}
\centering
   \begin{tikzpicture}[>=Stealth, scale=1.4]
        \draw[dotted] ([shift={(0,0)}]120:2) arc[radius=2, start angle=120, end angle= 0];
        \draw[->] (0,0) -- node[left]{$\vec{r}_i$} (60:2);
        \draw[->] (60:2) -- node[above right]{$\vec{v}_i$} +(-30:0.51763809) coordinate (A);
        \draw[->] (0,0) -- node[below]{$\vec{r}_f$} (30:2);
        \draw[->] (30:2) -- node[right]{$\vec{v}_f$}+(-60:0.51763809);
        \draw[dashdotted] (0,0) -- (45:2.6);
        
        \draw[->, dashed] (60:2) -- node[below left]{$\vec{v}_f$} +(-60:0.51763809) coordinate (B);
        \draw[->] (A) -- node[right]{$\Delta \vec{v}$} (B);
        
   \end{tikzpicture}
   \caption{Em um movimento circular com velocidade constante, o vetor $\Delta\vec{v}$ aponta para o centro da trajetória quando disposto exatamente no ponto intermediário entre as posições inicial e final. Essa é a mesma direção que a aceleração média, consequentemente, quando tomamos o limite $\Delta t \to 0$ e aproximamos os pontos, verificamos que a aceleração instantânea aponta para o centro da trajetória.\label{Fig:Acel_mov_cir_unif}}
\end{marginfigure}

Analisando o movimento circular ---~restrito ao caso de velocidade constante em módulo~---, verificamos que temos uma alteração constante da direção do vetor velocidade. Na Figura~\ref{Fig:Acel_mov_cir_unif} vemos uma parte da trajetória seguida por uma partícula. Em dois instantes diferentes, temos dois vetores velocidade que têm o mesmo módulo, porém direções diferentes. Se calcularmos geometricamente a diferença entre esses vetores, vemos que $\Delta \vec{v}$ aponta perpendicularmente à trajetória quando disposto no ponto central entre as posições inicial e final, isto é, ele aponta para o centro da trajetória circular.

Sabemos, que o vetor aceleração média é dado por
\begin{equation}
  \vec{\mean{a}} = \frac{\Delta \vec{v}}{\Delta t}.
\end{equation}
%
Portanto, mesmo no caso de $v$ constante, temos uma aceleração caso ocorram mudanças na direção do vetor velocidade. Podemos calcular o módulo desta aceleração se considerarmos a Figura~\ref{Fig:Acel_mov_cir_unif_mod}. Inicialmente uma partícula ocupa a posição $\vec{r}_i$, com velocidade $\vec{v}_i$ no instante $t_i$. Após um intervalo de tempo, ela passa a ocupar a posição $\vec{r}_f$, com velocidade $\vec{v}_f$ no instante $t_f$. Vamos assumir que $v_i = v_f = v$ e que $r_i = r_f = r$. 


\begin{marginfigure}
\centering
   \begin{tikzpicture}[>=Stealth, scale=1.4]
        \draw[dotted] ([shift={(0,0)}]120:2) arc[radius=2, start angle=120, end angle= 0];
        \coordinate (O) at (0,0);
        
        \draw[->] (O) -- node[left]{$\vec{r}_i$} (60:2) coordinate (A);
        \draw[->] (O) -- node[below]{$\vec{r}_f$} (30:2) coordinate (B);
        \draw[->] (A) -- node[above right]{$\Delta \vec{r}$} (B);
        
        \pic [draw, "$\theta$", angle eccentricity=1.5] {angle = B--O--A};
        \pic [draw, "$\alpha$", angle eccentricity=1.5, angle radius = 3mm] {angle = O--A--B};
        \pic [draw, "$\alpha$", angle eccentricity=1.5, angle radius = 3mm] {angle = A--B--O};
        
        \coordinate (O) at (0, -1);    
        
        \draw[->] (O) -- node[above]{$\vec{v}_i$} +(-30:1.5) coordinate (A);
        \draw[->] (O) -- node[left]{$\vec{v}_f$} +(-60:1.5) coordinate (B);
        \draw[->] (A) -- node[right]{$\Delta \vec{v}$} (B);
        
        \pic [draw, "$\theta$", angle eccentricity=1.5] {angle = B--O--A};
        \pic [draw, "$\alpha$", angle eccentricity=1.5, angle radius = 3mm] {angle = O--A--B};
        \pic [draw, "$\alpha$", angle eccentricity=1.5, angle radius = 3mm] {angle = A--B--O};
    
   \end{tikzpicture}
   \caption{Triângulos formados pelos vetores $\vec{r}_i$, $\vec{r}_f$, e $\Delta \vec{r}$ e pelos vetores $\vec{v}_i$, $\vec{v}_f$, e $\Delta\vec{v}$. Note que este último foi ampliado em relação à Figura~\ref{Fig:Acel_mov_cir_unif} unicamente para facilitar a visualização.\label{Fig:Acel_mov_cir_unif_mod}}
\end{marginfigure}

Verificamos que existe um ângulo $\theta$ entre os vetores $\vec{r}_i$ e $\vec{r}_f$. Além disso, como $r_i = r_f = r$, temos que os outros dois ângulos do triângulo são $\alpha$. Podemos utilizar a lei dos senos para estabelecer a seguinte relação:
\begin{equation}
    \frac{\Delta r}{\sen \theta} = \frac{r}{\sen \alpha},
\end{equation}
%
ou, equivalentemente,
\begin{equation}
    \frac{\Delta r}{r} = \frac{\sen \theta}{\sen \alpha}.
\end{equation}

Os vetores velocidade inicial e final são perpendiculares aos vetores posição inicial e final, respectivamente. Portanto, o ângulo formado pelos vetores velocidade é o mesmo ângulo formado pelos vetores posição, isto é, o ângulo $\theta$.\footnote{Imagine o seguinte: o vetor $\vec{r}_i$ é girado por um ângulo $\theta$ para se tornar o vetor $\vec{r}_f$. Essa rotação também afeta o vetor velocidade $\vec{v}_i$ o transformando no vetor $\vec{v}_f$, pois a relação de perpendicularidade entre o vetor velocidade e o vetor posição se mantém para todos os pontos em um movimento circular.} Além disso, como $v_i = v_f = v$, os demais ângulos são iguais entre si e são iguais ao mesmo ângulo $\alpha$ que aparece no triângulo formado pelos vetores posição e deslocamento. Aplicando novamente a lei dos senos, obtemos
\begin{equation}
    \frac{\Delta v}{\sen \theta} = \frac{v}{\sen \alpha},
\end{equation}
%
ou
\begin{equation}
    \frac{\Delta v}{v} = \frac{\sen\theta}{\sen\alpha}.
\end{equation}

A partir desses resultados, temos que
\begin{equation}
    \frac{\Delta v}{v} = \frac{\Delta r}{r}.
\end{equation}
%
Isolando $\Delta v$ e substituindo na expressão para a aceleração média, obtemos para o módulo
\begin{equation}
  \mean{a} = \frac{v}{r} \frac{\Delta r}{\Delta t}.
\end{equation}
%
Tomando o limite $\Delta t \to 0$, obtemos a aceleração instantânea:
\begin{equation}
  a = \frac{v}{r} \lim_{\Delta t \to 0} \frac{\Delta r}{\Delta t},
\end{equation}
%
onde usamos a propriedade
\begin{equation}
    \lim_{\epsilon \to \xi} \lambda f(\epsilon) = \lambda \lim_{\epsilon\to \xi} f(\epsilon),
\end{equation}
%
$\lambda$ e $\xi$ representandos constantes quaisquer. Notamos que o limite que resta é a razão entre a distância percorrida pela partícula e o tempo necessário para efetuar tal deslocamento, ou seja, é a velocidade $v$. Logo,
\begin{equation}
  a = \frac{v^2}{r}.
\end{equation}

Verificamos no início desta seção que a aceleração média aponta para o centro da trajetória quando a dispomos exatamente na região central entre os pontos inicial e final (para o cálculo da aceleração média em questão). Quando tomamos o limite $\Delta t \to 0$, o que fazemos é mover tais pontos de forma que eles se tornam infinitamente próximos e se tornem o mesmo ponto, o que faz com que \emph{a aceleração instantânea aponte para o centro da trajetória circular}. Denominamos essa aceleração como \emph{aceleração centrípeta} $a_c$, cujo módulo é dado por
\begin{equation}\label{Eq:AceleracaoCentripeta}
  a_c = \frac{v^2}{r}. \mathnote{Módulo da aceleração centrípeta}
\end{equation}
%
Essa aceleração é responsável por alterar constantemente a direção do vetor velocidade, possibilitando que a partícula execute um movimento curvilíneo. Veremos adiante que ela está relacionada à força resultante que deve ser exercida por um agente externo para que uma partícula execute um movimento circular com velocidade $v$ e raio $r$.

%%%%%%%%%%%%%%%%%%%%%%%%%%%%%%%%%%%%%%%%%%%%%%%%%%%%%%%%%%%%%%%%%%%%%%%%%%%%%%
\subsection{Decomposição da aceleração em componentes tangencial e centrípeta}
%%%%%%%%%%%%%%%%%%%%%%%%%%%%%%%%%%%%%%%%%%%%%%%%%%%%%%%%%%%%%%%%%%%%%%%%%%%%%%

Verificamos que a aceleração é dada por
\begin{displaymath}
  \vec{a} = \lim_{\Delta t \to 0} \frac{\Delta \vec{v}}{\Delta t},
\end{displaymath}
%
e no caso específico de um movimento circular com velocidade constante, ela aponta para o centro da trajetória circular, com módulo dado pela Equação~\eqref{Eq:AceleracaoCentripeta}. No entanto, o caso de velocidade constante não é geral: podemos ter um movimento circular com velocidade que muda constantemente, ou mesmo ter um movimento curvilíneo que não tem um formato circular.

No caso de termos um movimento curvilíneo com velocidade constante, porém não circular, temos que a aceleração centrípeta não é constante. Se temos uma curvatura que se ``fecha'' paulatinamente, isto é, se o raio de curvatura diminui progressivamente, a aceleração centrípeta aumenta. Se ocorre o contrário, a curvatura se ``abre'' e o raio aumenta, temos que a aceleração centrípeta diminui. De qualquer forma, podemos calcular \emph{instantaneamente} a aceleração através da Expressão~\ref{Eq:AceleracaoCentripeta}.

Quando ocorre uma alteração do módulo da velocidade, podemos determinar a aceleração simplesmente calculando a razão $\Delta v/\Delta t$. Temos então dois efeitos possíveis da aceleração: um deles é alterar o módulo da velocidade, o outra é alterar a direção do vetor velocidade. Tais efeitos podem ocorrer isoladamente, como num movimento retilíneo com velocidade variável -- onde ocorre somente o primeiro --, ou num movimento circular com velocidade constante -- onde ocorre somente o segundo --, ou podem ocorrer \emph{ao mesmo tempo}.

O caso de ambos os efeitos ocorrerem conjuntamente é, na prática, o mais comum: um carro que trafega por uma rodovia, por exemplo, executa curvas e altera sua velocidade a todo momento, e em inúmeras vezes, essas mudandas de direção e de módulo da velocidade acontecem concomitantemente. Apesar de termos dois papéis distintos para a aceleração, temos \emph{somente um vetor aceleração}. No entanto, \emph{podemos decompor o vetor aceleração em duas componentes} cujos efeitos correspondem aos discutidos acima, isto é, mudar a direção e o módulo da velocidade. 

% adaptado de https://tex.stackexchange.com/questions/25928/how-to-draw-tangent-line-of-an-arbitrary-point-on-a-path-in-tikz
\begin{figure}
\centering
\begin{tikzpicture}[>=Stealth,scale=2.6,
    tangent/.style={
        decoration={
            markings,% switch on markings
            mark=
                at position #1
                with
                {
                    \coordinate (tangent point-\pgfkeysvalueof{/pgf/decoration/mark info/sequence number}) at (0pt,0pt);
                    \coordinate (tangent unit vector-\pgfkeysvalueof{/pgf/decoration/mark info/sequence number}) at (1,0pt);
                    \coordinate (tangent orthogonal unit vector-\pgfkeysvalueof{/pgf/decoration/mark info/sequence number}) at (0pt,1);
                }
        },
        postaction=decorate
    },
    use tangent/.style={
        shift=(tangent point-#1),
        x=(tangent unit vector-#1),
        y=(tangent orthogonal unit vector-#1)
    },
    use tangent/.default=1
]

%%%

	\path[name path=traj, draw, dashdotted, tangent=0.30] (0,2) .. controls (1,0.3) and (2,2.2) .. (3,1.5) coordinate [pos=0.24] (point);
    \draw [densely dotted, thin, use tangent] (-1.5,0) -- (1.5,0);

    \draw [->, use tangent, scale = 2] (0,0) -- node[below]{$a_t$}(0.5,0) coordinate (at);
    \draw [->, use tangent, scale = 2] (0,0) -- node[above left]{$a_c$}(0,0.5) coordinate (ac);
    \draw [use tangent, thin, densely dotted] (0,-1.5) -- (0,1.5);
    \draw[thin, use tangent, gray] ([shift={(0,2)}]210:2) arc[radius=2, start angle=210, end angle= 330];
    \draw[use tangent, fill] (0,2) node[right]{$C$} circle (0.4pt);
    \draw[use tangent, thick, ->, scale = 2] (0,0) coordinate (origin) -- node[above]{$\vec{a}$}(0.5,0.5) coordinate (a);
    
	\draw[dotted] (a) -- (at) (a) -- (ac);
\end{tikzpicture}
\caption{Podemos aproximar uma região qualquer de uma curva por um círculo. Assim, é possível se decompor o vetor aceleração em uma componente \emph{centrípeta} (que aponto para o centro $C$ do círculo), cujo papel é o de alterar a direção, e uma componente tangencial (que aponta tangencialmente à trajetória, na mesma direção da velocidade instantânea), cujo papel é o de alterar o módulo da velocidade.\label{Fig:Decomp_acel_tan_centrip}}
\end{figure}

Qualquer ponto de uma trajetória curvilínea pode ser aproximado por uma trajetória circular, com um centro e um raio bem definido\footnote{Até mesmo uma reta pode ser interpretada dessa maneira, nesse caso temos um círculo de raio infinito!}. Levando isso em conta, de forma geral, podemos assumir que o vetor aceleração não aponta para o centro da trajetória, porém podemos o decompor em duas partes:
\begin{description}
  \item[Componente radial] Uma das componentes é a projeção do vetor aceleração na direção do eixo radial que liga a partícula ao centro da trajetória circular. Temos então uma componente radial $a_r$ cujo módulo é dado pela aceleração centrípeta:
  \begin{equation}
    a_r = a_c = \frac{v^2}{r}.
  \end{equation}
  Podemos atribuir a esta componente o papel exclusivo de alterar a \emph{direção} do vetor velocidade. Podemos entender isso se considerarmos que se só existisse essa componente, então teríamos um movimento circular, onde somente a direção do vetor velocidade é alterada.
  \item[Componente tangencial] A outra componente da aceleração é a projeção do vetor na direção tangencial à trajetória -- ou seja, na direção perpendicular ao eixo radial --. Esta componente tem o papel exclusivo de alterar o \emph{módulo} da velocidade. Isso pode ser entendido se imaginarmos que houvesse somente essa componente da aceleração. Nesse caso, a aceleração seria na própria direção da velocidade, portanto ao calcularmos a velocidade final após um intervalo $\delta t$,
  \begin{equation}
    \vec{v}_f = \vec{v}_i + \vec{a} \delta t,
  \end{equation}
%
percebemos que estamos somando dois vetores colineares. Logo, o resultado $\vec{v}_f$ de tal soma se mantém na mesma direção que os termos do lado direito na equação acima. Assim, temos que
\begin{equation}
  v_f = v_i + a_t t. \\
\end{equation}
\end{description}

\begin{marginfigure}
\centering

\begin{tikzpicture}[>=Stealth]
    \draw[->] (0,0) -- node[above]{$\vec{v}_0$} +(1,0);
    \draw[->] (1,0) -- node[above]{$\vec{a} \delta t$} +(0.5,0);
    \draw[->] (0,-0.3) -- node[below]{$\vec{v}_f$} +(1.5,0);
\end{tikzpicture}
\caption{Caso o vetor aceleração seja colinear à velocidade instantânea, o vetor velocidade final também o é. Isto é, a aceleração só poderá mudar o módulo da velocidade. Esse raciocínio também vale quando estamos tratando o eixo tangencial à trajetória, que é justamente o eixo da velocidade instantânea; nesse caso utilizamos a aceleração tangencial para determinar a variação da velocidade (em módulo).}
\end{marginfigure}

Caso conheçamos ambas as componentes da aceleração -- o que é bastante comum, uma vez que é mais fácil determinar o módulo da velocidade e eventuais alterações desse valor, o que nos dará os valores de $a_c$ e de $a_t$ -- podemos determinar o vetor $\vec{a}$ utilizando as relações
\begin{align}
  \mod{\vec{a}} &= \sqrt{a_t^2 + a_c^2} \\
  \theta &= \arctan \frac{a_c}{a_t},
\end{align}
%
onde $\theta$ é o ângulo que o vetor $\vec{a}$ faz com a direção tangencial. Essas relações são oriundas das propriedades vetoriais discutidas no capítulo anterior. Da mesma forma, podemos conhecer as componentes a partir do vetor $\vec{a}$ através de
\begin{align}
  a_t &= a\cos\theta \\
  a_c &= a\sen\theta.
\end{align}

% adaptado de https://tex.stackexchange.com/questions/25928/how-to-draw-tangent-line-of-an-arbitrary-point-on-a-path-in-tikz
\begin{marginfigure}
\centering
\begin{tikzpicture}[>=Stealth, scale=3,
    tangent/.style={
        decoration={
            markings,% switch on markings
            mark=
                at position #1
                with
                {
                    \coordinate (tangent point-\pgfkeysvalueof{/pgf/decoration/mark info/sequence number}) at (0pt,0pt);
                    \coordinate (tangent unit vector-\pgfkeysvalueof{/pgf/decoration/mark info/sequence number}) at (1,0pt);
                    \coordinate (tangent orthogonal unit vector-\pgfkeysvalueof{/pgf/decoration/mark info/sequence number}) at (0pt,1);
                }
        },
        postaction=decorate
    },
    use tangent/.style={
        shift=(tangent point-#1),
        x=(tangent unit vector-#1),
        y=(tangent orthogonal unit vector-#1)
    },
    use tangent/.default=1
]

%%%

	\clip (0.2,2) rectangle (1.5, 1);

    \path[name path=traj, draw, dashdotted, tangent=0.30] (0,2) .. controls (1,0.3) and (2,2.2) .. (3,1.5) coordinate [pos=0.24] (point);
    \draw [densely dotted, thin, use tangent] (-1.5,0) -- (1.5,0);

    \draw [->, use tangent, scale = 2] (0,0) -- node[below]{$a_t$}(0.5,0) coordinate (at);
    \draw [->, use tangent, scale = 2] (0,0) -- node[above left]{$a_c$}(0,0.5) coordinate (ac);
    \draw [use tangent, thin, densely dotted] (0,-1.5) -- (0,1.5);
    \draw[thin, use tangent, gray] ([shift={(0,2)}]210:2) arc[radius=2, start angle=210, end angle= 330];
    \draw[use tangent, fill] (0,2) node[right]{$C$} circle (0.4pt);
    \draw[use tangent, thick, ->, scale = 2] (0,0) coordinate (origin) -- node[above]{$\vec{a}$}(0.5,0.5) coordinate (a);
    
	\pic[draw, "$\theta$", angle eccentricity = 1.5]{angle = at--origin--a};
	\draw[dotted] (a) -- (at) (a) -- (ac);

\end{tikzpicture}
\caption{Se conhecermos as componentes tangencial e centrípeta, através das propriedades dos vetores podemos determinar o módulo e o ângulo do vetor aceleração em relação à direção tangente à trajetória.\label{Fig:Decomp_acel_tan_centrip_mod_vet_acel}}
\end{marginfigure}

%%%%%%%%%%%%%%%%%%%%%%%%%%%%%%%%%%%%%%%%%%%%%%%%%
\subsection{Posição em uma trajetória curvilínea}
%%%%%%%%%%%%%%%%%%%%%%%%%%%%%%%%%%%%%%%%%%%%%%%%%

Através das componentes radial e tangencial, se ---~por exemplo~--- necessitarmos descrever o movimento de um carro que ganha velocidade com aceleração tangencial constante em uma pista circular, podemos utilizar as equações da cinemática para calcular a evolução temporal da velocidade.

Em uma análise análoga àquela realizada no Capítulo~\ref{Chap:MovimentoUnidimensional}, verificamos que a posição $s$ de um corpo sujeito a uma aceleração tangencial $a_t$ constante é dada por
\begin{equation}
  s_f = s_i + v_i t + \frac{a_t t^2}{2}.
\end{equation}
%
De fato, podemos reobter todos os resultados que verificamos para a cinemática unidimensional nesse contexto de deslocamento em uma trajetória curvilínea, sendo que a única diferença é a de que devemos substituir a aceleração $a$ pela componente tangencial $a_t$. Note porém que nesse movimento a velocidade não é constante e, portanto, temos que a componente centrípeta $a_c$ da aceleração aumenta ou diminui à medida que velocidade e/ou o raio da trajetória muda.

%%%%%%%%%%%%%%%%%%%%%%%%%%%%%%%%%%%%%%%%%%%%%%%%%%%%%%%%%%%%%%%%%%%%%%%%%%%%%%%%
\paragraph{Exemplo: Componentes da aceleração de um carro em uma pista circular}
%%%%%%%%%%%%%%%%%%%%%%%%%%%%%%%%%%%%%%%%%%%%%%%%%%%%%%%%%%%%%%%%%%%%%%%%%%%%%%%%

\begin{quote}
Em um certo instante, um carro se desloca em uma pista circular com velocidade $v = \np[m/s]{7,0}$, sendo que ele está sujeito a uma aceleração cuja componente na direção tangencial à trajetória é dada por $a_t = \np[m/s^2]{0,60}$. O raio $r$ da pista é de \np[m]{500,0}.
\begin{itemize}
	\item[(a)] Quais são o módulo e o ângulo do vetor aceleração $\vec{a}$ em relação à direção tangencial à trajetória nesse instante?
	\item[(b)] Quais são o módulo e o ângulo do vetor aceleração em relação à direção tangencial após \np[s]{120}?
	\item[(b)] Supondo que o piloto do veículo suporte uma aceleração máxima de módulo $5g$, isto é, cinco vezes a aceleração da gravidade, quanto tempo ele pode manter a aceleração tangencial mencionada antes de perder a consciência?
\end{itemize}
\end{quote}

Na Figura~\ref{Fig:FiguraProblemaMovCircCarroAcelerando} temos uma representação da situação descrita no problema. Adotamos um sistema de referência onde o eixo $y$ liga a posição ocupada pelo carro ao centro da trajetória circular. Se o movimento tivesse velocidade constante em módulo, teríamos um vetor aceleração apontando nessa direção e sentido. No entanto, devido ao fato de que há variação no módulo da velocidade, sabemos que o vetor aceleração não aponta para o centro da trajetória, mas sim em uma direção que faz um ângulo $\theta$ com a direção tangente à trajetória.

\begin{marginfigure}
\centering
\begin{tikzpicture}[>=Stealth, scale = 1.5]


	\draw[dashdotted] (0:2) arc [start angle = 0, end angle = -100, radius = 2];
	\draw[fill] (0,0) circle (1pt);

	\draw[<->] (-7:1pt) -- node[above, sloped]{$R = \np[m]{500}$} (-7:2cm);

	\draw[fill] (-60:2) circle (0.5mm);

	\draw[densely dotted, <-] (-60:2) +(30:2) node[below right]{$x$} -- +(-150:1.5);
	\draw[->] (-60:2) -- +(30:1.5) node[below]{$\vec{v}$} coordinate (vf);
	\path (-60:2) -- +(30:0.75) coordinate (atf);
	\draw[|<->|] (-60:2.2) -- node[below, sloped]{$a_t$} +(30:0.75);
	\draw[|<->|] (-60:2) ++(30:-0.2) -- node[below, sloped]{$a_c$} +(120:1);

	\draw[->, thick] (-60:2) coordinate (ai) -- +(83.13:1.25) coordinate (af) node[right]{$\vec{a}$};
	\draw[dotted] (-60:2)++(30:0.75) -- (af);
	\draw[dotted] (-60:1) -- (af);

	\pic [draw, "$\theta$", angle eccentricity = 1.5, angle radius = 3mm]{angle = vf--ai--af};

	\draw[densely dotted, ->] (-60:2) -- (-60:0.5) node[below left]{$y$};

\end{tikzpicture}
\caption{lorem ipsum dolor sihet amet contribus longus captionus co tambe curtas palavrorum.\label{Fig:FiguraProblemaMovCircCarroAcelerando}}
\end{marginfigure}

Como vimos ao analisar o movimento circular, as componentes da aceleração nos eixos tangencial e radial têm papeis distintos, e vemos que o problema nos dá o valor da componente tangencial:
\begin{equation}
	a_t = \np[m/s^2]{0,6}.
\end{equation}
%
A componente centrípeta, por sua vez, pode ser determinada através de
\begin{equation}
	a_c = \frac{v^2}{r}.
\end{equation}
%
No instante do item (a), a velocidade do veículo é de $\np[m/s]{7,0}$, e o raio $r = \np[m]{500}$. Logo,
\begin{equation}
	a_c = \np[m/s]{0.098}.
\end{equation}
%
Para determinarmos o módulo do vetor aceleração e o ângulo que ele faz em relação à direção tangencial ao movimento, basta utilizarmos as propriedades da decomposição de vetores:
\begin{align}
	a &= \sqrt{a_x^2 + a_y^2} \\
	\theta &= \arctan\frac{a_y}{a_x},
\end{align}
%
o que em termos das componentes $a_t$ e $a_c$, resulta em 
\begin{align}
	a &= \sqrt{a_t^2 + a_c^2} \\
	\theta &= \arctan\frac{a_c}{a_t}.
\end{align}
%
Assim, obtemos
\begin{align}
	a &= \np[m/s^2]{0.608} \\
	\theta &= \np[\tcdegree]{9.28}.
\end{align}

Após \np[s]{120}, devido à aceleração tangencial, temos um aumento de velocidade, obtendo
\begin{align}
	v_f &= v_i + a_t t \\
	&= \np[m/s]{79}.
\end{align}
%
Com essa nova velocidade, obtemos um novo valor de aceleração centrípeta, dado por
\begin{equation}
	a_c = \np[m/s^2]{12.48}.
\end{equation}
%
Calculado as propriedades do vetor aceleração nesse instante, obtemos
\begin{align}
	a &= \np[m/s^2]{12.50} \\
	\theta &= \np[\tcdegree]{87.25}.
\end{align}

Para determinarmos o tempo até que o piloto perca a consciência, devemos determinar o tempo até que a aceleração atinja o valor crítico $a_{\textrm{max}} = 5g = \np[m/s^2]{49,0}$. Sabemos que
\begin{equation}
	a = \sqrt{a_c^2 + a_t^2},
\end{equation}
%
e que
\begin{equation}
	a_c = \frac{v^2}{r}.
\end{equation}
%
Além disso,
\begin{equation}
	v = v_i + a_t t.
\end{equation}
%
Substituindo a expressão acima na anterior, obtemos para o quadrado da aceleração centripeta
\begin{equation}
	a_c^2 = \frac{(v_i + a_t t)^4}{r^2},
\end{equation}
%
que, por sua vez, pode ser substituida na expressão para aceleração, resultando em
\begin{equation}
	a_{\textrm{max}} = \sqrt{\frac{(v_i + a_t t)^4}{r^2} + a_t^2}.
\end{equation}
%
Basta agora isolar $t$, o que resulta em
\begin{equation}
	t = \frac{\sqrt[4]{r^2(a_{\textrm{max}}^2 - a_t^2)} - v_i}{a_t},
\end{equation}
%
de onde obtemos um tempo de
\begin{equation}
	t = \np[s]{249,20}.
\end{equation}

%%%%%%%%%%%%%%%%%%%%%%%%%%%%%%%%%%%%%%%%%%%%%%%%%%%%%%%%%%%%%%%%
\paragraph{Exemplo: Tempo transcorrido em um movimento circular}
%%%%%%%%%%%%%%%%%%%%%%%%%%%%%%%%%%%%%%%%%%%%%%%%%%%%%%%%%%%%%%%%

\begin{quote}
	Uma partícula descreve um movimento circular com raio $r = \np[m]{12.0}$. Em um certo instante, o vetor aceleração tem módulo $a = \np[m/s^2]{30,0}$, sendo que sua direção faz um ângulo $\theta = \np[\tcdegree]{60.0}$ em relação à direção tangente à trajetória. Se a aceleração tangencial é constante, quanto tempo transcorreu entre o início do movimento circular e o momento atual, assumindo que a partícula partiu do repouso.
\end{quote}

Decompondo o vetor aceleração, temos
\begin{align}
	a_c &= a\sen\theta \\
	&= \np[m/s^2]{25.98} \\
	a_t &= a\cos\theta \\
	&= \np[m/s^2]{15,0}.
\end{align}
%
Sabemos que a aceleração centripeta está associada ao módulo da velocidade e ao raio da trajetória através de
\begin{equation}
	a_c = \frac{v^2}{r},
\end{equation}
%
logo,
\begin{align}
	v &= \sqrt{r a_c} \\
	&= \np[m/s]{17.66}.
\end{align}

Utilizando a relação para o módulo da velocidade em um movimento com aceleração tangencial constante,
\begin{equation}
	v_f = v_i + a_t t,
\end{equation}
%
temos
\begin{equation}
	t = \frac{v_f}{a_t},
\end{equation}
%
onde usamos o fato de que $v_i = 0$. Obtemos então
\begin{equation}
	t = \np[s]{1.18}.
\end{equation}

%%%%%%%%%%%%%%%%%%%%%%%%%%%%
\section{Movimento Relativo}
%%%%%%%%%%%%%%%%%%%%%%%%%%%%

Geralmente escolhemos o referencial para descrever um fenômeno como sendo fixo no solo. Assim, quando um ônibus passa, a velocidade que atribuímos a seus passageiros é a mesma do próprio ônibus, caso eles estejam sentados. Se fixarmos o referencial no piso do ônibus, veremos que a velocidade dos passageiros é nula. Portanto, a velocidade que medimos depende do referencial adotado. Isso se deve ao fato de que a própria posição depende do referencial.

\begin{figure}[!h]\forceversofloat
\centering
\begin{tikzpicture}[>=Stealth]

    \coordinate (OS) at (0,0) node[below left]{$S$};
    
    \draw[->] (OS) -- +(4,0) node[below left]{$x$};
    \draw[->] (OS) -- +(0,3) node[below left]{$y$};
    
    \coordinate (OSP) at (2,1.1);
    
    \draw[->] (OSP) node[below]{$S'$} -- +(4,0) node[below left]{$x'$};
    \draw[->] (OSP) -- +(0,3) node[below left]{$y'$};
    
    \coordinate (P) at (3,2.5);
    
    \draw[fill] (P) circle (1pt) node[above right]{$P$};

    \draw[->] (OS) -- node[above]{$\vec{r}_S$} (P);
    \draw[->] (OSP) -- node[right]{$\vec{r}_{S'}$} (P);
    \draw[->] (OS) -- node[below right]{$\vec{r}_{S'S}$}(OSP);
    
    \draw[->] (OSP) ++(0,2.5) -- node[above]{$\vec{v}_{S'S}$} +(1,0);
\end{tikzpicture}
\caption{Posição de uma partícula $P$ em dois referenciais diferentes. \label{Fig:Ref_mov_relativo}}
\end{figure}

Vamos analisar a situação mostrada na Figura~\ref{Fig:Ref_mov_relativo}. A posição da partícula $P$ no referencial $S'$ é dada pelo vetor $\vec{r}_{S'}$. Se estivermos interessados em calcular a posição desta partícula em um referencial $S$, sendo que a posição do referencial $S'$ é dada por $\vec{r}_{S'S}$ em relação a $S$, temos
\begin{equation}
  \vec{r}_S = \vec{r}_S' + \vec{r}_{S'S},\mathnote{Transformação galileana de posição.}
\end{equation}
%
ou seja, temos uma simples adição de dois vetores, como pode ser visto na própria figura. Esta transformação, juntamente com as equivalentes para velocidade e aceleração, são denominadas \emph{transformações galileanas}. Dentro de Mecânica Clássica essas são as transformações que devem ser utilizadas, já para o caso da mecânica relativística, devemos utilizar as \emph{transformações de Lorentz}, que levam em conta o fato de que a velocidade da luz tem um valor absoluto (igual em todos os referenciais) e que não pode ser ultrapassado.

\begin{marginfigure}
\centering
\begin{tikzpicture}[>=Stealth]

    \draw[->] (0,0) -- (4,0);
    \draw[->] (0,0) -- (0,3);
    
    \draw[->] (0.5,0.5) -- (3,0.5);
    \draw[->] (0.5,0.5) -- (0.5,2.5);
    
    \draw[->] (0.5,2) -- node[above]{$\vec{v}_{SS'}$} +(0.75,0);
    \draw[->] (1.5, 1.5) -- +(1.25,-0.75) node[above]{$\vec{v}_{S'}$};
    
    \begin{scope}[shift={(1,-1)}]
        \draw[->] (0,0) -- node[above]{$\vec{v}_{SS'}$} (1,0);
        \draw[->] (1,0) -- node[above]{$\vec{v}_{S'}$} +(1.25,-0.75) coordinate (P);
        \draw[->] (0,0) -- node[below]{$\vec{v}_{S}$} (P);
    \end{scope}
    
\end{tikzpicture}
\caption{Transformação galileana de velocidades.}
\end{marginfigure}

De maneira semelhante, se a partícula tem velocidade $v_{S'}$ em relação ao referencial $S'$ e temos que esse referencial se move com velocidade $\vec{v}_{S'S}$ em relação a $S$, temos
\begin{equation}
  \vec{v}_S = \vec{v}_{S'} + \vec{v}_{S'S}.
\end{equation}
Para obter esta relação, basta utilizarmos a definição da velocidade instantânea $\vec{v} = \lim_{\Delta t \to 0} \Delta \vec{r} / \Delta t$, escrevendo
\begin{equation}
  \vec{v}_S = \lim_{\Delta t \to 0} \frac{\Delta \vec{r}_s}{\Delta t}.
\end{equation}
%
Através da relação para a posição, podemos calcular o deslocamento no referencial $S$ como
\begin{align}
  \Delta \vec{r}_S &= \vec{r}_S^f - \vec{r}_S^i \\
  &= (\vec{r}_{S'}^f + \vec{r}_{S'S}^f) - (\vec{r}_{S'}^i + \vec{r}_{S'S}^i) \\
  &= (\vec{r}_{S'}^f - \vec{r}_{S'}^i) + (\vec{r}_{S'S}^f - \vec{r}_{S'S}^i) \\
  &= \Delta \vec{r}_{S'} + \Delta \vec{r}_{S'S}.
\end{align}
%
Substituindo esse resultado na equação anterior para a velocidade, temos
\begin{equation}
  \vec{v}_S = \lim_{\Delta t \to 0} \frac{\Delta \vec{r}_{S'} + \Delta \vec{r}_{S'S}}{\Delta t}.
\end{equation}
%
Separando os termos do numerador em duas frações e sabendo que $\lim_{\epsilon \to \xi} f(\epsilon) + g(\epsilon) = \lim_{\epsilon \to \xi} f(\epsilon) + \lim_{\epsilon \to \xi} g(\epsilon)$, temos
\begin{equation}
  \vec{v}_S = \lim_{\Delta t \to 0} \frac{\Delta\vec{r}_{S'}}{\Delta t} + \lim_{\Delta t \to 0} \frac{\Delta \vec{r}_{S'S}}{\Delta t}.
\end{equation}
%
Os limites acima definem as velocidades $\vec{v}_{S'}$ e $\vec{v}_{S'S}$, logo,
\begin{equation}
  \vec{v}_S = \vec{v}_{S'} + \vec{v}_{S'S}. \mathnote{Transformação galileana de velocidade}
\end{equation}

Podemos ainda calcular as transformações para a aceleração através de um cálculo análogo ao utilizado para o caso da velocidade, obtendo
\begin{equation}
  \vec{a}_S = \vec{a}_{S'} + \vec{a}_{S'S}.  \mathnote{Transformação galileana de aceleração}
\end{equation}
%
Temos um interesse particular nessa equação devido ao conceito de \emph{referencial inercial}. Veremos adiante que as Leis de Newton só têm validade dentro de um referencial inercial, que é um referencial que não está submetido a acelerações. Dessa forma, se ---~por exemplo~--- o referencial $S$ for um referencial inercial, o referencial $S'$ só será inercial se a aceleração $\vec{a}_{S'S}$ for igual a zero. Um referencial não inercial pode ser identificado quando surgem forças ``inexplicáveis''. Um exemplo disso é quando estamos em carro que acelera para a frente e verificamos que um objeto suspenso se desloca para trás, como se tivesse sido puxado através de um fio invisível. Dentro do referencial do carro, esta força não tem origem física aparente, só podendo ser explicada quando levamos em conta o fato de que o referencial sofre uma aceleração em relação ao solo. Mesmo no caso do solo, como temos uma rotação da Terra em torno do próprio eixo, não temos um referencial verdadeiramente inercial. Este impasse pode ser resolvido através da Primeira Lei de Newton, que veremos no próximo capítulo.

%%%%%%%%%%%%%%%%%%%%%%
%\section{Questionário}
%%%%%%%%%%%%%%%%%%%%%%

%\begin{question}[type={exam}]
%Uma questão de cinemática.
%\end{question}
