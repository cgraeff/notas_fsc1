%%%%%%%%%%%%%%%%%%%%%%%%%%%%%%%%%%%%%%%%%%%%%%%%%%%%%%%%%%%%%%%%%%%%%%%%%%%%%%%%
\chapter{Rotações}\label{Chap:Rotacoes}
%%%%%%%%%%%%%%%%%%%%%%%%%%%%%%%%%%%%%%%%%%%%%%%%%%%%%%%%%%%%%%%%%%%%%%%%%%%%%%%%

%\minitoc

%\clearpage

\begin{fullwidth}
{\it
No Capítulo~\ref{Chap:CentroDeMassaEMomentoLinear} verificamos que um corpo extenso pode ser considerado como uma partícula localizada no centro de massa, sendo que já haviamos tratado da translação da partícula nos demais capítulos anteriores. Considerar somente o movimento de translação --~isto é, o deslocamento no espaço~-- não descreve completamente o movimento de corpos extensos: sabemos que existe também a possibilidade de um movimento de rotação. Vamos agora analisar esse tipo de movimento dos pontos de vista de cinemática, dinâmica e energia. Inicialmente, vamos tratar de rotações puras, depois verificaremos um caso específico de movimento combinado de rotação e de translação, o rolamento.
}
\end{fullwidth}

%%%%%%%%%%%%%%%%%%%%
\section{Introdução}
%%%%%%%%%%%%%%%%%%%%

Em tese, já sabemos como tratar o movimento de um corpo rígido: basta tratar o movimento de cada partícula indepedentemente. Claramente isso é impossível, afinal temos uma quantidade muito grande de partículas mesmo em um corpo com dimensões pequenas. Verificamos que podemos descrever o movimento \emph{coletivo} de um corpo rígido devido ao fato de que todas as partículas apresentam velocidades idênticas, sendo que tal observação nos levou a definir o que chamamos de \emph{centro de massa} de um corpo: uma partícula ``virtual'' cuja massa, posição, velocidade, e aceleração substituem aquelas do corpo rígido em questão. Tal conceito é de tamanha utilidade, que definimos sua existência mesmo para um sistema de partículas que não constituem um corpo rígido. Existe, no entanto, um tipo de movimento que não pode ser descrito por uma partícula: \emph{as rotações}.

A rotação de um corpo rígido pode ser tratada como o movimento coletivo exibido por um sistema de partículas sujeitas a um conjunto de forças internas, de modo que suas trajetórias são curvilíneas. Novamente, tratar o movimento de cada partícula individualmente é impossível, uma vez que temos um número muito grande delas.

\begin{marginfigure}[-4cm]
\centering
\begin{tikzpicture}[>=Stealth, scale = 0.8]

    \draw[fill] (0,0) coordinate (O) circle (1pt);
    
    \foreach \x in {0, 60, 120, 180, 240, 300}
    {
        \draw[fill] (\x:1.5) circle (2pt);
        \draw[densely dotted] (O) -- (\x:2);
    }
        
    \begin{scope}[shift={(0,-4)}]
    
        \draw[fill] (0,0) coordinate (O) circle (1pt);
        
        \foreach \x in {0, 60, 120, 180, 240, 300}
        {
            \draw[fill = white, draw = black] (\x:1.5) circle (2pt);
            \draw[loosely dotted] (O) -- (\x:2);
        }
            
        \foreach \x in {10, 70, 130, 190, 250, 310}
        {
            \draw[fill] (\x:1.5) circle (2pt);
            \draw[densely dotted] (O) -- (\x:2);
        }
        
        \coordinate (R1) at (0:2);
        \coordinate (R1R) at (10:2);
        
        \pic[draw, "$\theta$", angle eccentricity = 1.2, angle radius = 14mm]{angle = R1--O--R1R};
    
    \end{scope}

\end{tikzpicture}
\caption{Para um grupo de partículas que constituem um corpo rígido, ao submetermos uma das partículas a uma rotação por um ângulo $\theta$, todas as demais partículas serão rotacionadas pelo mesmo ângulo, em torno do mesmo eixo.}
\end{marginfigure}

Entretanto, como estamos tratando de um corpo rígido, temos que em um movimento de rotação pura, as diversas partículas efetuam um movimento coletivo em torno de um \emph{eixo de rotação}\footnote{Esse resultado é geral para qualquer rotação, e é conhecido como Teorema de Euler para a Rotação.}. Verificamos então que existem variáveis coletivas para o movimento das partículas: como a distância relativa entre as diversas partículas é constante \emph{por definição}, se uma partícula qualquer efetua uma rotação por um ângulo $\theta$ em torno do eixo de rotação, todas as demais devem efetuar rotações com o mesmo ângulo $\theta$, em torno do mesmo eixo; A taxa de variação de tal ângulo em relação ao tempo, durante um movimento de rotação também deve ser comum a todas as partículas. Verificaremos na próxima seção que seremos capazes de determinar valores de \emph{posição, velocidade, e aceleração angulares} para corpos rígidos.

%\textbf{Rever abaixo, o teorema de mozzi-chasles vai ser ``discutido'' ao falar de CM, depois provado no final do capítulo de momento. A ideia é deixar todo o convencimento e explicação lá, aqui só destacar o uso do teorema e tocar pra frente (até pq o exemplo do bastão é mencionado lá).}

Finalmente, devemos lembrar que alguns movimentos não podem ser descritos como sendo simplesmente translações ou simplesmente rotações. No entanto, nas Seções~\ref{Sec:TeoremaDeMozziChasles} e~\ref{Sec:DeducaoTeoremaDeMozziChasles} verificamos que o movimento descrito pelo centro de massa do corpo equivale ao movimento que uma partícula executaria se fosse submetida ao conjunto de forças que atuam sobre o corpo. Já a rotação efetuada pelo corpo será aquela que ele efetuaria se os torques atuassem da mesma maneira, porém o corpo não transladasse. Os dois movimentos são, em tese, completamente independentes. Em alguns casos, no entanto, algum tipo de vínculo pode existir de maneira que tais movimentos ocorram de maneira sincronizada. Vericaremos, na Seção~\ref{Sec:MovCombRotTrans} alguns casos desse tipo.

\begin{marginfigure}
\centering
\begin{tikzpicture}[>=Stealth]
    
%    \draw[dotted, smooth, samples=1000, domain=0:3.33] plot ({1*\x}, {2*\x - 0.6*\x*\x});
    \draw[smooth, samples=1000, domain=0:3.1] plot ({1*\x-0.3*sin(8*\x r)}, {2*\x - 0.6*\x*\x - 0.3*cos(8*\x r)});
    
    \begin{scope}
        \draw [rotate = 90, densely dotted] (-1,-0.1) rectangle (1,0.1);
        \draw[fill] (0,-0.3) circle (0.6pt);
%        \draw[fill] (0,0) circle (0.8pt);
    \end{scope}

    \begin{scope}[shift = {(1.25,1.5625)}, rotate = 572.96]
        \draw [rotate = 90, densely dotted] (-1,-0.1) rectangle (1,0.1);
        \draw[fill] (0,-0.285) circle (0.6pt);
%        \draw[fill] (0,0) circle (0.8pt);
    \end{scope}
    
    \begin{scope}[shift = {(2.75,0.9625)}, rotate = 1260.51]
        \draw [rotate = 90] (-1,-0.1) rectangle (1,0.1);
        \draw[fill] (0,-0.3) circle (0.6pt);
%        \draw[fill] (0,0) circle (0.8pt);
    \end{scope}
        
\end{tikzpicture}
\caption{Quando arremessamos um bastão com uma velocidade de rotação, o movimento efetuado por cada partícula é bastante complexo. Cada uma das partículas estará sujeita a um conjunto de forças internas que será capaz de alterar a sua trajetória. \label{Fig:LancamentoObliquoCorpoGirando}}
\end{marginfigure}

\begin{marginfigure}
\centering
\begin{tikzpicture}[>=Stealth]
    
    \draw[dotted, smooth, samples=1000, domain=0:3.33] plot ({1*\x}, {2*\x - 0.6*\x*\x});
    \draw[smooth, samples=1000, domain=0:3.1] plot ({1*\x-0.3*sin(8*\x r)}, {2*\x - 0.6*\x*\x - 0.3*cos(8*\x r)});
    
    \begin{scope}
        \draw [rotate = 90] (-1,-0.1) rectangle (1,0.1);
        \draw[fill] (0,-0.3) circle (0.6pt);
        \draw[fill] (0,0) circle (0.8pt);
        
        \draw[->] (240:0.4) arc[start angle = 240, end angle = 120, radius = 0.4]node[left, midway]{$\omega$};
    \end{scope}

    \begin{scope}[shift = {(1.25,1.5625)}, rotate = 572.96]
        \draw [rotate = 90, gray] (-1,-0.1) rectangle (1,0.1);
        \draw[fill] (0,-0.285) circle (0.6pt);
        \draw[fill, gray] (0,0) circle (0.8pt);
    \end{scope}
    
    \begin{scope}[shift = {(2.75,0.9625)}, rotate = 1260.51]
        \draw [rotate = 90, gray] (-1,-0.1) rectangle (1,0.1);
        \draw[fill] (0,-0.3) circle (0.6pt);
        \draw[fill,gray] (0,0) circle (0.8pt);
    \end{scope}
        
\end{tikzpicture}
\caption{Segundo o Teorema de Mozzi-Chasles, todo movimento de um corpo rígido pode ser interpretado como uma rotação em torno do centro de massa e uma translação do centro de massa. No exemplo do arremesso de um bastão, o movimento do centro de massa continua sendo uma parábola, uma vez que $\vec{F}_R^{\rm{ext}} = \vec{P}$. \label{Fig:LancamentoObliquoCorpoGirandoDecomposto}}
\end{marginfigure}

Vamos iniciar o tratamento de rotações considerando um corpo rígido que gira em torno de um eixo cuja direção é fixa no espaço. Além disso, vamos considerar que a origem do sistema de coordenadas está fixada no centro de massa de tal corpo. Dessa maneira, seremos capazes de verificar alguns resultados importantes, incluindo a descrição cinemática, dinâmica, e de energia em rotações. Após isso, etilizando o Teorema de Mozzi-Chasles, poderemos descrever o movimento de rolamento simplesmente considerando um movimento combinado de rotação em torno de um eixo que passa pelo centro de massa e de uma translação do centro de massa. Posteriormente, no próximo capítulo, vamos tratar os casos onde a direção do eixo de rotação varia. Veremos que isso dá origem a fenômenos bastante interessantes.

%%%%%%%%%%%%%%%%%%%%%%%%%%%%%%%%%%%%%%%%%%%%%%%% 
\subsection{Variáveis cinemáticas para rotações}
%%%%%%%%%%%%%%%%%%%%%%%%%%%%%%%%%%%%%%%%%%%%%%%% 

De maneira similar ao que fizemos com a translação, vamos iniciar a análise de movimentos de rotação através da cinemática: precisamos encontrar variáveis que permitam descrever os movimentos de rotação. Obteremos variáveis análogas àquelas da translação, resultando em expressões equivalentes às obtidas na Seção~\ref{Sec:EqCinematicasParaAcelConstTransl} para o caso de uma translação unidimensional com aceleração constante. Verificaremos também que existe uma relação simples entre as variáveis de translação e de rotação para as partículas que compõe um corpo rígido. Para que possamos obter tais resultados, no entanto, devemos iniciar discutindo os métodos para medir ângulos, em especial as medidas em radianos. 

%%%%%%%%%%%%%%%%%%%%%%%%%%%%%%%%%%%%%%%%%%%%%
\paragraph{Unidades para a medida de ângulos}
%%%%%%%%%%%%%%%%%%%%%%%%%%%%%%%%%%%%%%%%%%%%%

Um ângulo $\theta$ pode ser descrito em qualquer unidade: revoluções, graus, grados\footnote{O grado faz parte da proposta inicial do SI, porém não encontrou grande aceitação cotidiana mesmo em países que de maneira geral adotam tal sistema.}, e também radianos. A primeira nada mais é do que o número de voltas efetuadas, sejam completas ou não. A segunda e a terceira unidades são divisões de um círculo em um número arbitário de partes: 360 e 400 partes, respectivamente.

Para medirmos um ângulo em revoluções, podemos contar o número de voltas completas realizadas, porém para uma volta incompleta é difícil de determinarmos exatamente qual é o ângulo em revoluções a partir de uma comparação com um ``padrão'', que nesse caso é uma volta completa. De maneira um tanto quanto intuitiva, no entanto, podemos imaginar frações de uma revolução completa: $\nicefrac{1}{2}$ revolução, $\nicefrac{1}{3}$ de revolução, $\nicefrac{1}{4}$ de revolução, $\nicefrac{1}{8}$ de revolução, etc. 

As medidas eu graus ou em grados são obtidas em comparação com um padrão pré-definido, porém menor do que uma revolução. De certa forma, continuamos adotando frações de um círculo, pois usamos múltiplos de uma pequena fração que corresponde a \degree{1}: $\nicefrac{1}{2}$ revolução equivale a $\np[\tcdegree]{360} / 2 = \np[\tcdegree]{180}$, $\nicefrac{1}{3}$ de revolução equivale a $\np[\tcdegree]{360} / 3 = \np[\tcdegree]{120}$, etc. Para a medição de um ângulo na prática, necessitamos usar um transferidor, ou seja, realizamos uma comparação com uma escala.

\begin{marginfigure}
\centering
\begin{tikzpicture}

\draw[dashdotted, -Stealth] (-2,0) -- (2,0) node[below left]{$x$};
\draw[dashdotted, -Stealth] (0,-2) -- (0,2) node[below left]{$y$};

\path[pattern = north east lines, pattern color = lightgray] (0,0) -- ([shift={(0,0)}]45:1.5) arc[radius=1.5, start angle=45, end angle= 0] -- cycle;

\draw[densely dotted] (0,0) circle (0.75cm);
\draw[densely dotted] (0,0) circle (1.5cm);
\draw[dashed] (0,0) -- (45:2);
\draw[thick] (0,0) -- (1.5,0);

\draw[|<->|] (0,-1) -- node[below]{$r_1$} (0.75,-1);
\draw[dotted] (0.75,0) -- (0.75,-1);
\draw[|<->|] (0.0,-1.75) -- node[below]{$r_2$} (1.5,-1.75);
\draw[dotted] (1.5,0) -- (1.5,-1.75);

\draw[thick] ([shift={(0,0)}]45:0.75) arc[radius=0.75, start angle=45, end angle= 0];
\node (s1) at (22.5:0.95){$s_1$};

\draw[thick] ([shift={(0,0)}]45:1.5) arc[radius=1.5, start angle=45, end angle= 0];
\node (s1) at (22.5:1.7){$s_2$};

\end{tikzpicture}
\caption{A figura acima mostra um \emph{setor} de um círculo, isto é, uma parte de um círculo. Note que o setor determina o ângulo entre a linha tracejada e o eixo $x$, sendo que podemos utilizar a razão entre o arco do setor e o seu raio para denotar o ângulo entre tais retas. \label{Fig:ExplicacaoAngRad}}
\end{marginfigure}

Já o radiano pode ser considerado como uma medida \emph{natural} para ângulos. Podemos entender como funciona esse sistema de medidas de ângulos através da Figura~\ref{Fig:ExplicacaoAngRad}: se tomarmos um setor\footnote{Uma fração de um círculo é denominado como um \emph{setor} de um círculo.} $f$ de um círculo ---~na figura tomamos $\nicefrac{1}{8}$~---, podemos calcular a razão entre o arco $s$ e o raio $r$ e obter
\begin{equation}
    \frac{s_1}{r_1} = \frac{f\cdot 2\pi r_1}{r_1},
\end{equation}
%
\begin{marginfigure}[2cm]
\centering
\begin{tikzpicture}
    \draw (-1.15, 0) -- (1.15, 0);
    \draw (0,-1.15) -- (0, 1.15);
    
    \draw[densely dotted] (0,0) coordinate (origin) circle (1);
    
    \draw[thick] (1,0) coordinate (right) arc[start angle = 0, end angle = 1 r, radius = 1] coordinate (top) -- (0,0);
    \draw[rotate = 1 r, |-|] (0,0.2) -- node[above]{$r$} (1,0.2);
    \draw[|-|] (1.3,0) arc[start angle = 0, end angle = 1 r, radius = 1.3] node [right, midway]{$\ell = r$};
    
    \pic [draw, "$\theta$", angle eccentricity = 1.5, angle radius = 3mm] {angle = right--origin--top};

\end{tikzpicture}
\caption{Um ângulo de \np[rad]{1,0} é o ângulo compreendido por um arco cujo comprimento é igual ao do raio do círculo, e equivale a aproximadamente \np[\tcdegree]{57.2958}.}
\end{marginfigure}

\noindent{}onde escrevemos o comprimento $s_1$ como uma fração do perímetro $p_1 = 2 \pi r_1$ do círculo\footnote{Se, por exemplo, temos $\nicefrac{1}{8}$ de um círculo, o arco $s$ correspondente é de $\nicefrac{1}{8}$ do perímetro total $2\pi r$ do círculo completo}. Temos então que 
\begin{equation}
    \frac{s_1}{r_1} = f\cdot 2\pi.
\end{equation}
%
Note que a fração não depende do raio do círculo. Se fizermos o mesmo cálculo utilizando o círculo com raio $r_2$, obteremos o mesmo resultado: a razão entre o arco e o raio depende simplesmente da fração $f$. O resultado da razão acima serve como uma medida do ângulo determinado pelo setor do círculo, sendo que a denominamos como um ângulo \emph{em radianos}:
\begin{equation}\label{Eq:DefRadianos}
    \theta = \frac{s}{r}.
\end{equation}
%
Para o caso específico do setor mostrado na Figura~\ref{Fig:ExplicacaoAngRad}, temos
\begin{align}
    \theta &= \frac{s}{r} \\
    &= \frac{\nicefrac{1}{8} \cdot 2\pi r}{r} \\
    &= \nicefrac{\pi}{4},
\end{align}
%
o seja, temos um ângulo $\theta = \nicefrac{\pi}{4}~\rm{rad}$, o que correspode a $\theta = \nicefrac{\np[\tcdegree]{360}}{8} = \np[\tcdegree]{45}$.

Note que em uma rotação completa, quando $f = 1$, a reta tracejada na Figura~\ref{Fig:ExplicacaoAngRad} descreve uma circunferência completa. Isto é, temos uma revolução. Assim, $f$ é um ângulo em revoluções, e a relação entre o ângulo em radianos e o ângulo em revoluções é
\begin{equation}
    \theta = f \cdot 2 \pi.
\end{equation}
%
Para o ângulo em particular da Figura~\ref{Fig:ExplicacaoAngRad}, temos $\theta = \nicefrac{1}{8}~\rm{rev}$.

Veremos adiante que as relações entre as variáveis de rotação e de translação só serão válidas para ângulos medidos em radianos: podemos verificar da definição dada pela Equação~\eqref{Eq:DefRadianos} que $\theta$ é adimensional, uma vez que temos uma razão entre duas distâncias. As relações entre as variáveis dependerão da adimensionalidade dos ângulos, uma vez que originam na própria Equação~\eqref{Eq:DefRadianos}. Destacamos ainda que no SI a unidade de ângulo é o radiano.


%%%%%%%%%%%%%%%%%%%
\paragraph{Posição}
%%%%%%%%%%%%%%%%%%%

Podemos tomar uma reta fixa no objeto e que faz um ângulo de \np[\tcdegree]{90} em relação ao eixo de rotação e utilizá-la para determinar a \emph{posição angular}: o ângulo $\theta$ entre tal reta e o eixo $x$ pode ser usado como variável para descrever a posição angular do objeto. Na Figura~\ref{Fig:DefPosAngular} mostramos como exemplo a aplicação de tal conceito a um disco que gira em torno do eixo $z$ que passa pelo centro de massa, perpendicularmente às suas faces planas.

% https://tex.stackexchange.com/questions/123158/tikz-using-the-ellipse-command-with-a-start-and-end-angle-instead-of-an-arc
\tikzset{
    partial ellipse/.style args={#1:#2:#3}{
        insert path={+ (#1:#3) arc (#1:#2:#3)}
    }
}

\begin{marginfigure}
\centering
\begin{tikzpicture}[>=Stealth, scale = 1.2,
     interface/.style={
        % superfície
        postaction={draw,decorate,decoration={border,angle=-45,
                    amplitude=0.2cm,segment length=2mm}}},
    ]

%%% Figura superior

\draw (0,0) ellipse (1.25 and 0.5);

\draw (-1.25,0) -- (-1.25,-0.4);
\draw (1.25,-0.4) -- (1.25,0);  

\draw (-1.25,-0.4) arc (180:360:1.25 and 0.5);
\draw[densely dotted] (-1.25,-0.4) arc (180:360:1.25 and -0.5);

\draw[dashdotted,->] (0,0) -- (0,1.5) node[below left]{$z$};
\draw[dashdotted] (0,-1.5) -- (0,-0.9);
\draw[dotted] (0,-0.9) -- (0,0);
\draw[fill] (0,0) circle (1pt);

\draw[thick] (0,0) -- (15:-1.05);
\draw[dashdotted, ->] (-15:-1.5) -- (-15:2) node[below left]{$y$};
\draw[dashdotted, <-] (15:-2) node[below left]{$x$} -- (15:-1.05);
\draw[dashdotted] (0:0) -- (15:1.5);


%%% Figura do meio

\draw (0,-4) ellipse (1.25 and 0.5);

\draw (-1.25,-4) -- (-1.25,-4.4);
\draw (1.25,-4.4) -- (1.25,-4);  

\draw (-1.25,-4.4) arc (180:360:1.25 and 0.5);
\draw[densely dotted] (-1.25,-4.4) arc (180:360:1.25 and -0.5);

\draw[dashdotted,->] (0,-4) -- (0,-2.5) node[below left]{$z$};
\draw[dashdotted] (0,-5.5) -- (0,-4.9);
\draw[dotted] (0,-4.9) -- (0,-4);
\draw[fill] (0,-4) circle (1pt);

\draw[thick] (0,-4) -- +(-135:0.65);
\draw[dashed] (0,-4)+(-135:0.65) -- +(-135:1.7);
\draw[dashdotted, ->] (0,-4)+(-15:-1.5) -- +(-15:2) node[below left]{$y$};
\draw[dashdotted, ->] (0,-4)+(15:1.5) -- +(15:-2) node[below left]{$x$};

\draw (0,-4) [partial ellipse=-155:-121:1.82cm and 1.1cm];
\node[right] (theta) at (-1.65,-4.9){$\theta$};

\draw[->] (0,-3.20) [partial ellipse=-260:-30:0.3125cm and 0.125cm];

\end{tikzpicture}
\caption{Podemos descrever a posição angular de um corpo rígido que gira em torno de um eixo através do ângulo formado entre uma linha fixa no corpo e um dos eixos coordenados perpendicular ao eixo de rotação. \label{Fig:DefPosAngular}}
\end{marginfigure}

%%%%%%%%%%%%%%%%%%%%%%%%%%%%%%%%
\paragraph{Deslocamento angular}
%%%%%%%%%%%%%%%%%%%%%%%%%%%%%%%%

\begin{marginfigure}
\centering
\begin{tikzpicture}[>=Stealth, scale = 1.2,
     interface/.style={
        % superfície
        postaction={draw,decorate,decoration={border,angle=-45,
                    amplitude=0.2cm,segment length=2mm}}},
    ]

\draw (0,0) ellipse (1.25 and 0.5);

\draw (-1.25,0) -- (-1.25,-0.4);
\draw (1.25,-0.4) -- (1.25,0);  

\draw (-1.25,-0.4) arc (180:360:1.25 and 0.5);
\draw[densely dotted] (-1.25,-0.4) arc (180:360:1.25 and -0.5);

\draw[dashdotted,-Stealth] (0,0) -- (0,1.5) node[below left]{$z$};
\draw[dashdotted] (0,-1.5) -- (0,-0.9);
\draw[dotted] (0,-0.9) -- (0,0);
\draw[fill] (0,0) circle (1pt);

\draw[dashed] (0,0) -- +(-135:1.95);
\draw[dashdotted, -Stealth] (0,0)+(-15:-1.5) -- +(-15:2) node[below left]{$y$};
\draw[dashdotted, -Stealth] (0,0)+(15:1.5) -- +(15:-2) node[below left]{$x$};

\draw[<->] (0,0) [partial ellipse=-155:-121:1.82cm and 1.1cm];
\node[right] (theta) at (-1.65,-0.9){$\theta_i$};

\draw[-Stealth] (0,0.8) [partial ellipse=-260:-30:0.3125cm and 0.125cm];

\draw[thick] (0,0) -- +(-60:0.55);
\draw[dashed] (0,0)+(-60:0.55) -- +(-60:1.8);
\draw[<->] (0,0) [partial ellipse=-155:-71:1.72cm and 1.0cm];
\node[right] (theta) at (-0.85,-1.14){$\theta_f$};

\draw[<->] (-135:1.75) arc[start angle = -135, end angle = -61, x radius = 1.7, y radius = 0.7];
\node (Dth) at (-0.3, -1.62) {$\Delta\theta$};

\end{tikzpicture}
\caption{O deslocamento angular pode ser descrito como a diferença entre duas posições angulares $\theta_f$ e $\theta_i$.}
\end{marginfigure}

Conhecendo a posição angular, podemos calcular o deslocamento angular de maneira bastante simples, bastando calcular a diferença entre duas posições quaisquer:
\begin{equation}
	\Delta\theta = \theta_2 - \theta_1.
\end{equation}

%%%%%%%%%%%%%%%%%%%%%%%%%%%%%%
\paragraph{Velocidade angular}
%%%%%%%%%%%%%%%%%%%%%%%%%%%%%%

A partir do deslocamento angular, podemos definir uma velocidade angular média através de
\begin{equation}
	\mean{\omega} = \frac{\Delta\theta}{\Delta t},
\end{equation}
%
de onde podemos tomar o limite de $\Delta t$ tendendo a zero para definir a velocidade instantânea:
\begin{equation}
	\omega = \lim_{\Delta t \to 0} \frac{\Delta\theta}{\Delta t} \equiv \frac{d\theta}{dt}.
\end{equation}
%
As unidades da velocidade angular serão as de ``ângulo por tempo'', em unidades do SI, $\textrm{rad}/\textrm{s}$.

%%%%%%%%%%%%%%%%%%%%%%%%%%%%%%
\paragraph{Aceleração angular}
%%%%%%%%%%%%%%%%%%%%%%%%%%%%%%

Conhecendo a velocidade angular, podemos definir a aceleração angular média através de
\begin{equation}
	\mean{\alpha} = \frac{\Delta \omega}{\Delta t},
\end{equation}
%
o que nos leva à definição de aceleração angular instantânea através de
\begin{equation}
	\alpha = \lim_{\Delta t \to 0} \frac{\Delta\omega}{\Delta t} \equiv \frac{d\omega}{dt}.
\end{equation}
%
A aceleração angular tem unidade de ``ângulo por tempo ao quadrado'', no SI, $\textrm{rad}/\textrm{s}^2$.

%%%%%%%%%%%%%%%%%%%
\subsection{Sinais}
%%%%%%%%%%%%%%%%%%%

No caso da translação, a escolha da direção e sentido do sistema de coordenadas era livre. Muitas vezes só uma escolha de direções é adequada, mas a escolha do sentido dos eixos é algo livre\footnote{Exceto no caso do cálculo da energia potencial associada à força peso, onde determinamos uma expressão específicas para o caso em que o eixo vertical tem como sentido positivo aquele no sentido inverso ao da aceleração da gravidade.}. Para as rotações, no entanto, existe uma convenção para o sentido positivo: rotações no sentido anti-horário são positivas.\footnote{É claro que se ``olharmos'' o sistema pelo lado oposto, veremos uma rotação no sentido oposto, o que significa que o sentido é arbitrário. No entanto, precisamos que rotações em sentidos opostos tenham sinais opostos, o que é garantido por tal convenção.} Essa definição é a mesma utilizada para o círculo trigonométrico.

Assim, considerando os sinais para as variáveis cinemáticas da rotação, temos que:
\begin{description}
    \item[Posição angular:] Uma posição angular é positiva se ao partimos do eixo de de referência indo em direção à posição da marca de referência  obtemos um deslocamento no sentido anti-horário;
    \item[Deslocamento angular:] Os deslocamentos positivos são aqueles que ocorrem no sentido anti-horário;
    \item[Velocidade angular:] Uma velocidade angular é positiva se ela tem o sentido anti-horário, isto é, se o deslocamento angular associado a tal velocidade é no sentido anti-horário;
    \item[Aceleração:] Já no caso da aceleração angular temos uma situação mais complexa, similar ao que verificamos no movimento unidimensional:
    \begin{itemize}
        \item Se uma aceleração angular causa um aumento do módulo da velocidade angular, ela tem o mesmo sinal que a velocidade angular;
        \item Se uma aceleração angular causa uma diminuição do módulo da velocidade angular, ela tem o sinal oposto ao da velocidade angular.
    \end{itemize}
%
Se, por exemplo, temos que um corpo efetua uma rotação no sentido horário, com velocidade angular que cresce em módulo, temos que a aceleração angular é também no sentido horário, por isso ela deve ser negativa, assim como a velocidade angular. Essa análise é importante pois para qualquer dos dois sentidos de rotação podemos ter uma aceleração positiva ou negativa, sendo que o papel de cada uma delas é diferente para cada caso.
\end{description}

%%%%%%%%%%%%%%%%%%%%%%%%%%%%%%%%%%%%%%%%%%%%%%%%%%%%%%%
\subsection{Equações para aceleração angular constante}
%%%%%%%%%%%%%%%%%%%%%%%%%%%%%%%%%%%%%%%%%%%%%%%%%%%%%%%

Ao estudar movimentos de translação, nos preocupamos com o caso da aceleração constante pois pretendíamos estudar um caso importante que pode ser descrito desta maneira: a aceleração gravitacional. No caso das rotações, supor que a aceleração seja constante não é algo muito geral ou mesmo de especial interesse no tratamento de sistemas físicos reais. No entanto, é interessante mostrar que as equações têm a mesma forma que no caso da translação.\emph{E, pra ser honesto, é o caso mais simples possível com algum valor de aceleração não nulo.}

Da própria definição da aceleração angular instantânea, temos
\begin{equation}
	d\omega = \alpha dt,
\end{equation}
%
que pode ser integrada entre valores iniciais e finais de velocidade angular e de tempo, obtendo
\begin{equation}
	\int_{\omega_i}^{\omega_f} d\omega = \int_{t_i}^{t_f} \alpha dt.
\end{equation}
%
Se a aceleração angular é constante, podemos retirá-la da integral:
\begin{equation}
		\int_{\omega_i}^{\omega_f} d\omega = \int_{t_i}^{t_f} \alpha dt.
\end{equation}
%
As integrais que restam correspondem a $\omega_f - \omega_i$ e $t_f - t_i$, o que nos permite escrever
\begin{equation}
	\omega_f = \omega_i + \alpha\Delta t.
\end{equation}
%
Adotando $t_f = t$ e $t_i = 0$, temos 
\begin{equation}\label{Eq:VelAngParaAcelConst}
	\omega_f = \omega_i + \alpha t.
\end{equation}
%
Podemos perceber que no caso de uma rotação com aceleração angular constante, obtivemos uma equação para a velocidade angular que é análoga aquela para o caso da translação.

Voltando à definição de velocidade, 
\begin{equation}
	d\theta = \omega dt,
\end{equation}
%
e utilizando a Equação~\ref{Eq:VelAngParaAcelConst} acima, podemos escrever,
\begin{equation}
	d\theta = (\omega_i + \alpha t) dt.
\end{equation}
%
Integrando entre $\theta_i$ e $\theta_f$ do lado esquerdo e entre $t_i$ e $t_f$ do lado direito, obtemos
\begin{align}
	\Delta \theta &= \int_{t_i}^{t_f} \omega_i + \alpha t dt \\
	&= \omega_i [t + C]_{t_i}^{t_f} + \left[\frac{\alpha t^2}{2} + C\right]_{t_i}^{t_f} \\
	&= \omega_i[t_f - t_i] + \frac{\alpha}{2}[t_f^2 + t_i^2].
\end{align}
%
Se tomarmos $t_i = 0$ e $t_f = t$, obtemos finalmente
\begin{equation}
	\theta_f = \theta_i + \omega_i t +\frac{\alpha t^2}{2}.
\end{equation}
%
Novamente temos uma equação que é anóloga àquela do caso translacional.
	
%%%%%%%%%%%%%%%%%%%%%%%%%%%%%%%%%%%%%%%%%%%%%
\paragraph{Analogia com o caso translacional}
%%%%%%%%%%%%%%%%%%%%%%%%%%%%%%%%%%%%%%%%%%%%%

Para cada relação da cinemática translacional, temos uma correspondente para o caso rotacional. Na seção acima, utilizamos cálculo para determinar duas dessas equações. Esse método é, na verdade, equivalente ao cálculo de áreas feito para o caso translacional. A partir dessas equações, podemos determinar outras, como fizemos no caso da translação. Na Tabela~\ref{Tab:CompEqsTransRot} podemos ver as equações lado a lado, evidenciando quais equações têm a mesma forma.

\begin{table}[!h]\forcerectofloat
\centering
\caption{Comparação entre as equações para aceleração constante nos casos da cinemática da translação e da rotação.\label{Tab:CompEqsTransRot}}
\begin{tabular}{ll}
\toprule
Translação & Rotação \\
\midrule
$v_f = v_i + at$ & $\omega_f = \omega_i + \alpha t$ \\
$x_f = x_i + v_i t + at^2 /2$ & $\theta_f = \theta_i + \omega_i t + \alpha t^2 / 2$ \\
$v_f^2 = v_i^2 + 2 a \Delta x$ & $\omega_f^2 = \omega_i^2 + 2\alpha \Delta\theta$ \\
$\Delta x = (v_i + v_f)\, t/2$ & $\Delta\theta = (\omega_i + \omega) \, t / 2$ \\
$x_f = x_i + v_ft - at^2 / 2$ & $\theta_f = \theta_i + \omega_f t - \alpha t^2 / 2$ \\
\bottomrule
\end{tabular}
\end{table}

%%%%%%%%%%%%%%%%%%%%%%%%%%%%%%%%%%%%%%%%%%%%%%%%%%%%%%%%%%%%%%
\paragraph{Exemplo: Deslocamento angular de um motor elétrico}
%%%%%%%%%%%%%%%%%%%%%%%%%%%%%%%%%%%%%%%%%%%%%%%%%%%%%%%%%%%%%%

\begin{quote}
    Um esmeril é ligado, utilizado por 20 segundos, e então desligado. Sabendo que a velocidade máxima que ele é capaz de atingir é de \np[rpm]{3000}, que ele atinge tal velocidade em \np[s]{3,2}, e que ele demora \np[s]{20} para parar completamente, determine qual é o seu deslocamento angular total. Assuma que as acelerações ao ligar e ao desligar o motor sejam constantes.
\end{quote}

Precisamos dividir o cálculo em três partes: a aceleração do motor, o período em que ele trabalha com velocidade constante, e o período de desaceleração. No primeiro período, temos que
\begin{align}
    \Delta \theta &= \frac{\omega_i + \omega_f}{2} t \\
    &= \frac{0 + (\np[rpm]{3000})}{2} \cdot(\np[s]{3,2}) \\
    &= \frac{(\np[rpm]{3000})}{2} \cdot(\np[s]{3,2}).
\end{align}
%
Note que as unidades na expressão acima não estão consistentes, pois estamos medindo o tempo em segundos e em minutos, e devemos escolher uma das duas unidades:
\begin{equation*}
    \frac{\np[min]{1}}{\np[s]{60}} \cdot \np[s]{3,2} = \frac{\np{3,2}}{60}~\textrm{min} = \np[min]{53E-3}.
\end{equation*}
%
Assim,
\begin{equation}
    \Delta \theta = \np[rev]{80}.
\end{equation}

Na segunda parte, temos um movimento com velocidade angular constante, logo
\begin{align}
    \Delta \theta &= \omega t \\
    &= (\np[rpm]{3000})\cdot(\np[s]{20}) \\
    &= (\np[rpm]{3000})\cdot(\np[min]{0,33}) \\
    &= \np[rev]{1000}.
\end{align}
%
No cálculo acima utilizamos $\np[s]{20} = 1/3~\textrm{min} \approx \np[min]{0,33}.$

Finalmente, durante a desaceleração temos
\begin{align}
    \Delta \theta &= \frac{\omega_i + \omega_f}{2} t \\
    &= \frac{(\np[rpm]{3000}) + 0}{2} \cdot(\np[s]{20}) \\
    &= \frac{(\np[rpm]{3000})}{2} \cdot(\np[min]{0,33}) \\
    &= \np[rev]{5,0E2}.
\end{align}
%
Temos, portanto, um deslocamento angular total $\Delta \theta = \np[rev]{1,6e3}.$

%%%%%%%%%%%%%%%%%%%%%%%%%%%%%%%%%%%%%%%%%%%%%%%%%%%%%%%%%%%%%%%
\subsection{Relação entre variáveis de translação e de rotação}
%%%%%%%%%%%%%%%%%%%%%%%%%%%%%%%%%%%%%%%%%%%%%%%%%%%%%%%%%%%%%%%

Se considerarmos um ponto qualquer de um corpo rígido, ao tratá-lo individualmente, as variáveis cinemáticas relevantes são aquelas para o movimento de translação que estudamos anteriormente. Veremos agora que podemos determinar uma relação entre tais variáveis e aquelas para o movimento coletivo de rotação efetuado por todas as partículas do corpo.

%%%%%%%%%%%%%%%%%%%%%%%%%%%%%%%%%%%%%%%%%%%%%%%%%%%%%%%%%
\paragraph{Velocidade angular e velocidade de translação}
%%%%%%%%%%%%%%%%%%%%%%%%%%%%%%%%%%%%%%%%%%%%%%%%%%%%%%%%%

Podemos determinar uma relação entre a velocidade angular e a velocidade de translação ao considerarmos a Figura~\ref{Fig:RelVelocidades}. Uma partícula do corpo, localizada a uma distância $r_\perp$ do eixo de rotação\footnote[][-4cm]{A distância entre a partícula e o eixo de rotação deve ser medida ao longo de uma reta \emph{perpendicular} a este eixo.}, descreve um arco de comprimento $s$ enquanto o corpo gira por um ângulo $\theta$. O comprimento do arco descrito pela partícula durante a rotação é dado por
\begin{equation}
    s = \theta r.
\end{equation}

\begin{marginfigure}[-3cm]
\centering
\begin{tikzpicture}[>=Stealth, scale = 1.4,
     interface/.style={
        % superfície
        postaction={draw,decorate,decoration={border,angle=-45,
                    amplitude=0.2cm,segment length=2mm}}},
    ]

%%% Figura superior

\draw (0,0) ellipse (1.25 and 0.5);

\draw (-1.25,0) -- (-1.25,-0.4);
\draw (1.25,-0.4) -- (1.25,0);  

\draw (-1.25,-0.4) arc (180:360:1.25 and 0.5);
%\draw[densely dotted] (-1.25,-0.4) arc (180:360:1.25 and -0.5);

\draw[dashdotted,->] (0,0) -- (0,1.5) node[below left]{$z$};
\draw[dashdotted] (0,-1.5) -- (0,-0.9);
\draw[dotted] (0,-0.9) -- (0,0);
\draw[fill] (0,0) circle (1pt);

% raio
\draw[|-|] (-0.05,0.15) -- node[above]{$r_\perp$} +(15:-0.85);
\draw[fill] (15:-0.85) circle (1pt);

% velocidade
\draw[dashed] ellipse (1 and 0.35);
\draw[->] (15:-0.85) -- +(-30:0.4) node[below]{$\vec{v}$};

\draw[->] (0,1.05) [partial ellipse=-225:60:0.3cm and 0.125cm];
\node (VelAng) at (0.3, 0.85) {$\omega$};
\end{tikzpicture}
\caption{Quando um corpo rígido executa uma revolução em torno de um eixo, cada ponto do corpo descreve um movimento circular com uma velocidade $\vec{v}$. A direção da velocidade se altera a cada ponto, sendo sempre tangente à trajetória circular. Já o módulo da velocidade está ligado à velocidade angular do corpo rígido e também à distância ao eixo de rotação. \label{Fig:RelVelocidades}}
\end{marginfigure}

A partir dessa equação simples, podemos encontrar a relação entre a velocidade de translação e a velocidade angular da partícula fazendo uma derivada em relação ao tempo:\footnote{Note que no limite $\Delta t \to 0$ o arco $s$ corresponde à distância percorrida pela partícula na direção tangencial ao círculo tracejado, ou seja, na direção do deslocamento e da velocidade instantâneos.}
\begin{align}
	v &= \frac{ds}{dt} \\
	&= \frac{d(\theta r_\perp)}{dt} \\
	&= r_\perp\frac{d\theta}{dt} \\
	&= r_\perp\omega,
\end{align}
%
onde assumimos que $r_\perp$ seja constante.


Uma aplicação interessante da relação acima é a determinação da relação entre as velocidades angulares de duas polias ligadas por uma correia inextensível, como ilustrado na Figura~\ref{Fig:RelVelPoliasLigadasPorCorreia}. Se não ocorre deslizamento entre as polias e a correia, então as velocidades dos pontos das bordas de ambas as polias são iguais à velocidade da própria correia:
\begin{equation}
    v_1 = v_2 = v_c.
\end{equation}
%
Usando a relação entre a velocidade angular e a velocidade de translação para um ponto na borda da polia, temos
\begin{equation}
    \omega_1 R_1 = \omega_2 R_2,
\end{equation}
%
onde utilizamos $R_1$ e $R_2$ para denotar os raios.

\begin{marginfigure}[-2cm]
\centering
\begin{tikzpicture}[>=Stealth,
     interface/.style={
        % superfície
        postaction={draw,decorate,decoration={border,angle=-45,
                    amplitude=0.2cm,segment length=2mm}}},
    ]


    \draw[pattern = north west lines] (0,0) circle (0.5cm);
    \draw[pattern = north west lines] (3,0) circle (1cm);
    
    \path[name path=up_polia1] (-0.5,0) arc[start angle = 180, end angle = 0, radius = 0.5];
    \path[name path=down_polia1] (-0.5,0) arc[start angle = 180, end angle = 360, radius = 0.5];
    
    \path[name path=up_polia2] (2,0) arc[start angle = 180, end angle = 0, radius = 1];
    \path[name path=down_polia2] (2,0) arc[start angle = 180, end angle = 360, radius = 1];
    
    \path[name path=up_from_origin_1] (0,0)--+(-0.1,1);
    \path[name path=down_from_origin_1] (0,0)--+(-0.1,-1);
    \path[name path=up_from_origin_2] (3,0)--+(-0.3,1.5);
    \path[name path=down_from_origin_2] (3,0)--+(-0.3,-1.5);

    \path[name intersections={of=up_polia1 and up_from_origin_1}] (intersection-1) coordinate (a1);
    \path[name intersections={of=down_polia1 and down_from_origin_1}] (intersection-1) coordinate (b1);
    
    \path[name intersections={of=up_polia2 and up_from_origin_2}] (intersection-1) coordinate (a2);
    \path[name intersections={of=down_polia2 and down_from_origin_2}] (intersection-1) coordinate (b2);
    
    \draw[thick] (a1) -- (a2);
    \draw[thick] (b1) -- (b2);
    
    \fill (0,0) circle (1pt);
    \fill (3,0) circle (1pt);
    
    
\end{tikzpicture}
\caption{Polias ligadas por uma correia.\label{Fig:RelVelPoliasLigadasPorCorreia}}
\end{marginfigure}

%%%%%%%%%%%%%%%%%%%%%%%%%%%%%%%%%%%%%%%%%%%%%%%%%%%%%%
\paragraph{Velocidade angular e aceleração centrípeta}
%%%%%%%%%%%%%%%%%%%%%%%%%%%%%%%%%%%%%%%%%%%%%%%%%%%%%%

Como a trajetória da partícula é circular, sabemos que ela deve ter uma aceleração centrípeta, mesmo que sua velocidade seja constante em módulo. Temos que tal aceleração é dada por
\begin{equation}
	a_c = \frac{v^2}{r},
\end{equation}
%
ou, substituindo a relação entre $v$ e $\omega$ que acabamos de obter,
\begin{equation}
	a_c = \omega^2 r_\perp.
\end{equation}

Se, no entanto, tivermos uma variação da velocidade angular, sabemos que o movimento possui uma aceleração angular. Derivando a equação $v = \omega r_\perp$ em relação ao tempo, temos
\begin{align}
	\frac{dv}{dt} &= \frac{d(\omega r_\perp)}{dt} \\
	&=r_\perp \frac{d\omega}{dt},
\end{align}
%
onde assumimos novamente que $r_\perp$ seja constante. Sabemos que $a=dv/dt$ e que $\alpha = d\omega/dt$, então
\begin{equation}
	a = \alpha r_\perp.
\end{equation}

%%%%%%%%%%%%%%%%%%%%%%%%%%%%%%%%%%%%%%%%%%%%%%%%%%%%%%
\paragraph{Aceleração angular e aceleração tangencial}
%%%%%%%%%%%%%%%%%%%%%%%%%%%%%%%%%%%%%%%%%%%%%%%%%%%%%%

A aceleração acima é responsável por variar o módulo da velocidade somente, já que se $\omega$ for constante, $d\omega/dt = 0$ e ---~consequentemente~--- $a=0$. Já haviamos identificado o efeito distinto das componentes radial e tangencial da aceleração ao estudarmos o movimento circular: a primeira é responsável pela variação da direção da velocidade, enquanto a segunda é responsável pela variação do módulo da velocidade. Concluímos que na equação acima se trata da segunda:
\begin{equation}
	a_t = \alpha r_\perp.
\end{equation}

%%%%%%%%%%%%%%%%%%%%%%%%%%%%%%%%%%%%%%%%%%%%%%%%%%%%%
\paragraph{Discussão: Quebra de uma chaminé que tomba}
%%%%%%%%%%%%%%%%%%%%%%%%%%%%%%%%%%%%%%%%%%%%%%%%%%%%%

\begin{marginfigure}
\centering
\begin{tikzpicture}[>=Stealth,
     interface/.style={
        % superfície
        postaction={draw,decorate,decoration={border,angle=-45,
                    amplitude=0.2cm,segment length=2mm}}},
    ]


    \draw[interface] (-2,0) -- (2,0);
    
    \draw (-0.05,0) rectangle (0.05,3);
    \fill[gray] (-0.05,0) rectangle (0.05, 0.5);
    \fill[gray] (-0.05, 1) rectangle (0.05, 1.5);
    \fill[gray] (-0.05, 2) rectangle (0.05, 2.5);

    \draw (-0.2,0) -- (-0.2, 1.5) -- (-2,1.5);
    \draw (-2, 1.5) -- (-1.4, 1.8) -- (-1.4, 1.5) -- (-0.8, 1.8) -- (-0.8, 1.5) -- (-0.2, 1.8) -- (-0.2,1.5); 
    
    \begin{scope}[rotate = -30]
        \draw[dashed] (-0.05,0) rectangle (0.05,3);
        \fill[lightgray] (-0.05,0) rectangle (0.05, 0.5);
        \fill[lightgray] (-0.05, 1) rectangle (0.05, 1.5);
        \fill[lightgray] (-0.05, 2) rectangle (0.05, 2.5);
    \end{scope}
    
\end{tikzpicture}
\caption{Uma chaminé que tomba tem a tendência a se partir antes de chegar ao solo devido às diferentes acelerações tangenciais ao longo de seu comprimento.}
\end{marginfigure}

Um efeito bastante interessante da equação acima é que quando uma chaminé feita de tijolos tomba, pode ocorrer uma quebra antes de ela atingir o solo. Qualquer estrutura que sofre uma queda, sendo que sua base permanece no mesmo ponto de apoio, estará sujeita a acelerações tangenciais diferentes ao longo de sua extensão: os pontos próximos à base têm acelerações muito pequenas, enquanto os pontos próximos da extremidade oposta têm acelerações tangenciais muito mais altas. Essa diferença entre as acelerações de cada um dos pontos só são possíveis devido às \emph{forças de cisalhamento}\footnote{Forças de cisalhamento são forças que são aplicadas a um corpo em direções paralelas e em sentidos diferentes, com uma pequena distância entre as direções de aplicação. É o que uma tesoura faz; isso tem a tendência de cortar o corpo.} que surgem entre as diversas partes que compõe a estrutura.

Se imaginarmos uma chaminé feita de tijolos,\footnote{Uma torre alta feita de Lego se parte ao tombar pelo mesmo motivo.} sabemos que cada um dos tijolos estará sujeito a uma aceleração diferente, e que para que isso ocorra deve haver uma força de cisalhamento exercida pela argamassa que os une. No entanto, apesar de esse material ter uma grande resistência a ser comprimido, ele não resiste bem a forças que tendem a o esticar ou a o cisalhar. Por isso é comum que o material não seja capaz de resistir às forças e ocorra uma quebra. Se o material for flexível, como uma árvore alta, veremos uma flexão ao longo de seu comprimento durante queda, ao invés de uma quebra.\footnote{Essa flexão também se deve à força de arrasto oferecida pelo ar.}

%%%%%%%%%%%%%%%%%%%%%%%%%%%%%
%%%%%%%%%%%%%%%%%%%%%%%%%%%%%
\section{Dinâmica da rotação}
%%%%%%%%%%%%%%%%%%%%%%%%%%%%%
%%%%%%%%%%%%%%%%%%%%%%%%%%%%%

Uma vez descritas as variáveis que caracterizam a cinemática da rotação, vamos agora nos preocupar com a dinâmica da rotação. Vamos iniciar definindo uma grandeza que cumprirá na rotação um papel análogo ao da força em uma translação, isto é, causar uma aceleração. Verificaremos que as expressões para a dinâmica da rotação também seguem um paralelo com equações equivalentes para a translação.

%%%%%%%%%%%%%%%%%%%
\subsection{Torque}
%%%%%%%%%%%%%%%%%%%

Experimentalmente podemos verificar que para abrir uma porta é muito mais fácil empurrar a extremidade mais distante das dobradiças do que o meio da porta, ou próximo das dobradiças. Além disso o ângulo de aplicação da força também é relevante. Se o ângulo entre a força e o plano da porta for de \degree{90}, é mais fácil empurrar a porta do que se ele for de \degree{30}.

\begin{marginfigure}[-2cm]
\centering
\begin{tikzpicture}[>=Stealth, scale = 1.4,
     interface/.style={
        % superfície
        postaction={draw,decorate,decoration={border,angle=-45,
                    amplitude=0.2cm,segment length=2mm}}},
    ]

%%% Figura superior

\draw (0,0) ellipse (1.25 and 0.5);

\draw (-1.25,0) -- (-1.25,-0.4);
\draw (1.25,-0.4) -- (1.25,0);  

\draw (-1.25,-0.4) arc (180:360:1.25 and 0.5);
\draw[densely dotted] (-1.25,-0.4) arc (180:360:1.25 and -0.5);

\draw[dashdotted,->] (0,0) -- (0,1.5) node[below left]{$z$};
\draw[dashdotted] (0,-1.5) -- (0,-0.9);
\draw[dotted] (0,-0.9) -- (0,0);
\draw[fill] (0,0) circle (1pt);

\draw[dashdotted] (15:-2) -- (0,0);

% raio
\draw[|-|] (-0.05,0.15) -- node[above]{$r_\perp$} +(15:-0.8);
\draw[fill] (15:-0.8) circle (1pt);

% força
\draw[thick, ->] (15:-0.8) -- +(-65:1) node[below]{$\vec{F}$};

% projeção radial
\draw[dashed] (15:-0.8) ++(-65:1) -- (15:-1.6); 
\draw[->] (15:-0.8) -- (15:-1.6) node[above]{$F_r$};

% projeção tangencial
\draw[dashed] (15:-0.8) +(-30:1.6) -- +(-30:-0.3);
\draw[dashed] (15:-0.8) ++(-65:1) -- +(18:0.8);
\draw[->] (15:-0.8) -- +(-30:1.35) node[above]{$F_t$};

% phi
\draw (15:-1.2) arc (150:117:1.25 and -0.5);
\node (theta) at (30:-1.1) {$\phi$};

\end{tikzpicture}
\caption{Ao exercermos uma força $\vec{F}$ sobre um corpo, causando uma rotação, somente uma das componentes mostradas acima --~a componente $F_t$~-- é responsável pela rotação. A componente $F_r$ só é capaz de causar uma translação do corpo. \label{Fig:DefinicaoTorque}}
\end{marginfigure}

Na Figura~\ref{Fig:DefinicaoTorque}, podemos ver que a componente $F_r$ da força $\vec{F}$, cuja direção é \emph{radial} ---~isto é, ao longo da reta que une o eixo de rotação ao ponto de aplicação da força~---, não tem efeito de causar uma rotação: ela somente é capaz de causar uma translação do corpo. Assim, somente a componente de força perpendicular a esta reta\footnote{E também ao eixo de rotação, uma vez que estamos considerando que este eixo é fixo no espaço.}, isto é, a componente $F_t$, é responsável por causar uma rotação. Definimos então uma gradeza denominada \emph{torque}, dada por
\begin{align}\label{Eq:DefModTorque}
	\tau &= r_\perp F_t \mathnote{Definição de torque} \\
	&= r_\perp F \sen\phi.
\end{align}
%
Note que o ângulo $\phi$ deve ser medido entre ao prolongamento\footnote{Isto é, o segmento de reta \emph{além} do ponto de aplicação da força quando partimos o eixo e nos dirigimos ao ponto de aplicação da força.} da reta que une o eixo de rotação ao ponto de aplicação da força $\vec{F}$ e a própria direção da força.

\begin{marginfigure}
\centering
\begin{tikzpicture}[>=Stealth, scale = 1.5,
     interface/.style={
        % superfície
        postaction={draw,decorate,decoration={border,angle=-45,
                    amplitude=0.2cm,segment length=2mm}}},
    ]

\draw (0,0) ellipse (1.25 and 0.5);

\draw (-1.25,0) -- (-1.25,-0.4);
\draw (1.25,-0.4) -- (1.25,0);  

\draw (-1.25,-0.4) arc (180:360:1.25 and 0.5);
%\draw[densely dotted] (-1.25,-0.4) arc (180:360:1.25 and -0.5);

\draw[dashdotted,->] (0,0) -- (0,1.5) node[below left]{$z$};
\draw[dashdotted] (0,-1.5) -- (0,-0.9);
\draw[dotted] (0,-0.9) -- (0,0);
\draw[fill] (0,0) circle (1pt);

% raio
%\draw[|-|] (-0.05,0.15) -- node[above]{$r_\perp$} +(15:-0.8);
\draw[fill] (15:-0.8) circle (1pt);
\draw[dashdotted] (15:-1.5) -- node[below, near end]{$r_\perp$} (0,0);
% phi
%\draw (15:-1.2) arc (150:117:1.25 and -0.5);
%\node (theta) at (30:-1.1) {$\phi$};

% força
\draw[thick, ->] (15:-0.8) coordinate (pt) -- +(-115:1) node[right]{$\vec{F}$};

% braço de alavanca
\draw[dashed, name path = direcF] (15:-0.8) -- +(-115:-0.5);
\path[name path = direcB] (0,0) coordinate (O) -- (-5:-1);

\path[name intersections={of=direcF and direcB}] (intersection-1) coordinate (a1);

\draw[densely dotted] (0,0) -- node[above]{$r_b$} (a1);

\pic[draw, "$\cdot$", angle eccentricity = 0.5, angle radius = 2mm]{angle = pt--a1--O};

\end{tikzpicture}
\caption{Podemos determinar o torque através do braço de alavanca $r_b$. Tal comprimento equivale à distância entre o eixo de rotação e o ponto onde a força $\vec{F}$ deveria ser aplicada de forma que a reta que liga o ponto de aplicação da força até o eixo de rotação formasse um ângulo de \degree{90} com a própria direção da força. Isso equivale a dizer que $r_b$ é a menor distância entre o eixo de rotação e a reta que denota a direção da força $\vec{F}$.\label{Fig:DefBracoDeAlavanca}}
\end{marginfigure}

Na equação acima, podemos reordenar os termos do produto e reescrever o torque em termos de um comprimento $r_b$ conhecido como \emph{braço de alavanca}:
\begin{align}
	\tau &= r_\perp F \sen\phi \\
	&= r_\perp \sen\phi F \\
	&= (r_\perp \sen\phi) F \\
	&= r_b F,
\end{align}
%
onde $r_b \equiv r_\perp \sen\phi$. O braço de alavanca corresponde à menor distância entre o eixo de rotação e a direção da força $\vec{F}$ (veja a Figura~\ref{Fig:DefBracoDeAlavanca}) e em algumas situações pode ser mais fácil determinar o torque através de tal comprimento.

%%%%%%%%%%%%%%%%%%%%
\paragraph{Unidades}
%%%%%%%%%%%%%%%%%%%%

Fazendo uma análise dimensional da definição para o torque acima, podemos verificar que as unidades são
\begin{align}
    [\tau] &= [r_\perp F \sen\phi] \\
    &= [r_\perp F_t] \\
    &= [r_\perp] [F_t] \\
    &= \rm{m} \cdot \rm{N},
\end{align}
%
ou, como é mais comum de se denotar,
\begin{equation}
    [\tau] = \rm{N}\cdot\rm{m}.
\end{equation}
%
Veja que dimensionalmente $\rm{N}\cdot\rm{m}$ é equivalente à unidade para o trabalho, isto é, o joule (J). No entanto, jamais devemos nos referir ao torque em joules: o joule é uma unidade utilizada para se referir exclusivamente a energia. Veremos futuramente que o torque pode ser definido como um vetor --~diferentemente da energia, que é um escalar~--. Temos, portanto, grandezas físicas completamente diferentes.

%%%%%%%%%%%%%%%%%%%%%%%%%%%%%%%%%%%%%%%%%%%%%%%%%%%
\subsection{Segunda lei de Newton para as rotações}
\label{Sec:SegundaLeiDeNewtonParaRotacoes}
%%%%%%%%%%%%%%%%%%%%%%%%%%%%%%%%%%%%%%%%%%%%%%%%%%%

Se aplicarmos uma força sobre uma porta em que as dobradiças têm pouco atrito e, após um breve momento, cessarmos a aplicação da força, perceberemos que a ela continua girando em torno das dobradiças. Da mesma forma que para o caso de um corpo que se desloca ao longo de um plano, concluímos que o torque não é responsável pela velocidade angular: assim como no caso da translação, a força está associada a uma aceleração. Logo, concluímos que
\begin{equation}
	\alpha \propto \tau.
\end{equation}

Se tentarmos abrir uma porta interna de uma casa, ou uma porta externa, percebemos que a externa exige mais força para obter uma mesma aceleração angular, pois ela é maciça, e -- portanto -- mais massiva. No entanto, veremos que para as rotações a forma como a massa está distribuida também influencia a aceleração angular obtida.

\begin{marginfigure}[-3cm]
\centering
\begin{tikzpicture}[>=Stealth, scale = 1.4,
     interface/.style={
        % superfície
        postaction={draw,decorate,decoration={border,angle=-45,
                    amplitude=0.2cm,segment length=2mm}}},
    ]

%%% Figura superior

\draw (0,0) ellipse (1.25 and 0.5);

\draw (-1.25,0) -- (-1.25,-0.4);
\draw (1.25,-0.4) -- (1.25,0);  

\draw (-1.25,-0.4) arc (180:360:1.25 and 0.5);
\draw[densely dotted] (-1.25,-0.4) arc (180:360:1.25 and -0.5);

\draw[dashdotted,->] (0,0) -- (0,1.5) node[below left]{$z$};
\draw[dashdotted] (0,-1.5) -- (0,-0.9);
\draw[dotted] (0,-0.9) -- (0,0);
\draw[fill] (0,0) circle (1pt);

\draw[dashdotted] (15:-2) -- (0,0);

% raio
\draw[|-|] (-0.05,0.15) -- node[above]{$r_\perp^i$} +(15:-0.8);
\draw[fill] (15:-0.8) circle (1pt);

% força
\draw[thick, ->] (15:-0.8) -- +(-65:1) node[below]{$\vec{F}_j^i$};

% projeção radial
\draw[dashed] (15:-0.8) ++(-65:1) -- (15:-1.6); 
\draw[->] (15:-0.8) -- (15:-1.6) node[above]{$F_{j,r}^i$};

% projeção tangencial
\draw[dashed] (15:-0.8) +(-30:1.6) -- +(-30:-0.3);
\draw[dashed] (15:-0.8) ++(-65:1) -- +(18:0.8);
\draw[->] (15:-0.8) -- +(-30:1.35) node[above]{$F_{j,t}^i$};

% phi
\draw (15:-1.2) arc (150:117:1.25 and -0.5);
\node (theta) at (30:-1.1) {$\phi_j^i$};

\end{tikzpicture}
\caption{Sobre cada uma das partículas que compõe um corpo rígido, age uma grande quantidade de forças internas e externas. Na figura tomamos uma partícula $i$ qualquer e analisamos o efeito de uma força $\vec{F}_j^i$ que atua sobre ela, obtendo as componentes nas direções radial e tangencial.\label{Fig:CalculoSegundaLeiRotacoes}}
\end{marginfigure}

Para verificar a relação entre o torque e aceleração, podemos analisar um objeto como o da Figura~\ref{Fig:CalculoSegundaLeiRotacoes}. Vamos supor que sobre cada partícula do corpo sejam exercidas forças internas e externas, sendo que cada força que atua sobre a $i$-ésima partícula é representada por $F_j^i$, onde $j$ é um índice que enumera cada uma das forças que atuam sobre tal partícula. Se analisarmos o torque devido a $j$-ésima força que atua sobre a $i$-ésima partícula, temos
\begin{align}
	\tau_j^i &= r_\perp^i F_j^i \sen\phi_j^i \\
	&= r_\perp^i F_{j,t}^i
\end{align}
%
mas
\begin{equation}
	F_{j,t}^{i} = m_i a_{j,t}^{i},
\end{equation}
%
onde $a_{j,t}^i$ é a aceleração  tangencial que a $i$-ésima partícula teria se fosse submetida à $j$-ésima força isoladamente.
%
Logo,
\begin{equation}
	\tau_j^i = m_i a_{j,t}^{i} r_\perp^i.
\end{equation}
%
Se somarmos todos os torques que atuam sobre todas as partículas que compõe o corpo, obtemos
\begin{align}
	\sum_i \sum_j \tau_j^i &= \sum_{i}\left[ \sum_j m_i r_\perp^i a_{j,t}^i\right] \\
	&= \sum_{i}\left[m_i r_\perp^i \sum_j a_{j,t}^i\right].
\end{align}
%
A soma das acelerações que cada força exerce isoladamente nada mais é do que a aceleração $\vec{a}^i$ causada pela força resultante, pois
\begin{align}
    \sum_j \vec{a}_{j}^i &= \sum_j \frac{\vec{F}_{j}^i}{m_i} \\
    &= \frac{\sum_j \vec{F}_j^i}{m_i} \\
    &= \frac{\vec{F}_R^i}{m_i} \\
    &= \vec{a}^i,
\end{align}
%
o que implica na relação
\begin{equation}
    \sum_j a_{j,t}^i = a_t^i
\end{equation}
%
para a componente da aceleração na direção tangencial. Assim,
\begin{equation}
	\sum_i \sum_j \tau_j^i = \sum_{i}[m_i r_\perp^i a_{t}^i].
\end{equation}

Podemos utilizar a relação $a_t = \alpha r$ na expressão acima para escrever $a_{t}^i$ como $\alpha r_\perp^i$, o que resulta em
\begin{equation}
	\sum_i \sum_j \tau_j^i = \sum_{i}[m_i (r_\perp^i)^2 \alpha].
\end{equation}
%
No resultado acima, a aceleração angular $\alpha$ é comum a todos os pontos do corpo rígido, por isso podemos escrever
\begin{equation}
	\sum_i \sum_j \tau_j^i =  \left[\sum_{i} m_i (r_\perp^i)^2\right] \alpha.
\end{equation}

\begin{marginfigure}
\centering
\begin{tikzpicture}[>=Stealth, scale = 2,
     interface/.style={
        % superfície
        postaction={draw,decorate,decoration={border,angle=-45,
                    amplitude=0.2cm,segment length=2mm}}},
    ]

    \draw[fill] (0,0) coordinate (O) circle (1pt) node[below right]{$O$};
    
    \draw[->] (O) -- node[below left]{$\vec{r}'$} (130:3) coordinate (A);
    \draw[dotted] (130:3) -- (130:3.5) coordinate (AC);
    \draw[->] (O) -- node[right]{$\vec{r}$} (90:2) coordinate (B);
    \draw[dotted] (90:2) -- (90:2.5) coordinate (BC);
    
    \draw[densely dotted] (A) -- (B);
    
    \draw[->](A) -- node[above, near end]{$\vec{F}'$} ($ (A) !0.3! (B) $) coordinate (AF);
    \draw[->](B) -- node[below, near end]{$\vec{F}$} ($ (B) !0.3! (A) $) coordinate (BF);
    
    \pic[draw, "$\theta$", angle eccentricity = 1.5, angle radius = 5mm]{angle = B--O--A};
    \pic[draw, "$\phi'$", angle eccentricity = 1.8, angle radius = 3mm]{angle = AF--A--AC};
    \pic[draw, "$\phi$", angle eccentricity = 1.4, angle radius = 5mm]{angle = BC--B--BF};
    \pic[draw, "$\alpha$", angle eccentricity = 1.4, angle radius = 5mm]{angle = O--A--AF};
    \pic[draw, "$\beta$", angle eccentricity = 1.8, angle radius = 3mm]{angle = BF--B--O};
    
\end{tikzpicture}
\caption{Os torques devidos aos pares de forças de ação e reação se cancelam.\label{Fig:CancelamentoTorquesInternos}}
\end{marginfigure}

Podemos observar que a soma à esquerda da igualdade inclui todos os torques realizados sobre todas as partículas por todas as forças que atuam no sistema, sejam elas internas ou externas. No entanto, como as forças internas estão presentes aos pares, sendo que cada força do par têm o mesmo módulo e direção, porém sentidos contrários, verificamos que as forças de ação e reação geram torques que se equilibram. Isso pode ser verificado observando a Figura~\ref{Fig:CancelamentoTorquesInternos}: Nela temos um par ação-reação devido à interação entre duas partículas, sendo que o eixo de rotação passa por $O$. Como os torques devidos a essas duas forças são em sentidos opostos, verificamos que para que haja equilíbrio entre os torques é necessário que
\begin{equation}
    Fr\sen\phi - F'r'\sen\phi' = 0,
\end{equation}
%
ou seja, que
\begin{equation}
    r\sen\phi = r'\sen\phi',
\end{equation}
%
\noindent{}onde usamos o fato de que $F' = F$. Aplicando a lei dos senos ao triângulo da Figura~\ref{Fig:CancelamentoTorquesInternos}, temos
\begin{equation}
    \frac{r'}{\sen\beta} = \frac{r}{\sen\alpha}.
\end{equation}
%
Note, no entanto, que
\begin{align}
    \alpha &= \degree{180} - \phi' \\
    \beta &= \degree{180} - \phi.
\end{align}
%
Além disso, $\sen (\degree{180} - \gamma) = \sen\gamma$. Logo,
\begin{align}
    \frac{r'}{\sen\phi} &= \frac{r}{\sen\phi'} \\
    r'\sen\phi' &= r\sen\phi.
\end{align}
%
A relação acima é justamente a condição necessária para que a soma dos torques devidos à força de ação e à força de reação se equilibrem. Assim, podemos escrever a soma dos torques como a \emph{soma dos torques externos}, ou seja, o \emph{torque resultante externo}:
\begin{equation}\label{Eq:ProtoSegundaLeiRotacoes}
    \tau_R^{\rm{ext}} = \left[\sum_{i} m_i (r_\perp^i)^2\right] \alpha.
\end{equation}
% 

Na expressão acima, o termo entre colchetes é denominado como \emph{momento de inércia}:
\begin{equation}\label{Eq:MomentoInerciaConjPart}
    I = \left[\sum_{i = 1}^N m_i (r_\perp^i)^2\right], \mathnote{Momento de inércia para um conjunto de partículas}
\end{equation}
%
e suas unidades são $[I] = \rm{M}\cdot\rm{L}^2$, ou, dentro do SI, $[I] = \rm{kg}\cdot\rm{m}^2$. Verificamos que quanto maior o valor de $I$, menor a aceleração a que o corpo estará sujeito, quando atua sobre ele um determinado valor de torque. A Equação~\eqref{Eq:ProtoSegundaLeiRotacoes} acima relaciona então a aceleração angular a que um objeto estará sujeito quando sobre ele atua uma força, dando origem a um torque. Concluímos, portanto, que a expressão acima é análoga à Segunda Lei de Newton, porém descrevendo o caso da rotação:
\begin{equation}
    \tau_R^{\rm{ext}} = I \alpha. \mathnote{Segunda Lei de Newton para a rotação}
\end{equation}

É importante notar que o momento de inércia, definido pela Equação~\eqref{Eq:MomentoInerciaConjPart}, tem uma dependência na \emph{forma} do objeto: devido ao fato de que existe uma dependência no quadrado da distância $r_i$ entre a partícula e o eixo de rotação, temos uma dependência na forma como a massa está distribuída. Se, por exemplo, temos um disco girando em torno de um eixo que passa por seu centro de massa, perpendicularmente à sua face plana, temos partículas distribuídas por toda a distância entre o eixo e a borda externa do disco. Se tomarmos um aro sustentado por raios finos, de forma que a massa e o diâmetro sejam iguais aos do disco, devemos ter um momento de inércia maior para uma rotação em torno de um eixo que também passe pelo centro de massa, perpendicularmente à face plana. Isso se deve ao fato de que a maior parte das partículas se localizará a uma distância maior do eixo de rotação.

%%%%%%%%%%%%%%%%%%
\paragraph{Sinais}
%%%%%%%%%%%%%%%%%%

Um ponto importante a ser notado é que o torque pode ser positivo ou negativo: quando temos um torque que tende a gerar uma rotação no sentido horário, isso implica em uma aceleração negativa: se ---~por exemplo~--- temos uma velocidade angular inicial nula, temos um aumento do módulo da velocidade angular no sentido horário, o que implica em uma aceleração angular negativa, pois velocidades angulares no sentido horário são negativas. Através da Segunda Lei de Newton para a rotação, sabemos que o sinal do torque deve ser o mesmo da velocidade angular, logo, \emph{torques que tendem a gerar rotações no sentido horário são negativos, enquanto torques que tendem a gerar rotações no sentido anti-horário são positivos}.

%%%%%%%%%%%%%%%%%%%%%%%%%%%%%%%%%%%%%%%%%%%%%%%%%%%%%%%%%%%%%%%
\paragraph{Discussão: Torque devido ao peso de um corpo extenso}
%%%%%%%%%%%%%%%%%%%%%%%%%%%%%%%%%%%%%%%%%%%%%%%%%%%%%%%%%%%%%%%

É bastante comum que estejamos interessados em calcular o torque exercido pela força peso de um corpo extenso, cuja massa é $M$. Nesse caso, basta determinarmos o torque exercido por uma partícula de massa $M$ que ocupa a posição do centro de massa do corpo. Para verificarmos isso, basta somarmos o torque de cada uma das partículas individualmente, obtendo o torque resultante:
\begin{align}
    \tau_{\text{Res}} &= \sum_i r_i F_i \sen \phi_i \\
    &= \sum_i r_i P_i \sen \phi_i.
\end{align}

\begin{marginfigure}
\centering
\begin{tikzpicture}[>=Stealth, scale = 1.2]
       
    % braço
    \draw[pattern = north west lines, draw = gray, pattern color = gray] (-0.5,-0.05) rectangle (3.5,0.05);
    \fill (0,0) coordinate (O) node[above left]{$O$} circle (1pt);
    
    % peso
    \path[fill = white] (2.5,-0.5) rectangle (3,0.5);
    \draw[pattern = crosshatch, draw = gray, pattern color = gray] (2.5,-0.5) rectangle (3,0.5);
    \fill (2.75,0) circle (1pt);
    
    % ponto
    \fill (10:2.55) coordinate (A) circle (1.5pt);
    \draw[->, thick] (A) -- node[left]{$\vec{P}_i$} +(0,-0.3);
    \draw[->] (O) -- node[above]{$\vec{r}_i$} (A);
    
    \begin{scope}[shift={(0,-2)}]
        
        \fill (0,0) node[above left]{$O$} circle (1pt);
            
        % ponto
        \fill (10:2.55) coordinate (A) circle (1.5pt);
        \draw[->, thick] (A) -- +(0,-0.3) coordinate (P);
        \draw[dotted] (0,0) coordinate (B) -- (10:3.25) coordinate (C);
        \draw[->] (0,0) -- node[above]{$\vec{r}_i$} (A);
        
        \pic[draw, "$\phi$", angle eccentricity = 1.5, angle radius = 3mm]{angle = P--A--C};
        \pic[draw, "$\gamma$", angle eccentricity = 1.65, angle radius = 2.5mm]{angle = B--A--P};
    
        \draw[dashed] (0,0) -- (2.51,0);
        \draw[|<->|] (0,-0.25) -- node[below]{$x_i$} (2.51,-0.25);
    
    \end{scope}
\end{tikzpicture}
\caption{O torque devido à força peso de uma partícula pode ser determinado em termos do braço de alavanca $x_i$. (O sistema é capaz de girar em torno de um eixo que passa no ponto $O$, perpendicular à página.)\label{Fig:TorquePeso}}
\end{marginfigure}


Sabendo que $\sen \phi = \sen (\degree{180} - \phi) = \sen \gamma$, podemos escrever (veja a Figura~\ref{Fig:TorquePeso})
\begin{equation}
    x_i = r_i \sen\phi,
\end{equation}
%
o que nos permite reescrever a expressão para o torque resultante como
\begin{align}
    \tau_{\text{Res}} &= \sum_i x_i m_i g \\
    &= \left(\sum_i x_i m_i\right) g.
\end{align}
%
Através da definição de centro de massa, temos
\begin{equation}
    x_{\text{CM}} = \frac{\sum x_i m_i}{M},
\end{equation}
%
o que nos permite escrever
\begin{align}
    \tau_{\text{Res}} &= \left(M x_{\text{CM}}\right) g \\
    &= x_{\text{CM}} Mg \\
    &= x_{\text{CM}} P.
\end{align}
%
Finalmente, é importante destacar que a distância $x_{\text{CM}}$ na Figura~\ref{Fig:TorquePeso} é o braço de alavanca, então em outras configurações (veja a Figura~\ref{Fig:TorquePeso2}) ela equivale à distância $r_{\text{CM}}$ entre o eixo de rotação e o centro de massa, porém multiplicada por $\sen\phi$. Assim, podemos escrever
\begin{equation}
    \tau_{P} = r_{\text{CM}} P \sen\phi.
\end{equation}
%
Note que a expressão acima tem exatamente a mesma forma que a expressão para o torque devido ao peso de uma partícula, ou seja, \emph{para determinarmos o torque devido ao peso de um corpo extenso, basta o substituir por uma partícula com a mesma massa que o corpo e ocupando a posição do centro de massa}.

\begin{marginfigure}[-4cm]
\centering
\begin{tikzpicture}[>=Stealth, scale = 1.2]
       
    \begin{scope}[rotate = 45]
        % braço
        \draw[pattern = crosshatch, draw = gray, pattern color = gray] (-0.5,-0.05) rectangle (2.5,0.05);
        \fill (0,0) coordinate (O) node[above left]{$O$} circle (1pt);
        
        % peso
        \draw[draw = gray] (3,0) coordinate (center) circle (0.5);
        \fill (center) circle (1pt);
        
    \end{scope}
    
    % vetor posição CM
    \draw[dotted] (O) -- (45:4.25) coordinate (A);
    \draw[->] (O) -- node[above, sloped]{$\vec{r}_{\text{CM}}$} (center);
    \draw[dotted] (2.1213,0) -- (center);
        
    \draw[->, thick] (center) -- +(0,-1) coordinate (B) node[right]{$\vec{P}$};
    \draw[dashed] (0,0) -- (2.1213,0);
    \draw[|<->|] (0,-0.25) -- node[below]{$x_{\text{CM}}$} (2.1213,-0.25);
    
    \pic[draw, "$\phi$", angle eccentricity = 1.5, angle radius = 3mm]{angle = B--center--A};
    \pic[draw, "$\gamma$", angle eccentricity = 1.25, angle radius = 8mm]{angle = O--center--B};
    
\end{tikzpicture}
\caption{O valor de $x_{\text{CM}}$ pode ser calculado através de $r_{\text{CM}}$ e do ângulo $\phi$.\label{Fig:TorquePeso2}}
\end{marginfigure}

%%%%%%%%%%%%%%%%%%%%%%%%%%%%%%%%
\paragraph{Discussão: Alavancas}
%%%%%%%%%%%%%%%%%%%%%%%%%%%%%%%%

\begin{marginfigure}[3cm]
\centering
\begin{tikzpicture}[>=Stealth, interface/.style={
        % superfície
        postaction={draw,decorate,decoration={border,angle=-45,
                    amplitude=0.2cm,segment length=2mm}}}
    ]
       
    \begin{scope}[rotate = 45]
        % braço
        \draw[pattern = crosshatch, draw = gray, pattern color = gray] (-0.5,-0.05) rectangle (3,0.05);
        \fill (0,0) coordinate (O) node[above left]{$O$} circle (1pt);
        
        % ponto de aplicação da força
        \coordinate (center) at (3,0);
        
    \end{scope}
    
    \draw[dotted] (O) -- (45:4.25) coordinate (A);
    \path (O) -- node[below, sloped]{$L$} (center);
    \draw[dotted] (45:-0.5) coordinate (extr) -- (45:-1) coordinate (fix);
    \path (O) -- node[below, sloped]{$\ell$} (extr);
        
    \draw[->, thick] (45:-0.5) -- +(0,-1.5) coordinate (E) node[right]{$\vec{P}$};
    \draw[->, thick] (center) -- +(0,-0.9) coordinate (B) node[right]{$\vec{F}$};
    
    \pic[draw, "$\phi_2$", angle eccentricity = 1.5, angle radius = 3mm]{angle = B--center--A};
    \pic[draw, "$\phi_1$", angle eccentricity = 1.5, angle radius = 3mm]{angle = fix--extr--E};
    \pic[draw, "$\gamma$", angle eccentricity = 1.25, angle radius = 6mm]{angle = O--center--B};

\end{tikzpicture}
\caption{alavanca. \label{Fig:Alavanca}}
\end{marginfigure}

Uma aplicação simples, porém útil da Segunda Lei de Newton para a rotação é a de uma \emph{alavanca}. Se, por exemplo, desejamos erguer um corpo muito pesado, podemos usar uma haste e um fulcro como mostrado na Figura~\ref{Fig:Alavanca}. Aplicando a a Segunda Lei de Newton para a rotação em torno do fulcro ($O$), obtemos
\begin{align}
    \tau_{\text{Res}} &= I \alpha \\
    P \ell \sen\phi_1 - F L \sen\phi_2 &= I\alpha.
\end{align}
%
Se considerarmos que a aceleração seja nula, ou seja, após iniciarmos o movimento, temos velocidade constante, então
\begin{align}
    F L \sen \phi_2 &= P\ell \sen\phi_1 \\
    F &= P \frac{\ell}{L} \frac{\sen\phi_1}{\sen\phi_2}.
\end{align}
%
Note que $\sen(\degree{180} - \alpha) = \sen\alpha$, por isso $\sen\phi_2 = \sen\gamma$. Como $\gamma = \phi_1$, já que as forças $\vec{F}$ e $\vec{P}$ são paralelas, concluímos que a razão entre os senos é igual a 1. Logo,
\begin{equation}
    F = P \frac{\ell}{L}.
\end{equation}
%
A razão $\ell/L$ é menor do que 1, o que nos mostra que 
\begin{equation}
    F < P.
\end{equation}

%%%%%%%%%%%%%%%%%%%%%%%%%%%%%%%%%%%%%%%%%
\paragraph{Balanças de braços}
%%%%%%%%%%%%%%%%%%%%%%%%%%%%%%%%%%%%%%%%%

\begin{marginfigure}[3cm]
\centering
\begin{tikzpicture}[interface/.style={
        % superfície
        postaction={draw,decorate,decoration={border,angle=-45,
                    amplitude=0.2cm,segment length=2mm}}}
    ]
       
    \draw[interface] (0,0) -- (4.5,0);
    
    % fulcro
    \draw[pattern = crosshatch] (1.5,0.5) -- (1.75,0) -- (1.25,0) -- cycle; 
    
    % braço
    \draw[pattern = north west lines] (0.5,0.5) rectangle (4,0.7);
    \fill (1,0.6) circle (1pt);
    \fill (2.75,0.6) circle (1pt);
    
    % peso
    \path[fill = white] (3,0.4) rectangle (3.7,0.8);
    \draw[pattern = crosshatch] (3,0.4) rectangle (3.7,0.8);
    \fill (3.35,0.6) circle (1pt);
    
    % prato
    \draw[pattern = north west lines] (0.1, 1.2) -- (0.1,1.1) -- (0.5,1.1) -- (0.5, 0.7) -- (0.7, 0.7) -- (0.7,1.1) -- (1.1,1.1) -- (1.1,1.2) -- cycle;
    
    \fill[white, draw = black] (0.6, 0.6) circle (1pt);
    \fill (0.6,1.05) circle (1pt);
    
    % bloco
    \draw[pattern = north east lines] (0.3, 1.2) rectangle +(0.6,0.6);
    \fill (0.6, 1.5) circle (1pt);
    
    % medidas
    \draw[dashed] (1.5,0.5) -- (1.5,2.5);
    \draw[|<->] (0.6, 2) -- node[above]{$\ell_c$} (1.5,2);
    
    \draw[dotted] (3.35,0.8) -- (3.35,2);
    \draw[|<->|] (1.5, 2) -- node[above]{$L$} (3.35, 2);
    
    \draw[|<->] (1,0.85) -- node[above]{$d$} (1.5,0.85);
    \draw[<->|] (1.5,0.85) -- node[above]{$D$} (2.75,0.85);
    
\end{tikzpicture}
\caption{Balança de braços. \label{Fig:BalancaDeBracos}}
\end{marginfigure}

Uma balança de braços é um dispositivo que se vale das propriedades do torque ao empregar uma alavanca, permitindo que possamos determinar a massa de um corpo qualquer. Na Figura~\ref{Fig:BalancaDeBracos} temos um desenho esquemático de uma balança desse tipo: nele podemos verificar a existência de um braço longo apoiado sobre um fulcro, de um prato sobre o qual apoiamos o corpo cuja massa desejamos aferir e que está ligado a uma das extremidades do braço, e uma massa que pode deslizar ao longo do braço. No equilíbrio, o torque exercido em torno do fulcro pela parte da balança à esquerda do ponto de contato é igual ao torque exercido pela parte à direita de tal ponto.

Assim,
\begin{equation}
    \tau_c + \tau_p + \tau_{b_1} + \tau_{b_2} + \tau_m = 0,
\end{equation}
%
onde $\tau_c$ representa o torque devido ao corpo cuja massa pretendemos determinar, $\tau_p$ representa o torque devido ao prato de apoio, $\tau_{b_1}$ e $\tau_{b_2}$ representam os torques devidos aos segmentos esquerdo e direito do braço,\footnote{Podemos substituir esses dois termos por um só se considerarmos que toda a massa do braço se encontra no centro de massa e calcularmos um único torque.} e $\tau_m$ representa o torque devido à massa deslizante. Também podemos observar que todos os ângulos entre as forças e o braço são de \degree{90}, o que nos permite escrever
\begin{equation}
    (m_c + m_p) \ell_c + m_{b_1} d - m_{b_2} D - m_m L = 0,
\end{equation}
%
onde usamos o fato de que os centros de massa do corpo e do prato de apoio estão à mesma distância horizontal em relação ao fulcro, além do fato de que os torques devidos às forças à direita do fulcro tendem a causar rotações no sentido horário e são ---~portanto~--- torques negativos. Como estamos interessados em calcular a massa $m_c$, basta a isolarmos, o que resulta em
\begin{align}
    m_c &= \frac{m_{b_2} D + m_m L - m_{b_1}d}{\ell_c} - m_p\\
    &= \frac{m_{m}}{\ell_c} L + \left(\frac{m_{b_2} D - m_{b_1}d}{\ell_c} - m_p\right).
\end{align}
%
Com exceção da distância $L$, todos os parâmetros à direita são constantes, o que implica em uma relação direta entre a distância e a massa sobre o prato o que permite que o próprio braço seja marcado com uma escala mostrando os valores de massa. Além disso, a relação entre $L$ e $m$ é \emph{linear}, isto é, ela segue uma equação da reta. Isso quer dizer que as marcações no braço são feitas em intervalos regulares: para calibrarmos a balança, basta escolhermos dois corpo de massas conhecidas, marcar ambas as posições de equilíbrio, e elaborar novas marcações com o mesmo tamanho que a primeira.\footnote[][-2cm]{Para medir valores intermediários e/ou realizar medidas mais ``finas'', basta realizarmos marcações mais próximas, sempre em intervalos regulares.}

%%%%%%%%%%%%%%%%%%%%%%%%%%%%%%%%%%
%\paragraph{Discussão: Engrenagens}
%%%%%%%%%%%%%%%%%%%%%%%%%%%%%%%%%%

%\textbf{Explicar como alavancas e engrenagens ampliam o torque. Lembre-se que ainda não vimos o cálculo do momento de inércia.}


%%%%%%%%%%%%%%%%%%%%%%%%%%%%%%%%%%%%%%%%%%
\paragraph{Discussão: Polia cilíndrica}
\label{Sec:PoliaCilindrica}
%%%%%%%%%%%%%%%%%%%%%%%%%%%%%%%%%%%%%%%%%%

\begin{marginfigure}
\centering
\begin{tikzpicture}[>=Stealth,  interface/.style={
        % superfície
        postaction={draw,decorate,decoration={border,angle=-45,
                    amplitude=0.2cm,segment length=2mm}}}
    ]
       
    \draw[fill] (0,0) circle (1pt);
    \draw[interface] ([shift={(0,0)}]121:2) arc[radius=2, start angle=121, end angle= 0];
    \draw ([shift={(0,0)}]120:1.86) arc[radius=1.86, start angle=120, end angle= -30];
    \draw (1.86,0) -- +(0,-1.5);
    \draw[interface] (2,0) -- +(0,-1.5);
    
    \draw[fill] (30:1.86) circle (1pt);
    \draw (0,0) -- (30:1.86);
    \draw[->, thick] (30:1.86) -- +(-60:1) node[above right]{$f_1$};
    
    \draw[fill] (60:1.86) circle (1pt);
    \draw (0,0) -- (60:1.86);
    \draw[->, thick] (60:1.86) -- +(-30:1) node[above]{$f_2$};
    
    \draw[fill] (90:1.86) circle (1pt);
    \draw (0,0) -- (90:1.86);
    \draw[->, thick] (90:1.86) -- +(0:1) node[above]{$f_3$};
    
\end{tikzpicture}
\caption{A força de atrito total que atua sobre a corda se deve a diversas forças em diferentes pontos de contato. Ainda assim, podemos determinar o torque de maneira simples pois as características dessas forças são comuns a todos os pontos de contato. \label{Fig:MaquinaDeAtwoodComPoliaDetalhePolia}}
\end{marginfigure}

Uma situação relativamente comum e que envolve rotação é a utilização de polias. O modelo mais simples que pode podemos fazer para uma polia e a de um disco que gira em torno de um eixo que passa por seu centro de massa, perpendicularmente à sua face plana. Em torno da polia uma corda é então enrolada por um certo número de voltas, ou mesmo menos que uma volta. Ao aplicarmos tensões diferentes nas extremidades livres da corda, teremos uma rotação da polia, com uma aceleração angular qualquer. Vamos determinar a relação entre tais tensões e o valor da aceleração angular.

O torque exercido sobre a polia se deve à interação com corda através do atrito. Podemos determinar o valor desse torque ao considerarmos que ---~como mostrado na Figura~\ref{Fig:MaquinaDeAtwoodComPoliaDetalhePolia}~--- as forças de interação entre a corda e a polia nos diversos pontos de contato fazem sempre um ângulo de \np[\tcdegree]{90} com a direção que liga o eixo de rotação ao ponto em que a força é exercida. Além disso, a distância entre o ponto de aplicação da força e o eixo de rotação é sempre igual ao raio da polia. Portanto,\footnote{O sinal negativo se deve ao fato de que as forças tendem a causar uma rotação da polia no sentido horário.}
%
\begin{align}
    \tau &= -f_1 R \sen\phi - f_2 R \sen\phi - \dots \\
    &= -f_1 R \sen\np[\tcdegree]{90} - f_2 R \sen\np[\tcdegree]{90} - \dots
\end{align}
%
A força de atrito total na corda é dada pela soma das forças exercidas nos diversos pontos de contato, isto é, $f_{\rm{at}} = \sum_{i} f_i$, o que nos permite reescrever a relação acima como
\begin{align}
    \tau &= -f_{\rm{at}} R \sen\np[\tcdegree]{90} \\
    &= -f_{\rm{at}} R.
\end{align}

Podemos determinar o valor da força de atrito analisando as forças que atuam sobre o segmento da corda que está em contato com a polia. As forças que atuam sobre tal segmento são as tensões $T_1$ e $T_2$ exercidas sobre os segmentos livres de corda, além da própria força de atrito (reação da força de atrito exercida sobre a polia). Considerando somente a aceleração tangencial, pois ela é a responsável por alterar a velocidade, podemos substituir a região curva da corda por uma região reta, como mostrado na Figura~\ref{Fig:MaquinaDeAtwoodComPoliaDetalheCorda}. Assim:
\begin{align}
    F_R &= m_{cc} a_t \\
    T_2 - T_1 - f_{\rm{at}} &= m_{cc} a_t.
\end{align}

\begin{marginfigure}
\centering
\begin{tikzpicture}[>=Stealth,  interface/.style={
        % superfície
        postaction={draw,decorate,decoration={border,angle=-45,
                    amplitude=0.2cm,segment length=2mm}}}
    ]

    \draw[fill = lightgray] (0,-0.075) rectangle (2,0.075);       
    \draw[pattern = north west lines] (0,-0.075) rectangle (2,0.075);
    \draw[pattern = north west lines, pattern color = gray, draw = gray] (-1,-0.075) rectangle (0,0.075);
    \draw[pattern = north west lines, pattern color = gray, draw = gray] (2,-0.075) rectangle (3,0.075);
    
    \draw[dashed, ->] (-1.5,0) -- (3.5,0) node[below left] {$x$};
    \draw[thick, ->] (2,0) -- (2.75,0) node[above left]{$\vec{T}_2$};
    \draw[thick, ->] (0,0) -- (-0.75,0) node[above right]{$\vec{T}_1$};
    
\end{tikzpicture}
\caption{Podemos determinar a aceleração da corda na direção tangencial à trajetória simplificando o problema ao adotarmos um sistema onde o segmento de corda em contato com a polia é substituido por um segmento retilíneo. Note que para fins de cálculo da aceleração tangencial, o resultado é o mesmo, por tal aceleração altera somente o módulo da velocidade. \label{Fig:MaquinaDeAtwoodComPoliaDetalheCorda}}
\end{marginfigure}

\noindent{}Em geral podemos desprezar a massa $m_{cc}$ da corda em contato com a polia, pois ela costuma ser muito menor que a massa dos demais corpos envolvidos no sistema. Nesse caso, podemos afirmar que
\begin{equation}
    f_{\rm{at}} = T_2 - T_1.
\end{equation}
%
Note que \emph{as tensões nos segmentos esquerdo e direito da corda são necessariamente diferentes}. Tal diferença é responsável por causar a aceleração da polia! 

Outra relação importante que devemos observar entre a polia e a corda é a de que a aceleração tangencial $a_t^b$ e a velocidade $v_b$ de um ponto na borda da polia são iguais a aceleração $a$ e a velocidade $v$ da corda. Caso isso não fosse verdade, teríamos um deslizamento entre a corda e a polia, o que não observamos.\footnote{Uma situação dessas é possível, mas complicaria ainda mais esse problema.} Isso nos permite determinar uma relação entre a aceleração da corda --~que é igual a aceleração dos blocos~-- e a aceleração angular da polia:
\begin{align}
    a_t^b &= \alpha r \\
    a &= \alpha R.
\end{align}
%
Note ainda que uma aceleração positiva dos blocos implica em uma aceleração angular negativa da polia, pois se partíssemos de uma velocidade angular nula, teríamos uma velocidade que cresce no sentido horário. Assim, ainda precisamos adicionar um sinal na expressão acima. \emph{Note que esse sinal é um reflexo da nossa escolha para o sentido positivo para a aceleração da corda. Se tivessemos adotado como positivo o sentido oposto, essa relação não teria um sinal negativo.} Logo,
\begin{equation}
        a = -\alpha R.
\end{equation}

Finalmente, obtemos para o torque a relação
\begin{equation}
    \tau = -(T_2 - T_1) R,
\end{equation}
%
o que --~através da Segunda Lei de Newton para a rotação~-- nos leva a
\begin{align}
    \tau_{R}^{\rm{ext}} &= I\alpha \\
    -(T_2 - T_1) R &= I \alpha.
\end{align}
%
Aqui notamos que existe uma dependência no momento de inércia, que por sua vez depende da massa e da distribuição de massa da polia ---~isto é, depende da \emph{forma} da polia~---. Estamos supondo que a polia tem formato de disco, cujo momento de inércia é dado por\footnote{Verificaremos nas seções seguintes como determinar o momento de inércia de um corpo rígido.}
\begin{equation}
    I = \frac{1}{2} MR^2.
\end{equation}
%
Podemos então escrever a seguinte relação para a polia:
\begin{equation}\label{Eq:ResPolia}
    -(T_2 - T_1) = \frac{MR}{2} \alpha.
\end{equation}

Nesse sistema, a questão da determinação do torque sobre a polia é bastante complexa. Se substituíssemos a corda por uma corrente como a de uma bicicleta e a polia por uma que tivesse dentes para encaixarem na corrente, teríamos uma interação diferente, através de forças normais nas superfícies dos dentes e dos encaixes da corrente. Essas forças são predominantes nos primeiros dentes onde ocorre o encaixe completo. Dessa forma, podemos imaginar que as forças de tensão são exercidas sobre a polia somente nesses pontos, o que resulta em algo como a Figura~\eqref{Fig:TorquesPolia}. Nesse caso, temos dois torques devidos às forças de tensão que são exercidas sobre a polia, logo
%
\begin{marginfigure}[5cm]
\centering
\begin{tikzpicture}[>=Stealth,  interface/.style={
        % superfície
        postaction={draw,decorate,decoration={border,angle=-45,
                    amplitude=0.2cm,segment length=2mm}}}
    ]

    \draw[pattern = north west lines, pattern color = lightgray] (0,0) circle (1cm);
    \draw[fill] (0,0) circle (1pt);
    
    \draw[dashed] (0,0) -- node[above]{$R$} (1,0);
    \draw[dashed] (0,0) -- node[above]{$R$} (-1,0);
    
    \draw (1,0) -- +(0,-2);
    \draw (-1,0) -- +(0,-2);
    
    \draw[fill] (1,0) circle (0.8pt);
    \draw[fill] (-1,0) circle (0.8pt);
    \draw[thick, ->] (1,0) -- +(0,-1) node[right] {$\vec{T}_2$};
    \draw[thick, ->] (-1,0) -- +(0,-1) node[left] {$\vec{T}_1$};
    
\end{tikzpicture}
\caption{Podemos supor que o torque sobre a polia se deve aos torques devido às tensões nos cabos, como se as tensões atuassem diretamente sobre a polia. Nesse caso, supomos que as forças atuam sobre os pontos de contato onde a direção da corda faz \np[\tcdegree]{90} em relação à reta que parte do eixo de rotação e vai em direção ao ponto de contato (pontos marcados com pequenos círculos na borda da polia).\label{Fig:TorquesPolia}}
\end{marginfigure}
%
\begin{align}
    \tau_R^{\rm{ext}} &= I \alpha \\
    T_1 R \sen\phi - T_2 R \sen\phi &= I \alpha \\
    (T_1 - T_2) R &= I\alpha,
\end{align}
%
onde utilizamos $\phi = \np[\tcdegree]{90}$ e o sinal se deve ao fato de que a tensão $T_2$ tende a causar uma rotação no sentido horário.

Observamos que o resultado acima é exatamente o mesmo que obtivemos para o caso em que fizemos uma análise através do atrito. De fato, o que ocorre é que em todos os casos onde não há deslizamento entre a polia e o cabo ---~seja ele uma corda, uma correia, ou uma corrente~--- podemos supor que as forças de tensão exercidas nas extremidades do cabo são exercida diretamente sobre a polia, no ponto onde a reta que liga o eixo de rotação ao ponto de contato faz um ângulo de \np[\tcdegree]{90} em relação à direção da tensão.

Finalmente, para determinarmos a aceleração, basta isolarmos $\alpha$ na  Equação~\eqref{Eq:ResPolia} acima, o que resulta em
\begin{equation}
    \alpha = (T_1 - T_2) \frac{2}{MR}.
\end{equation}

%%%%%%%%%%%%%%%%%%%%%%%%%%%%%%%%%%%%%%%
\section{Cálculo do momento de inércia}
%%%%%%%%%%%%%%%%%%%%%%%%%%%%%%%%%%%%%%%

Para que possamos aplicar a Segunda Lei de Newton para a rotação, é fundamental que saibamos qual é o momento de inércia do corpo em questão para o eixo em torno do qual a rotação ocorre. Como é possível verificar na Expressão~\ref{Eq:MomentoInerciaConjPart}, o momento de inércia depende da distância de cada uma das partículas em relação ao eixo de rotação. Isso nos indica que é importante determinar uma série de características particulares a cada situação que estivermos analisando.

\begin{marginfigure}
\centering
\begin{tikzpicture}[>=Stealth, rotate = -90,
     interface/.style={
        % superfície
        postaction={draw,decorate,decoration={border,angle=-45,
                    amplitude=0.2cm,segment length=2mm}}},
    ]

%%% Disco

\draw (0,0) ellipse (1.25 and 0.75);

\draw (-1.25,0) -- (-1.25,-0.4);
\draw (1.25,-0.4) -- (1.25,0);  

\draw (-1.25,-0.4) arc (180:360:1.25 and 0.75);
\draw[dotted] (-1.25,-0.4) arc (180:360:1.25 and -0.75);

\draw[dashdotted,<-] (-2,-0.2) node[below left]{$z$} -- (-1.25,-0.2);
\draw[loosely dotted] (-1.25,-0.2) -- (1.25,-0.2);
\draw[dashdotted] (1.25,-0.2) -- (1.75, -0.2);

\draw[->] (-1.5,-0.2) [partial ellipse=-120:150:0.125cm and 0.3cm];

%%% aro

\draw (4,0) ellipse (1.25 and 0.75);
\draw (4,0) ellipse (1 and 0.55);
\draw[dotted] (4,-0.4) ellipse (1 and 0.55);
\draw (4,-0.4) [partial ellipse=22:158:1cm and 0.55cm];

\draw (2.75,0) -- (2.75,-0.4);
\draw (5.25,-0.4) -- (5.25,0);  

\draw (2.75,-0.4) arc (180:360:1.25 and 0.75);
\draw[dotted] (2.75,-0.4) arc (180:360:1.25 and -0.75);

\draw[dashdotted,<-] (2,-0.2) node[below left]{$z$} -- (2.75,-0.2);
\draw[dotted] (2.75,-0.2) -- (3.15,-0.2);
\draw[dashdotted] (3.15,-0.2) -- (4.9,-0.2);
\draw[loosely dotted] (4.9,-0.2) -- (5.25,-0.2);
\draw[dashdotted] (5.25,-0.2) -- (5.75, -0.2);

\draw[->] (2.5,-0.2) [partial ellipse=-120:150:0.125cm and 0.3cm];

\end{tikzpicture}
\caption{Um disco e um anel de mesma massa e raio têm momentos de inércia diferentes em virtude das diferentes \emph{distribuições} de massa.\label{Fig:CompMomInerciaDiscoAnel}}
\end{marginfigure}

Dentre as principais características do momento de inércia, temos as seguintes dependências, cuja origem é a própria dependência na distância entre o eixo de rotação e a posição de cada partícula do corpo:
\begin{description}
    \item[Dependência na forma do objeto:] Se tomarmos um disco e um anel de mesma massa e raio, como na Figura~\ref{Fig:CompMomInerciaDiscoAnel}, o momento de inércia em torno do eixo $z$ mostrado acima é maior para o anel. Podemos compreender esse fato ao verificarmos que no caso do disco uma parte da massa, localizada na região central, se encontra próxima ao eixo de rotação. No caso do anel, essa massa está localizada a uma distância maior do eixo de rotação. Note que para que ambos os corpos possam ter a mesma massa e o mesmo raio, o material de que é feito o anel deve ser mais denso que o material do disco.
\begin{marginfigure}[0.5cm]
\centering
\begin{tikzpicture}[>=Stealth, scale = 1.2,
     interface/.style={
        % superfície
        postaction={draw,decorate,decoration={border,angle=-45,
                    amplitude=0.2cm,segment length=2mm}}},
    ]

\draw[fill] (0,0) circle (0.6pt);
\draw[fill] (0,-0.2) circle (1pt) node[above left] {CM};
\draw (0,0) ellipse (1.25 and 0.5);
\draw[fill] (0,-0.4) circle (0.6pt);

\draw (-1.25,0) -- (-1.25,-0.4);
\draw (1.25,-0.4) -- (1.25,0);  

\draw (-1.25,-0.4) arc (180:360:1.25 and 0.5);
\draw[densely dotted] (-1.25,-0.4) arc (180:360:1.25 and -0.5);

% Eixo z
\draw[dashdotted,->] (0,0) -- (0,1.5) node[below left]{$z$};
\draw[dashdotted] (0,-1.5) -- (0,-0.9);
\draw[loosely dotted] (0,-0.9) -- (0,0);

\draw[->] (0,1.05) [partial ellipse=-225:60:0.3cm and 0.125cm];

% Eixo x
\draw[dashdotted, ->] (0,-0.2)++(15:-1.05) -- +(15:-0.75) node[below right]{$x$};
\draw[fill] (0,-0.2)+(15:-1.05) circle (0.6pt);
\draw[loosely dotted] (0,-0.2)+(15:-1.05) -- (0,-0.2) -- +(15:1.05);
\draw[fill] (0,-0.2) +(15:1.05) circle (0.6pt);

\draw[dashdotted] (0,-0.2) ++(15:1.25) -- +(15:0.75);

\draw[->, rotate=15] (0,-0.2)++(0:-1.5) [partial ellipse=-130:160:0.105cm and 0.3cm];

% Eixo w
\draw[dashdotted, <-] (0,-0.2)+(60:-1.38) node[below right]{$w$} -- +(60:-0.78);
\draw[fill] (0,-0.2)+(60:-0.78) circle (0.6pt);
\draw[loosely dotted] (0,-0.2)+(60:-0.78) -- +(60:0.78);
\draw[fill] (0,-0.2)+(60:0.78) circle (0.6pt);
\draw[dashdotted] (0,-0.2)+(60:0.78) -- +(60:1.18);

\draw[->, rotate=60] (0,-0.2)++(-4:-1.2) [partial ellipse=-150:150:0.15cm and 0.3cm];

\end{tikzpicture}
\caption{A orientação do eixo de rotação em relação ao objeto determina a distância entre as partículas e o próprio eixo, o que faz com que o momento de inércia seja diferente para cada orientação. Na figura, os pequenos círculos pretos mostram pontos onde os eixos saem/entram no objeto. \label{Fig:MomInerciaDiscoVariosEixos}}
\end{marginfigure}
    \item[Dependência no eixo em que ocorre a rotação:] Mesmo para um corpo só, podemos ter diversos momentos de inércia diferentes. Na Figura~\ref{Fig:MomInerciaDiscoVariosEixos}, temos três eixos que atravessam um disco de maneiras diferentes, e que passam pelo centro de massa do corpo. Como a orientação do eixo muda a distância entre as partículas que compõe o corpo e o eixo de rotação, ocorre uma mudança do momento de inércia. Dentre os três eixos mostrados, o momento de inércia é maior em relação ao eixo $z$, uma vez que ele minimiza a quantidade de massa próxima ao eixo.
    \item[Dependência na distância ao eixo de rotação:] Mesmo que tenhamos o cuidado de escolher eixos cuja orientação em relação ao corpo é a mesma, a distância entre o eixo de rotação e o próprio corpo é um fator determinante no cálculo do momento de inércia, já que isso causa uma variação da distância entre as partículas e o eixo. Na Figura~\ref{Fig:MomInerciaRotEixosParalelos} temos três eixos distintos, paralelos uns aos outros, em torno dos quais o corpo pode efetuar uma rotação. Devido a alteração das distâncias entre as partículas e o eixo, cada um deles terá um momento de inércia diferente. Devemos destacar em especial o eixo $z''$, pois ele não atravessa o corpo. Esse tipo de situação é relativamente comum quando analisamos corpos compostos de diversas partes, e tem grande influência no valor do momento de inércia. Verificaremos adiante que os momentos de inércia em eixos paralelos estão relacionados de forma bastante simples e essa relação deixará evidente que o momento de inércia mínimo é aquele associado ao eixo que passa pelo centro de massa do corpo.
\end{description}

\begin{marginfigure}
\centering
\begin{tikzpicture}[>=Stealth, scale = 1.1,
     interface/.style={
        % superfície
        postaction={draw,decorate,decoration={border,angle=-45,
                    amplitude=0.2cm,segment length=2mm}}},
    ]

%%% Figura superior

\draw[fill] (0,0) circle (0.6pt);
\draw[fill] (0,-0.2) circle (1pt) node[above left] {CM};
\draw (0,0) ellipse (1.25 and 0.5);
\draw[fill] (0,-0.4) circle (0.6pt);

\draw (-1.25,0) -- (-1.25,-0.4);
\draw (1.25,-0.4) -- (1.25,0);  

\draw (-1.25,-0.4) arc (180:360:1.25 and 0.5);
\draw[densely dotted] (-1.25,-0.4) arc (180:360:1.25 and -0.5);

% Eixo z
\draw[dashdotted,->] (0,0) -- (0,1.5) node[below left]{$z$};
\draw[dashdotted] (0,-1.5) -- (0,-0.9);
\draw[dotted] (0,-0.9) -- (0,0);

\draw[->] (0,1.05) [partial ellipse=-225:60:0.3cm and 0.125cm];

% Eixos z', z''

\draw[dashdotted, ->] (-1.25,-1.5) -- (-1.25,1.5) node[below left]{$z'$};
\draw[->] (-1.25,1.05) [partial ellipse=-225:60:0.3cm and 0.125cm];

\draw[dashdotted, ->] (-2.5,-1.5) -- (-2.5,1.5) node[below left]{$z''$};
\draw[->] (-2.5,1.05) [partial ellipse=-225:60:0.3cm and 0.125cm];

\end{tikzpicture}
\caption{Mesmo que a orientação de diversos eixos em relação a um corpo seja a mesma, a distância em relação ao corpo também determina a distância das partículas em relação ao eixo de rotação. Veja que quando um corpo faz parte de um conjunto mais complexo, podemos ter uma rotação em relação a um eixo que não o atravessa. \label{Fig:MomInerciaRotEixosParalelos}}
\end{marginfigure}

Outra característica importante do processo de determinação do momento de inércia é o fato de que --~assim como para o cálculo da posição do centro de massa~-- temos uma quantidade muito grande de partículas. Através da Expressão~\ref{Eq:MomentoInerciaConjPart}, em tese podemos determinar o momento de inércia para qualquer corpo. No entanto, isso é claramente inadequado para o caso de um corpo rígido, devido ao fato de que a soma teria um número muito grande de termos. Além disso, não sabemos precisar quais são as posições de cada uma das partículas, ou quais são as suas massas. Diferentemente da determinação do centro de massa, não podemos utilizar argumentos de simetria para determinar o momento de inércia\footnote{Isso se deve ao fato de que a distância ao eixo de rotação aparece ao quadrado na Expressão~\ref{Eq:MomentoInerciaConjPart}. No caso do cálculo do centro de massa nos valíamos do sinal que aparece na distância ao eixo de simetria para garantir que os termos simétricos se cancelassem.}, por isso vamos ter que utilizar técnicas de Cálculo para que possamos considerar uma distribuição contínua de massa.

Nas próximas seções discutiremos alguns métodos de cálculo que nos permitirão determinar o momento de inércia para corpos rígidos extensos: em muitos casos as expressões são bastante simples e --~como verificaremos~-- uma das propriedades do momento de inércia é a aditividade, o que nos permitirá tratar um corpo complexo através da decomposição em formas mais simples. Além disso, verificaremos que o momento de inércia para um eixo qualquer pode ser determinado a partir do momento de inércia em torno de um eixo que passa pelo centro de massa do objeto, desde que ambos os eixos sejam paralelos. Finalmente, veremos que em alguns casos o momento de inércia em torno de três eixos perpendiculares entre si não são independentes, o que nos permite calcular um deles se conhecemos os outros dois.

%%%%%%%%%%%%%%%%%%%%%%%%%%%%%%%%%%%%%%%%%%%%%%
\subsection{Aditividade do momento de inércia}
%%%%%%%%%%%%%%%%%%%%%%%%%%%%%%%%%%%%%%%%%%%%%%

\begin{marginfigure}[5cm]
\centering
\begin{tikzpicture}[>=Stealth]

%%% Figura superior

\draw[fill] (0,0) circle (0.6pt);
\draw[fill] (0,-0.2) circle (1pt) node[right] {CM};
\draw (0,0) ellipse (1.25 and 0.5);
\draw[fill] (0,-0.4) circle (0.6pt);

\draw (-1.25,0) -- (-1.25,-0.4);
\draw (1.25,-0.4) -- (1.25,0);  

%% arco inferior
\draw (-1.25,-0.4) arc (180:219:1.25 and 0.5);
\draw (1.25,-0.4) arc (0:-135:1.25 and 0.5);
\draw[densely dotted] (-1.25,-0.4) arc (180:360:1.25 and -0.5);

% Eixo z
\draw[dashdotted,->] (0,0) -- (0,1.5) node[below left]{$z$};
\draw[dashdotted] (0,-1.5) -- (0,-0.9);
\draw[dotted] (0,-0.9) -- (0,0);

\draw[->] (0,1.05) [partial ellipse=-225:60:0.3cm and 0.125cm];

% haste
\draw (0,-0.2)+(30:-2) ellipse (0.45mm and 0.5mm);
\draw (0,-0.25)+(30:-0.82) arc[start angle = -90, end angle = 90, radius = 0.5mm];
\draw (0,-0.146)+(30:-2) -- +(30:-0.82);
\draw (0,-0.254)+(30:-2) -- +(30:-0.82);

\end{tikzpicture}
\caption{Para um corpo complexo, podemos determinar o momento de inércia separando-o em partes simples, cujo momento de inércia sabemos determinar. \label{Fig:AditividadeMomInercia}}
\end{marginfigure}

Através da Equação~\ref{Eq:MomentoInerciaConjPart}, podemos mostrar que o momento de inércia de um corpo extenso é aditivo: sabemos que
\begin{equation}
    I = \sum_i m_i (r_\perp^i)^2,
\end{equation}
%
ou, se separarmos a soma em diferentes regiões,
\begin{align}
    I &= \sum_{i}^{R_1} m_i (r_\perp^i)^2 + \sum_{i}^{R_1} m_i (r_\perp^i)^2 + \sum_{i}^{R_1} m_i (r_\perp^i)^2 + \dots \\
    &= I_1 + I_2 + I_3 + \dots \\
    &= \sum_i I_i.
\end{align}

Isso significa que podemos dividir o corpo em regiões cujo momento de inércia sabemos calcular, e, após isso, determinamos o momento de inércia total simplesmente somando os resultados. É importante destacar que a posição das regiões em relação ao eixo de rotação não muda, isto é, apesar de separarmos o corpo de diversas partes, a posição que elas ocupam no corpo altera o valor obtido para o momento de inércia. No caso da Figura~\ref{Fig:AditividadeMomInercia}, por exemplo, podemos calcular separadamente o momento de inércia do disco e da haste, mas devemos levar em conta que a haste gira em torno de um eixo que está distante da extremidade.

%%%%%%%%%%%%%%%%%%%%%%%%%%%%%%%%%%%%%%%%%%%%%%%%%%%%%%%%%%%%
\subsection{Momento de inércia de uma distribuição contínua de massa}
%%%%%%%%%%%%%%%%%%%%%%%%%%%%%%%%%%%%%%%%%%%%%%%%%%%%%%%%%%%%

\begin{marginfigure}[4cm]
\centering
\begin{tikzpicture}[interface/.style={
        % superfície
        postaction={draw,decorate,decoration={border,angle=-45,
                    amplitude=0.2cm,segment length=2mm}}},
    ]
    
% topo
\draw (0,0) ellipse (1.25 and 0.5);
\draw (0,0) ellipse (1.3 and 0.55);

\draw[<->] (0,0) -- node[above]{$R$} (1.25,0);

% laterais
\draw[dotted] (-1.25,0) -- (-1.25,-3);
\draw[dotted] (1.25,-3) -- (1.25,0);
\draw (-1.3,0) -- (-1.3,-3);
\draw (1.3,-3) -- (1.3,0);  

% fundo
\draw[dotted] (-1.25,-3) arc (180:360:1.25 and 0.5);
\draw[dotted] (-1.25,-3) arc (180:360:1.25 and -0.5);
\draw[dotted] (-1.3,-3) arc (180:360:1.3 and -0.55);
\draw (-1.3,-3) arc (180:360:1.3 and 0.55);

% Eixo z
\draw[dashdotted,-Stealth] (0,-0.5) -- (0,1.5) node[below left]{$z$};
\draw[loosely dotted] (0,-3.55) -- (0,-0.5);
\draw[dashdotted] (0,-3.55) -- (0,-4);

\draw[fill] (0,-1.5) circle (1pt) node[above left] {CM};

\draw[-Stealth] (0,1.05) [partial ellipse=-225:60:0.3cm and 0.125cm];

\end{tikzpicture}
\caption{Tubo cilindrico formado por paredes finas. \label{Fig:MomInerciaTubo}}
\end{marginfigure}

Sabemos que o momento de inércia de um conjunto de partículas é dado pela Expressão~\eqref{Eq:MomentoInerciaConjPart}, porém, não determinamos até agora o momento de inércia de nenhum objeto que possa ser interessante de alguma maneira. Vamos então realizar esse cálculo para um tubo formado por uma parede extremamente fina. Nesse caso, temos que
\begin{equation}
    I = \sum_i m_i (r_\perp^i)^2,
\end{equation}
%
onde utilizamos $r_\perp^i$ para denotar a distância de cada partícula ao eixo de rotação. Note que, como a parede do tubo é muito fina, podemos em boa aproximação assumir que a distância das partículas ao eixo é igual ao raio do próprio tubo:\footnote{Qualquer tubo real tem um raio interno e um raio externo. Se utilizarmos o raio interno, obtemos um valor mínimo para $I$, se usarmo o externo, obtemos um valor máximo. Estamos assumindo que a a espessura da parede é realmente muito fina, de forma que essa diferença seja desprezível.}
\begin{equation}
    r_\perp^i = R.
\end{equation}
%
Assim, temos para o momento de inércia
\begin{align}
    I &= \sum_i m_i (R)^2 \\
    &= \left[\sum_i m_i\right] R^2,
\end{align}
%
ou seja,
\begin{equation}
    I = MR^2. \mathnote{Momento de inércia de um tubo/anel}
\end{equation}

O momento de inércia obtido acima também é válido para um anel fino, pois nesse caso também é válida a consideração de que a distância de cada partícula ao eixo de rotação é a mesma, e seu valor é o próprio raio do objeto. Dada uma quantidade de massa $M$ qualquer, se limitamos a distância ao eixo de rotação a um valor qualquer, a forma que maximiza o momento de inércia é a de um tubo/anel. Nesse caso, assumimos que todas as partículas estão exatamente à distância limite em relação ao eixo.

\begin{marginfigure}[2cm]
\centering
\begin{tikzpicture}[scale = 1.2,
    interface/.style={
        % superfície
        postaction={draw,decorate,decoration={border,angle=-45,
                    amplitude=0.2cm,segment length=2mm}}},
    ]
    
% topo
\draw (0,0) ellipse (1.3 and 0.55);
\draw (-1.3, 0) -- (-1.3,-0.25);
\draw (1.3, 0) -- (1.3, -0.25);
\draw (-1.3,-0.25) arc (180:360:1.3 and 0.55);
\draw[dotted] (-1.3,-0.25) arc (180:360:1.3 and -0.55);

%\draw[<->] (0,0) -- node[above]{$R$} (1.25,0);

% laterais
\draw[densely dotted] (-1.3,-0.25) arc[start angle = 90, end angle = -14, radius = 0.25];
\draw[densely dashed] (-1.3,-0.75) arc[start angle = -90, end angle = -14, radius = 0.25];

\draw[densely dotted] (1.3,-0.25) arc[start angle = 90, end angle = 194, radius = 0.25];
\draw[densely dashed] (1.3,-0.75) arc[start angle = -90, end angle = -166, radius = 0.25];

% fundo
\draw (-1.3,-0.75) arc (180:153:1.3 and 0.55);
\draw (1.3,-0.75) arc (0:27:1.3 and 0.55);
\draw (-1.3,-0.75) arc (180:360:1.3 and 0.55);
\draw[dotted] (-1.3,-0.75) arc (180:360:1.3 and -0.55);
\draw (-1.3,-0.75) -- (-1.3,-1);
\draw (1.3,-0.75) -- (1.3,-1);
\draw (-1.3,-1) arc (180:360:1.3 and 0.55);
\draw[dotted] (-1.3,-1) arc (180:360:1.3 and -0.55);
\draw[fill] (0,-1) circle (0.6pt);

% Eixo z
\draw[fill] (0,0) circle (0.6pt);
\draw[dashdotted,-Stealth] (0,0) -- (0,1.5) node[below left]{$z$};
\draw[densely dotted] (0,0) -- (0,-1.55);
\draw[dashdotted] (0,-1.55) -- (0,-2);

\draw[fill] (0,-0.5) circle (1pt);

\draw[-Stealth] (0,1.05) [partial ellipse=-225:60:0.3cm and 0.125cm];

% dm
\begin{scope}[scale = 0.15, rotate around y = -7, shift = {(-5, -1)}]

    \draw[fill, lightgray, draw = black] (1,1,1) -- (1,1,0) -- (0,1,0) -- (0,1,1) -- cycle;
    \draw[fill, lightgray, draw = black] (1,1,1) -- (1,1,0) -- (1,0,0) -- (1,0,1) -- cycle;
    \draw[fill, lightgray, draw = black] (1,1,1) -- (0,1,1) -- (0,0,1) -- (1,0,1) -- cycle;

\end{scope}

\draw[-Stealth] (0,-0.18) coordinate (brcmc) -- node[above]{$r_\perp$} (-0.68,-0.09) coordinate (cmc);
\draw[fill] (brcmc) circle (0.3pt);
\draw[fill] (cmc) circle (0.6pt);
\coordinate (O) at (0,0);

\pic [draw, "$\cdot$", angle radius = 1.7mm, angle eccentricity = 0.5] {angle = O--brcmc--cmc};

% Visão lateral

\draw (-1.3,-3) rectangle (1.3,-3.25);
\draw (-1.3,-3.25) arc[start angle = 90, end angle = -90, radius = 0.25];
\draw (1.3,-3.25) arc[start angle = 90, end angle = 270, radius = 0.25];
\draw (-1.3,-4) rectangle (1.3, -3.75);

\draw[dashdotted, -Stealth] (0,-3) -- +(0,0.75) node[below left]{$z$};
\draw[dotted] (0,-3) -- +(0,-1);
\draw[dashdotted] (0,-4) -- +(0,-0.75);

\draw[fill] (0,-3.5) circle (1pt);

\end{tikzpicture}
\caption{Roldana formada por um cilindro com uma ``calha'' semi-circular.}
\end{marginfigure}

Exceto para o caso do tubo/anel acima, não podemos empregar a Expressão~\eqref{Eq:MomentoInerciaConjPart} para determinar o momento de inércia de um corpo rígido. Nesse caso, podemos fazer algo similar ao que fizemos para determinar a posição do centro de massa de um objeto extenso: dividimos o corpo em uma série de regiões cúbicas e vamos considerar que toda a massa está localizada no centro de massa do cubo. Assim, temos que
\begin{equation}
    I = \sum_{i = 1}^N M_{R_i} (r_\perp^i)^2.
\end{equation}

Como no caso do centro de massa, a divisão em cubos pode não ser muito precisa para um corpo qualquer cujos limites não se alinhem com os cubos, porém sempre podemos melhorar a precisão ao aumentar o número de cubos, diminuindo assim o volume de cada um deles. Ao tomarmos o limite de infinitos cubos, recaímos novamente na definição de uma integral em duas ou três dimensões:\footnote[][3cm]{O caso bidimensional é um caso especial do tridimensional, e se aplica aos casos onde temos objetos em que uma das dimensões é constante para todo o objeto --~ou seja, uma figura plana, como uma placa metálica, por exemplo~--.}
\begin{align}
    I &= \lim_{N \to \infty} \sum_{i = 1}^N M_{R_i} (r_\perp^i)^2 \\
    &= \int r_\perp^2 dm. \label{Eq:MomInerciaDistContinua}
\end{align}
%
A interpretação da expressão acima é mais simples se escrevermos $dm$ em termos de densidades, como fizemos na determinação da posição do centro de massa:
\begin{align}
    dm &= \lambda(x) \; dx \\
    dm &= \sigma(\vec{r}) \;dA \\
    dm &= \rho(\vec{r}) \; dV.
\end{align}

Mais uma vez, o uso de expressões integrais como a Equação~\eqref{Eq:MomInerciaDistContinua} acima depende de podermos descrever a forma de um corpo matematicamente, o que só pode ser feito facilmente para formas simples. Nesse caso, as expressões resultantes para o momento de inércia dos corpos serão diferentes para cada tipo de corpo, porém são razoavelmente simples.

%%%%%%%%%%%%%%%%%%%%%%%%%%%%%%%%%%%%%%%%%%%%%%%%%%%%%%%%%
\paragraph{Exemplo: Momento de inércia de uma haste fina}
%%%%%%%%%%%%%%%%%%%%%%%%%%%%%%%%%%%%%%%%%%%%%%%%%%%%%%%%%

Um exemplo simples de aplicação da Equação~\eqref{Eq:MomInerciaDistContinua} é a determinação do momento de inércia de uma haste fina  e homogênea que gira em torno de um eixo que passa por seu centro de massa, perpendicularmente ao eixo da haste --~como mostrado na Figura~\ref{Fig:CompMomInerciaHasteFina}~--.

\begin{marginfigure}[-2cm]
\centering
\begin{tikzpicture}[>=Stealth,
     interface/.style={
        % superfície
        postaction={draw,decorate,decoration={border,angle=-45,
                    amplitude=0.2cm,segment length=2mm}}},
    ]
    
%%% Haste

\draw[fill] (0,0) circle (0.4pt);

\draw[dashdotted, ->] (0,0)+(0,0.1) -- +(0, 1.5) coordinate (top) node[below left]{$z$};
\draw[dashdotted] (0,0)+(0,-0.1) -- +(0,-1);

\draw (15:-1.25) circle (0.05);
\draw (0,0.05) coordinate (angmid) -- +(15:-1.25) coordinate (angstart);
\draw (0,-0.05) -- +(15:-1.25) node[below]{$\nicefrac{L}{2}$};

\draw (0,0.05)+(15:1.2) arc[start angle = 90, end angle = -90, radius = 0.05];
\draw (0,0.05) -- +(15:1.2);
\draw (0,-0.05) -- +(15:1.2) node[below]{$-\nicefrac{L}{2}$};

\draw[fill] (15:-1.25) circle (0.4pt);
\draw[dashdotted, ->] (15:-1.25) -- (15:-2.15) node[below right] {$x$};
\draw[dashdotted] (15:1.25) -- (15:2);

\pic [draw, "$\cdot$", angle eccentricity = 0.5, angle radius = 1.5mm]{angle = top--angmid--angstart};

\draw[->] (100:1) arc[start angle = -240, end angle = 60, x radius = 0.3, y radius = 0.15];

\draw[draw = black, fill] (0,0.05) ++(15:-0.5) arc[start angle = 90, end angle = -90, radius = 0.05] -- +(15:0.1) arc[start angle = -90, end angle = 90, radius = 0.05] -- node[above]{$dx$} cycle;

\draw (0, -0.1) -- +(0,-0.1);
\draw (0,-0.15) -- node[below]{$x$} +(15:-0.4);
\draw (0,-0.1) ++(15:-0.4) -- +(0,-0.1);

\end{tikzpicture}
\caption{Uma haste fina pode ser considerada como uma distribuição contínua de massa em que somente uma das dimensões é relevante.\label{Fig:CompMomInerciaHasteFina}}
\end{marginfigure}

Podemos dividir a massa $M$ da haste por seu comprimento $L$, obtendo uma densidade linear de massa $\lambda$:
\begin{equation}
    \lambda = \frac{M}{L}.
\end{equation}
%
Se tomarmos um segmento da haste com comprimento $dx$, temos que a massa correspondente a este segmento é dada por
\begin{equation}
    dm = \lambda \; dx,
\end{equation}
%
o que nos permite escrever a expressão para o momento de inércia em torno do eixo $z$ como
\begin{equation}
    I_z = \int r_\perp^2 \lambda \;dx.
\end{equation}
%
Note que a distância de um segmento $dx$ ao eixo de rotação --~isto é, $r_\perp$~-- nada mais é que a própria variável $x$. Logo
\begin{equation}
    I_z = \int_{-\nicefrac{L}{2}}^{\nicefrac{L}{2}} \lambda x^2 \; dx,
\end{equation}
%
onde já substituimos os limites de integração, que correspondem às posições das extremidades da haste no eixo $x$. Realizando a integração, temos
\begin{align}
    I_z &= \int_{-\nicefrac{L}{2}}^{\nicefrac{L}{2}} \lambda x^2 \; dx \label{Eq:MomInerciaHasteFinaInteg}\\
    &= \lambda \int_{-\nicefrac{L}{2}}^{\nicefrac{L}{2}} x^2 \; dx \\
    &= \lambda \left[\frac{x^3}{3} + C\right]_{-\nicefrac{L}{2}}^{\nicefrac{L}{2}} \\
    &= \frac{\lambda}{3} \left\{\left[\left(\frac{L}{2}\right)^3 + C\right] - \left[\left(\frac{-L}{2}\right)^3 + C\right] \right\} \\
    &= \frac{\lambda}{3} \left[\frac{L^3}{8} + \frac{L^3}{8}\right] \\
    &= \frac{\lambda}{3}\frac{L^3}{4} \\
    &= \frac{\lambda L^3}{12}.
\end{align}
%
Finalmente, substituindo a expressão para a densidade, temos
\begin{equation}
    I_z = \frac{ML^2}{12}. \mathnote{Momento de inércia de uma haste fina}
\end{equation}

Se desejarmos determinar o momento de inércia de uma haste fina de comprimento $L$ em torno de um eixo que passa por sua extremidade, perpendicularmente ao eixo da haste, basta substituirmos os limites de integração na Equação~\eqref{Eq:MomInerciaHasteFinaInteg}. Fazendo isso, obtemos
\begin{align}
    I_z &= \int_{0}^{L} \lambda x^2 \; dx\\
    &= \lambda \int_{0}^{L} x^2 \; dx \\
    &= \lambda \left[\frac{x^3}{3} + C\right]_{0}^{L} \\
    &= \frac{\lambda}{3} \{[(L)^3 + C] - [(0)^3 + C] \} \\
    &= \frac{\lambda L^3}{3},
\end{align}
%
de onde obtemos
\begin{equation}
    I_z = \frac{ML^2}{3}.
\end{equation}

%%%%%%%%%%%%%%%%%%%%%%%%%%%%%%%%%%%%%%%%%%%%%%%%%%%%%%
\paragraph{Momentos de inércia de sólidos geométricos}
%%%%%%%%%%%%%%%%%%%%%%%%%%%%%%%%%%%%%%%%%%%%%%%%%%%%%%

Através da Equação~\eqref{Eq:MomInerciaDistContinua}, podemos obter os momentos de inércia de diversos sólidos geométricos simples. Não vamos determinar cada um desses resultados pois em geral é necessário um conhecimento de cálculo de integrais em duas e três dimensões.

\begin{description}

%%
%% Esfera
%%
    \item[Esfera:] A esfera é um caso especialmente interessante pois além de ser um sólido simples, ela tem a propriedade de o momento de inércia para todos os eixos que passam pelo centro de massa é o mesmo. Além disso, podemos distinguir entre o caso de esferas maciças e de cascas esféricas, ou seja, de esferas ocas formadas por uma parede fina. Os momentos de inércia são dados por
\begin{marginfigure}[-5mm]
\centering
\begin{tikzpicture}[>=Stealth]

    \draw (0,0) circle (1.03cm);
    \draw[fill] (0,0) circle (1pt) node [above left]{CM};
    
    \draw[dashdotted, <-] (60:1.8) node [below right]{$z$} -- (60:0.83) coordinate (furosup);
    \draw[fill] (60:0.8) circle (0.6pt);
    
    \draw[dotted] (furosup) -- (60:-1);
    \draw[fill] (60:-0.82) circle (0.6pt);
    
    \draw[dashdotted] (60:-1) -- (60:-1.75);
    
    \draw[dashed, rotate = -30] (-1.03,0) arc [start angle = -180, end angle = 0, x radius = 1.03, y radius = 0.5];
    \draw[densely dotted, rotate = -30] (1.03,0) arc [start angle = 0, end angle = 180, x radius = 1.03, y radius = 0.5];
    
    \draw[->, scale = 0.8, rotate = -30] (100:1.8) arc[start angle = -210, end angle = 60, x radius = 0.4, y radius = 0.2];
    
\end{tikzpicture}
\caption{Esfera que pode girar em torno de um eixo que passa por seu centro de massa.}
\end{marginfigure}  
        \begin{align}
            I^{\textrm{esf}}_{\textrm{oca}} &= \frac{2}{3} MR^2 \\
            I^{\textrm{esf}}_{\textrm{mac}} &= \frac{2}{5} MR^2.
        \end{align}
%%
%% Disco
%%  
           
\begin{marginfigure}
\centering
\begin{tikzpicture}[>=Stealth, scale = 1.2,
     interface/.style={
        % superfície
        postaction={draw,decorate,decoration={border,angle=-45,
                    amplitude=0.2cm,segment length=2mm}}},
    ]

% topo
\draw[fill] (0,0) circle (0.6pt);
\draw[fill] (0,-0.2) circle (1pt) node[above left] {CM};
\draw (0,0) ellipse (1.25 and 0.5);
\draw[fill] (0,-0.4) circle (0.6pt);

% Laterais
\draw (-1.25,0) -- (-1.25,-0.4);
\draw (1.25,-0.4) -- (1.25,0);  

% Raio
\draw[dotted] (1.25,-0.4) -- (1.25,-1.25);
\draw[|<->|] (0,-1.25) -- node[above]{$R$} (1.25,-1.25);

% Fundo
\draw (-1.25,-0.4) arc (180:360:1.25 and 0.5);
\draw[dotted] (-1.25,-0.4) arc (180:360:1.25 and -0.5);

% Eixo z
\draw[dashdotted,->] (0,0) -- (0,1.5) node[below left]{$z$};
\draw[dashdotted] (0,-1.5) -- (0,-0.9);
\draw[loosely dotted] (0,-0.9) -- (0,0);

\draw[->] (0,1.05) [partial ellipse=-225:60:0.3cm and 0.125cm];

% Eixo x
\draw[dashdotted, ->] (0,-0.2)++(15:-1.05) -- +(15:-0.75) node[below right]{$x$};
\draw[fill] (0,-0.2)+(15:-1.05) circle (0.6pt);
\draw[loosely dotted] (0,-0.2)+(15:-1.05) -- (0,-0.2) -- +(15:1.05);
\draw[fill] (0,-0.2) +(15:1.05) circle (0.6pt);

\draw[dashdotted] (0,-0.2) ++(15:1.25) -- +(15:0.75);

\draw[->, rotate=15] (0,-0.2)++(0:-1.5) [partial ellipse=-130:160:0.105cm and 0.3cm];

%\draw[->, rotate=60] (0,-0.2)++(-4:-1.2) [partial ellipse=-150:150:0.15cm and 0.3cm];

\end{tikzpicture}
\caption{Disco. \label{Fig:MomInerciaDisco}}
\end{marginfigure}

    \item[Disco, cilindro:] Vamos considerar somente dois eixos, que corresponem às situações mais simples:
        \begin{align}
            I_z &= \frac{1}{2}MR^2 \\
            I_x &= \frac{1}{4} MR^2.
        \end{align}
           
%%
%% Anel fino
%% 
       
\begin{marginfigure}[5mm]
\centering
\begin{tikzpicture}[interface/.style={>=Stealth,
        % superfície
        postaction={draw,decorate,decoration={border,angle=-45,
                    amplitude=0.2cm,segment length=2mm}}},
    ]
    
% topo
\draw (0,0) ellipse (1.25 and 0.5);
\draw (0,0) ellipse (1.3 and 0.55);

\draw[Stealth-Stealth] (0,0) -- node[below]{$R$} (-1.25,0);

% laterais
\draw[dotted] (-1.25,0) -- (-1.25,-0.2);
\draw[dotted] (1.25,-0.2) -- (1.25,0);
\draw (-1.3,0) -- (-1.3,-0.2);
\draw (1.3,-0.2) -- (1.3,0);  

% fundo
\draw[dotted] (-1.25,-0.2) arc (180:360:1.25 and 0.5);
\draw[dotted] (-1.25,-0.2) arc (180:360:1.25 and -0.5);
\draw (0,-0.2)+(0,0.5) arc (90:12:1.25 and 0.5);
\draw (0,-0.2)+(0,0.5) arc (90:168:1.25 and 0.5);
\draw[dotted] (-1.3,-0.2) arc (180:360:1.3 and -0.55);
\draw (-1.3,-0.2) arc (180:360:1.3 and 0.55);

% Eixo z
\draw[dashdotted,-Stealth] (0,-0.5) -- (0,1.5) node[below left]{$z$};
\draw[dotted] (0,-0.5) -- +(0,-0.2);
\draw[dashdotted] (0,-0.75) -- (0,-1.5);

% Eixo x
\draw[dashdotted] (0,-0.1) -- +(25:-0.75);
\draw[fill] (0,-0.1) +(25:-0.9) circle (0.3pt);
\draw[fill] (0,-0.1) +(25:-0.95) circle (0.3pt);
\draw[dashdotted,-Stealth] (0,-0.1)++(25:-0.95) -- +(25:-0.75) node[below right]{$x$};

\draw[dashdotted] (0,-0.1) -- +(25:0.95);
\draw[fill] (0,-0.1) +(25:0.9) circle (0.3pt);
\draw[fill] (0,-0.1) +(25:0.95) circle (0.3pt);
\draw[dashdotted] (0,-0.1)++(25:1.08) -- +(25:0.75);

\draw[fill] (0,-0.1) circle (1pt) node[below right] {CM};

\draw[-Stealth] (0,1.05) [partial ellipse=-225:60:0.3cm and 0.125cm];
\draw[rotate = 120, -Stealth] (0,1.4) [partial ellipse=-225:60:0.3cm and 0.125cm];

\end{tikzpicture}
\caption{Anel/Tubo cilindrico formado por paredes finas. \label{Fig:MomInerciaTuboAnelFinoTab}}
\end{marginfigure}

    \item[Anel, tubo (finos):] Tanto para o anel, quanto para o tubo, quando o eixo de rotação é perpendicular à face circular, temos para um eixo que passa pelo centro de massa
        \begin{equation}
            I_z = MR^2.
        \end{equation}
        %
        Para o anel, em particular, quando o eixo passa pelo centro de massa e está contido no plano determinado pelo próprio anel, temos
        \begin{equation}
            I_x = \frac{1}{2} MR^2.
        \end{equation}
        
%%
%% Anel, tubo
%%
  
    \item[Anel, tubo:] Para o caso de um anel ou tubo onde a espessura da parede não é desprezível, o momento de inércia em torno de um eixo perpendicular à face plana e que passa pelo centro de massa é dado por
        \begin{equation}
            I_z = M \frac{R_i^2 + R_e^2}{2}.
        \end{equation}
        
\begin{marginfigure}[-15mm]
\centering
\begin{tikzpicture}[interface/.style={>=Stealth,
        % superfície
        postaction={draw,decorate,decoration={border,angle=-45,
                    amplitude=0.2cm,segment length=2mm}}},
    ]
    
% topo
\draw (0,0) ellipse (1.25 and 0.5);
\draw (0,0) ellipse (1.5 and 0.7);

\draw[Stealth-Stealth] (0,0) -- node[above]{$R_i$} (-1.25,0);
\draw[Stealth-Stealth] (0,0) -- node[above]{$R_e$} (-20:1.25);

% laterais
\draw[dotted] (-1.25,0) -- (-1.25,-1.5);
\draw[dotted] (1.25,-1.5) -- (1.25,0);
\draw (-1.5,0) -- (-1.5,-1.5);
\draw (1.5,-1.5) -- (1.5,0);  

% fundo
\draw[dotted] (-1.25,-1.5) arc (180:360:1.25 and 0.5);
\draw[dotted] (-1.25,-1.5) arc (180:360:1.25 and -0.5);
\draw[dotted] (-1.5,-1.5) arc (180:360:1.5 and -0.7);
\draw (-1.5,-1.5) arc (180:360:1.5 and 0.7);

% Eixo z
\draw[dashdotted,-Stealth] (0,-0.5) -- (0,1.5) node[below left]{$z$};
\draw[dotted] (0,-0.5) -- (0,-2.2);
\draw[dashdotted] (0,-2.2) -- (0,-2.9);


\draw[fill] (0,-0.75) circle (1pt) node[below right] {CM};

\draw[-Stealth] (0,1.05) [partial ellipse=-225:60:0.3cm and 0.125cm];

\end{tikzpicture}
\caption{Anel/Tubo cilíndrico. \label{Fig:MomInerciaTuboAnelTab}}
\end{marginfigure}
          
%%
%% Haste fina
%%
   
\begin{marginfigure}[5mm]
\centering
\begin{tikzpicture}[>=Stealth,
     interface/.style={
        % superfície
        postaction={draw,decorate,decoration={border,angle=-45,
                    amplitude=0.2cm,segment length=2mm}}},
    ]
    
\draw[fill] (0,0) circle (0.4pt);

\draw[dashdotted, ->] (0,0)+(0,0.1) -- +(0, 1.5) coordinate (top) node[below left]{$z$};
\draw[dashdotted] (0,0)+(0,-0.1) -- +(0,-1);

\draw (15:-1.25) circle (0.05);
\draw (0,0.05) coordinate (angmid) -- +(15:-1.25) coordinate (angstart);
\draw (0,-0.05) -- +(15:-1.25);

\draw (0,0.05)+(15:1.2) arc[start angle = 90, end angle = -90, radius = 0.05];
\draw (0,0.05) -- +(15:1.2);
\draw (0,-0.05) -- +(15:1.2);

\draw[fill] (15:-1.25) coordinate (icomp) circle (0.4pt);
\draw[dashdotted, ->] (15:-1.25) -- (15:-2.15) node[below right] {$x$};
\draw[dashdotted] (15:1.25) -- (15:2);

\pic [draw, "$\cdot$", angle eccentricity = 0.5, angle radius = 1.5mm]{angle = top--angmid--angstart};

\draw[->] (100:1) arc[start angle = -240, end angle = 60, x radius = 0.3, y radius = 0.15];

\draw[|-|] (0,-0.3)+(15:-1.25) -- node[below right]{$L$} +(15:1.25);

\end{tikzpicture}
\caption{Haste fina.}
\end{marginfigure}

    \item[Haste (fina):] Como verificamos anteriormente, o momento de inércia de uma haste fina e que gira em torno de um eixo que passa pelo seu centro de massa, perpendicularmente ao eixo da haste, é dado por
        \begin{equation}
            I_z = \frac{1}{12} ML^2.
        \end{equation}
    %
    Note que a forma da seção reta da haste não é relevante, uma vez que estamos considerando que ela seja fina o suficiente para que somente o comprimento seja uma dimensão relevante.
    
%%
%% Haste
%%
 
\begin{marginfigure}[5mm]
\centering
\begin{tikzpicture}[>=Stealth,
     interface/.style={
        % superfície
        postaction={draw,decorate,decoration={border,angle=-45,
                    amplitude=0.2cm,segment length=2mm}}},
    ]
    
\draw[fill] (0,0) circle (0.4pt);

\draw[dashdotted, ->] (0,0)+(0,0.2) -- +(0, 1.5) coordinate (top) node[below left]{$z$};
\draw[dashdotted] (0,0)+(0,-0.2) -- +(0,-1);

\draw[rotate around={10:(15:-1.25)}, fill = white] (15:-1.25) circle[x radius = 0.1, y radius = 0.2];
\draw (0,0.2) coordinate (angmid) -- +(15:-1.25) coordinate (angstart);
\draw (0,-0.2) -- +(15:-1.25);

\draw[rotate around={10:(15:1.25)}] (0,0.2)+(15:1.2) arc[start angle = 90, end angle = -88, x radius = 0.1, y radius = 0.19];
\draw (0,0.2) -- +(15:1.18);
\draw (0,-0.2) -- +(15:1.25);

\draw[fill] (15:-1.25) coordinate (icomp) circle (0.4pt);
\draw[dashdotted, ->] (15:-1.25) -- (15:-2.15) node[below right] {$x$};
\draw[dashdotted] (15:1.4) -- (15:1.6);
\draw (0,0.2)+(15:2) -- node[right]{$R$} (15:2);
\draw (0,0.2)+(15:1.95) -- +(15:2.05);
\draw (15:1.95) -- (15:2.05);
\draw[dashdotted] (15:1.5) -- (15:1.9);
\draw[dotted] (0,0.2)+(15:1.3) -- +(15:1.9);

\pic [draw, "$\cdot$", angle eccentricity = 0.5, angle radius = 1.5mm]{angle = top--angmid--angstart};

\draw[->] (100:1) arc[start angle = -240, end angle = 60, x radius = 0.3, y radius = 0.15];

\path (0,-0.5)+(15:-1.25) coordinate (A) -- +(15:1.25) coordinate (B);
\draw (A)+(0,0.1) -- (A)+(0,-0.1);
\draw (B)+(0,0.1) -- (B)+(0,-0.1);

\draw (0,-0.45)+(15:-1.25) coordinate (A) -- node[below right]{$L$} +(15:1.25) coordinate (B);

\end{tikzpicture}
\caption{Haste cilíndrica.}
\end{marginfigure}  

    \item[Haste cilíndrica:] Se temos uma haste cilíndrica cujo raio não pode ser desprezado, temos que o momento de inércia em torno de um eixo perpendicular ao eixo da haste, e que passa por seu centro de massa é dado por
    \begin{equation}
        I_z = \frac{1}{4} MR^2 + \frac{1}{12} ML^2.
    \end{equation}
    %
    Note que esse resultado se reduz ao primeiro termo, se temos um disco --~isto é, um cilindro cuja altura é desprezível~-- ou ao caso de uma haste fina --~se o raio é desprezível.  
%% 
%% Placa, cubo
%%
    
    \item[Placa, prisma retangular:] Se temos uma placa retangular --~ou mesmo um prisma retangular~-- que gira em torno de um eixo que passa por seu centro de massa, perpendicularmente ao plano, o momento de inércia é dado por
    \begin{equation}
        I_z = \frac{1}{12}M (a^2 + b^2).
    \end{equation}
    
\begin{marginfigure}
\centering
\begin{tikzpicture}[
     interface/.style={
        % superfície
        postaction={draw,decorate,decoration={border,angle=-45,
                    amplitude=0.2cm,segment length=2mm}}},
    ]
    
    % topo
    \draw (1.4,1,1) -- (1.4,1,0) -- (0,1,0) -- (0,1,1) -- cycle;
    
    % Laterais
    \draw (1.4,1,1) -- +(0, -1.5, 0);
    \draw (0,1,1) -- +(0, -1.5, 0);
    \draw (1.4,1,0) -- +(0, -1.5, 0);
    \draw[densely dotted] (0,1,0) -- +(0, -1.5, 0);
    
    % fundo
    \draw[densely dotted] (0,-0.5,0) ++(0,0,1) -- ++(0,0,-1) -- +(1.4,0,0);
    \draw (0,-0.5,0)++(0,0,1) -- ++(1.4,0,0) -- +(0,0,-1);
    
    % Eixo z
    \draw[Stealth-, dashdotted] (0.7, 2.2, 0.5) node[below right]{$z$}-- (0.7, 1, 0.5);
    \draw[fill] (0.7,1,0.5) circle (0.4pt);
    \draw[dotted] (0.7,1,0.5) -- +(0,-1.7,0) coordinate (lowz);
    \draw[fill] (0.7,1,0.5) +(0,-1.5,0) circle (0.4pt);
    \draw[dashdotted] (lowz) -- +(0,-0.5);
    
    \tikzset{zxplane/.style={canvas is zx plane at y=#1}}
    \begin{scope}[zxplane=1.7]
        \draw[-Stealth] (0.5,0.7)+(-120:0.3) arc[start angle = -120, end angle = 180, radius = 0.3];
    \end{scope}
     
    % CM
    \draw[fill] (0.7,0.3,0.5) circle (1pt) node[below left]{CM};
    
    % medidas
    \draw (-0.2, 1, 1) -- (-0.2, 1, 0);
    \draw (-0.1, 1, 1) -- (-0.3, 1, 1);
    \draw (-0.1, 1, 0) -- (-0.3, 1, 0);
    \node (a) at (-0.4, 1, 0.3) {$a$};
    
    \draw (0, 1, -0.3) -- node[above]{$b$} (1.4, 1, -0.3);
    \draw (0, 1, -0.2) -- (0,1,-0.4);
    \draw (1.4, 1, -0.2) -- (1.4, 1, -0.4);
    
\end{tikzpicture}
\caption{Placa/prisma retangular.}
\end{marginfigure}
    
\end{description}

%%%%%%%%%%%%%%%%%%%%%%%%%%%%%%%%%%%%%%%%
\subsection{Teorema dos eixos paralelos}
%%%%%%%%%%%%%%%%%%%%%%%%%%%%%%%%%%%%%%%%

Todos os eixos para os quais apresentamos resultados para o momento de inércia sempre passam pelo centro de massa. Isso não é por acaso: caso conheçamos o momento de inércia em torno de um eixo que passa pelo centro de massa, podemos calcular o momento de inércia em torno de qualquer eixo que seja paralelo a ele.

\begin{marginfigure}[1cm]
\centering
\begin{tikzpicture}

    
    % topo
    \draw (0,0) ellipse (1.25 and 0.5);
    
    % laterais
    \draw (-1.25,0) -- +(0,-2.5);
    \draw (1.25,0) -- +(0,-2.5);
    
    % fundo
    \draw (-1.25,-2.5) arc[start angle = -180, end angle = 0, x radius = 1.25, y radius = 0.5];
    \draw[dotted] (1.25,-2.5) arc[start angle = 0, end angle = 180, x radius = 1.25, y radius = 0.5];
    
    % Eixo z
    \draw[dashdotted, -Stealth] (0,0) -- +(0,1.5) node[below right]{$z$};
    \draw[fill] (0,0) circle (0.4pt);
    \draw[loosely dotted] (0,0) -- (0,-3);
    \draw[fill] (0,-2.5) circle (0.4pt);
    \draw[dashdotted] (0,-3) -- +(0,-1);
    
    \draw[fill] (0,-1.25) circle (1pt) node[below right]{CM};
    
    % Eixos x e y
    \draw[dashdotted, -Stealth] (0,0) -- (15:-2) node[below right]{$x$};
    \draw[dashdotted, -Stealth] (0,0) -- (-15:2) node [below left]{$y$};
    
    % Eixo P
    \draw[dashed, -Stealth] (-135: 0.3) -- +(0,1.5) node[below left]{$p$};
    \draw[fill] (-135:0.3) circle (0.4pt);
    \draw[loosely dotted] (-135:0.3) -- +(0,-3);
    \draw[fill] (0,-2.5) +(-135:0.3) circle (0.4pt);
    \draw[dashed] (0,-3)++(-135:0.3) -- +(0,-1);
    
    
\end{tikzpicture}
\caption{Rotação em torno de um eixo $p$ que não passa pelo centro de massa. \label{Fig:CilindoEixoParalelo}}
\end{marginfigure}

Na Figura~\ref{Fig:CilindoEixoParalelo}, temos dois eixos que passam por um objeto de massa $M$. O eixo $z$ passa pelo centro de massa do objeto, enquanto o eixo $p$ está deslocado em relação ao eixo $z$ por uma ditância $h$, porém é paralelo a ele. Estamos interessados em calcular o momento de inércia em torno do eixo que passa por $p$ e vamos considerar que o momento de inércia em torno do eixo que passa pelo centro de massa seja conhecido. 

Utilizando a Equação~\eqref{Eq:MomInerciaDistContinua}, podemos calcular o momento de inércia em torno dos eixos $z$ e $p$:
\begin{align}
    I_z &= \int r_\perp^2 \;dm \\
    I_p &= \int r_\perp^{\prime \, 2} \;dm. \label{Eq:TeorEixosParalelosDefIp}
\end{align}
%
Vamos escolher como origem do sistema de coordenadas a própria posição do centro de massa, o que resulta em uma visão como a da Figura~\ref{Fig:TeoremaEixosParalelosDM} ao olharmos na mesma direção dos eixos $z$ e $p$. 

\begin{marginfigure}[2cm]
\centering
\begin{tikzpicture}[>=Stealth, scale = 2]

    % Eixos x e y
    \draw[dashdotted, ->] (0,0) -- (1.75, 0) node[below left]{$x$};
    \draw[dashdotted, ->] (0,0) -- (0, 1.75) node[below left]{$y$};
    \draw (0,0)+(-15:1.25) arc[start angle = -15, end angle = 105, radius = 1.25];
        
    % raios
    \draw (0,0) -- (45:1.15);
    \draw (30:0.75) -- node[right]{$r'_\perp$}(45:1.15);
    \draw (0,0) -- (30:0.75);
    \node (r) at (55:0.6) {$r_\perp$};
    \node (h) at (15:0.4) {$h$};
    
    % dm
    \draw[fill = white, draw = black] (45:1.15) circle (0.02) node[above left]{$dm$};

    % Pontos p e z
    \draw[fill = white, draw = black] (30:0.75) circle (1pt);
    \draw[fill] (30:0.75) circle (0.4pt) node[below right]{$p$};
    \draw[fill = white, draw = black] (0,0) circle (1pt);
    \draw[fill] (0,0) circle (0.4pt) node[below left]{$z$};


\end{tikzpicture}
\caption{A contribuição de um elemento de massa para o momento de inércia depende de sua distância ao eixo. Para o eixo $z$, o elemento destacado está a uma distância $r_\perp$, já para o eixo $p$, a distância é $r'_\perp$. Note que os eixos $z$ e $p$ são perpendiculares ao plano da página.\label{Fig:TeoremaEixosParalelosDM}}
\end{marginfigure}

Tomando a contribuição para o momento de inércia de um ponto qualquer, representado por um círculo vazado na figura, verificamos que existe uma relação entre as distâncias entre o ponto e o eixo $z$ e entre o ponto e o eixo $p$: as projeções nos eixos $x$ e $y$ de  $r_\perp$ podem ser calculadas em termos das projeções de $r'_\perp$ e de $h$ através das equações:
\begin{align}
    r_{\perp, \,x} &= r'_{\perp, \, x} + h_x \\
    r_{\perp, \,y} &= r'_{\perp, \, y} + h_y.
\end{align}
%
Através do Teorema de Pitágoras, temos que $r_\perp^{\prime \,2} = r_{\perp, \,x}^{\prime \,2} + r_{\perp, \, y}^{\prime \, 2}$ podemos escrever a Equação~\eqref{Eq:TeorEixosParalelosDefIp}, utilizando as relações acima, como
\begin{align}
	I_p &= \int (r'_{\perp})^2 \;dm \\
	&= \int (r'_{\perp, \,x})^2 + (r'_{\perp, \, y})^2 \;dm \\
	&= \int (r_{\perp, \,x} - h_x)^2 + (r_{\perp, \, y} - h_y)^2 \;dm.
\end{align}
%
Desenvolvendo os quadrados e reagrupando os termos, podemos escrever
\begin{equation}
\begin{split}
	I_p = & \int (r_{\perp, \,x})^2 + (r_{\perp, \,y})^2 \;dm - 2h_x \int r_{\perp, \,x} \;dm \\
	&- 2h_y \int r_{\perp, \,y} \;dm + \int (h_x^2 + h_y^2) \;dm.
\end{split}
\end{equation}
%
A segunda e a terceira integrais acima são as expressões para o cálculo da posição do centro de massa nos eixos $x$ e $y$, respectivamente. Devido à escolha da posição da origem do sistema de coordenadas, temos que $r_x^{\textrm{CM}} = r_y^{\textrm{CM}} = 0$. Além disso, podemos ver da figura que $r_{\perp, \, x}^2 + r_{\perp, \, y}^2 = r_\perp^2$ e que $h_x^2 + h_y^2 = h^2$. Logo
\begin{equation}
	I_p = \int r_\perp^2 \;dm + \int h^2 \;dm.
\end{equation}
%
A primeira integral nada mais é do que o cálculo do momento de inércia em torno do eixo $z$, ou seja, em torno do eixo que passa pelo centro de massa. Já na segunda integral, $h$ é a distância entre os eixos $z$ e $p$, que é constante. Retirando-a da integral, obtemos a seguinte expressão, conhecida como \emph{Teorema do Eixos Paralelos}
\begin{equation}\label{Eq:TeoremaEixosParalelos}
	I_p = I_{\textrm{CM}} + h^2 M, \mathnote{Teorema dos eixos paralelos}
\end{equation}
%
onde usamos o fato de que
\begin{equation}
	\int dm = M.
\end{equation}

Concluímos, portanto, que conhecendo o momento de inércia de um objeto em torno de um eixo qualquer que passa pelo centro de massa, podemos calcular o momento de inércia em torno de qualquer eixo $p$ paralelo ao primeiro, bastando conhecer a distância $h$ entre os dois eixos e a massa do objeto. É importante destacar que o eixo $p$ não precisa necessariamente atravessar o corpo em algum ponto, podendo passar fora dele. Finalmente, é interessante notar que, dentro do conjunto infinito de eixos que apontam na mesma direção do espaço, o momento de inércia de um corpo tem seu valor mínimo para o eixo de rotação que passa pelo centro de massa.

%%%%%%%%%%%%%%%%%%%%%%%%%%%%%%%%%%%%%%%
\paragraph{Discussão: Limite $h \to \infty$}
%%%%%%%%%%%%%%%%%%%%%%%%%%%%%%%%%%%%%%%

\begin{marginfigure}
\centering
\begin{tikzpicture}[>=Stealth]

    \draw[Stealth-, dashdotted] (0,1.25) node[below right]{$p$} -- (0,-1.25); 
    
    \draw (0,-0.05) -- +(4,0);
    \draw (0,0.05) -- +(4,0);
    
    \draw[dashdotted, ->] (2, 0.05) -- +(0,1.2) node[below right]{$z_h$};
    \draw[dashdotted] (2, -0.05) -- +(0,-1.2);
    
    \draw (4.25, 0) circle (0.25);
    \draw[dashed] (4, 0) arc[start angle = -180, end angle = 0, x radius = 0.25, y radius = 0.1];
    \draw[dotted] (4, 0) arc[start angle = 180, end angle = 0, x radius = 0.25, y radius = 0.1];
    \draw[fill] (4.25, 0) circle (0.4pt);
    
    \draw[dashdotted,->] (4.25, 0.25) -- +(0,1) node[below right]{$z_e$};
    \draw[dashdotted] (4.25, -0.25) -- +(0,-1);
    
\end{tikzpicture}
\caption{Para calcular o momento de inércia do objeto composto mostrado acima em torno do eixo $p$ devemos calcular o momento de inércia da haste e da esfera, ambos em torno do eixo $p$. No entanto, devido ao fato de que a distância entre o eixo de rotação e o eixo que passa pelo centro de massa da esfera é muito maior que o raio da esfera, podemos desprezar o termo referente à rotação da esfera em torno do próprio eixo ($I_{\rm{CM}}$ no Teorema dos Eixos Paralelos).}
\end{marginfigure}

Muitas vezes o segundo termo na Equação~\eqref{Eq:TeoremaEixosParalelos} é muito maior que $I_{\textrm{CM}}$, devido a um grande valor de $h$. Nesses casos, podemos desprezar o primeiro termo, restando somente
\begin{equation}
  I_P = h^2 M,
\end{equation}
%
que corresponde ao caso de utilizarmos a Equação~\eqref{Eq:MomentoInerciaConjPart} para uma partícula girando em torno do eixo $p$, com toda a massa concentrada no centro de massa. Esta análise nos permite ---~por exemplo~--- tratar uma esfera de raio $r_e$ girando a uma distância $d$ em torno de um eixo como se fosse uma partícula, se $d \gg r_e$.

%%%%%%%%%%%%%%%%%%%%%%%%%%%%%%%%%%%%%%%%%%%%%%%%%%%%%%%%%%%%%%%%%%%%%%%%%%%%%%%%%%%%%%%%%
\paragraph{Exemplo: Momento de inércia de uma haste fina que gira em torno de um eixo que passa por sua extremidade}
%%%%%%%%%%%%%%%%%%%%%%%%%%%%%%%%%%%%%%%%%%%%%%%%%%%%%%%%%%%%%%%%%%%%%%%%%%%%%%%%%%%%%%%%%

Podemos determinar o momento de inércia de uma haste fina que gira em torno de um eixo $p$ que passa por sua extremidade, perpendicularmente ao eixo da haste. Sabemos que o momento de inércia em torno de um eixo que passa pelo centro de massa, perpendicularmente ao eixo da haste, é dado por
\begin{equation}
    I_{\rm{CM}} = \frac{1}{12} ML^2.
\end{equation}

\begin{marginfigure}
\centering
\begin{tikzpicture}[>=Stealth,
     interface/.style={
        % superfície
        postaction={draw,decorate,decoration={border,angle=-45,
                    amplitude=0.2cm,segment length=2mm}}},
    ]
    
\draw[fill] (0,0) circle (0.4pt);

\draw[dashdotted, ->] (0,0)+(0,0.1) -- +(0, 1.5) coordinate (top) node[below left]{$z$};
\draw[dashdotted] (0,0)+(0,-0.1) -- +(0,-1);

\draw (15:-1.25) circle (0.05);
\draw (0,0.05) coordinate (angmid) -- +(15:-1.25) coordinate (angstart);
\draw (0,-0.05) -- +(15:-1.25);

\draw (0,0.05)+(15:1.2) arc[start angle = 90, end angle = -90, radius = 0.05];
\draw (0,0.05) -- +(15:1.2);
\draw (0,-0.05) -- +(15:1.2);

\draw[fill] (15:-1.25) coordinate (icomp) circle (0.4pt);
\draw[dashdotted, ->] (15:-1.25) -- (15:-2.15) node[below right] {$x$};
\draw[dashdotted] (15:1.25) -- (15:2);

\pic [draw, "$\cdot$", angle eccentricity = 0.5, angle radius = 1.5mm]{angle = top--angmid--angstart};

\draw[->] (1.17,0.3)+(100:1) arc[start angle = -240, end angle = 60, x radius = 0.3, y radius = 0.15];

\draw (0,-0.3)+(15:-1.25) coordinate (A) -- node[below right]{$L$} +(15:1.2) coordinate (B);

\draw (A)++(0, 0.07) -- +(0,-0.14);
\draw (B)++(0, 0.07) -- +(0,-0.14);

% Eixo p
\draw[dashdotted, ->] (0,0.05)++(15:1.2) -- +(0,1.5) node[below right]{$p$};
\draw[dashdotted] (0,-0.05)++(15:1.2) -- +(0,-1); 

\end{tikzpicture}
\caption{Haste fina girando em torno da extremidade.\label{Fig:TeoremaEixosParaleleosExHaste}}
\end{marginfigure}

O eixo $p$ em torno do qual estamos interessados em calcular o momento de inércia é paralelo ao eixo que passa pelo centro de massa (eixo $z$, Figura~\ref{Fig:TeoremaEixosParaleleosExHaste}). Logo, podemos usar o Teorema dos Eixos Paralelos, considerando que a distância entre os dois eixos é dada por
\begin{equation}
    h = \frac{L}{2}.
\end{equation}
%
Assim:
\begin{align}
    I_p &= I_{\rm{CM}} + h^2 M \\
    &= \frac{1}{12} ML^2 + \left(\frac{L}{2}\right)^2 M \\
    &= \frac{1}{12} ML^2 + \frac{1}{4} ML^2 \\
    &= \left(\frac{1}{12} + \frac{1}{4}\right) ML^2,
\end{align}
%
o que resulta em 
\begin{equation}
    I_p = \frac{1}{3} ML^2.
\end{equation}
%
Note que esse é o mesmo resultado que havíamos obtido através de uma integração na distribuição de massa.

%%%%%%%%%%%%%%%%%%%%%%%%%%%%%%%%%%%%%%%%%%%%%%%%%%%%%%%%%%%%%%%%%%%
\paragraph{Exemplo: Momento de inércia de um conjunto de cilindros}
%%%%%%%%%%%%%%%%%%%%%%%%%%%%%%%%%%%%%%%%%%%%%%%%%%%%%%%%%%%%%%%%%%%

\begin{quote}
A Figura~\ref{Fig:Q:teste} mostra uma visão superior de um objeto formado por quatro cilindros e que pode girar em torno de um eixo perpendicular à página, passando pelo centro de um dos cilindros (indicado pelo ponto). Se todos os cilindros têm a mesma massa $m$ e o mesmo raio $r$, qual é o momento de inércia em torno do eixo indicado?
\end{quote}

\begin{marginfigure}[-1cm]
\centering
\begin{tikzpicture}[>=Stealth]
    \draw[pattern = north west lines, pattern color = gray] (0,0) circle (0.5);
    \draw[pattern = north west lines, pattern color = gray] (1,0) circle (0.5);
    \draw[pattern = north west lines, pattern color = gray] (0,1) circle (0.5);
    \draw[pattern = north west lines, pattern color = gray] (1,1) circle (0.5);
    \draw[fill] (0,0) circle (0.8pt);
    \draw[->] (-90:1) arc[start angle = -90, end angle = -180, radius = 1] node[midway, below left]{$\omega$};
\end{tikzpicture}
\caption{Visão lateral de um objeto formado por quatro cilindros e que pode girar em torno do eixo de um deles.\label{Fig:Q:teste}}
\end{marginfigure}

Sabemos que o momento de inércia é aditivo, por isso podemos simplesmente determinar o momento de inércia de cada cilindro girando em torno do eixo indicado individualmente e então somar os resultados obtidos. Para o cilindro inferior esquerdo, basta utilizarmos o resultado tabelado para um cilindro que gira em torno do próprio eixo:
\begin{equation}
    I_1 = \frac{1}{2} mr^2.
\end{equation}
%
Para os cilindros superior esquerdo e inferior direito, devemos utilizar o teorema dos eixos paralelos, uma vez que ambos giram em torno de um eixo que dista $2r$ do eixo que passa por seus centros de massa. Assim
\begin{align}
    I_2 &= I_3 \\
    &= I_{\text{CM}} + h^2 M \\
    &= \frac{1}{2} mr^2 + (2r)^2 m \\
    &= \frac{9}{2} mr^2.
\end{align}
%
Finalmente, também precisamos utilizar o teorema dos eixos paralelos para determinar o momento de inércia do cilindro superior direito. Para determinar a distância entre o eixo de rotação e o eixo paralelo que passa pelo centro de massa de tal cilindro, devemos utilizar o Teorema de Pitágoras:
\begin{align}
    h^2 &= a^2 + b^2 \\
    &= (2r)^2 + (2r)^2 \\
    &= 8r^2.
\end{align}
%
Assim,
\begin{align}
    I_4 &= I_{\text{CM}} + h^2M \\
    &= \frac{1}{2} mr^2 + 8r^2 m \\
    &= \frac{17}{2} mr^2.
\end{align}
%
Somando as contribuições dos quatro cilindros, obtemos
\begin{align}
    I &= I_1 + I_2 + I_3 + I_4 \\
    &= \frac{1}{2} mr^2 + \frac{9}{2} mr^2 + \frac{9}{2} mr^2 + \frac{17}{2} mr^2 \\
    &= \left[\frac{1}{2} + \frac{9}{2} + \frac{9}{2} + \frac{17}{2}\right]mr^2 \\
    &= 18 mr^2.
\end{align}

%%%%%%%%%%%%%%%%%%%%%%%%%%%%%%%%%%%%%%%%%%%%%%
\subsection{Teorema dos eixos perpendiculares}
%%%%%%%%%%%%%%%%%%%%%%%%%%%%%%%%%%%%%%%%%%%%%%

% https://en.wikipedia.org/wiki/Perpendicular_axis_theorem

Para um objeto que se aproxima de uma figura plana, como a placa da Figura~\ref{Fig:TeoremaDosEixosPerpendicularesPlaca}, temos um resultado bastante interessante: os momentos de inércia em torno dos eixos $x$, $y$, e $z$ seguem a relação
\begin{equation}
    I_z = I_x + I_y.
\end{equation}

\begin{marginfigure}[2cm]
\centering
\begin{tikzpicture}[>=Stealth]

    % Placa
    \draw (1,0,1) -- (1,0,-1) -- (-1,0,-1) -- (-1,0,1) -- cycle;
    \draw (1,-0.2,-1) -- (1,-0.2,1) -- (-1,-0.2,1);
    \draw (1,0,1) -- (1,-0.2,1);
    \draw (-1,0,1) -- (-1,-0.2,1);
    \draw (1,0,-1) -- (1,-0.2,-1);
    
    % eixos
    \draw[dashdotted, ->] (0,0,0) -- +(0,1.5,0) node[below right]{$z$};
    \draw[dashdotted] (0,-0.6,0) -- +(0,-1,0);
    \draw[fill] (0,0,0) circle (0.4pt);
    
    \draw[dashdotted, ->] (0,-0.1,1) -- +(0,0,1.5) node[below right]{$x$};
    \draw[fill] (0,-0.1,1) circle (0.4pt);
    \draw[dashdotted] (0,-0.1,-1.25) -- +(0,0,-1);
    
    \draw[dashdotted, ->] (1,-0.1,0) -- +(1.5,0,0) node [below left]{$y$};
    \draw[fill] (1,-0.1,0) circle (0.4pt);
    \draw[dashdotted] (-1.1,-0.1,0) -- +(-1,0,0);
    
\end{tikzpicture}
\caption{Para uma placa fina, os momentos de inércia em relação aos eixos mostrados na figura não são independentes: temos que $I_z = I_x + I_y$. \label{Fig:TeoremaDosEixosPerpendicularesPlaca}}
\end{marginfigure}

Para entender a origem dessa expressão, basta analisarmos a Figura~\ref{Fig:TeoremaDosEixosPerpendicularesAnalise}. No cálculo do momento de inércia em torno do eixo $x$, temos que a distância entre uma partícula $P$ qualquer e o eixo é dada por
\begin{equation}
    r_\perp = y_P.
\end{equation}
%
Assim, utilizando a Equação~\eqref{Eq:MomentoInerciaConjPart} para o momento de inércia de um conjunto de partículas, temos
\begin{equation}
    I_x = \sum_{i = 1}^{N_P} m_i y_i^2,
\end{equation}
%
onde $N_P$ representa o número de partículas da placa.

\begin{marginfigure}[2cm]
\centering
\begin{tikzpicture}[>=Stealth, scale = 1.2]

    % Placa
    \draw (-1,-1) rectangle (1,1);
    
    % eixos
    \draw[->, dashdotted] (-1.5,0) -- (2,0) node[below left]{$x$};
    \draw[->, dashdotted] (0,-1.5) -- (0,2) node[below left]{$y$};
    
    % distâncias
    \draw[dotted] (0.75,0.5) -- (0.75,1.3);
    \draw[dotted] (0.75,0.5) -- (1.3,0.5);
    \draw[<->|] (0,1.3) -- node[above]{$x_P$} (0.75,1.3);
    \draw[<->|] (1.3,0) -- node[right]{$y_P$} (1.3,0.5);
    
    %dm
    \draw (0,0) -- (0.75,0.5);
    \node (rp) at (15:0.5) {$r_P$};
    \draw [fill, lightgray, draw = black] (0.75, 0.5) circle (1.4pt);
    \path (0.75, 0.5) circle (1.4pt) node[above left]{$P$};
    \draw[fill, white, draw = black] (0,0) circle (1.4pt) node[below left]{$z$};
    \draw[fill] (0,0) circle (0.3pt);
    
\end{tikzpicture}
\caption{Os momentos de inércia em torno dos eixos $x$ e $y$ dependem exclusivamente das distâncias das partículas ao eixo de rotação. No entanto, como a dimensão em $z$ é desprezível, podemos as denotar por $y_P$ e $x_P$, respectivamente. \label{Fig:TeoremaDosEixosPerpendicularesAnalise}}
\end{marginfigure}

Já para o momento de inércia em torno do eixo $y$, temos que
\begin{equation}
    r_\perp = x_P.
\end{equation}
%
Logo,
\begin{equation}
    I_y = \sum_{i = 1}^{N_P} m_i x_i^2.
\end{equation}

Se somarmos as expressões para os momentos de inércia em torno desses eixos, obtemos
\begin{align}
    I_x + I_y &= \sum_{i = 1}^{N_P} m_i y_i^2 + \sum_{i = 1}^{N_P} m_i x_i^2 \\
    &= \sum_{i = 1}^{N_P} \left[ m_i y_i^2 + m_i x_i^2\right] \\
    &= \sum_{i = 1}^{N_P} \left[ m_i (y_i^2 + x_i^2)\right].
\end{align}
%
Analisando a Figura~\ref{Fig:TeoremaDosEixosPerpendicularesAnalise}, verificamos que
\begin{equation}
    r_P^2 = x_P^2 + y_P^2.
\end{equation}
%
Portanto,
\begin{equation}
    I_x + I_y = \sum_{i = 1}^{N_P} m_i r_i^2.
\end{equation}

Se calcularmos o momento de inércia em torno do eixo $z$, verificamos que a distância entre o ponto $P$ e esse eixo é dada por
\begin{equation}
    r_\perp = r_P,
\end{equation}
%
o que implica em um momento de inércia dado por
\begin{equation}
    I_z = \sum_{i = 1}^{N_P} m_i r_i^2.
\end{equation}
%
Concluímos então que
\begin{equation}
    I_z = I_x + I_y.
\end{equation}

%%%%%%%%%%%%%%%%%%%%%%%%%%%%%
%\paragraph{Tensor de inércia}
%%%%%%%%%%%%%%%%%%%%%%%%%%%%%

%\textbf{Falar do tensor de inércia?}

%%%%%%%%%%%%%%%%%%%%%%%%%%%%%%%%%%%%%%%%%%%%%%%%%%
\section{Trabalho e energia cinética para rotações}
%%%%%%%%%%%%%%%%%%%%%%%%%%%%%%%%%%%%%%%%%%%%%%%%%%

Assim como no caso das translações, é interessante utilizarmos os conceitos de trabalho, energia cinética e energia mecânica no caso das rotações. Nas próximas seções vamos determinar expressões adequadas para o cálculo dessas grandezas em um contexto de rotações. Após isso, verificaremos alguns exemplos de sistemas cuja análise é mais simples de um ponto de vista energético, do que de um ponto de vista dinâmico.

%%%%%%%%%%%%%%%%%%%%%%%%%%%%%%%%%%%%%%%%%%%%%%
\subsection{Trabalho em rotações}
%%%%%%%%%%%%%%%%%%%%%%%%%%%%%%%%%%%%%%%%%%%%%%

Como primeiro passo para que possamos utilizar conceitos de trabalho e energia cinética, precisamos determinar uma expressão para o trabalho em termos das variáveis que descrevem a rotação. Através da Segunda Lei de Newton para a rotação, sabemos que um corpo sujeito a um torque resultante externo estará sujeito a uma aceleração. Devido a essa aceleração, ele sofrerá uma variação de sua velocidade angular, que está ligada a uma variação da energia cinética do corpo.

\begin{marginfigure}[6cm]
\centering
\begin{tikzpicture}[>=Stealth, scale = 1.2,
     interface/.style={
        % superfície
        postaction={draw,decorate,decoration={border,angle=-45,
                    amplitude=0.2cm,segment length=2mm}}},
    ]

%%% Figura superior

\draw (0,0) ellipse (1.25 and 0.5);

\draw (-1.25,0) -- (-1.25,-0.4);
\draw (1.25,-0.4) -- (1.25,0);  

\draw (-1.25,-0.4) arc (180:360:1.25 and 0.5);
\draw[densely dotted] (-1.25,-0.4) arc (180:360:1.25 and -0.5);

\draw[dashdotted,->] (0,0) -- (0,1.5) node[below left]{$z$};
\draw[dashdotted] (0,-1.5) -- (0,-0.9);
\draw[dotted] (0,-0.9) -- (0,0);
\draw[fill] (0,0) circle (0.4pt);

\draw[dashdotted] (15:-2) -- (0,0);

% raio
\draw[|-|] (-0.05,0.15) -- node[above]{$r_\perp$} +(15:-0.8);
\draw[fill] (15:-0.8) circle (0.4pt);

% força
\draw[thick, ->] (15:-0.8) -- +(-65:1) node[below]{$\vec{F}$};

% projeção radial
\draw[dashed] (15:-0.8) ++(-65:1) -- (15:-1.6);
\draw[fill] (15:-0.8) circle (1pt);
\draw[->] (15:-0.8) -- (15:-1.6) node[above]{$F_r$};

% projeção tangencial
\draw[dashed] (15:-0.8) +(-30:1.6) -- +(-30:-0.3);
\draw[dashed] (15:-0.8) ++(-65:1) -- +(18:0.8);
\draw[->] (15:-0.8) -- +(-30:1.35) node[above]{$F_t$};

% phi
\draw (15:-1.2) arc (150:117:1.25 and -0.5);
\node (theta) at (30:-1.1) {$\phi$};

\end{tikzpicture}
\caption{Se uma força atua sobre uma partícula de um corpo, ela realiza um trabalho, cedendo energia ao sistema de partículas. Devido às demais interações da partícula com o resto do corpo rígido, todo o corpo ganha energia. Em termos das variáveis da rotação, podemos determinar o trabalho como um produto do torque pelo deslocamento angular. \label{Fig:DeducaoTeoremaTrabEnergiaRotacoes}}
\end{marginfigure}

Para determinar uma expressão para o trabalho efetuado por uma força em uma rotação, vamos considerar a Figura~\ref{Fig:DeducaoTeoremaTrabEnergiaRotacoes}, que é similar àquela utilizada para obter a Segunda Lei de Newton para as Rotações. Nela, consideramos que uma força $\vec{F}$ de módulo constante e que faz um ângulo $\phi$ também constante em relação ao raio que liga o eixo de rotação ao ponto de aplicação da força, e que atua sobre um ponto $P$ do disco.

Podemos calcular o trabalho realizado no deslocamento de tal ponto se considerarmos a componente $F_t$: essa componente da força é sempre na direção do vetor deslocamento instantâneo. Assim,
\begin{equation}
    W = F_t s,
\end{equation}
%
onde $s$ é o comprimento do arco descrito pelo ponto $P$. Devido ao fato de que o comprimento do arco está ligado ao deslocamento angular em radianos através de
\begin{equation}
    \theta = \frac{s}{r_\perp},
\end{equation}
%
onde $r_\perp$ é o raio da trajetória circular, podemos reescrever a expressão para o trabalho como
\begin{equation}
    W = F_t r_\perp \Delta\theta,
\end{equation}
%
onde utilizamos $\Delta \theta$ para evidenciar que estamos tratando do deslocamento angular sofrido sob influência da força $\vec{F}$, e não de uma posição angular específica\footnote{É claro que se escolhermos $\theta_i = 0$, temos que $\theta_f = \theta$.}.

Da própria definição de torque, temos que
\begin{equation}
    \tau = F_t r_\perp,
\end{equation}
%
portanto,
\begin{equation}\label{Eq:TrabalhoTorqueConstante}
    W = \tau \Delta\theta, \mathnote{Trabalho realizado por um torque constante}
\end{equation}
%
Note que o trabalho pode ser positivo ou negativo, sendo que tal sinal é determinado tanto pelo sinal do torque, quanto pelo sinal do deslocamento angular.

%%%%%%%%%%%%%%%%%%%%%%%%%%%%%%%%%%%%%%%%%%
\paragraph{Trabalho de um torque variável}
%%%%%%%%%%%%%%%%%%%%%%%%%%%%%%%%%%%%%%%%%%

Caso a força $\vec{F}$ não seja constante, podemos considerar o trabalho realizado em um deslocamento infinitesimal $ds$, considerando que $\vec{F}$ se mantenha constante\footnote{Como o deslocamento é infinitesimal, podemos considerar que as variações de intensidade ou direção são desprezíveis.}, o que nos leva a um trabalho infinitesimal $dW$ dado por
\begin{align}
    dW &= F_t \, ds \\
    &= F_t \, r_\perp d\theta \\
    &= \tau \,d\theta,
\end{align}
%
onde usamos o fato de que um deslocamento infinitesimal $ds$ corresponde a um deslocamento angular infinitesimal
\begin{equation}
    \frac{ds}{r_\perp} = d\theta.
\end{equation}
%
Finalmente, para obtermos o trabalho total, basta integrarmos entre os valores inicial e final das variáveis:
\begin{equation}\label{Eq:TrabalhoTorqueVariavel}
    W = \int_{\theta_i}^{\theta_f} \tau \;d\theta. \mathnote{Trabalho realizado por um torque variável}
\end{equation}
%
Na expressão acima, a determinação do sinal do deslocamento está ligada aos limites de integração: se $\theta_f > \theta_i$, então temos um deslocamento angular positivo; caso contrário, o deslocamento angular é negativo. Note que nesse caso não estamos nos referindo ao módulo da variável: se, por exemplo, $\theta_f = -\np[rad]{2,0}$ e $\theta_i = -\np[rad]{10,0}$, então $\theta_f > \theta_i$.

Finalmente, é importante destacar que tanto a expressão obtida para o trabalho realizado por um torque constante, quanto a obtida para o trabalho realizado por um torque variável são análogas àquelas que obtivemos para o caso da translação.

%%%%%%%%%%%%%%%%%%%%%%%%%%%%%%%%%%%%%%%%%%%%%%%
\subsection{Teorema trabalho--energia-cinética}
%%%%%%%%%%%%%%%%%%%%%%%%%%%%%%%%%%%%%%%%%%%%%%%

Definida uma experssão para o trabalho, precisamos demostrar no contexto de rotações a validade da expressão
\begin{equation}
    \Delta K = W.
\end{equation}
%
Para isso, vamos analisar um conjunto de $N$ partículas que formam um corpo rígido e que podem girar em torno de um eixo. Cada uma das partículas que compõe o corpo está sujeita a um número $n_f$ de forças, sejam elas devidas a interações com outras partículas ou a agentes externos ao sistema de partículas. De um ponto de vista translacional, se analisarmos a variação da energia cinética da $i$-ésima partícula, obtemos a seguinte expressão
\begin{align}
    \Delta K^i &= \sum_{j=1}^{n_f} W_j \\
    K_f^i - K_i^i &= \sum_{j=1}^{n_f} W_j.
\end{align}
%
Para cada partícula teremos uma equação como a acima, totalizando $N$ equações. Somando todas elas, obtemos
\begin{align}
    \sum_{i = 1}^N K_f^i - \sum_{i = 1}^N K_i^i &= \sum_{i = 1}^N\sum_{j=1}^{n_f} W_j \\
    \sum_{i = 1}^N \frac{1}{2}m_i (v_f^i)^2 - \sum_{i = 1}^N \frac{1}{2}m_i(v_i^i)^2 &= \sum_{i = 1}^N\sum_{j=1}^{n_f} W_j.
\end{align}
%
O fato de que as partículas compõe um corpo rígido que efetua uma rotação em torno de um eixo nos permite relacionar a velocidade de cada uma delas à velocidade angular do corpo através de
\begin{equation*}
v = r_\perp \omega,
\end{equation*}
%
o que nos permite escrever:\footnote{Note que as constantes podem ser postas em evidência, ou seja, para fora do somatório.}
\begin{align}
    \sum_{i = 1}^N \frac{1}{2}m_i (r_\perp^i)^2 \omega_f^2 - \sum_{i = 1}^N \frac{1}{2}m_i (r_\perp^i)^2\omega_i^2 &= \sum_{i = 1}^N\sum_{j=1}^{n_f} W_j \\
    \frac{1}{2}\left[\sum_{i = 1}^N m_i (r_\perp^i)^2\right]\omega_f^2 - \frac{1}{2}\left[\sum_{i = 1}^N m_i (r_\perp^i)^2\right]\omega_i^2 &= \sum_{i = 1}^N\sum_{j=1}^{n_f} W_j. \label{Eq:TrabEnergiaRotacoesQuase}
\end{align}

Note que os termos à esquerda da igualdade são iguais às energias cinéticas inicial e final do do conjunto de partículas que compõe o corpo rígido
\begin{align}
    K_i &= \frac{1}{2}\left[\sum_{i = 1}^N m_i (r_\perp^i)^2\right]\omega_i^2 \\
    K_f &= \frac{1}{2}\left[\sum_{i = 1}^N m_i (r_\perp^i)^2\right]\omega_f^2,
\end{align}
%
e que o termo entre colchetes é igual em ambos os casos e é o próprio momento de inércia do corpo. Assim, podemos denotar a energia cinética em rotações como
\begin{equation}
    K = \frac{1}{2} I\omega^2. \mathnote{Energia cinética em rotações.}
\end{equation}
    
Finalmente, resta analisarmos a soma dos trabalhos no termo à direita da igualdade na Equação~\eqref{Eq:TrabEnergiaRotacoesQuase}. Podemos dividir as forças que atuam sobre as partículas em \emph{forças internas} e em \emph{forças externas}, o que nos permite reescrever a soma dos trabalhos como\footnote{A primeira soma à direita é sobre os trabalhos devidos a forças internas, enquanto a segunda soma à direita é sobre os trabalhos devidos a forças externas.}
\begin{equation}
    \sum_{i = 1}^N\sum_{j=1}^{n_f} W_j = \sum_{i = 1}^{n_f^{\text{int}}} W_i + \sum_{j = 1}^{n_f^{\text{ext}}} W_j
\end{equation}
%
 Sabemos que a maior parte das forças que atuam sobre as partículas que compõe um corpo rígido são devidas a interações das próprias partículas umas com as outras, ou seja, são forças internas. Sabemos que os torques devidos aos pares ação-reação são iguais, porém têm sentidos opostos. Logo, os trabalhos devidos a forças internas devem ter sinais opostos e se cancelam, uma vez que sempre formam pares ação-reação contidos no corpo rígido.\footnote[][1cm]{Verificamos a questão dos torques na Seção~\ref{Sec:SegundaLeiDeNewtonParaRotacoes}. Além disso, estamos tratando de um corpo rígido, as partículas estão sujeitas ao mesmo deslocamento angular \emph{por definição}. Consequentemente, o trabalho total de pares ação-reação é nulo.} \emph{O trabalho total devido a forças internas é, portanto, nulo} 
 \begin{equation}
    \sum_{i = 1}^{n_f^{\text{int}}} W_i = 0,
 \end{equation}
 %
 e então
 \begin{equation}
    \sum_{i = 1}^N\sum_{j=1}^{n_f} W_j = \sum_{j = 1}^{n_f^{\text{ext}}} W_j
 \end{equation}
 
 Assim, restam somente os trabalhos devidos a forças externas que atuam sobre o corpo. Tais trabalhos externos devem ser calculados da maneira que for mais apropriada, seja através das expressões para o trabalho de uma força em uma translação, seja através das expressões para o trabalho de um torque. Em todo caso, podemos denotar o trabalho total por\footnote{Note que no índice se refere a externo ao corpo rígido, não ao sistema, seja ele qual for.}
\begin{equation}
    W_{\text{Res}}^{\text{Ext}} = \sum_{i = 1}^N\sum_{j=1}^{n_f} W_j,
\end{equation}
%
o que nos permite escrever
\begin{equation}
    \frac{1}{2}I\omega_f^2 - \frac{1}{2}I\omega_i^2 = W_{\text{Res}}^{\text{Ext}},
\end{equation}
%
ou ainda
\begin{equation}
    \Delta K = W_{\text{Res}}^{\text{Ext}}.
\end{equation}




%%%%%%%%%%%%%%%%%%%%%%%%%%%%%%%%%%%%%%%%%%%%%%%%%%%%%%%%%%%%%%%%%%%%%%%%
\paragraph{Exemplo: Velocidade angular de uma haste após sofrer um deslocamento sujeito à força peso}
%%%%%%%%%%%%%%%%%%%%%%%%%%%%%%%%%%%%%%%%%%%%%%%%%%%%%%%%%%%%%%%%%%%%%%%%

\begin{quote}
A Figura~\ref{Fig:TeoremaTrabalhoEnergiaRotacaoHaste} mostra uma haste disposta horizontalmente e que pode girar em torno de um eixo que passa por sua extremidade, perpendicularmente ao próprio eixo da haste. Ao liberarmos a haste para girar, ela sofre uma rotação devida ao trabalho realizado pela sua própria força peso. Determine a velocidade angular final da haste sabendo que ela partiu do repouso e que tem um comprimento $L = \np[cm]{100}$.
\end{quote}

\begin{marginfigure}
\centering
\begin{tikzpicture}[>=Stealth]

    \draw[pattern = north west lines] (0,-0.1) rectangle (3, 0.1);
    \draw[fill] (1.5,0) circle (0.6pt);
    \draw[dotted,rotate = -45] (0,-0.1) rectangle (3, 0.1);
    \draw[fill, rotate = -45] (1.5,0) circle (0.6pt);
    \draw[pattern = north east lines, rotate = -90, pattern color = gray, draw = gray] (0,-0.1) rectangle (3, 0.1);
    \draw[fill, rotate = -90] (1.5,0) circle (0.6pt);
    
    \draw[loosely dotted] (1.5,0) arc[start angle = 0, end angle = -90, radius = 1.5];
    \draw[->] (3.3, 0) arc[start angle = 0, end angle = -15, radius = 3.3];
    
    \draw[fill] (0,0) circle (0.8pt);
    
\end{tikzpicture}
\caption{Variação da velocidade angular de uma haste devido ao trabalho da força peso. \label{Fig:TeoremaTrabalhoEnergiaRotacaoHaste}}
\end{marginfigure}

Para determinarmos a velocidade final, basta utilizarmos o teorema Trabalho--Energia-cinética:
\begin{align}
    \Delta K & = W \\
    \frac{1}{2} I \omega_f^2 - \frac{1}{2} I \omega_i^2 &= W.
\end{align}
%
Assumindo um eixo $y$ apontando verticalmente para baixo, podemos determinar o trabalho realizado pelo peso através de 
\begin{align}
    W_g &= mg\Delta y \\
    &= mg\frac{L}{2},
\end{align}
%
o que resulta em
\begin{align}
     \frac{1}{2} I \omega_f^2 - \frac{1}{2} I \omega_i^2 &= mg\frac{L}{2} \\
      \frac{1}{2} I \omega_f^2 &= mg\frac{L}{2} \\
      \omega_f &= \sqrt{\frac{mgL}{I}} \\
      &= \sqrt{\frac{mgL}{mL^2/3}} \\
      &= \sqrt{\frac{3g}{L}}.
\end{align}
%
Substituindo o valor de $L$, obtemos
\begin{align}
    \omega_f &= \sqrt{\frac{3g}{L}} \\
    &= \np[rad/s]{5,42}.
\end{align}

Alternativamente, podemos calcular o trabalho realizado pelo torque devido à força peso:
\begin{align}
    W &= \int \tau \,d\theta \\
    &= \int_{\degree{0}}^{\degree{90}} F r \sen\phi \,d\theta.
\end{align}
%
Substituindo $F \equiv P = mg$, $r \equiv L/2$ e $\phi = \degree{90} - \theta$, obtemos
\begin{align}
    W &= \int_{\degree{0}}^{\degree{90}} mg\frac{L}{2} \sin(\degree{90} - \theta)\,d\theta \\
    &= mg\frac{L}{2} \int_{\degree{0}}^{\degree{90}} \cos \theta \,d\theta \\
    &= mg\frac{L}{2} [\sen \theta + C]_{\degree{0}}^{\degree{90}} \\
    &= mg\frac{L}{2}.
\end{align}

O resultado acima reproduz aquele obtido através da expressão para o trabalho na translação. É claro que ao calcularmos o trabalho através do torque temos uma complicação que é, nesse caso, completamente desnecessária. Em outras situações, no entanto, o cálculo do trabalho através do torque pode ser o caminho mais simples.

%\textbf{Mais um exemplo, se possível}

%%%%%%%%%%%%%%%%%%%%
\subsection{Potência}
%%%%%%%%%%%%%%%%%%%%

Da mesma forma que definimos a potência para o caso da translação, podemos defini-la para a rotação como
\begin{equation}
    \mean{P} = \frac{W}{\Delta t},
\end{equation}
%
no caso da potência média, ou
\begin{equation}
    P = \frac{dW}{dt}
\end{equation}
%
no caso da potência instantânea.

Assim como no caso da translação, é interessante notarmos que para um torque constante, temos
\begin{align}
    P &= \frac{d}{dt} \tau \theta \\
    &= \tau \frac{d\theta}{dt} \\
    &= \tau \omega,
\end{align}
%
de onde verificamos que ao aplicarmos um torque constante, o trabalho é tanto maior quanto maior for a velocidade angular. Esse resultado impôe um limite no torque efetuado por um dispositivo como um motor elétrico, cuja potência seja constante, ou impôe um valor de potência para um valor de velocidade angular específico, se o torque for constante. Essa equação mostra que o torque e a potência de um motor a combustão interna, por exemplo, não são independentes: se considerarmos que o torque seja aproximadamente constante para toda a faixa de rotação que o motor é capaz de trabalhar, verificamos que a potência não é constante, mas aumenta conforme a rotação aumenta.\footnote{Os valores de potência divulgados por fabricantes de veículos são sempre os valores máximos.}

%%%%%%%%%%%%%%%%%%%%%%%%%%%%%%%%%%%%%%%%%%%%%
%\paragraph{Exemplo: Desenrolando um carretel}
%%%%%%%%%%%%%%%%%%%%%%%%%%%%%%%%%%%%%%%%%%%%%

% \textbf{podemos verificar o trabalho e a potência efetuadas ao puxar um fio de um carretel, desenrolando-o.}

%%%%%%%%%%%%%%%%%%%%%%%%%%%%%%%%%%%%%%%%%%%%%%%%%%%%%%%%%%
\section{Movimentos combinados de rotação e de translação}
\label{Sec:MovCombRotTrans}
%%%%%%%%%%%%%%%%%%%%%%%%%%%%%%%%%%%%%%%%%%%%%%%%%%%%%%%%%%

%\textbf{Nas seções abaixo, a ideia é a seguinte: em casos como o da queda da haste discutida no exemplo acima, existe o complicador de que podemos tratar como uma rotação pura em torno da extremidade ou podemos tratar o movimento de rotação em torno do CM concomitante com uma translação do CM. O jeito mais certo é o segundo, o mais fácil é o primeiro. O problema é que tratar do jeito certo exige considerar tanto a translação quanto a rotação. Como eu só queria mostrar que existe uma expressão simples pra energia cinética, deixei do jeito mais simples acima, mas agora vamos colocar o resto no movimento combinado de translação e de rotação. O script é: mostrar que a energia cinética do conjunto de partículas pode ser dividida em uma parte de translação e outra de rotação (deveríamos fazer tanto para a energia quanto para a dinâmica, mas como, exatamente? Como o teorema de mozzi-chasles entra na nisso?),  mostrar a equivalência da rotação em torno do CM + mov. do CM e da rotação em torno do eixo que passa pela extremidade, resolver o problema da barra que gira em torno da extremidade sujeita ao próprio peso usando energia mecânica, mostrar o conceito de energia mecânica considerando a rotação tb, resolver a máquina de atwood. Note ainda que a lista de exercícios precisa ser modificada pra refletir essa organização. Os exercícios de dinâmica na maioria dos casos envolve translação de blocos ou coisa do gênero.}

Já verificamos como descrever a translação de um corpo através do centro de massa, e também a sua  rotação em torno de um eixo que passa pelo centro de massa, sendo que a direção do eixo é fixa. Notamos ainda que separar o movimento nessas componentes \emph{translacional} e \emph{rotacional} facilita muito a sua descrição: se, por exemplo, atiramos um bastão em um lançamento oblíquo, sendo que o bastão tem uma velocidade angular em torno do centro de massa, sabemos que o movimento de qualquer ponto pode ser descrito como uma rotação em torno do centro de massa e por um deslocamento do centro de massa.

A descrição de movimento desse tipo, no entanto, depende de conhecermos diversas informações: módulo e direção da velocidade do centro de massa, módulo da velocidade angular, direção do eixo de rotação, aceleração gravitacional, forças de arrasto, dimensões e massa do objeto, e potencialmente outras quantidades. Isso se deve ao fato de que os movimentos de rotação e de translação são \emph{independentes}.

Por outro lado, o fato de que são movimentos de um mesmo corpo implica em interações com outros corpos cujos comportamentos dependem tanto da rotação, quanto da translação do corpo. Um exemplo bastante claro disso é o jogo de tênis: quando o jogador bate na bola com a raquete, é comum que ele o faça de forma a imprimir uma rotação na bola; tal rotação não afeta\footnote{Na verdade, pode afetar sim: o fato de que a bola gira enquanto se desloca faz com que surja uma força aerodinâmica que provoca um deslocamento lateral da bola. Esse efeito é o que possibilita um gol de escanteio no futebol.} a trajetória da bola entre a raquete e o ponto de contato com o solo no lado oposto da quadra, porém faz com que surja uma força de atrito que causa um desvio da trajetória da bola quando ela toca o chão. O intuito do tenista com esse desvio é dificultar a reação do adversário.

Nas seções seguintes vamos verificar algumas propriedades gerais dos movimentos combinados de rotação e de translação de um sistema com um ou mais corpos. Depois, verificaremos um movimento em especial, o de \emph{rolamento}, que possui um vínculo simples entre as variáveis de rotação e de translação, permitindo que exploremos mais as suas propriedades e obtenhamos alguns resultados interessantes.

%%%%%%%%%%%%%%%%%%%%%%%%%%%%%%%%%%%%%%%%%%%%%%%%%%%%%%%%%%%%%%%%%%%%%%%%%%
\subsection{Dinâmica de um movimento combinado de rotação e de translação}
%%%%%%%%%%%%%%%%%%%%%%%%%%%%%%%%%%%%%%%%%%%%%%%%%%%%%%%%%%%%%%%%%%%%%%%%%%

%\textbf{Na real tem a questão de que a direção do eixo de rotação pode mudar tb, então seria algo mais complicado do que o que está posto aí em baixo. Melhor não se comprometer. só usar o papo de que temos que aplicar a segunda lei da rotação ou da translação pra cada corpo, ou nos dois aspectos, se ambos forem relevantes.}
Sempre que um corpo estiver sujeito a rotações e/ou a translações, devemos
\begin{itemize}
    \item determinar as forças que atuam sobre ele e aplicar as Leis de Newton para a translação, obtendo assim a aceleração do centro de massa $\vec{a}_{\text{CM}}(t)$;
    \item determinar os torques que atuam sobre ele e aplicar as Lei de Newton para a rotação, obtendo a aceleração angular $\alpha(t)$;
    \item determinar informações sobre a posição $\vec{r}_{\text{CM}}(t_0)$ e a velocidade $\vec{v}_{\text{CM}}(t_0)$ iniciais do centro de massa para um tempo $t_0$ qualquer, o que nos permite determinar completamente a função $\vec{r}_{\text{CM}}(t)$;
    \item determinar informações sobre a posição angular $\theta(t_0)$ e a velocidade angular $\omega(t_0)$ iniciais, o que nos permite determinar completamente a função $\theta(t)$.
\end{itemize}
%
Ao obtermos essas $\vec{r}_{\text{CM}}(t)$ e $\theta(t)$, o movimento fica completamente determinado. Caso exista mais que um corpo, devemos fazer o mesmo procedimento para cada um deles. Devemos destacar que uma mesma força pode ser relevante tanto para a translação do corpo, quanto para sua rotação.

É claro que em muitos casos ou o aspecto de rotação, ou o aspecto de translação do movimento não é relevante, o que permite que simplifiquemos bastante o tratamento do problema em questão: um exemplo disso é a Máquina de Atwood que discutiremos abaixo, dessa vez tratando a polia de maneira correta. Nela, os blocos só exibiam translação e a polia somente rotação.

Em outros casos, existe um vínculo entre as variáveis de rotação e de translação, eliminando várias incógnitas do tratamento matemático oriundo das Leis de Newton. Alguns exemplos desse tipo de movimento são o movimento de um ioiô, a Roda de Maxwell (que do ponto de vista físico é similar a um ioiô), e o rolamento. Vamos tratar a Roda de Maxwell abaixo, após discutirmos a Máquina de Atwood, e também o movimento de rolamento, que optamos por deixar para mais tarde para que possamos fazer uma análise um pouco mais detalhada.

\pagebreak
%%%%%%%%%%%%%%%%%%%%%%%%%%%%%%%%%%%%%%%%
\paragraph{Discussão: Máquina de Atwood}
%\label{Par:AcelMaqAtwood}
%%%%%%%%%%%%%%%%%%%%%%%%%%%%%%%%%%%%%%%%

Anteriormente discutimos sistemas envolvendo roldanas determinando ou a aceleração dos blocos suspensos através da Segunda Lei de Newton para a translação, ou suas velocidades após percorrerem uma distância $d$ através da conservação da energia mecânica. Em ambos os casos assumimos que a massa da roldana era desprezível. Agora vamos passar a levar em conta o fato de que isso não é verdade.\footnote[][-2cm]{A análise em termo de energia e a determinação da velocidade será feita na Seção~\ref{Sec:EnergiaMovCombRotTrans}.}

\begin{marginfigure}
\centering
\begin{tikzpicture}[>=Stealth,  interface/.style={
        % superfície
        postaction={draw,decorate,decoration={border,angle=-45,
                    amplitude=0.2cm,segment length=2mm}}}
    ]
       
    \draw[interface] (2,-0.3) -- (-2,-0.3);
    
    \draw[pattern = north west lines] (0,-1) circle (0.5);
    \draw[fill] (0,-1) circle (1pt);
    \path[fill=white] (0,-1) -- (0.2,-0.3) -- (-0.2,-0.3) -- cycle;
    \draw[pattern = crosshatch] (0,-1) -- (0.2,-0.3) -- (-0.2,-0.3) -- cycle;
    
    \draw[pattern = north west lines] (-0.8,-4) rectangle (-0.2, -4.6);
    \draw (-0.5,-1) -- (-0.5,-4);
    \draw[fill] (-0.5, -4.3) circle (1pt);
    \draw[thick, ->] (-0.5,-4.3) -- +(0,-0.6) node[right]{$\vec{P}_1$};
    \draw[thick, ->] (-0.5,-4) -- +(0,0.75) node[right]{$\vec{T}_1$};
    
    \draw[pattern = north west lines] (0.2,-2) rectangle (0.8,-2.6);
    \draw (0.5,-1) -- (0.5,-2);
    \draw[fill] (0.5, -2.3) circle (1pt);
    \draw[thick, ->] (0.5,-2.3) -- +(0,-1) node[right]{$\vec{P}_2$};
    \draw[thick, ->] (0.5,-2) -- +(0,0.75) node[right]{$\vec{T}_2$};
    
    \draw[->] (-1,-5) -- (-1, -2) node[left]{$y$};
    \draw (-0.95, -4.3) -- (-1.05,-4.3) node[left]{$y=0$};
\end{tikzpicture}
\caption{Uma solução para o sistema mostrado acima deve levar em conta o fato de que é necessário um torque resultante seja exercido sobre a polia, causando uma aceleração. Verificaremos que nesse caso as tensões nos segmentos suspensos esquerdo e direito da corda não podem ser iguais. \label{Fig:MaquinaDeAtwoodComPolia}}
\end{marginfigure}

Verificamos na Seção~\ref{Sec:PoliaCilindrica} que para que haja uma aceleração angular da polia, é necessário que haja um torque resultante externo que atue sobre ela, e que tal torque se deve à diferença entre as tensões nas duas extremidades da corda. Obtivemos então a relação
\begin{equation}
    \alpha = (T_1 - T_2) \frac{2}{MR}
\end{equation}
%
para a aceleração angular $\alpha$ da polia. Note que esse resultado depende do momento de inércia, e por isso dependa da forma da polia. A equação acima é válida para uma polia cilíndrida, ou em formato de disco.

Resta considerarmos agora a translação dos blocos. Adotaremos ambos os eixos $y$ ---~um para cada bloco~--- como sendo verticais, com sentido positivo para cima. Assim, as acelerações dos blocos são $a_1^y = a$ e $a_2^y = -a$, onde $a$ é a aceleração da corda.
\begin{description}
    \item[Bloco 1:] Aplicando a Segunda Lei de Newton para cada eixo:
        \begin{description}
            \item[Eixo $x_1$:] Não há nenhuma força nesse eixo.
            \item[Eixo $y_1$:]
                \begin{align}
                    F_R^{y_1} &= m_1 a_1^{y_1} \\
                    T_1 - P_1 &= m_1 a_1^{y_1} \\
                    T_1 - P_1 &= m_1 a.
                \end{align}
        \end{description}
    \item[Bloco 2:] Novamente, aplicando a Segunda Lei de Newton para cada eixo:
        \begin{description}
            \item[Eixo $x_2$:] Não há nenhuma força nesse eixo.
            \item[Eixo $y_2$:]
                \begin{align}
                    F_R^{y_2} &= m_2 a_2^{y_2} \\
                    T_2 - P_2 &= m_2 a_2^{y_2} \\
                    T_2 - P_2 &= m_2 (-a) \\
                    - T_2 + P_2 &= m_2 a.
                \end{align}
        \end{description}
\end{description}

Antes que possamos resolver as equações e determinar a aceleração do sistema, precisamos de uma relação entre a aceleração angular da polia e a aceleração de translação dos blocos. Verificamos, também na Seção~\ref{Sec:PoliaCilindrica}, que a aceleração da corda é a mesma que a de um ponto na borda da polia, se não houver deslizamento, o que implica em
\begin{equation}
    a = \alpha R,
\end{equation}
%
onde $R$ representa o raio da polia. Note, no entanto, que devido à nossa escolha do sistema de coordenadas, uma aceleração para cima do bloco 1 corresponde a um valor positivo de $a$. Isso corresponde a uma rotação da polia no sentido horário, o que implica em uma aceleração angular negativa, por isso devemos colocar um sinal negativo na relação acima, obtendo
\begin{equation}
    a = -\alpha R.
\end{equation}

Finalmente, podemos escrever o sistema
\begin{equation}
\begin{system}
    T_1 - P_1 &= m_1 a \\
    - T_2 + P_2 &= m_2 a \\
    T1 - T_2 &= \frac{MR}{2} \alpha \\
    a &= -\alpha R,
\end{system}
\end{equation}
%
cuja solução para a aceleração é
\begin{equation}
    a = \frac{m_2 - m_1}{m_1+m_2+\nicefrac{M}{2}} g.
\end{equation}
%
Note que se $M \to 0$ o resultado acima se reduz ao caso estudado anteriormente, quando ignorávamos a massa da polia.

%%%%%%%%%%%%%%%%%%%%%%%%%%%
\paragraph{Roda de Maxwell}
%%%%%%%%%%%%%%%%%%%%%%%%%%%

% Pra essa figura ficar boa, precisaria fazer
% os traços sobre uma foto, assim a perspectiva
% ficaria correta
\begin{marginfigure}[-1.5cm]
\centering
\begin{tikzpicture}[>=Stealth, rotate = -95,
     interface/.style={
        % superfície
        postaction={draw,decorate,decoration={border,angle=-45,
                    amplitude=0.2cm,segment length=2mm}}},
    ]

%%% Teto

    \draw[interface] (-2.6,2.25) -- (-2.45,-2.5);
    
%%% Disco

\draw (0,0) ellipse (1.25 and 0.75);

\draw (-1.25,0) -- (-1.25,-0.4);
\draw (1.25,-0.4) -- (1.25,0);  

\draw (-1.25,-0.4) arc (180:360:1.25 and 0.75);
\draw[dotted] (-1.25,-0.4) arc (180:360:1.25 and -0.75);

%%% Semieixo direito
\fill[white] (0.125,0) rectangle (-0.125,2);
\draw (0.125,0) -- +(0,2);
\draw (-0.125,0) -- +(0,2);
\draw (0,2) ellipse (0.125 and 0.075);
\draw (-0.125,0) arc[start angle = 180, end angle = 360, x radius = 0.125, y radius = 0.075];
% Corda
\draw (-0.125,1.75) arc[start angle = 180, end angle = 360, x radius = 0.125, y radius = 0.075];
\draw (-0.125,1.70) arc[start angle = 180, end angle = 360, x radius = 0.125, y radius = 0.075];
\draw (-0.125,1.65) arc[start angle = 180, end angle = 360, x radius = 0.125, y radius = 0.075];
\draw (-0.125,1.60) arc[start angle = 180, end angle = 360, x radius = 0.125, y radius = 0.075];
\draw (-0.125,1.55) arc[start angle = 180, end angle = 360, x radius = 0.125, y radius = 0.075];
\draw (-0.125,1.50) arc[start angle = 180, end angle = 360, x radius = 0.125, y radius = 0.075];
\draw (-0.125,1.45) arc[start angle = 180, end angle = 360, x radius = 0.125, y radius = 0.075];
\draw (-0.125,1.40) arc[start angle = 180, end angle = 360, x radius = 0.125, y radius = 0.075];
\draw (-0.125,1.8) -- (-2.58,1.55);


%%% Semieixo esquerdo
\draw[dotted] (0,-0.4) ellipse (0.125 and 0.075);
\draw[dotted] (0.125, -0.4) -- (0.125,-1.15);
\draw[dotted] (-0.125, -0.4) -- (-0.125,-1.15);
\draw (0.125, -1.15) -- (0.125,-2.13);
\draw (-0.125, -1.15) -- (-0.125,-2.13);
\draw (-0.125,-2.13) arc[start angle = 180, end angle = 360, x radius = 0.125, y radius = 0.075];
% Corda
\draw (-0.125,-1.85) arc[start angle = 180, end angle = 360, x radius = 0.125, y radius = 0.075];
\draw (-0.125,-1.80) arc[start angle = 180, end angle = 360, x radius = 0.125, y radius = 0.075];
\draw (-0.125,-1.75) arc[start angle = 180, end angle = 360, x radius = 0.125, y radius = 0.075];
\draw (-0.125,-1.70) arc[start angle = 180, end angle = 360, x radius = 0.125, y radius = 0.075];
\draw (-0.125,-1.65) arc[start angle = 180, end angle = 360, x radius = 0.125, y radius = 0.075];
\draw (-0.125,-1.60) arc[start angle = 180, end angle = 360, x radius = 0.125, y radius = 0.075];
\draw (-0.125,-1.55) arc[start angle = 180, end angle = 360, x radius = 0.125, y radius = 0.075];
\draw (-0.125,-1.50) arc[start angle = 180, end angle = 360, x radius = 0.125, y radius = 0.075];
\draw (-0.125,-1.9) -- (-2.45,-2.1);

\end{tikzpicture}
\caption{Roda de Maxwell.\label{Fig:RodaDeMaxwell}}
\end{marginfigure}

\begin{marginfigure}[1.5cm]
\centering
\begin{tikzpicture}[>=Stealth,
     interface/.style={
        % superfície
        postaction={draw,decorate,decoration={border,angle=-45,
                    amplitude=0.2cm,segment length=2mm}}},
    ]

    \draw (0,0) circle (2cm);
    \draw (0,0) circle (0.25cm);
    
    \draw (-0.25,0) -- (-0.25,3);
    \draw[interface] (1,3) -- (-1,3);
    
    \draw[dotted] (0,0.25) -- (1,0.25);
    \draw[dotted] (0,-0.25) -- (1,-0.25);
    \draw[|<->|] (1,0.25) -- node[right]{$2r$} (1,-0.25);
    
    \draw[dotted] (0,2) -- (2.5,2);
    \draw[dotted] (0,-2) -- (2.5,-2);
    \draw[|<->|] (2.5,2) -- node[right]{$2R$} (2.5,-2);

\end{tikzpicture}
\caption{Roda de Maxwell, visão lateral.\label{Fig:RodaDeMaxwellVisaoLateral}}
\end{marginfigure}

Um dos sistemas mais simples que exibe rotação e translação é o que denominamos como ``Roda de Maxwell'', e que consiste em um disco ligado a um eixo suspenso por cordas. As cordas são enroladas no eixo e o sistema é liberado a partir do repouso, deixando o disco descer e desenrolar a corda, de maneira similar a um ioiô. Apesar de o disco exibir uma rotação e uma translação, esses movimentos não são independentes, uma vez que a distância percorrida pelo centro de massa está ligada ao comprimento de corda que é desenrolado do eixo. A relação entre o deslocamento do centro de massa e o deslocamento angular, que determina a quantidade $\ell$ de corda desenrolada, é
\begin{equation}
    \Delta y_{\text{CM}} \equiv \ell = r \Delta \theta.
\end{equation}
%
A partir dessa relação, também temos
\begin{align}
    v_{\text{CM}} &= r \omega \\
    a_{\text{CM}} &= r \alpha.
\end{align}
%
Note que as relações acima podem ou não ter um sinal negativo, dependendo da escolha do sistema de referência. Se adotarmos um eixo vertical apontando para cima, as expressões acima estão adequadas. %\textbf{Se tirar a máquina de atwood lá do começo, vou precisar explicar a questão dos sinais aqui.}

Aplicando a Segunda Lei de Newton para translação ao centro de massa, obtemos
\begin{description}
\item[Eixo $y$:]
\begin{align}
    F_R^y &= m_t a_{\text{CM}, y} \\
    T - P &= m_t a,
\end{align}
\end{description}
%
onde adotamos $a \equiv a_{\text{CM}, y}$ e $m_t$ representa a massa total do sistema (soma da massa do disco e dos dois semieixos).

Aplicando a Segunda Lei de Newton para a rotação, temos
\begin{align}
    \tau &= I\alpha \\
    -Tr &= I\alpha.
\end{align}
%
O momento de inércia total é dado pela soma do momento de inércia do disco central e dos dois semieixos:
\begin{equation}
    I = \frac{1}{2} MR^2 + 2\frac{1}{2} m r^2.
\end{equation}

Através dos resultados obtidos acima, podemos montar um sistema de equações
\begin{equation}
\begin{system}
    T - P &= m_t a \\
    -Tr &= I\alpha \\
    a &= r \alpha,
\end{system}
\end{equation}
%
cuja solução é
\begin{align}
    a &= \frac{m_t g}{m_t + I / r^2} \label{Eq:AcelRodaDeMaxwellGenerica}\\
    &= \frac{m_tg}{m_t + M(R/r)^2/2 + m} \\
    &= \frac{M + 2m}{M + 2m + M(R/r)^2/2 + m} \cdot g \\
    &= \frac{M + 2m}{3m + M \cdot [1 + (R/r)^2/2]} \cdot g \\
    &= \frac{2 + M/m}{3 + (M/m) \cdot [1 + (R/r)^2/2]} \cdot g
\end{align}

\begin{marginfigure}[1.5cm]
\centering
\begin{tikzpicture}[>=Stealth, scale = 0.95,
     interface/.style={
        % superfície
        postaction={draw,decorate,decoration={border,angle=-45,
                    amplitude=0.2cm,segment length=2mm}}},
    ]

    \def\hL{0.5};
    \def\l{2};
    \def\R{2};
    \def\r{0.15};
    \def\d{0.3}
    \draw (-\hL,-\R) rectangle (\hL,\R);
    \draw (-\hL,\r) rectangle +(-\l,-\d);
    \draw (\hL,-\r) rectangle +(\l,\d);
    
    \draw[|<->|] (-\hL, 2.2) -- node[above]{$L$} (\hL, 2.2);
    \draw[|<->] (-\hL, 0.35) +(-\l,0) -- node[above]{$\ell$} (-\hL,0.35);
    \draw[|<->] (\hL,0.35) +(\l,0) -- node[above]{$\ell$} (\hL,0.35);
    
\end{tikzpicture}
\caption{Roda de Maxwell, visão frontal.\label{Fig:RodaDeMaxwellVisaoFrontal}}
\end{marginfigure}

Se assumirmos que os eixos laterais e o disco central são feitos do mesmo material, podemos escrever as massas como
\begin{align}
    M &= \rho V_d \\
    &= \rho \pi R^2 L \\
    m &= \rho V_e \\
    &= \rho \pi r^2 \ell,
\end{align}
%
onde $L$ e $\ell$ representam a espessura do disco e o comprimento dos semieixos, respectivamente, e $\rho$ representa a densidade do material. Calculando a razão $M/m$, obtemos
\begin{align}
    \frac{M}{m} &= \frac{\rho \pi R^2 L}{\rho \pi r^2 \ell} \\
    &= \left(\frac{R}{r}\right)^2\frac{L}{\ell}.
\end{align}
%
Se substituirmos os resultados acima na expressão para a aceleração, podemos escrever
\begin{equation}
    a = \frac{2 + \lambda^2\gamma}{3 + (1 + \lambda^2/2) \lambda^2 \gamma} \cdot g,
\end{equation}
%
onde definimos
\begin{align}
    \lambda &\equiv R/r \\
    \gamma &\equiv L/\ell.
\end{align}

Um caso particular da expressão obtida acima para a aceleração é aquele em que $\lambda = R/r \to 1$, ou seja, temos um simples cilindro. Nesse caso temos que
\begin{align}
    a &\to \frac{2 + \gamma}{3 + (1 + 1/2)\gamma} \cdot g \\
    &\to \frac{2 + \gamma}{3 + 3\gamma/2} \cdot g \\
    &\to \frac{2}{3} \cdot \frac{ 1 + \gamma/2}{1 + \gamma/2} \cdot g \\
    &\to \frac{2}{3} \cdot g.
\end{align}
%
Note que esse é o mesmo resultado que se obtém ao usar $I = MR^2 / 2$, com $M = m_t$ e $R = r$ na Equação~\eqref{Eq:AcelRodaDeMaxwellGenerica}.

%%%%%%%%%%%%%%%%%%%%%%%%%%%%%%%%%%%%%%%%%%%%%%%%%%%%%%%%%%%%%%%%%%%%%%%%%
\subsection{Energia em um movimento combinado de rotação e de translação}
\label{Sec:EnergiaMovCombRotTrans}
%%%%%%%%%%%%%%%%%%%%%%%%%%%%%%%%%%%%%%%%%%%%%%%%%%%%%%%%%%%%%%%%%%%%%%%%%

%\textbf{Simplificar aqui. Provar que a energia cinética pode ser dada pela soma da energia cinética de translação com a de rotação (está pronto, abaixo). Depois simplesmente argumentar que se $\Delta E = W_{NC}$ e o sistema pode ser sempre considerado como um conjunto de inúmeras partículas (os átomos), então precisamos só somar as energias cinéticas das partículas, o que é equivalente a somar a energia cinética de translação do cm com a energia cinética de rotação em torno do eixo de rotação que passa pelo cm, e as energias potenciais. Como a soma das energias potenciais gravitacionais é igual à energia potencial do centro de massa, basta usarmos $E = K_{t, CM} + K_{r} + U_{g,CM} + \sum U_i$.}

Anteriormente determinamos uma expressão para a energia cinética de um corpo rígido simplesmente considerando que suas partículas executavam uma rotação em torno de um eixo. No caso mais geral, no entanto, precisamos considerar a possibilidade de que o corpo como um todo sofra um deslocamento. Nesse caso, podemos escrever a velocidade de cada partícula como
\begin{equation}
    \vec{v}_i = \vec{v}_{\text{CM}} + \vec{v}^i_{\text{CM}},
\end{equation}
%
onde $\vec{v}_{\text{CM}}$ representa a velocidade do centro de massa do sistema de partículas e $\vec{v}^i_{\text{CM}}$ representa a velocidade da partícula em relação ao centro de massa. A energia cinética do corpo como um todo é dada pela soma das energias cinéticas das partículas individualmente
\begin{align}
    K &= \frac{1}{2}\sum_{i = 1}^{N} m_i v_i^2 \\
    &= \frac{1}{2}\sum_{i = 1}^{N} m_i \vec{v}_i \cdot \vec{v}_i,
\end{align}
%
onde usamos $v_i^2 = (\vec{v}_i)^2 = \vec{v}_i\cdot\vec{v}_i$. Substituindo a expressão para o vetor velocidade da partícula, obtemos
\begin{align}
    K &= \frac{1}{2}\sum_{i = 1}^N m_i (\vec{v}_{\text{CM}} + \vec{v}^i_{\text{CM}})\cdot (\vec{v}_{\text{CM}} + \vec{v}^i_{\text{CM}}) \\
    &= \frac{1}{2}\sum_{i = 1}^N m_i \vec{v}_{\text{CM}}^2 + \frac{1}{2}\sum_{i = 1}^N m_i (\vec{v}^i_{\text{CM}})^2 + \sum_{i=1}^N m_i \vec{v}^i_{\text{CM}} \cdot \vec{v}_{\text{CM}}.
\end{align}
%
O primeiro termo da expressão acima nada mais é do que a energia cinética de translação do centro de massa do corpo:
\begin{equation}
    K_{\text{trans}} = \frac{1}{2}\sum_{i = 1}^N m_i \vec{v}_{\text{CM}}^2.
\end{equation}
%
Já o segundo pode ser escrito como
\begin{align}
    \frac{1}{2}\sum_{i = 1}^N m_i (\vec{v}^i_{\text{CM}})^2 &= \frac{1}{2}\sum_{i = 1}^N m_i (\omega r_\perp^i)^2 \\
    &= \frac{1}{2} \left[\sum_{i=1}^N m_i r_\perp^i\right]\omega^2,
\end{align}
%
onde usamos $\vec{v}^i_{\text{CM}} = \omega r_\perp^i$. Identificamos esse termo como a energia cinética de rotação:
\begin{equation}
    K_{\text{rot}} = \frac{1}{2} I\omega^2.
\end{equation}

O terceiro termo pode ser reescrito como
\begin{align}
    \sum_{i=1}^N m_i \vec{v}^i_{\text{CM}} \cdot \vec{v}_{\text{CM}} &= \left[\sum_{i=1}^N m_i \vec{v}^i_{\text{CM}}\right] \cdot \vec{v}_{\text{CM}} \\
    &= \left[\sum_{i = 1}^N \vec{p}^i_{\text{CM}}\right]\cdot \vec{v}_{\text{CM}}.
\end{align}
%
Note, no entanto, que a soma dos momentos lineares das partículas no referencial do centro de massa é igual ao momento linear total do sistema nesse referencial, porém o momento linear total de um sistema no referecial de seu próprio centro de massa é zero.\footnote[][-3cm]{O momento linear total de um sistema é igual ao momento linear de seu centro de massa e pode ser escrito como $\vec{P}_{\text{sis}} = M\vec{v}_{\text{CM}}$, mas a velocidade do centro de massa em relação ao referencial fixado no próprio centro de massa é obviamente zero!} Concluímos então que para um corpo rígido qualquer, podemos determinar sua energia cinética como
\begin{equation}
    K = K_{\text{rot}} + K_{\text{trans}}. \mathnote{Energia cinética de um corpo que efetua um movimento combinado de rotação e de translação.}
\end{equation}

%\textbf{texto antigo, ver se tem algo proveitoso. Acho que pelo menos o exemplo é algo interessante, pra podermos dizer que é equivalente tratar o movimento da haste como uma rotação em torno do eixo, ou como uma rotação em torno do cm mais uma translação do cm. Acredito que a questão aqui é que é equivalente tratar como uma rotação em torno do eixo instantâneo de rotação e a rotação em torno do cm mais a translação do cm. Preciso só achar uma explicação e usar esse problema como gancho pra poder discutir isso. A única coisa que achei a respeito é que se temos um eixo de rotação, então $\vec{v} = \vec{\omega}\times\vec{r}$, onde $\vec{r}$ dá a posição em relação ao eixo de rotação (o que é estranho, deveria ser em relação à origem); fazendo o produto $\vec{\omega}\times\vec{v}$ é possível mostrar que $\vec{r}_{\text{inst}} = -\vec{r} = \vec{\omega}\times\vec{v} / \omega^2$, isto é, sabemos que uma rotação pura pode ser usada para descrever o movimento de um corpo com velocidade $\vec{v}$ e velocidade angular $\vec{\omega}$ se adotarmos o eixo de rotação determinado por essa expressão.} % ver esse link: https://physics.stackexchange.com/questions/291234/instantaneous-axis-of-rotation
\begin{marginfigure}[4cm]
\centering
\begin{tikzpicture}[>=Stealth]

    \draw[pattern = north west lines] (0,-0.1) rectangle (3, 0.1);
    \draw[fill] (1.5,0) circle (0.6pt);
    \draw[dotted,rotate = -45] (0,-0.1) rectangle (3, 0.1);
    \draw[fill, rotate = -45] (1.5,0) circle (0.6pt);
    \draw[pattern = north east lines, rotate = -90, pattern color = gray, draw = gray] (0,-0.1) rectangle (3, 0.1);
    \draw[fill, rotate = -90] (1.5,0) circle (0.6pt);
    
    \draw[loosely dotted] (1.5,0) arc[start angle = 0, end angle = -90, radius = 1.5];
    \draw[->] (3.3, 0) arc[start angle = 0, end angle = -15, radius = 3.3];
    
    \draw[fill] (0,0) circle (0.8pt);
    
\end{tikzpicture}
\caption{Em um sistema formado por uma haste que pode girar em torno de uma extremidade, sujeita à força peso, temos uma situação onde a energia potencial gravitacional é transformada em energia cinética de rotação. \label{Fig:EnergiaMecanicaNaRotacao}}
\end{marginfigure}

Verificaremos agora que podemos estender o conceito de energia mecânica a sistemas que envolvem rotação: basta considerarmos que energia cinética presente na energia mecânica corresponde à energia cinética de rotação.

Podemos utilizar o que sabemos de energia mecânica e o que verificamos sobre energia cinética na rotação para analisar um sistema mecânico que envolve uma rotação. Se tomarmos o sistema da Figura~\ref{Fig:EnergiaMecanicaNaRotacao} ---~em que uma haste pode girar em torno de um eixo que passa por sua extremidade~---, temos a atuação das forças peso, arrasto, atrito, e normal.

Vamos considerar que o arrasto e o atrito são desprezíveis, porém não podemos dizer o mesmo da força normal: tal força é exercida pelas paredes do eixo de rotação nas paredes da cavidade onde ele se encaixa\footnote{Essa cavidade é denominada \emph{mancal}.}, e é responsável pela aceleração centrípeta no movimento de rotação efetuado pela haste. No entanto, o ponto onde a força é aplicada não sofre nenhum deslocamento na direção da força normal, o que implica em um trabalho nulo realizado por essa força. Como o peso é uma força conservativa, podemos escrever
\begin{align}
    \Delta E &= W_{\rm{NC}} \\
    &= 0,
\end{align}
%
o que resulta em
\begin{equation}
    E_i = E_f.
\end{equation}
%
Sabemos que esse resultado se aplica ao sistema de partículas que compõe o corpo rígido. Verificamos no capítulo anterior que a energia potencial de um corpo rígido pode ser calculada através da posição do centro de massa, e na seção anterior que a energia cinética associada a um corpo rígido pode ser calculada através de 
\begin{equation}
    K = \frac{1}{2} I\omega^2.
\end{equation}
%
Portanto, para um sistema que está sujeito a uma simples rotação, se $W_{\rm{NC}} = 0$, podemos utilizar a conservação da energia mecânica: basta interpretarmos a energia cinética como uma energia cinética para a rotação.
%\textbf{Falar que se um corpo gira e translada, precisamos levar em conta ambos os termos. Dizer que vamos ver isso no caso do rolamento.}

%%%%%%%%%%%%%%%%%%%%%%%%%%%%%%%%%%%%%%%%%%%%%%%%%%%%%%%%%%%%%%%%%%%%%%%%%%%%%%%%%%%%%%%%%
\paragraph{Discussão: velocidade angular de uma haste após sofrer um deslocamento sujeita à força peso (solução utilizando conservação da energia mecânica)}
%%%%%%%%%%%%%%%%%%%%%%%%%%%%%%%%%%%%%%%%%%%%%%%%%%%%%%%%%%%%%%%%%%%%%%%%%%%%%%%%%%%%%%%%%

%\textbf{Resolver usando a rotação em torno do centro de massa mais a translação do centro de massa}

\begin{marginfigure}
\centering
\begin{tikzpicture}[>=Stealth]

    \draw[pattern = north west lines] (0,-0.1) rectangle (3, 0.1);
    \draw[fill] (1.5,0) circle (0.6pt);
    \draw[dotted,rotate = -45] (0,-0.1) rectangle (3, 0.1);
    \draw[fill, rotate = -45] (1.5,0) circle (0.6pt);
    \draw[pattern = north east lines, rotate = -90, pattern color = gray, draw = gray] (0,-0.1) rectangle (3, 0.1);
    \draw[fill, rotate = -90] (1.5,0) circle (0.6pt);
    
    \draw[loosely dotted] (1.5,0) arc[start angle = 0, end angle = -90, radius = 1.5];
    \draw[->] (3.3, 0) arc[start angle = 0, end angle = -15, radius = 3.3];
    
    \draw[fill] (0,0) circle (0.8pt);
    
    \draw[->] (-0.5, -3) -- (-0.5,0.5) node[below left]{$y$};
    \draw[-|] (-0.5,-3) -- (-0.5, -1.5) node[left]{0};
    
\end{tikzpicture}
\caption{Sistema de referência para a determinação da energia mecânica.}
\end{marginfigure}

Utilizando a conservação da energia mecânica, podemos determinar a velocidade da haste na Figura~\ref{Fig:EnergiaMecanicaNaRotacao} após ela sofrer um deslocamento de \np[\tcdegree]{90,0}, partindo do repouso:
\begin{align}
    E_i &= E_f \\
    K_i + U_g^i &= K_f + U_g^f \\
    mg\frac{L}{2} &= \frac{1}{2} I \omega_f^2,
\end{align}
%
onde usamos o fato de que a velocidade angular inicial é zero, e escolhemos a posição do centro de massa da haste no final da rotação como sendo o zero do eixo vertical. Além disso, assumimos que o comprimento da haste é $L$, o que implica em uma posição inicial do centro de massa dada por $L/2$ . Substituindo a expressão para o momento de inércia, obtemos
\begin{equation}
    mg\frac{L}{2} = \frac{1}{2} \frac{mL^2}{3} \omega_f^2,
\end{equation}
%
o que resulta em
\begin{equation}
    \omega_f = \sqrt{\frac{3g}{L}}.
\end{equation}

%%%%%%%%%%%%%%%%%%%%%%%%%%%%%%%%%%%%%%%%%%%%%%%%%%%%%%%%%%%%%%%%%%%%%%%%%%%%%%%%%%%%%%%%%%%%%%%%
\paragraph{Discussão: Equivalência das rotações em torno do centro de massa e em torno de um eixo paralelo}
%%%%%%%%%%%%%%%%%%%%%%%%%%%%%%%%%%%%%%%%%%%%%%%%%%%%%%%%%%%%%%%%%%%%%%%%%%%%%%%%%%%%%%%%%%%%%%%%
%\textbf{revisar}
É importante notar que no problema acima tratamos o movimento da haste como uma rotação pura em torno de um eixo que passa por sua extremidade, perpendicularmente ao seu eixo. Dessa forma, não consideramos a translação do centro de massa. Isso, no entanto, equivale a tratarmos o movimento como uma translação do centro de massa simultânea a uma rotação em torno do centro de massa. Tal equivalência pode ser entendida do posto de vista da energia cinética de rotação através de\footnote{Note que a velocidade angular em torno do eixo que passa pelo centro de massa é igual à velocidade angular em torno do eixo paralelo $p$.}
\begin{align}
    K_{\rm{rot}}^p &= \frac{I_p\omega^2}{2} \\
    &= \frac{(I_{\rm{CM}} + h^2M)\omega^2}{2} \\
    &= \frac{I_{\rm{CM}}\omega^2}{2} + \frac{Mh^2\omega^2}{2}.
\end{align}
%
O primeiro termo na expressão acima corresponde à energia cinética de rotação em torno do centro de massa. Já o segundo, se considerarmos que $v_{\rm{CM}} = h\omega$, nos dá a energia cinética associada à translação do centro de massa. Assim,
\begin{equation}
    K_{\rm{rot}}^p = K_{\rm{rot}}^{\rm{CM}} + K_{\rm{trans}}^{\rm{CM}}.
\end{equation}

%%%%%%%%%%%%%%%%%%%%%%%%%%%%%%%%%%%%%%%%
\paragraph{Discussão: Máquina de Atwood}
%%%%%%%%%%%%%%%%%%%%%%%%%%%%%%%%%%%%%%%%
%\textbf{revisar}
Podemos utilizar a expressão para a energia cinética de um corpo rígido juntamente com a energia mecânica para analisar um sistema que contém partes que estão sujeitas a movimentos de translação e de rotação, conjuntamente. No caso do sistema mostrado na Figura~\ref{Fig:MaquinaDeAtwoodComPoliaEnergia}, temos dois blocos de massas $m_1$ e $m_2$, presos a uma corda de massa desprezível, e que passa por uma polia. A polia tem momento de inércia $I$ e raio $r$. Ao liberarmos o sistema a partir do repouso, qual é a velocidade dos blocos após terem percorrido uma distância $d$?

\begin{marginfigure}[3cm]
\centering
\begin{tikzpicture}[>=Stealth,  interface/.style={
        % superfície
        postaction={draw,decorate,decoration={border,angle=-45,
                    amplitude=0.2cm,segment length=2mm}}}
    ]
       
    % polia
    \draw[interface] (2,1.2) -- (-2,1.2);
    
    \draw[pattern = north west lines] (0,0.5) circle (0.5);
    \draw[fill] (0,0.5) circle (1pt);
    \fill[white] (0,0.5) -- (0.2,1.2) -- (-0.2,1.2) -- cycle;
    \draw[pattern = crosshatch] (0,0.5) -- (0.2,1.2) -- (-0.2,1.2) -- cycle;
    
    % bloco esquerdo
    \draw[pattern = north west lines] (-0.8,-4) rectangle (-0.2, -4.6);
    \draw[dotted, pattern = north west lines] (-0.2,-2) rectangle (-0.8,-2.6);
    \draw (-0.5,0.5) -- (-0.5,-4);
    \draw[fill] (-0.5,0.5) circle (0.8pt);
    \draw[thick, ->] (-0.5,0.5) -- node[left]{$\vec{T}_1^s$} +(0,-0.75);
    \draw[fill] (-0.5, -4.3) circle (1pt);
    \draw[thick, ->] (-0.5,-4.3) -- +(0,-0.6) node[right]{$\vec{P}_1$};
    \draw[thick, ->] (-0.5,-4) -- node[right]{$\vec{T}_1^i$} +(0,0.75);
    
    % bloco direito
    \draw[pattern = north west lines] (0.2,-2) rectangle (0.8,-2.6);
    \draw[dotted, pattern = north west lines] (0.8,-4) rectangle (0.2, -4.6);
    \draw (0.5,0.5) -- (0.5,-2);
    \draw[thick, ->] (0.5,0.5) -- node[right]{$\vec{T}_2^s$}+(0, -1);
    \draw[fill] (0.5,0.5) circle (0.8pt);
    \draw[dotted] (0.5, -2.6) -- (0.5,-4);
    \draw[fill] (0.5, -2.3) circle (1pt);
    \draw[thick, ->] (0.5,-2.3) -- +(0,-1) node[right]{$\vec{P}_2$};
    \draw[thick, ->] (0.5,-2) -- node[right]{$\vec{T}_2^i$} +(0,1);
    
    % eixo y
    \draw[|-|] (1,-2.3) -- node[right]{$d$} (1,-4.3);
    \draw[->] (-1,-5) -- (-1, -2) node[left]{$y$};
    \draw (-0.95, -4.3) -- (-1.05,-4.3) node[left]{$y=0$};
\end{tikzpicture}
\caption{Se não pudermos desprezar a massa da polia, ao determinar a energia mecânica no sistema acima, devemos levar em conta a energia cinética associada à rotação da polia. Note também que os trabalhos das forças de tensão e de atrito se cancelam (na figura não mostramos as reações das tensões).\label{Fig:MaquinaDeAtwoodComPoliaEnergia}}
\end{marginfigure}

As forças que atuam nesse problema são as forças peso, tensões, e normal (no eixo da polia). Vamos desconsiderar o efeito de forças de arrasto e de atrito. A força normal não realiza trabalho pois o ponto onde a força é aplicada não se move (não sofre translação). 

No caso das tensões, sabemos que elas aparecem aos pares, atuando em corpo diferentes\footnote{Para uma corda real devemos dividi-la em segmentos que interagem uns com os outros através de forças de ação e reação. Imagine a interação entre os elos de uma corrente.}. Como as forças do par ação-reação têm mesmo módulo e direção, porém têm sentidos diferentes, sabemos que seus trabalhos se cancelam, uma vez que o deslocamento é igual para ambos os corpos.

Precisamos também considerar o trabalho feito pela força de atrito entre a corda e a polia. Novamente, podemos simplesmente lembrar que existe um par ação-reação onde uma das forças atua sobre a polia, enquanto a outra atua sobre a corda. Como ambas têm o mesmo módulo e direção, porém sentidos opostos, e o deslocamento de ambos os corpos é o mesmo, verificamos que o trabalho total devido ao atrito também é zero.

Finalmente, como concluímos que $W_{\rm{NC}} = 0$, sabemos que a energia mecânica se conserva. Assim,
\begin{align}
    E_i &= E_f \\
    K_{t,1}^i + K_{t,2}^i + K_r^i + U_{g,1}^i + U_{g,2}^i &= K_{t,1}^f + K_{t,2}^f + K_r^f + U_{g,1}^f + U_{g,2}^f.
\end{align}
%
Como o sistema parte do repouso, as energias cinéticas iniciais são nulas. A posição inicial do bloco 1 coincide com a origem do eixo vertical $y$, assim como a posição final do bloco 2. Assim, podemos simplificar a expressão acima, obtendo
\begin{equation}
    U_{g,2}^i = K_{t,1}^f + K_{t,2}^f + K_r^f + U_{g,1}^f.
\end{equation}
%
ou, substituindo as expressões para as energias cinéticas e potenciais,
\begin{equation}
    m_2 g d = \frac{m_1 v_{1, f}^2}{2} + \frac{m_2 v_{2, f}^2}{2} + \frac{I\omega_f^2}{2} + m_1 g d.
\end{equation}

As velocidades dos blocos são as mesmas, pois estão ligados por uma corda inextensível. Além disso, a velocidade da borda da polia é a mesma que a da corda, se não há deslizamento. Podemos então utilizar essas observações para reescrever a expressão acima como
\begin{align}
    (m_2 - m_1) gd &= \frac{(m_1 + m_2) v_f^2}{2} + \frac{I v_f^2}{2r^2} \\
    &= \frac{(m_1 + m_2 + I/r^2) v_f^2}{2},
\end{align}
%
o que resulta na seguinte expressão para a velocidade:
\begin{equation}
    v_f = \sqrt{\frac{2(m_2-m_1)gd}{m_1 + m_2 + I/r^2}}.
\end{equation}

Note que obtivemos uma velocidade menor do que se tivessemos desprezado a massa da polia. Isso se deve ao fato de que parte da energia potencial armazenada no sistema foi transformada em energia cinética associada à rotação da polia.


%%%%%%%%%%%%%%%%%%%%%%
\section{Rolamento}
%%%%%%%%%%%%%%%%%%%%%%

Assim como no caso da Roda de Maxwell, no rolamento há um vínculo entre o movimento de rotação e o de translação, caso o movimento ocorra \emph{sem deslizamento}. Nesse caso, o deslocamento do centro de massa está vinculado ao deslocamento angular. Vamos considerar aqui somente rolamentos que ocorrem sobre superfícies planas.\footnote[][-3cm]{Para superfícies sujeitas a curvaturas, o ponto de contato entre a superfície e o corpo mudará de posição, o que complica a análise do problema.}

Matematicamente, o vínculo entre o deslocamento angular e o deslocamento do centro de massa, pode ser descrito através de (veja a Figura~\ref{Fig:DeslRolamento})
\begin{equation}
    \Delta x_{\rm{CM}} = r \, \theta,
\end{equation}
%
\begin{marginfigure}
\centering
\begin{tikzpicture}[>=Stealth,  interface/.style={
        % superfície
        postaction={draw,decorate,decoration={border,angle=-45,
                    amplitude=0.2cm,segment length=2mm}}}
    ]
    
    % piso
    \draw[gray, interface] (-3,0) -- (1,0);
    
    % eixo x
    \draw[densely dotted, ->] (-3.0,0.5) -- (-2.46,0.5) (-1.46,0.5) -- (-0.5,0.5) (0.5,0.5) -- (1,0.5) node[below left]{$x$};
    
    % corpo inicial
    \draw[dashed, pattern = north west lines] (-1.96, 0.5) circle (0.5);
    \draw[densely dashed, ->] (-1.96, 0.5)+(135:0.6) arc[start angle = 135, end angle = 80, radius = 0.6] node[above, midway]{$\omega$};
    \draw[fill, gray] (-1.96,0.5) circle (0.8pt);
    
    % corpo final
    \draw[pattern = north west lines] (0, 0.5) circle (0.5);
    \draw[->] (0, 0.5)+(135:0.6) arc[start angle = 135, end angle = 80, radius = 0.6] node[above, midway]{$\omega$};
    \draw[fill] (0,0.5) coordinate(CM) circle (0.8pt);
    
    % arco percorrido
    \draw[thick] (0,0) coordinate (origin) arc[start angle = -90, end angle = -315, radius = 0.5];
    \draw[fill] (0,0) circle (0.8pt);
    \draw[fill] (0,0.5)+(45:0.5) coordinate (fim) circle (0.8pt);
    
    % \theta
    \draw (origin) -- (CM);
    \draw (CM) -- (fim);
    \pic[draw, "$\theta$", angle eccentricity = 2, angle radius = 1.1mm]{angle = fim--CM--origin};
    
    % deslocamento
    \draw[thick] (-1.96,0) -- (0,0);
    \draw[fill] (-1.96,0) circle (0.8pt);
    \draw[|<->|] (-1.96, -0.5) -- node[below]{$\Delta x_{\rm{CM}}$}(0,-0.5);

\end{tikzpicture}
\caption{Se um corpo rola sem deslizar, o deslocamento do centro de massa está ligado ao deslocamento angular através de $\Delta x_{\rm{CM}} = r \, \theta$. \label{Fig:DeslRolamento}}
\end{marginfigure}
%
\noindent{}onde o eixo $x$ é um eixo paralelo à superfície sobre a qual o corpo rola, $r$ é o raio do corpo, e os deslocamento translacional e rotacional são representados por $\Delta x_{\rm{CM}}$ e $\theta$, respectivamente. Essa relação se deve ao fato de que, uma vez que não há deslizamento entre a superfície do objeto que rola e a superfície de apoio, o arco percorrido $s$ deve ter o mesmo comprimento que o deslocamento $\Delta x$ (podemos pensar nas distâncias que cobrem os pontos dos dois corpos que tiveram contato entre si). Por outro lado, vemos que durante o rolamento, o centro de massa se desloca por uma distância $\Delta x$. Note que o deslocamento angular deve ser medido em radianos, uma vez que a equação acima é uma consequência da própria definição para um ângulo em radianos. 

Para determinarmos as relações entre as velocidades e acelerações, basta derivarmos a expressão acima em relação ao tempo, lembrando que $r$ é uma constante. Obtemos assim:
\begin{align}
    v_{\rm{CM}} &= \frac{d}{dt} \Delta x_{\rm{CM}} \\
    &= \frac{d}{dt} r \, \theta \\
    &= r \, \omega
\end{align}
%
e
\begin{align}
    a_{\rm{CM}} &= \frac{d}{dt} v_{\rm{CM}} \\
    &= \frac{d}{dt} r \, \omega \\
    &= r \, \alpha.
\end{align}
%
É importante destacar que essas relações podem ter um sinal que depende da escolha que adotamos para o sentido positivo do eixo $x$. Como convencionamos que rotações no sentido anti-horário são positivas, devemos analisar os sistemas que estamos tratando caso a caso: se uma rotação no sentido anti-horário implica em um deslocamento no sentido negativo de $x$, as relações para deslocamento, velocidade e posição apresentadas acima ganham um sinal negativo.

Devemos destacar ainda que no referencial do centro de massa do corpo que rola, a velocidade em módulo de um ponto na borda da roda é dada por
\begin{align}
    v_b &= r_\perp \, \omega \\
    &= r \, \omega,
\end{align}
%
e que tal resultado é exatamente igual ao módulo da velocidade do centro de massa no refencial do solo:
\begin{equation}
    v_b = v_{\rm{CM}}.
\end{equation}


\begin{marginfigure}
\centering
\begin{tikzpicture}[>=Stealth,  interface/.style={
        % superfície
        postaction={draw,decorate,decoration={border,angle=-45,
                    amplitude=0.2cm,segment length=2mm}}}
    ]
    
    %% Figura superior
    % piso
    \draw[gray, interface] (-3,0) -- (1,0);
    
    % ref cm
    \draw[dashed, ->] (0,0.5) -- +(1,0) node[below left]{$x$};
    \draw[dashed, ->] (0,0.5) -- +(0,1) node[below left]{$y$};
       
    % corpo final
    \draw[draw = gray, pattern color = gray, pattern = north west lines] (0, 0.5) circle (0.5);
    \draw[gray, ->] (0, 0.5)+(180:0.6) arc[start angle = 180, end angle = 135, radius = 0.6] node[left, midway]{$\omega$};
    \draw[fill] (0,0.5) coordinate(CM) circle (0.8pt);
    
    % velocidades
    \draw[fill] (0,0) circle (0.8pt);
    \draw[->] (0,0) -- +(-1,0) node[above right]{$\vec{v}_b$};
    
    \draw[fill] (0,-0.2) circle (0.8pt);
    \draw[->] (0,-0.2) -- +(-1,0) node[below right]{$\vec{v}_{s, \,\rm{CM}}$};
    
    %% Figura inferior
    % piso
    \draw[gray, interface] (-3,-2.5) -- (1,-2.5);
          
    % corpo final
    \draw[draw = gray, pattern color = gray, pattern = north west lines] (0, -2) circle (0.5);
    \draw[gray, ->] (0, -2)+(180:0.6) arc[start angle = 180, end angle = 135, radius = 0.6] node[left, midway]{$\omega$};
    \draw[fill] (0,-2) coordinate(CM) circle (0.8pt);
    
    % ref solo
    \draw[dashed, ->] (-2,-2.5) -- +(1,0) node[below left]{$x$};
    \draw[dashed, ->] (-2,-2.5) -- +(0,1) node[below left]{$y$};

    % velocidades
    \draw[->] (0,-2) -- +(1,0) node[below]{$\vec{v}_{\rm{CM}, \,s}$};

\end{tikzpicture}
\caption{Velocidades do solo e da borda no referencial do centro de massa, e do centro de massa no referencial do solo. Note em especial que $\vec{v}_{s,\,\rm{CM}} = -\vec{v}_{\rm{CM},\,s}$. \label{Fig:RolamentoRefereciais}}
\end{marginfigure}

Podemos compreender essa igualdade de maneira simples: se estivermos andando com uma bicicleta, estamos no referencial do centro de massa das rodas, uma vez que o quadro está preso aos eixos que passam por seus centros de massa. Quando olhamos para o solo, vemos que ele se desloca para trás no eixo horizontal com uma velocidade $\vec{v}_{s,\,\rm{CM}} = -\vec{v}_{s,\,\rm{CM}}$ em relação ao centro de massa da roda (Figura~\ref{Fig:RolamentoRefereciais} superior). Se não ocorre deslizamento, a velocidade da borda da roda no ponto de contato também é para trás, com a mesma velocidade. Logo,
\begin{equation}
    \vec{v}_b = \vec{v}_{s,\,\rm{CM}}.
\end{equation}
%
Já no referencial do solo (Figura~\ref{Fig:RolamentoRefereciais} inferior), temos claramente que o centro de massa de desloca com uma velocidade $\vec{v}_{\rm{CM},\,s}$ dada por
\begin{equation}\label{Eq:RelVelCMSoloESoloCM}
    \vec{v}_{\rm{CM},\,s} = - \vec{v}_{s,\,\rm{CM}},
\end{equation}
%
isto é, a velocidade do centro de massa no referencial do solo é igual em direção e em módulo à velocidade do solo no referencial do centro de massa, porém tem sentido contrário. No ponto inferior, em particular, temos uma igualdade entre o módulo da velocidade da borda, e o módulo da velocidade do centro de massa em relação ao solo:
\begin{equation}
    v_b = v_{\rm{CM},\,s}.
\end{equation}
%
Essa igualdade, no entanto, é válida para todos os pontos da borda, uma vez que estamos tratando de um corpo rígido.

Finalmente, devemos notar que, se a velocidade da borda da roda no ponto de contato é a mesma que a do solo no referencial do centro de massa, essa igualdade também é válida no referencial do solo. Como o referencial do solo é fixado no solo, temos que tal velocidade é \emph{nula}: para um rolamento que ocorre sem deslizar, a velocidade da borda da roda no ponto de contato com o solo é \emph{zero}.


%%%%%%%%%%%%%%%%%%%%%%%%%%%%%%%%%%%%%%%%%%%%%%%%%%%%%%%%%%%%%%%%%%%%%%%%%%%%
\subsection{Movimento das partículas de um corpo rígido durante o rolamento}
%%%%%%%%%%%%%%%%%%%%%%%%%%%%%%%%%%%%%%%%%%%%%%%%%%%%%%%%%%%%%%%%%%%%%%%%%%%%

O movimento das partículas de um corpo que rola só pode ser descrito de uma maneira simples no referencial do centro de massa, pois temos que as partículas executam círculos em torno do eixo de rotação. Uma descrição de tal movimento em relação a um sistema de coordenadas fixado no solo é mais complexa, mas pode ser descrita em termos de dois movimentos simples: um de rotação em torno do centro de massa, e uma translação do centro de massa. Usaremos tal separação para obter a trajetória e a velocidade de cada ponto em um movimento de rolamento.

%%%%%%%%%%%%%%%%%%%%%%%%%%%%%%%%%%%%%%%%%%%%%%%
\paragraph{Rotação em torno do centro de massa}
%%%%%%%%%%%%%%%%%%%%%%%%%%%%%%%%%%%%%%%%%%%%%%%

A determinação da posição de um ponto qualquer no referencial do centro de massa é bastante simples. Vamos considerar que o ponto $P$ na Figura~\ref{Fig:PosPRelCMRol}, cuja distância em relação ao eixo de rotação é $r_\perp$, parte da posição inferior e, durante o rolamento, se desloca por um ângulo $\theta$. Após tal deslocamento, ele se encontra em uma posição que pode ser descrita em termos dos eixos $x$ e $y$ como
%
\begin{marginfigure}[-2cm]
\centering
\begin{tikzpicture}[>=Stealth]
    
    % eixos de referência
    \draw[->, dashdotted] (0,0) coordinate (origin) -- +(1,0) node[below left]{$x$};
    \draw[->, dashdotted] (0,0) -- +(0,1) node[below left]{$y$};
    \draw[fill] (0,0) circle (0.8pt);
    
    % Trajetória
    \draw[dotted] (140:2) arc[start angle = 140, end angle = 300, radius = 2];
    \draw[<->, dashed] (0,0) -- node[above]{$r_\perp$} (160:2);
    
    % arco percorrido
    \draw (220:2) arc[start angle = 220, end angle = 270, radius = 2];
    \draw[fill] (270:2) coordinate (bottom) circle (0.8pt);
    \draw[fill] (220:2) coordinate (point) circle (0.8pt) node[left]{$P$};
    \draw[->] (0,0) -- node[above left]{$\vec{r}_{P, \,\rm{CM}}$}(220:2);
    
    % posições em x e y
    \draw[loosely dotted] (0,0) -- (0,-1.2856);
    \pic[draw, "$\theta$", angle eccentricity = 1.5] {angle = point--origin--bottom};
    \draw[<->|] (0.2,0) -- node[right]{$r_\perp \cos\theta$}(0.2,-1.2856);
    \draw[|<->|] (0, -2.2) -- node[below]{$r_\perp \sen\theta$} (-1.53, -2.2);
    \draw[loosely dotted] (-1.53,-2.2) -- (point);
    
\end{tikzpicture}
\caption{No referencial do centro de massa, a posição do ponto $P$ após sofrer um deslocamento $\theta$ a partir da posição inferior é dado por $\vec{r}_{P,\,\rm{CM}} = r_\perp \sen\theta \, \versi + r_\perp \cos\theta \, \versj$. \label{Fig:PosPRelCMRol}}
\end{marginfigure}
%
\begin{align}
    r_{P, \,\rm{CM}}^x &= -r_\perp \sen\theta \\
    r_{P, \,\rm{CM}}^y &= -r_\perp \cos\theta,
\end{align}
%
ou, em notação vetorial,
\begin{equation}
    \vec{r}_{P,\,\rm{CM}} = - r_\perp \sen\theta \, \versi - r_\perp \cos\theta \, \versj.
\end{equation}

%%%%%%%%%%%%%%%%%%%%%%%%%%%%%%%%%%%%%%%%
\paragraph{Movimento em relação ao solo}
%%%%%%%%%%%%%%%%%%%%%%%%%%%%%%%%%%%%%%%%

Vamos agora determinar a posição do ponto $P$ em relação ao solo. Podemos determiná-la através de uma transformação de posição, já que sabemos $\vec{r}_{P, \,\rm{CM}}$:
\begin{equation}
    \vec{r}_{P, \, s} = \vec{r}_{\rm{CM}, \, s} + \vec{r}_{P, \,\rm{CM}},
\end{equation}
%
porém precisamos determinar a posição $\vec{r}_{\rm{CM}, \, s}$ do centro de massa em relação ao solo.

\begin{marginfigure}[2cm]
\centering
\begin{tikzpicture}[>=Stealth,  interface/.style={
        % superfície
        postaction={draw,decorate,decoration={border,angle=-45,
                    amplitude=0.2cm,segment length=2mm}}}
    ]
    
    %% Figura inferior
    % piso
    \draw[gray, interface] (-3,-2.5) -- (1,-2.5);
          
    % corpo final
    \draw[draw = gray, pattern color = gray, pattern = north west lines] (0, -2) circle (0.5);
    \draw[gray, ->] (0, -2)+(180:0.6) arc[start angle = 180, end angle = 135, radius = 0.6] node[left, midway]{$\omega$};
    \draw[fill] (0,-2) coordinate(CM) circle (0.8pt);
    
    % ref solo
    \draw[dashed, ->] (-2,-2.5) -- +(1,0) node[below left]{$x$};
    \draw[dashed, ->] (-2,-2.5) -- +(0,1) node[below left]{$y$};
    \draw[fill] (-2,-2.5) circle (0.8pt);

    % velocidades
    \draw[->] (0,-2) -- +(1,0) node[below]{$\vec{v}_{\rm{CM}, \,s}$};

    % posição
    \draw[->, thick] (-2,-2.5) -- node[above]{$\vec{r}_{\rm{CM}, \, s}$} (0,-2);
    
    % deslocamento
    \draw[|<->|] (-2,-2.9) -- node[below]{$d$} (0,-2.9);
    
    % raio
    \draw[<->] (0,-2) -- node[right]{$R$} +(0,-0.5);
\end{tikzpicture}
\caption{No referencial do solo, a posição do centro de massa é descrita por um valor constante no eixo $y$, e pela própria distância percorrida, no eixo $x$. \label{Fig:PosCMDetCicloideRol}}
\end{marginfigure}

Na Figura~\ref{Fig:PosCMDetCicloideRol}, verificamos que a posição do centro de massa é dada pela distância horizontal percorrida pelo centro de massa no eixo $x$, e pelo próprio valor do raio do corpo que efetua o rolamento, no eixo $y$:
\begin{align}
    r_{\rm{CM}, \, s}^x &= d \\
    r_{\rm{CM}, \, s}^y &= R,
\end{align}
%
ou seja,
\begin{equation}
    \vec{r}_{\rm{CM}, \, s} = d \, \versi + R \, \versj.
\end{equation}
%
Assim,
\begin{align}
    \vec{r}_{P, \, s} & = \vec{r}_{\rm{CM}, \, s} + \vec{r}_{P, \,\rm{CM}} \\
    &= d \, \versi + R \, \versj - r_\perp \sen\theta \, \versi - r_\perp \cos\theta \, \versj \\
    &= (d - r_\perp \sen\theta)\,\versi + (R - r_\perp \cos\theta)\,\versj.
\end{align}

Finalmente, devemos lembrar que em um rolamento a distância percorrida pelo centro de massa e o deslocamento angular estão vinculados através de
\begin{equation}
    \Delta x = r\,\theta,
\end{equation}
%
o que resulta em
\begin{equation}
    d = R\,\theta.
\end{equation}
%
Assim, obtemos
\begin{equation}\label{Eq:TrajetoriaPontosQueRolam}
    \vec{r}_{P, \, s} = (R\,\theta - r_\perp \sen\theta)\,\versi + (R - r_\perp \cos\theta)\,\versj.
\end{equation}
%
Veja que a equação acima não é uma função do tempo, mas sim uma função do deslocamento angular $\theta$. Esse tipo de expressão é denominada \emph{equação paramétrica}, pois ela descreve uma curva no espaço em função de um parâmetro ---~isto é, ela descreve tanto a posição no eixo $x$, quanto no eixo $y$ em termos de um parâmetro; normalmente descrevemos trajetórias\footnote[][-4cm]{Veja que esse conceito não é tão estranho assim: quando descrevemos uma trajetória como uma função do tempo $\vec{r}(t)$, isso é um exemplo de uma equação paramétrica com um parâmetro $t$.} na forma $y(x)$~---. Na Figura~\ref{Fig:GraficosCicloideEAfins}, temos exemplos das trajetórias descritas pela equação acima. Para o caso $r_\perp = R$, em particular, temos uma curva conhecida como \emph{cicloide}. 

\begin{marginfigure}[-2cm]
\centering
\begin{tikzpicture}[>=Stealth,  interface/.style={
        % superfície
        postaction={draw,decorate,decoration={border,angle=-45,
                    amplitude=0.2cm,segment length=2mm}}}
    ]
    
    % piso
    \draw[interface, gray] (-0.5,0) -- (4,0);
    
    % corpo
    \draw[draw = gray, pattern color = gray, pattern = north west lines] (0, 0.5) circle (0.5);
    \draw[gray, ->] (0, 0.5)+(135:0.6) arc[start angle = 135, end angle = 80, radius = 0.6] node[above, midway]{$\omega$};
    \draw[fill, gray] (0,0.5) circle (0.8pt);
    
    % cicloides
    \draw[domain=-1.5:8, samples=100] plot ({0.5 * \x - 0.5 * sin(\x r)}, {0.5 * 1 - 0.5 * cos(\x r)} );
    \draw[densely dotted,  domain=-1.5:8, samples=100] plot ({0.5 * \x - 0.325 * sin(\x r)}, {0.5 * 1 - 0.325 * cos(\x r)} );
    \draw[dotted, domain=-1.5:8, samples=100] plot ({0.5 * \x - 0.15 * sin(\x r)}, {0.5 * 1 - 0.15 * cos(\x r)} );
    
    % pontos
    \draw[fill] (0,0) circle (0.8pt);
    \draw[fill] (0,0.175) circle (0.8pt);
    \draw[fill] (0, 0.35) circle (0.8pt);

    % corpo 2
    \draw[densely dotted, gray] (1.3, 0.5) circle (0.5);
    \draw[fill, gray] (1.3,0.5) circle (0.8pt);
    
    \draw[fill] (1.0442, 0.9284) circle (0.8pt);
    \draw[fill] (1.1325,0.7785) circle (0.8pt);
    \draw[fill] (1.2227,0.6285) circle (0.8pt);
    
    % corpo 3
    \draw[densely dotted, gray] (2.6, 0.5) circle (0.5);
    \draw[fill, gray] (2.6,0.5) circle (0.8pt);
    
    \draw[fill] (3.0417,0.2657) circle (0.8pt);
    \draw[fill] (2.8871,0.3477) circle (0.8pt);
    \draw[fill] (2.7325,0.4297) circle (0.8pt);
    
\end{tikzpicture}
\caption{Se tomarmos um ponto qualquer de um corpo que executa um rolamento, sem que haja deslizamento, a sua trajetória em relação a um referencial fixado no solo será descrita pela Equação~\ref{Eq:TrajetoriaPontosQueRolam}. No caso de um ponto na borda do corpo, a trajetória será aquela representada por uma linha cheia na figura acima, sendo que tal curva é conhecida como \emph{cicloide}. \label{Fig:GraficosCicloideEAfins}}
\end{marginfigure}

%%%%%%%%%%%%%%%%%%%%%%%%%%%%%%%%%%%%%%%%%%%%%%%%%%%%%%%%%%%%%%%%%%%%%%%%
\paragraph{Velocidade dos pontos de um corpo rígido durante o rolamento}
%%%%%%%%%%%%%%%%%%%%%%%%%%%%%%%%%%%%%%%%%%%%%%%%%%%%%%%%%%%%%%%%%%%%%%%%

Para determinar a velocidade de um ponto $P$ qualquer em relação ao solo durante o rolamento, basta derivarmos a expressão para a posição em relação ao tempo:
\begin{align}
    \vec{v}_{P, \, s} &= \frac{d}{dt}\vec{r}_{P, \, s} \\
    &= \frac{d}{dt}\left[(R\,\theta - r_\perp \sen\theta)\,\versi + (R - r_\perp \cos\theta)\,\versj\right] \\
    &= \left(R \frac{d\theta}{dt} - r_\perp \cos\theta\frac{d\theta}{dt}\right)\versi + \left(r_\perp \sen\theta\frac{d\theta}{dt}\right)\versj.
\end{align}
%
A derivada da posição angular nada mais é do que a própria velocidade angular:
\begin{equation}
    \frac{d\theta}{dt} = \omega.
\end{equation}
Assim, temos que
\begin{equation}\label{Eq:VelPSoloOmega}
    \vec{v}_{P, \, s} = \omega(R - r_\perp \cos\theta)\versi + r_\perp \omega \sen\theta \,\versj.
\end{equation}

Verificamos anteriomente que a velocidade angular de um corpo que rola está ligada ao módulo da velocidade do centro de massa em relação ao solo por
\begin{equation}
    v_{\rm{CM},\, s} = r\,\omega,
\end{equation}
%
onde $r$ é o raio do corpo que rola e, no presente caso, é representado por $R$. Utilizando tais resultados, podemos escrever a velocidade do ponto $P$ em relação ao solo como\footnote{Note que temos uma separação bastante clara entre a translação e a rotação nessa expressão para a velocidade: o primeiro termo no eixo $x$ se refere à translação, os demais à rotação.}
\begin{equation}\label{Eq:VelPSoloVCM}
    \vec{v}_{P, \, s} = v_{\rm{CM}, \, s}\left(1 - \frac{r_\perp}{R} \cos\theta\right)\versi + v_{\rm{CM},\, s} \frac{r_\perp}{R} \sen\theta \,\versj.
\end{equation}

As Equações~\eqref{Eq:VelPSoloOmega} e~\eqref{Eq:VelPSoloVCM} acima nos permitem calcular o vetor velocidade de um ponto $P$ qualquer cuja distância em relação ao eixo de rotação é $r_\perp$, durante o rolamento de um corpo com raio $R$. Se calcularmos o módulo da velocidade, obtemos
\begin{equation}
    v_{P,\,s} = \sqrt{\left(v_{\rm{CM}} - \frac{r_\perp}{R} v_{\rm{CM}}\cos\theta\right)^2 + \left(\frac{r_\perp}{R}\sen\theta\right)^2},
\end{equation}
%
que após densenvolvermos os termos quadráticos e usarmos o fato de que $\sen^2\theta + \cos^2\theta = 1$, pode ser escrita como
\begin{equation}\label{Eq:VelEscalarPontoRolamento}
    v_{P,\,s} = v_{\rm{CM}} \sqrt{1 + \frac{r_\perp^2}{R^2} - 2\frac{r_\perp}{R}\cos\theta}.
\end{equation}
%
Como podemos notar, a expressão acima nos dá o módulo da velocidade para qualquer ponto ao assumir uma dada posição $\theta$. Tomandos alguns pontos mais interessantes, verificamos que:
\begin{description}
    \item[Para o caso $r_\perp = 0$:] Para o centro de massa, temos que $r_\perp = 0$, logo,
        \begin{align}
            v_{P,\,s} &= v_{\rm{CM}} \sqrt{1 + \frac{r_\perp^2}{R^2} - 2\frac{r_\perp}{R}\cos\theta} \\
            &= v_{\rm{CM}} \sqrt{1 + 0 - 0} \\
            & = v_{\rm{CM}}.
        \end{align}
        
    \item[Para o caso $r_\perp = R$:] Para os pontos da borda da roda,
        \begin{align}
            v_{P,\,s} &= v_{\rm{CM}} \sqrt{1 + \frac{r_\perp^2}{R^2} - 2\frac{r_\perp}{R}\cos\theta} \\
            &= v_{\rm{CM}} \sqrt{1 + 1 - 2\cos\theta} \\
            & = v_{\rm{CM}} \sqrt{2(1-\cos\theta)}.
        \end{align}
        %
        Se substituirmos os valores \np[\tcdegree]{0}, \np[\tcdegree]{180}, \np[\tcdegree]{90}, e \np[\tcdegree]{270} ---~que correspondem aos pontos inferior, superior, esquerdo e direito da roda, em relação ao centro de massa~---, obtemos:
        \begin{align}
            v_{P,\,s}^{\rm{inf}} &= v_{\rm{CM}} \sqrt{2(1-1)} \\
            &= 0 \\
            v_{P,\,s}^{\rm{sup}} &= v_{\rm{CM}} \sqrt{2(1-(-1))} \\
            &= 2 \, v_{\rm{CM}} \\
            v_{P,\,s}^{\rm{esq}} &= v_{\rm{CM}} \sqrt{2(1-0)} \\
            &= \sqrt{2}\, v_{\rm{CM}} \\
            v_{P,\,s}^{\rm{dir}} &= v_{\rm{CM}} \sqrt{2(1-0)} \\
            &= \sqrt{2}\, v_{\rm{CM}}.
        \end{align}
        %
\end{description}

%%%%%%%%%%%%%%%%%%%%%%%%%%%%%%%%%%%%%%%%%%%%%%%%%%%%%%%%%%%%%%%%%%%%%%%%%%%%%%%%%%%
\subsection{Forças no rolamento}
%%%%%%%%%%%%%%%%%%%%%%%%%%%%%%%%%%%%%%%%%%%%%%%%%%%%%%%%%%%%%%%%%%%%%%%%%%%%%%%%%%%

Se ignorarmos a força de arrasto oferecida pelo ar, quando um corpo rola sem deslisar sobre uma superfície horizontal, temos
\begin{description}
    \item[Eixo $x$:] Se considerarmos a possibilidade de que haja uma força de atrito no ponto de contato com o solo,
        \begin{align}
            F_{R,\,x}^{\rm{ext}} &= m a_x \\
            f_{\rm{at}} &= m a_x.
        \end{align}
        
    \item[Eixo $y$:] Considerando que não há aceleração no eixo vertical, temos
    \begin{align}
        F_{R,\,y}^{\rm{ext}} &= m a_y \\
        N - P &= 0.
    \end{align}
\end{description}

\begin{marginfigure}
\centering
\begin{tikzpicture}[>=Stealth,  interface/.style={
        % superfície
        postaction={draw,decorate,decoration={border,angle=-45,
                    amplitude=0.2cm,segment length=2mm}}}
    ]

    %Fig superior
    \draw[interface, gray] (0,0) -- (4,0);
    
    \draw[gray, pattern color = gray, pattern = north west lines] (2,0.5) circle (0.5cm);
    \draw[->] (2,0.5) +(160:0.7) arc[start angle = 160, end angle = 110, radius = 0.7] node[midway, left]{$\omega$};
    \draw[fill] (2,0.5) circle (1pt);
    
    \draw[->, thick] (2,0.5) -- +(0,-1) node[right]{$\vec{P}$};    
    \draw[->, thick] (2,1) -- +(0,1) node[right]{$\vec{N}$};
    
    \path (3.25, 1) -- node[above]{$\vec{a} = 0$}(2.75,1);
    
    \draw[dashdotted,->] (0.15,0) -- +(1,0) node[below left]{$x$};
    \draw[dashdotted,->] (0.15,0) -- +(0,1) node[below left]{$y$};
    \draw[fill] (0.15,0) circle (0.8pt);
    
    % Fig meio
    
    \draw[interface, gray] (0,-3) -- (4,-3);
    
    \draw[gray, pattern color = gray, pattern = north west lines] (2,-2.5) circle (0.5cm);
    \draw[->] (2,-2.5) +(160:0.7) arc[start angle = 160, end angle = 110, radius = 0.7] node[midway, left]{$\omega$};
    \draw[fill] (2,-2.5) circle (1pt);
    
    \draw[->, thick] (2,-2.5) -- +(0,-1) node[right]{$\vec{P}$};    
    \draw[->, thick] (2,-2) -- +(0,1) node[right]{$\vec{N}$};
    
    \draw[<-] (3.25, -2) -- node[above]{$\vec{a}$}(2.75,-2);
    
    \draw[->,thick] (2,-3) -- +(0.75,0) node[below]{$\vec{f}_{\rm{at}}$};
    \draw[fill] (2,-3) circle (0.8pt);
    
    \draw[dashdotted,->] (0.15,-3) -- +(1,0) node[below left]{$x$};
    \draw[dashdotted,->] (0.15,-3) -- +(0,1) node[below left]{$y$};
    \draw[fill] (0.15,-3) circle (0.8pt);
    
    % Fig inferior
    
    \draw[interface, gray] (0,-6) -- (4,-6);
    
    \draw[gray, pattern color = gray, pattern = north west lines] (2,-5.5) circle (0.5cm);
    \draw[->] (2,-5.5) +(160:0.7) arc[start angle = 160, end angle = 110, radius = 0.7] node[midway, left]{$\omega$};
    \draw[fill] (2,-5.5) circle (1pt);
    
    \draw[->, thick] (2,-5.5) -- +(0,-1) node[right]{$\vec{P}$};    
    \draw[->, thick] (2,-5) -- +(0,1) node[right]{$\vec{N}$};
    
    \draw[->] (3.25, -5) -- node[above]{$\vec{a}$}(2.75,-5);
    
    \draw[->,thick] (2,-6) -- +(-0.75,0) node[below right]{$\vec{f}_{\rm{at}}$};
    \draw[fill] (2,-6) circle (0.8pt);
    
    \draw[dashdotted,->] (0.15,-6) -- +(1,0) node[below left]{$x$};
    \draw[dashdotted,->] (0.15,-6) -- +(0,1) node[below left]{$y$};
    \draw[fill] (0.15,-6) circle (0.8pt);
    
\end{tikzpicture}
\caption{Forças de atrito no ponto de contato com o solo para um corpo que rola sem deslizar. Note que as figuras correspondem ao caso em que o próprio atrito é o responsável pela aceleração, como no caso da roda motriz de um veículo (a roda traseira de uma bicicleta, por exemplo).}
\end{marginfigure}

Verificamos através do eixo horizontal que a existência da força de atrito está condicionada à existência de uma aceleração horizontal:
\begin{description}
    \item[Caso $a_x = 0$:] Se a aceleração horizontal é nula, concluímos que $f_{at} = 0$, pois não há nenhuma outra força que eventualmente possa equilibrá-la.
    \item[Caso $a_x \neq 0$:] Se o objeto que rola for uma roda de bicicleta, por exemplo, ao frearmos deve agir sobre algum ponto da roda uma força dirigida para trás, que será responsável por desacelerar o centro de massa do sistema\footnote{As forças exercidas sobre as rodas seja pelos freios da bicicleta, seja pelo ciclista, são forças internas e não são capazes de alterar a velocidade do centro de massa da roda/bicicleta.}. Claramente essa força será uma força de atrito que atuará no ponto de contato da roda com o solo. Se, por outro lado, o ciclista resolver acelerar a bicicleta, deverá aparecer uma força de atrito no ponto de contato da roda com o solo, dirigida para frente\footnote{Na roda motriz, ou seja, a traseira; na da frente o atrito é para trás, pois deve causar uma aceleração angular da roda, veja a discussão abaixo.}.
\end{description}
%
Concluímos então que no caso de um rolamento em um plano horizontal, do ponto de vista de translação, as forças que aceleram ou desaceleram o centro de massa são \emph{forças de atrito no ponto de contato com o solo}.

De um ponto de vista de rotação, a análise pode ser muito complexa pois existe uma série de fatores a serem analisados. Vamos considerar, por exemplo, o que acontece com as rodas de uma bicicleta em algumas situações:
\begin{description}
    \item[Frenagem:] Ao aplicarmos os freios, sabemos que o sistema composto pela bicicleta e pelo ciclista desacelera. Isso, como vimos, se deve à força de atrito entre os pneus e o solo. Do ponto de vista de rotações, os torques devidos ao atrito são no sentido de causar um \emph{aumento da velocidade angular}. A diminuição da velocidade angular se deve ao fato de que existe um torque exercido pelos freios da bicicleta. Note ainda que se considerarmos um eixo que passa pelo centro de massa do sistema, verificamos que as forças de atrito tendem a causar uma rotação do sistema bicicleta+ciclista como um todo. Para que o sistema não sofra uma rotação, é necessário que a força normal exercida pelo solo sobre a roda da frente aumente, enquanto a força normal exercida sobre a roda de trás diminui.\footnote{Isso também explica por que a frente de um carro, ou qualquer outra veículo que tenha suspensão dianteira, abaixa nas frenagens.}
    \item[Aceleração:] Ao acelerarmos, sabemos que deve haver uma força de atrito na roda motriz e que ela deve acelerar o sistema. O torque realizado pela força de atrito sobre a roda, no entanto, tende a desacelerá-la. Como o ciclista exerce força sobre os pedais, no entanto, temos um torque no sentido de acelerar a roda que se deve a tal força\footnote{Transmitida através dos pedais, pedivela, coroas, corrente e pinhões do sistema de transmissão da bicicleta.}. Já na roda da frente, que não é motriz e deve ser ``arrastada'', o atrito surge para trás, provocando um torque que acelera a roda. Novamente, considerando o sistema como um todo está sujeito a um torque que tende a girar a bicicleta+ciclista para trás, ``empinando'', deve haver uma diminuição da força normal que atua sobre a roda dianteira e um aumento da normal que atua sobre a roda traseira.\footnote{Note que se a normal aumenta, a força de atrito estático máximo, e também a aceleração máxima aumentam. Por isso um veículo com tração traseira tende a ter melhores ``arrancadas'' que um com tração dianteira. Note também que se há uma diminuição da força de atrito nas rodas da frente, ao acelerar ficamos mais suscetíveis a perder o controle de um veículo. Isso vale para qualquer veículo.}
    \item[Desaceleração devido à resistência à rolagem:] Se o ciclista não freia nem pedala, eventualmente acaba parando devido à própria resistência à rolagem oferecida pelos pneus. No ponto de contato com o solo, o pneu sofre uma deformação que se deve a uma força exercida pelo solo. Devido às características do pneu, ele se comporta como uma mola, exercendo a mesma força sobre a roda. Devido ao fato de que a deformação não é totalmente elástica, quando o pneu volta à forma original, a força exercida na expansão é menor que na compressão, o que faz com que surja uma força resultante que tende a desacelerar a roda. Essa desaceleração tende a fazer com que a velocidade da borda da roda fique em descompasso com a velocidade do centro de massa, o que causa uma tendência ao deslisamento da roda para frente. Essa tendência, por sua vez, causa uma força de atrito que desacelera o sistema.
    \item[Desaceleração devida ao arrasto:] A força de arrasto atua principalmente sobre o ciclista, devido à maior área de seção reta. Assim, a força se transmite ao quadro da bicicleta pelos pontos de apoio. O quadro, por sua vez, está ligado ao eixo de rotação das rodas pelos eixos. As forças exercidas através do eixo tendem a transladar a roda como um todo, o que tende a causar um deslisamento da roda em relação ao solo. Essa tendência ao deslizamento tende a causar um atrito que atua para frente no ponto de contato, pois é necessário que a roda diminua sua velocidade angular. É claro que, como a força de arrasto é maior, a resultante tende a causar uma desaceleração do sistema.
    
    \item[Bicicleta sendo empurrada:] De uma maneira similar ao caso em que a força de arrasto atua sobre o ciclista, se empurramos a bicicleta por algum ponto do quadro, ou ligado a ele, transmitimos uma força às rodas através do eixo do cubo da roda. Ao fazermos isso, criamos uma tendência de deslizamento no ponto de contato com o solo, o que faz com que surjam forças de atrito dirigidas para trás, e que causam a rotação das rodas.
\end{description}

Através dos casos acima percebemos que as forças que atuam no rolamento estão ligadas não só à rotação do corpo, mas também ao movimento do centro de massa. Portanto, devemos aplicar tanto a Segunda Lei de Newton para a translação, quando a Segunda Lei de Newton para a rotação a um corpo que efetua um rolamento.\footnote{O que é similar ao que fizemos para a Roda de Maxwell.} 

%%%%%%%%%%%%%%%%%%%%%%%%%%%%%%%%%%%%%%%%%%%
\paragraph{Exemplo: Rolamento em uma Rampa}
%%%%%%%%%%%%%%%%%%%%%%%%%%%%%%%%%%%%%%%%%%%

Um problema que pode ser tratado a partir das observações acima e que trás um resultado bastante interessante é o de um objeto que rola rampa abaixo:
\begin{quote}
    Na Figura~\ref{Fig:RolamentoRampaFigQuestao}, um corpo está disposto em uma rampa. Ao liberarmos para que ele se mova, inicia-se um rolamento, uma vez que existe atrito entre o corpo e a rampa. Considerando que a massa do corpo é $m$, o raio é $r$, e que seu o momento de inércia é $I = f mr^2$, determine uma expressão para a aceleração do centro de massa em função do ângulo $\beta$ de inclinação do plano.
\end{quote}

\begin{marginfigure}[-2cm]
\centering
\begin{tikzpicture}[>=Stealth,  rotate = -30, interface/.style={
        % superfície
        postaction={draw,decorate,decoration={border,angle=-45,
                    amplitude=0.2cm,segment length=2mm}}}
    ]
    
    \draw[interface] (0,0) coordinate (O) -- (4,0) coordinate (Or);
    \draw (Or) -- +(-150:1) coordinate (h);
    
    \pic [draw, "$\beta$", angle eccentricity = 1.2, angle radius = 8mm]{angle = O--Or--h};
    
    \draw[pattern = north west lines] (1,0.5) circle (0.5);
    
\end{tikzpicture}
\caption{Corpo sujeito a um movimento de rolamento.\label{Fig:RolamentoRampaFigQuestao}}
\end{marginfigure}

Analisando o corpo do ponto de vista da translação (como se ele fosse um bloco e não pudesse girar) através da Segunda Lei de Newton para a Translação, temos
\begin{description}
    \item[Eixo $x$:] Adotando um eixo $x$ na direção da aceleração do centro de massa, isto é, paralelo ao plano inclinado, temos:
    \begin{align}
        F_{R,\,x}^{\rm{ext}} &= ma_x \\
        mg \sen\theta - f_{\rm{at}} &= m a_{\textrm{CM}}.
    \end{align}

    \item[Eixo $y$:] Para o eixo $y$ temos
    \begin{align}
        F_{R,\,y}^{\rm{ext}} &= m a_y \\
        N - P_y &= 0,
    \end{align}
    %
    onde usamos o fato de que não há aceleração perpendicular ao plano inclinado. Essa equação não será particularmente útil, pois não podemos assumir que $f_{\rm{at}} = \mu_e N$, uma vez que nada garante que o corpo esteja na iminência de deslizar.
\end{description}

\begin{marginfigure}[5cm]
\centering
\begin{tikzpicture}[>=Stealth,  rotate = -30, interface/.style={
        % superfície
        postaction={draw,decorate,decoration={border,angle=-45,
                    amplitude=0.2cm,segment length=2mm}}}
    ]
    
    \draw[interface, gray] (0,0) coordinate (O) -- (4,0) coordinate (Or);
    \draw[gray] (Or) -- +(-150:1) coordinate (h);
    
    \pic [draw, "$\beta$", angle eccentricity = 1.2, angle radius = 8mm]{angle = O--Or--h};
    
    \draw[pattern = north west lines, draw = gray, pattern color = gray] (1,0.5) circle (0.5);
    \draw[fill] (1,0.5) circle (0.8pt);
    
    \draw[->, thick] (1,1) -- node[right]{$\vec{N}$} +(0,0.866);
    \draw[->, thick] (1,0.5) -- +(-60:1) node[left]{$\vec{P}$};
    %\draw[dashed, thick, ->] (1,0.5) -- +(0,-1.08)node[left]{$P_y$};
    %\draw[dotted, thick, ->] (1,0.5) -- +(0.625,0) node[above right]{$P_x$};
    %\draw[loosely dotted] (1,0.5) ++(0,-1.08) -- +(0.625,0);
    %\draw[loosely dotted] (1,0.5) ++(0.625,0) -- +(0,-1.08);
    \draw[->, thick] (1,0) -- +(-0.75,0) node[below]{$\vec{f}_{\rm{at}}$};
    \draw[fill] (1,0) circle (0.8pt);
%    \draw[dashed,->] (1,0.5) -- node[above]{$r$} +(220:0.5);
    \draw[|<->|] (0.3, 0.5) -- +(0,0.5);
    \node (r) at (0.1, 0.75) {$r$};
    \draw[dotted] (0.3,1) -- +(0.5,0);
    
    \draw[dashdotted] (0,0.5) -- (0.5,0.5);
    \draw[dashdotted,->] (1.5,0.5) -- +(1.5,0) node[below left]{$x$};
    \draw[dashdotted,->] (1,1)++(0,0.866) -- +(0,0.5)node[left]{$y$};
    \draw[dashdotted] (1,0) -- +(0,-1);
    
\end{tikzpicture}
\caption{Rolamento de um corpo sobre uma superfície plana inclinada. Note que a força de atrito tem o papel tanto de causar \emph{desaceleração} da translação, quanto de causar \emph{aceleração} da rotação, isto é, de causar \emph{aceleração angular}.}
\end{marginfigure}

%\textbf{\textbf{Seria interessante provar que o torque do peso em torno de um eixo de rotação é o torque efetuado pelo centro de massa em torno do eixo, sendo que a força é a própria força peso.}}
Analisando a rotação do sistema, temos que
\begin{equation}
  \tau_R^{\rm{ext}} = I\alpha.
\end{equation}
%
Devemos considerar os torques em relação ao eixo de rotação, sendo que temos três forças atuando no sistema: a força peso, a força normal, e a força de atrito. Para o caso da força peso, em particular, devemos considerar os torques devidos às forças peso de cada uma das partículas que compõe o sistema. Como o corpo é simétrico em relação ao eixo de rotação, o torque resultante é nulo. Caso o corpo não seja simétrico, devemos considerar o torque exercido pela força peso em torno do eixo de rotação, considerando que tal força atua no centro de massa do corpo. Já para o caso da força normal, verificamos que ela atua perpendicularmente ao eixo de rotação, o que resulta em um torque nulo devido a essa força. Logo, resta o torque devido à força de atrito:
\begin{equation}
  -f_{\rm{at}} \, r = I\alpha,
\end{equation}
%
onde usamos que $\tau = r f_{\rm{at}} \sen\degree{90}$. Note que o torque tende a causar uma rotação no sentido horário e é, portanto, negativo. Adicionamos um sinal para refletir tal convenção. Além disso, para o rolamento temos que 
\begin{equation}
  a_{\textrm{CM}} = r\,\alpha.
\end{equation}
%
Devido à escolha do sistema de coordenadas, no entanto, uma aceleração positiva no eixo $x$ implica em uma \emph{aceleração angular negativa}\footnote{Lembre-se que convencionamos que acelerações angulares que tendem a causar rotações no sentido anti-horário são positivas.}. Logo, devemos acertar essa diferença de sinais adicionando um sinal negativo:
\begin{equation}
  a_{\textrm{CM}} = -r\,\alpha.
\end{equation}

Podemos agora montar um sistema de equações dado por
\begin{equation}
\begin{system}
    mg\sen\theta - f_{\rm{at}}&= m a_{\textrm{CM}} \\
    -f_{\rm{at}} \,r &= I\alpha \\
    a_{\textrm{CM}} &= -r\,\alpha
\end{system}
\end{equation}
%
Resolvendo para a aceleração, obtemos
\begin{equation}
  a_{\textrm{CM}} = \frac{\sen\theta}{1+I/(mr^2)} g.
\end{equation}
%
Esse resultado é interessante pois mostra que a aceleração a que um corpo será submetido ao descer uma rampa executando um rolamento sem deslizar é diferente para cada objeto, dependendo de seu momento de inércia. Utilizando a expressão $I = f mr^2$, obtemos:\footnote{O momento de inércia para alguns objetos com seção reta circular pode ser escrito como $I = f mr^2$, onde $f$ é um número menor que 1.}
\begin{equation}
  a_{\textrm{CM}} = \frac{\sen\theta}{1+f} g.
\end{equation}
%
Isto é, cada tipo de objeto tem uma aceleração diferente, sendo tanto menor quanto maior for o valor de $f$. Se, por exemplo, tomarmos três objetos com formas distintas ---~um aro, um cilindro, e uma esfera maciça, por exemplo~--- e os soltarmos a partir do topo de uma rampa, eles levarão tempos diferentes para percorrer a distância até a base. Considerando os três objetos tomados como exemplo, temos que a ordem de chegada será: esfera, cilindro e aro, devido aos valores de $f$ para esses três objetos: $f_a = 1$, $f_c = \nicefrac{1}{2}$ e $f_e = \nicefrac{2}{5}$. Veja ainda que a aceleração não depende da massa ou do raio do objeto em questão, mas sim do fator $f$ associado à \emph{forma} do objeto.

%%%%%%%%%%%%%%%%%%%%%%%%%%%%%%%%%%%%%%%%%%
\subsection{Energia no rolamento}
%%%%%%%%%%%%%%%%%%%%%%%%%%%%%%%%%%%%%%%%%%

No rolamento é válida a expressão
\begin{equation}
    K_{\rm{rol}} = K_{\rm{rot}} + K_{\rm{trans}}
\end{equation}
%
para a energia cinética em um movimento combinado de rotação e de translação. Além disso, também podemos utilizar a conservação da energia mecânica. Verificaremos abaixo dois exemplos de aplicação dos conceitos de energia ao rolamento.

%%%%%%%%%%%%%%%%%%%%%%%%%%%%%%%%%%%%%%%%%%%%%%%%%%%%%%%%%%%%%%%%%%%%%%%%
\paragraph{Razão entre as energias cinéticas de rotação e de translação}
%%%%%%%%%%%%%%%%%%%%%%%%%%%%%%%%%%%%%%%%%%%%%%%%%%%%%%%%%%%%%%%%%%%%%%%%

No caso de não haver deslizamento, como temos um vínculo entre as velocidades de translação e de rotação, as energias cinéticas também têm valores vinculados um ao outro: a razão entre elas é constante:
\begin{align}
    \frac{K_{\rm{rot}}}{K_{\rm{trans}}} &= \frac{\nicefrac{1}{2} \, I \omega^2}{\nicefrac{1}{2} \, m v_{\rm{CM}}^2} \\
    &= \frac{I \omega^2}{m v_{\rm{CM}}^2} \\
    &= \frac{I v_{\rm{CM}}^2/R^2}{m v_{\rm{CM}}^2} \\
    &= \frac{I /R^2}{m},
\end{align}
%
onde usamos
\begin{equation}
    v_{\rm{CM}} = r\,\omega,
\end{equation}
%
com $r = R$.

Assumindo que o momento de inércia possa ser escrito na forma
\begin{equation}
    I = f MR^2,
\end{equation}
%
como fizemos ao calcular a aceleração do centro de massa durante o rolamento em um plano inclinado, temos finalmente
\begin{align}
    \frac{K_{\rm{rot}}}{K_{\rm{trans}}} &= \frac{f mR^2 /R^2}{m} \\
    &= f.
\end{align}

%%%%%%%%%%%%%%%%%%%%%%%%%%%%%%%%%%%%%%%%%%%%%%%%%%%%%%%%%
\paragraph{Velocidade no rolamento em uma rampa}
%%%%%%%%%%%%%%%%%%%%%%%%%%%%%%%%%%%%%%%%%%%%%%%%%%%%%%%%%

Se tomarmos um corpo e o dispusermos no topo de um plano inclinado ---~como retratado na Figura~\ref{Fig:ConsEnergiaRolamento}~---, liberando-o a partir do repouso, podemos determinar a velocidade do centro de massa após ele percorrer uma distância $d$. Para considerar a aplicação da conservação da energia, em primeiro lugar, precisamos determinar se o trabalho exercido por forças não conservativas é nulo. No problema em questão, atuam duas forças não conservativas: a normal e a força de atrito.

No caso da força normal, temos que a direção da força é sempre perpendicular ao deslocamento, o que implica em um trabalho nulo. No caso da força de atrito, temos que o ponto onde a força atua não se desloca: conforme o rolamento progride, temos uma troca de pontos de apoio, porém não temos o deslocamento dos pontos, fazendo com que o trabalho seja nulo\footnote{Se um bloco é posto sobre uma esteira que é puxada, acelerando o bloco através da força de atrito, por exemplo, temos que o ponto de aplicação da força de atrito se desloca no espaço. Consequentemente, temos um trabalho realizado pelo atrito. No caso do rolamento, no entanto, esse deslocamento do ponto onde a força é aplicada não ocorre. (Lembre-se que o ponto de contato tem velocidade nula em relação ao solo, portanto não se desloca enquanto está sujeito à força de atrito).}. Assim, podemos utilizar a conservação da energia:
\begin{align}
    \Delta E &= W_{\rm{NC}} \\
    &= W_N + W_{f_{\rm{at}}} \\
    &=0,
\end{align}
%
o que implica em
\begin{equation}
    E_i = E_f.
\end{equation}

\begin{marginfigure}
\centering
\begin{tikzpicture}[>=Stealth,  rotate = -30, interface/.style={
        % superfície
        postaction={draw,decorate,decoration={border,angle=-45,
                    amplitude=0.2cm,segment length=2mm}}}
    ]
    
    \draw[interface, gray] (0,0) coordinate (O) -- (4,0) coordinate (Or);
    \draw[gray] (Or) -- +(-150:1) coordinate (h);
    
    \pic [draw, "$\beta$", angle eccentricity = 1.2, angle radius = 8mm]{angle = O--Or--h};
    
    \draw[pattern = north west lines, draw = gray, pattern color = gray] (1,0.5) circle (0.5);
    \draw[fill] (1,0.5) circle (0.8pt);
    
    \draw[->, thick] (1,1) -- node[right]{$\vec{N}$} +(0,0.866);
    \draw[->, thick] (1,0.5) -- +(-60:1) node[left]{$\vec{P}$};
    %\draw[dashed, thick, ->] (1,0.5) -- +(0,-1.08)node[left]{$P_y$};
    %\draw[dotted, thick, ->] (1,0.5) -- +(0.625,0) node[above right]{$P_x$};
    %\draw[loosely dotted] (1,0.5) ++(0,-1.08) -- +(0.625,0);
    %\draw[loosely dotted] (1,0.5) ++(0.625,0) -- +(0,-1.08);
    \draw[->, thick] (1,0) -- +(-0.75,0) node[below]{$\vec{f}_{\rm{at}}$};
    \draw[fill] (1,0) circle (0.8pt);
%    \draw[dashed,->] (1,0.5) -- node[above]{$r$} +(220:0.5);
%    \draw[|<->|] (0.3, 0.5) -- +(0,0.5);
%    \node (r) at (0.1, 0.75) {$r$};
%    \draw[dotted] (0.3,1) -- +(0.5,0);
    
    \draw[dashdotted] (0,0.5) -- (0.5,0.5);
    \draw[dashdotted,->] (1.5,0.5) -- +(1,0) node[below left]{$x$};
    \draw[dashdotted,->] (1,1)++(0,0.866) -- +(0,0.5)node[left]{$y$};
    \draw[dashdotted] (1,0) -- +(0,-1);
    
    \draw[|<->|] (1,1.2) -- node[above]{$d$} (3.5,1.2);
    \draw[dotted] (3.5,0.5) circle (0.5);
    \draw[gray, fill] (3.5,0.5) circle (0.5pt);
    \draw[->,gray] (3.5,0.5)++(70:0.7) arc[start angle = 70, end angle = 20, radius = 0.7] node[midway, right]{$\omega$};

\end{tikzpicture}
\caption{Podemos utilizar a conservação da energia mecânica para determinar a velocidade de translação de um rolamento após o corpo percorrer uma distância $d$. \label{Fig:ConsEnergiaRolamento}}
\end{marginfigure}

Sabemos que a energia mecânica é dada pela soma das energias cinética e potencial:
\begin{equation}
    K^i + U^i = K^f + U^f.
\end{equation}
%
No sistema em questão, temos somente uma força conservativa ---~o peso~---, portanto só temos um potencial. No caso da energia cinética, no entanto, temos duas formas: a de translação e a de rotação:
\begin{equation}
    K_r^i + K_t^i + U_g^i = K_r^f + K_r^t + U_g^f.
\end{equation}
%
Como o corpo é liberado a partir do repouso no topo da rampa, temos que as energias cinéticas iniciais são nulas. Além disso, podemos escolher a origem do eixo vertical na posição final do centro de massa, o que implica que a posição inicial é dada por $d\sen\beta$. Assim,
\begin{equation}
    mgd\sen\beta = \frac{1}{2} mv_f^2 + \frac{1}{2} I\omega^2.
\end{equation}
%
Se utilizarmos o vínculo entre a velocidade angular e a velocidade de translação válida para movimentos de rolamento (que ocorrem sem deslizar), temos
\begin{equation}
    mgd\sen\beta = \frac{1}{2} mv_f^2 + \frac{1}{2} I \frac{v_f^2}{r^2}.
\end{equation}
%
Assumindo que $I = f MR^2$, temos
\begin{align}
    mgd\sen\beta = \frac{1}{2} mv_f^2 + \frac{1}{2} f mr^2 \frac{v_f^2}{r^2} \\
    gd\sen\beta = \frac{1}{2} v_f^2 + \frac{1}{2} f v_f^2 \\
    2gd\sen\beta = (1 + f) v^2 \\
    \frac{2gd\sen\beta}{1 + f} = v^2.
\end{align}
%
Finalmente,
\begin{equation}
    v = \sqrt{\frac{2gd\sen\beta}{1 + f}}.
\end{equation}


%%%%%%%%%%%%%%%%%%%%
\section{Exercícios}
%%%%%%%%%%%%%%%%%%%%

%%%%%%%%%%%%%%%%%%%%%%%%%%%%%%
\paragraph{Momento de inércia}
%%%%%%%%%%%%%%%%%%%%%%%%%%%%%%

\begin{question}[type={exam}]\label{Q:MomInerciaDiscosLigadosPorHastes}
A Figura~\ref{Q:MomInerciaDiscosLigadosPorHastes} mostra um objeto construído com discos de material leve, ligados por barras metálicas. As massas dos discos e das barras são iguais, sendo denotadas por $m$. O raio dos discos é $R$, e o comprimento das barras é $L = 2R$. Determine o momento de inércia do objeto em torno do eixo que passa perpendicularmente à face plana do disco central, passando pelo centro de massa, isto é, no ponto indicado na figura. {\it R: $(\nicefrac{253}{6}) mR^2$.}
\begin{marginfigure}[-3cm]
\centering
\begin{tikzpicture}[>=Stealth]
    \draw[pattern = north west lines, pattern color = gray] (0,0) circle (0.5);
    \draw[fill] (0,0) circle (0.8pt);
    \draw[very thick] (-0.5,0) -- +(-1,0);
    \draw[very thick] (0.5,0) -- +(1,0);
    \draw[pattern = north west lines, pattern color = gray] (-2,0) circle (0.5);
    \draw[pattern = north west lines, pattern color = gray] (2,0) circle (0.5);
    
    \draw[->] (110:0.8) arc[start angle = 110, end angle = 70, radius = 0.8] node[midway, above]{$\omega$};
    
\end{tikzpicture}
\caption{Questão~\ref{Q:MomInerciaDiscosLigadosPorHastes}. \label{Fig:Q:MomInerciaDiscosLigadosPorHastes}}
\end{marginfigure}
\end{question}

\begin{question}[type={exam}]
A Figura~\ref{Fig:Q:DiscosFormandoTriangulo} mostra um objeto composto de três discos com massa $m$ e raio $r$ e que pode girar em torno do eixo $z$ mostrado e que está contido no mesmo plano que os discos. Determine o momento de inércia do objeto em torno do eixo $z$. {\it R: $I = (\nicefrac{11}{4}) mr^2$.}
\begin{marginfigure}[-0.5cm]
\centering
\begin{tikzpicture}[>=Stealth]\label{Q:DiscosFormandoTriangulo}
    \draw[pattern = north west lines, pattern color = gray] (0,0) circle (0.5);
    \draw[pattern = north west lines, pattern color = gray] (240:1) circle (0.5);
    \draw[pattern = north west lines, pattern color = gray] (-60:1) circle (0.5);
    
    \draw[<-] (0,1.5) node[right]{$z$} -- (0,0.5);
    \draw[dashed] (0,0.5) -- (0, -0.5);
    \draw (0,-0.5) -- (0,-2);
    
    \draw[->] (-0.25,1.2) arc[start angle = -240, end angle = 60, x radius = 0.5, y radius = 0.2] node[near end,below]{$\omega$};
\end{tikzpicture}
\caption{Questão~\ref{Q:DiscosFormandoTriangulo}.\label{Fig:Q:DiscosFormandoTriangulo}}
\end{marginfigure}
\end{question}

%%%%%%%%%%%%%%%%%%%%%%%%%%%%%%%%%%%%%%%%
\paragraph{Energia cinética em rotações}
%%%%%%%%%%%%%%%%%%%%%%%%%%%%%%%%%%%%%%%%

\begin{question}[type={exam}]\label{Q:HasteHorizontalDescendo}
Uma haste de comprimento $L$ disposta horizontalmente pode girar em torno de um eixo que passa por sua extremidade, como indicado na Figura~\ref{Fig:Q:HasteHorizontalDescendo}.
\begin{enumerate}[label=(\alph*)]
    \item Determine a velocidade angular da haste quando ela chega à posição mais baixa, assumindo que ela partiu do repouso. {\it R: $\omega = \sqrt{3g/L}$.}
    \item Apesar de termos somente uma rotação, é interessante notarmos que a energia cinética associada a rotação é diferente daquela que a haste teria se ela estivesse simplesmente transladando: Utilize a relação entre a velocidade de translação e a velocidade angular de um ponto e determine a razão entre a energia cinética de rotação da haste e a energia cinética de translação que seria associada a ela em uma translação pura. Considere que a velocidade de translação seja aquela que o centro de massa tem na situação do item (a). {\it R: $K_r/K_t = \nicefrac{4}{3}$. Esse resultado mostra que não podemos tratar o movimento da barra como se fosse o movimento de uma partícula.}
\end{enumerate}
\end{question}

\begin{marginfigure}
\centering
\begin{tikzpicture}[>=Stealth]

    \draw[pattern = north west lines] (0,-0.1) rectangle (3, 0.1);
    \draw[fill] (1.5,0) circle (0.6pt);
    \draw[dotted,rotate = -45] (0,-0.1) rectangle (3, 0.1);
    \draw[fill, rotate = -45] (1.5,0) circle (0.6pt);
    \draw[pattern = north east lines, rotate = -90, pattern color = gray, draw = gray] (0,-0.1) rectangle (3, 0.1);
    \draw[fill, rotate = -90] (1.5,0) circle (0.6pt);
    
    \draw[loosely dotted] (1.5,0) arc[start angle = 0, end angle = -90, radius = 1.5];
    \draw[->] (3.3, 0) arc[start angle = 0, end angle = -15, radius = 3.3];
    
    \draw[fill] (0,0) circle (0.8pt);
    
\end{tikzpicture}
\caption{Questão~\ref{Q:HasteHorizontalDescendo}. \label{Fig:Q:HasteHorizontalDescendo}}
\end{marginfigure}



