%%%%%%%%%%%%%%%%%%%%%%%%%%%%%%%%%%%%%%%%%%%%%%%%%%%%%%%%%%%%%%%%%%%%%%%%%%%%%%%%
\chapter{Rotações e Momento Angular}\label{Chap:Rotacoes}
%%%%%%%%%%%%%%%%%%%%%%%%%%%%%%%%%%%%%%%%%%%%%%%%%%%%%%%%%%%%%%%%%%%%%%%%%%%%%%%%

%\minitoc

%\clearpage

%%%%%%%%%%%%%%%%%%%%
\section{Introdução}
%%%%%%%%%%%%%%%%%%%%

{\it
Intro ...
}

%%%%%%%%%%%%%%%%%%%%%%%%%%%%%%%
\section{Cinemática da Rotação}
%%%%%%%%%%%%%%%%%%%%%%%%%%%%%%%

\comment{TODO texto introdutório e falar sobre corpos rígidos}
Vamos analisar o movimento de rotação de um corpo rígido (isto é, um corpo em que a distância relativa entre cada uma das partículas que o constituem é constante o que faz com que ele não mude de forma) em torno de um eixo.

%%%%%%%%%%%%%%%%%%%%%%%%%%%%%%%%%%%%%%%%%%%%%%%% 
\subsection{Variáveis cinemáticas para rotações}
%%%%%%%%%%%%%%%%%%%%%%%%%%%%%%%%%%%%%%%%%%%%%%%% 

Necessitamos variáveis permitam descrever os movimentos de rotação. Podemos tomar uma reta fixa no objeto e que faz um ângulo de \np[\tcdegree]{90} em relação ao eixo de rotação. O ângulo $\theta$ entre tal reta e o exito $x$ pode ser usado para descrever a posição angular do objeto.

\comment{TODO figura de um disco em 3d que descreve um arco (marginfig)}

O ângulo $\theta$ pode ser descrito em qualquer unidade (graus ou revoluções, por exemplo), porém é comum se utilizar medidas em \emph{radianos}. Veremos adiante que algumas relações só serão válidas para ângulos utilizando essa unidade de medida. No SI, os ângulos devem ser descritos em radianos

\comment{TODO Figura mostrando a definição de ângulo em radianos (melhor uma inline no texto e que mostrasse um "close" da região interessante).}

Conhecendo a posição angular, podemos calcular o deslocamento angular de maneira bastante simples, bastando calcular a diferença entre duas posições quaisquer:
\begin{equation}
	\Delta\theta = \theta_2 - \theta_1.
\end{equation}

A partir do deslocamento angular, podemos definir uma velocidade angular média através de
\begin{equation}
	\mean{\omega} = \frac{\Delta\theta}{\Delta t},
\end{equation}
%
de onde podemos tomar o limite de $\Delta t$ tendendo a zero para definir a velocidade instantânea:
\begin{equation}
	\omega = \lim_{\Delta t \to 0} \frac{\Delta}{\Delta t} \equiv \frac{d\theta}{dt}.
\end{equation}
%
As unidades da velocidade angular serão as de ``ângulo por tempo'', dentro do SI, $\textrm{rad}/\textrm{s}$.

Conhecendo a velocidade angular, podemos definir a aceleração angular média através de
\begin{equation}
	\mean{\alpha} = \frac{\Delta \omega}{\Delta t},
\end{equation}
%
o que nos leva à definição de aceleração angular instantânea através de
\begin{equation}
	\alpha = \lim_{\Delta t \to 0} \frac{\Delta\omega}{\Delta t} \equiv \frac{d\omega}{dt}.
\end{equation}
%
A aceleração angular tem unidade de ``ângulo por tempo ao quadrado'', no SI, $\textrm{rad}/\textrm{s}^2$.

%%%%%%%%%%%%%%%%%%%
\subsection{Sinais}
%%%%%%%%%%%%%%%%%%%

Discutir convenção de sinais para $\theta$ crescente.

%%%%%%%%%%%%%%%%%%%%%%%%%%%%%%%%%%%%%%%%%%%%%%%%%%%%%%%
\subsection{Equações para aceleração angular constante}
%%%%%%%%%%%%%%%%%%%%%%%%%%%%%%%%%%%%%%%%%%%%%%%%%%%%%%%

Ao estudar movimentos de translação, nos preocupamos com o caso da aceleração constante pois pretendíamos estudar um caso importante que pode ser descrito desta maneira: a aceleração gravitacional. No caso das rotações, supor que a aceleração seja constante não é algo muito geral ou mesmo de especial interesse para tratar sistemas físicos reais. No entanto, é interessante mostrar que as equações têm a mesma forma que no caso da translação.

Da própria definição da aceleração angular instantânea, temos
\begin{equation}
	d\omega = \alpha dt,
\end{equation}
%
que pode ser integrada entre valores iniciais e finais de velocidade angular e de tempo, obtendo
\begin{equation}
	\int_{\omega_i}^{\omega_f} d\omega = \int_{t_i}^{t_f} \alpha dt.
\end{equation}

Se a aceleração angular é constante, podemos retirá-la da integral:
\begin{equation}
		\int_{\omega_i}^{\omega_f} d\omega = \int_{t_i}^{t_f} \alpha dt.
\end{equation}
%
As integrais que restam correspondem a $\omega_f - \omega_i$ e $t_f - t_i$, o que nos permite escrever
\begin{equation}
	\omega_f = \omega_i + \alpha\Delta t.
\end{equation}
%
Adotanto $t_f = t$ e $t_i = 0$, temos 
\begin{equation}\label{Eq:VelAngParaAcelConst}
	\omega_f = \omega_i + \alpha t.
\end{equation}
%
Podemos perceber que no caso de uma rotação com aceleração angular constante, obtivemos uma equação para a velocidade angular que é análoga aquela para o caso da translação.

Voltando à definição de velocidade, 
\begin{equation}
	d\theta = \omega dt,
\end{equation}
%
e utilizando a Equação~\ref{Eq:VelAngParaAcelConst} acima, podemos escrever,
\begin{equation}
	d\theta = (\omega_0 + \alpha t) dt.
\end{equation}
%
Integrando entre $\theta_i$ e $\theta_f$ do lado esquerdo e entre $t_i$ e $t_f$ do lado direito, obtemos
\begin{align}
	\Delta \theta &= \int_{t_i}^{t_f} \omega_0 + \alpha t dt \\
	&= \omega_0 |_{t_i}^{t_f} + \frac{\alpha t^2}{2}.
\end{align}
%
Se tomarmos $t_i = 0$ e $t_f = t$, obtemos finalmente
\begin{equation}
	\theta = \theta_0 + \omega_0 t +\frac{\alpha t^2}{2}.
\end{equation}
%
Novamente temos uma equação que é anóloga àquela do caso translacional.
	
%%%%%%%%%%%%%%%%%%%%%%%%%%%%%%%%%%%%%%%%%%%%%%%%%
\newthought{Analogia com o caso translacional:} Para cada relação da cinemática translacional, temos uma correspondente para o caso rotacional. Na seção acima, utilizamos cálculo para determinar duas dessas equações. Esse método é, na verdade, equivalente ao cálculo de áreas feito para o caso translacional. A partir dessas equações, podemos determinar outras, como fizemos no caso da translação. Na Tabela~\ref{Tab:CompEqsTransRot} podemos ver as equações lado a lado, evidenciando quais equações têm a mesma forma.

\begin{table}[!h]
\centering
\caption{Comparação entre as equações para aceleração constante nos casos da cinemática da translação e da rotação.\label{Tab:CompEqsTransRot}}
\begin{tabular}{ll}
\toprule
Translação & Rotação \\
\midrule
$v = v_0 + at$ & $\omega = \omega_0 + \alpha t$ \\
$x = x_0 + v_0 t +\frac{at^2}{2}$ & $\theta = \theta_0 t + \frac{\alpha t^2}{2}$ \\
$v^2 = v_0^2 + 2 a \Delta x$ & $\omega^2 = \omega_0^2 + 2\alpha \Delta\theta$ \\
$\Delta x = \frac{v_0 + v}{2} t$ & $\Delta\theta = \frac{\omega_0 + \omega}{2} t$ \\
$x = x_0 + vt - \frac{at^2}{2}$ & $\theta = \theta_0 + \omega t = \frac{\alpha t^2}{2}$ \\
\bottomrule
\end{tabular}
\end{table}

%%%%%%%%%%%%%%%%%%%%%%%%%%%%%%%%%%%%%%%%%%%%%%%%%%%%%%%%%%%%%%%
\subsection{Relação entre variáveis de translação e de rotação}
%%%%%%%%%%%%%%%%%%%%%%%%%%%%%%%%%%%%%%%%%%%%%%%%%%%%%%%%%%%%%%%

% marginfig da situação
Um passageiro de um carrossel descreve um arco de comprimento $s$ enquanto gira em torno do eixo de rotação. Em uma volta, temos que $s = 2\pi r$, onde $r$ é a distância entre o passageiro e o eixo de rotação. Se necessitamos calcular o comprimento do arco para menos que uma volta, podemos utilizar o ângulo $\theta$ e o raio $r$, pois, da definição do ângulo em radianos temos
\begin{equation}
	\theta = s/r,
\end{equation}
%
de onde temos
\begin{equation}
	s = \theta r.
\end{equation}

A partir dessa equação simples, podemos encontrar a relação entre a velocidade do passageiro e a velocidade angular do carrossel fazendo uma derivada em relação ao tempo:
\begin{align}
	v &= \frac{ds}{dt} \\
	&= \frac{d(\theta r)}{dt} \\
	&= r\frac{d\theta}{dt} \\
	&= r\omega,
\end{align}
%
onde assumimos que $r$ seja constante para o retirar da derivada.

Como o passageiro descreve um círculo, sabemos que ele deve ter uma aceleração centrípeta, mesmo que sua velocidade seja constante. Temos que tal aceleração é dada por
\begin{equation}
	a_c = \frac{v^2}{r}
\end{equation}
%
e substituindo a relação entre $v$ e $\omega$ que acabamos de obter, encontramos
\begin{equation}
	a_c = \omega^2 r.
\end{equation}

Se, no entanto, tivermos uma variação da velocidade angular, temos uma aceleração angular. Derivando a equação $v = \omega r$ em relação ao tempo, temos
\begin{align}
	\frac{dv}{dt} &= \frac{d(\omega r)}{dt} \\
	&=r \frac{d\omega}{dt},
\end{align}
%
onde assumimos novamente que $r$ seja constante. Sabemos que $a=dv/dt$ e que $\alpha = d\omega/dt$, então
\begin{equation}
	a = \alpha r.
\end{equation}
%
A aceleração acima é responsável por variar o módulo da velocidade somente, já que se $\omega$ for constante, $d\omega/dt = 0$ e -- consequentemente -- $a=0$. Já haviamos identificado o efeito distinto das componentes radial e tangencial da aceleração ao estudarmos o movimento circular: a primeira é responsável pela variação da direção da velocidade, enquanto a segunda é responsável pela variação do módulo da velocidade. Concluímos que na equação acima estamos tratando da segunda:
\begin{equation}
	a_t = \alpha r.
\end{equation}

%%%%%%%%%%%%%%%%%%%%%%%%%%%%%
%%%%%%%%%%%%%%%%%%%%%%%%%%%%%
\section{Dinâmica da rotação}
%%%%%%%%%%%%%%%%%%%%%%%%%%%%%
%%%%%%%%%%%%%%%%%%%%%%%%%%%%%

%%%%%%%%%%%%%%%%%%%
\subsection{Torque}
%%%%%%%%%%%%%%%%%%%

Experimentalmente podemos verificar que para abrir uma porta é muito mais fácil empurrar a extremidade mais distante das dobradiças do que o meio da prota, ou próximo das dobradiças. Além disso o ângulo de aplicação da força trambém é relevante. Se o ângulo entre a força eo plano da prota for de \grau{90}, é mais fácil empurrar a porta do que se ele for de \grau{30}.

% margin figure

Na figura, podemos ver que a componente da força ao longo do eixo pontilhado não tem efeito de fazer a porta girar, sometne a componente perpendicular tem. Definimos então uma gradeza denominada torque, dada por
\begin{equation}\label{Eq:DefModTorque}
	\tau = r F \sen\phi.
\end{equation}

\comment{TODO Discutir o braço de alavanca aqui}

%%%%%%%%%%%%%%%%%%%%%%%%%%%%%%%%%%%%%%%%%%%%%%%%%%%
\subsection{Segunda lei de Newton para as rotações}
%%%%%%%%%%%%%%%%%%%%%%%%%%%%%%%%%%%%%%%%%%%%%%%%%%%

Se aplicarmos uma força sobre uma porta em que as dobradiças têm pouco atrito e, após um breve momento, cessarmos a aplicação da força, perceberemos que a porta continua girando em torno das dobradiças. Da mesma forma que para o caso de uj corpo que se desloca ao longo de um plano, concluímos que a o troque não é responsável pela velocidade angular, assim como no corpo que se desloca no plna, a força é responsável pela aceleração. Logo, concluímos que
\begin{equation}
	\alpha \propto \tau.
\end{equation}

Se tentarmos abrir uma porta interna de uma casa, ou uma porta externa, percebemos que a externa exige mais força para obter uma mesma aceleração angular, pois ela é maciça, e -- portanto -- mais massiva. Se analisarmos somente uma partícula que compõe a porta, temos
\begin{equation}
	\tau_i = F_t r_i,
\end{equation}
%
mas
\begin{equation}
	F_{t_i} = m_i a_{t_i}.
\end{equation}
%
Logo,
\begin{equation}
	\tau_i = m_i a_{t_i} r_i.
\end{equation}
%
Temos também que $a_{t_i} = \alpha r_i$, o que resulta em
\begin{equation}
	\tau_i = m_ir_i^2\alpha.
\end{equation}
%
Se somarmos para todas as partículas que compõe a porta, obtemos
\begin{equation}\label{Eq:DefMomInercia}
	\tau = \left[\sum_{i = 1}^N m_i r_i^2\right] \alpha.
\end{equation}
%
Já vimos, no entanto, que a soma na equação acima define o que chamamos de momento de inércia. Substituindo essa definição, temos
\begin{equation}
	\tau = I\alpha.
\end{equation}

Verificamos que quanto maior o valor de $I$, menor a aceleração a que a porta estará sujeita, quando atua sobre ela um determinado valor de torque. A equação acima relaciona então a aceleração angular a que um objeto estará sujeito quando sobre ele atua uma força, dando origem a um torque. Concluímos, portanto, que a expressão acima é análoga a segunda lei de Newton, porém descrevendo o caso da rotação.

%%%%%%%%%%%%%%%%%%%%%%%%%%%%%%%%%%%%%%%%%%%%%%%%%%
\section{Trabalho e energia cinética para rotações}
%%%%%%%%%%%%%%%%%%%%%%%%%%%%%%%%%%%%%%%%%%%%%%%%%%

%%%%%%%%%%%%%%%%%%%%%%%%%%%%%%%%%%%%%%%%
\subsection{Energia cinética de rotação}
%%%%%%%%%%%%%%%%%%%%%%%%%%%%%%%%%%%%%%%%

A energia cinética de um conjunto de partículas pode ser escrita como
\begin{equation}
	K = \sum_{i=1}^N \frac{1}{2} m_i v_i^2,
\end{equation}
%
onde $N$ é o número de partículas. Se, no caso de um corpo rígido, somarmos a energia cinética das partículas ao descreverem círculos ao redor do eixo de rotação, temos que a velocidade de cada uma delas é $v = \omega r$ -- onde $\omega$ é a velocidade angular do corpo rígido e $r$ é a distância de cada uma das partículas ao eixo de rotação --. Logo,
\begin{align}
	K &= \sum_{i=1}^{N} \frac{1}{2} m_i (\omega r_i)^2 \\
	&= \sum_{i=1}^N \frac{1}{2} m_i r_i^2 \omega^2 \\
	&= \frac{1}{2} \left[\sum_{i=1}^N m_i r_i^2\right] \omega^2,
\end{align}
%
onde $\omega$ pôde sair do somatório pois a velocidade angular é a mesma para todas as partículas que compõe um corpo rígido. Notamos que o termo entre colchetes é uma constante que só depende das características do corpo rígido. Além da dependência na massa das partículas, também temos uma depenência na posição que elas ocupam em relação ao eixo de rotação. Definimos então uma grandeza denominada \emph{momento de inércia}, representada por $I$:
\begin{equation}\label{Eq:DefMomentoDeInercia}
	I = \sum_{i=1}^N m_i r_i^2.
\end{equation}

Utilizando essa definição, temos que a energia cinética é dada por
\begin{equation}
	K = \frac{1}{2} I \omega^2.
\end{equation}
%
Temos mais uma vez uma equação que segue uma analogia com o caso da translação. O momento de inércia I cumpre um papel similar ao da massa nessa situação -- e em muitas outras, como veremos adiante --, sendo também conhecido como \emph{inércia rotacional}, da mesma forma que a massa pode ser denominada como \emph{inércia translacional}.

\comment{TODO discutir que o momento de inércia depende de onde a gente escolhe o eixo, ou pelo menos fazer isso na seção de cálculo do momento de inércia}

%%%%%%%%%%%%%%%%%%%%%%%%%%%%%%%%%%%%%%%%%%%%%%%%%%%
\subsection{Teorema trabalho-energia para rotações}
%%%%%%%%%%%%%%%%%%%%%%%%%%%%%%%%%%%%%%%%%%%%%%%%%%%

explicar

%%%%%%%%%%%%%%%%%%%%%
\subsection{Potência}
%%%%%%%%%%%%%%%%%%%%%

explicar

%%%%%%%%%%%%%%%%%%%%%%%%%%%%%%%%%%%%%%%
\section{Cálculo do momento de inércia}
%%%%%%%%%%%%%%%%%%%%%%%%%%%%%%%%%%%%%%%

%%%%%%%%%%%%%%%%%%%%%%%%%%%%%%%%%%%%%%%%%%%%%%%%%%%%%%%%%%%
\subsection{Momento de inércia de um sistema de partículas}
%%%%%%%%%%%%%%%%%%%%%%%%%%%%%%%%%%%%%%%%%%%%%%%%%%%%%%%%%%%

\begin{equation}
  I = \sum m_i r_i^2.
\end{equation}

%%%%%%%%%%%%%%%%%%%%%%%%%%%%%%%%%%%%%%%%%%%%%%%
\subsection{Propriedades do momento de inércia}
%%%%%%%%%%%%%%%%%%%%%%%%%%%%%%%%%%%%%%%%%%%%%%%

Sei lá, deixar claro que muda conforme o eixo e que pode ser somado.


%%%%%%%%%%%%%%%%%%%%%%%%%%%
\subsection{Discretização?}
%%%%%%%%%%%%%%%%%%%%%%%%%%%

Fazer para a barra fina em analogia ao cálculo do CM.

%%%%%%%%%%%%%%%%%%%%%%%%%%%%%%%%%%%%%%%%%%%%%%%%%%%%%%%%%%%%
\subsection{Momento de inércia de uma distribuição contínua}
%%%%%%%%%%%%%%%%%%%%%%%%%%%%%%%%%%%%%%%%%%%%%%%%%%%%%%%%%%%%

Quando necessitamos relacionar a aceleração a que um objeto estava sujeito com a força resultante que nele atuava, é necessário se obter conhecimento acerca da massa do objeto. O processo de aferição da massa é relativamente simples: basta utilizarmos um dispositivo que compare a massa do objeto em questão com a de um corpo de referência, conforme discutimos no Capítulo ??.

Para o cálculo do momento de inércia, é necessário utilizar a definição dada pela Equação~\ref{Eq:DefMomentoDeInercia}. Entretanto, utilizá-la no caso de um corpo real não é algo factível: o número de partículas (átomos) que compõe um objeto é muito grande.\footnote{Por exemplo, \np[g]{12,0} de $^12\textrm{C}$ contém 1 mol de átomos de carbono, isto é, \np{6.02e23} átomos.} Podemos, entretanto, escrever a contribuição de uma pequena parte de um corpo para o momento de inércia como
\begin{equation}
	dI = r^2 dm
\end{equation}
\comment{Figura do lado pra mostrar isso}
onde $dm$ é a massa correspondente à pequena parte do corpo e $r$ é sua distância em relação ao eixo de rotação. Podemos então calcular o momento de inércia total do corpo fazendo uma integral sobre a distribuição de massa do objeto:
\begin{equation}\label{Eq:MomInerciaIntegral}
	I = \int r^2 dm.
\end{equation}
%
Em geral, realizar esse cálculo é bastante complicado e trabalhoso, porém para algumas formas geométricas simples -- considerando que a densidade dos objetos é uniforme, ou varia na posição de uma forma conhecida --, podemos realizar uma substituição de variáveis na equação acima e integrar nas coordenadas espaciais. Alguns casos permitem que a integração seja feita até mesmo em uma variável somente. Nas Seções ?? a ?? tratamos alguns exemplos utilizando esse método.

As expressões resultantes para o momento de inércia dos corpos serão diferentes para cada tipo de corpo, porém são razoavelmente simples. A Tabela~?? mosta os casos mais comuns.

%%%%%%%%%%%%%%%%%%%%%%%%%%%%%%%%%%%%%%%%
\subsection{Teorema dos eixos paralelos}
%%%%%%%%%%%%%%%%%%%%%%%%%%%%%%%%%%%%%%%%

Caso conheçamos o momento de inércia em torno de um eixo que passa pelo centro de massa, podemos calcular o momento de inércia em torno de qualquer eixo que seja paralelo ao primeiro.

\comment{TODO Figura ao lado mostrando o esquema pra deduzir o teorema dos eixos paralelos}

Na Figura~??, temos dois eixos que saem do plano da página, passando por um objeto de massa $M$ nos pontos $O$ e $P$. O ponto $O$ denota a origem do sistema de coordenadas $xy$ e reside exatamente no centro de massa do objeto. Estamos interessados em calcular o momento de inércia em torno do eixo que passa por $P$ e vamos considerar que o momento de inércia em torno do eixo que passa pelo centro de massa seja conhecido. Utilizando a Equação~\ref{Eq:MomInerciaIntegral}, considerando que $r = (x-a)^2 + (y-b)^2$, como pode ser visto da Figura e utilizando o teorema de Pitágoras, temos que
\begin{equation}
	I_p = \int (x-a)^2 + (y-b)^2 dm.
\end{equation}
%
Desenvolvendo os quadrados e reagrupando os termos, podemos escrever
\begin{equation}
	I_p = \int (x^2 + y^2) dm - 2a \int x dm - 2b \int y dm + \int (a^2 + b^2) dm.
\end{equation}
%
A segunda e a terceira integrais acima são justamente as expressões para o cálculo da posição do centro de massa nos eixos $x$ e $y$, respectivamente. Devido à escolha da posição da origem do sistema de coordenadas, temos que $x_{\textrm{CM}} = y_{\textrm{CM}} = 0$. Além disso, podemos ver da figura que $x^2 + y^2 = r'^2$ e que $a^2 + b^2 = h^2$. Logo
\begin{equation}
	I_p = \int r'^2 dm + \int h^2 dm.
\end{equation}
%
A primeira integral nada mais é do que o cálculo do momento de inércia em torno da origem, ou seja, em torno do eixo que passa pelo centro de massa. Já na segunda integral, $h$ é a distância entre os eixos que passam por $O$ e por $P$ que é constante. Retirando-a da integral, obtemos a seguinte expressão, conhecida como \textbf{Teorema do Eixos Paralelos}
\begin{equation}\label{Eq:TeoremaEixosParalelos}
	I_p = I_{\textrm{CM}} + h^2 M,
\end{equation}
%
onde usamos o fato de que
\begin{equation}
	\int dm = M.
\end{equation}

Concluímos, portanto, que conhecendo o momento de inércia de um objeto em torno de um eixo qualquer que passa pelo centro de massa, podemos calcular o momento de inércia em torno de qualquer eixo $P$ paralelo ao primeiro, bastando conhecer a distância $h$ entre os dois eixos e a massa do objeto. É importante destacar que o eixo $P$ não precisa necessariamente atravessar o corpo em algum ponto, podendo passar fora dele, como no caso de uma esfera que gira em torno de um eixo estando ligada por um fio fino. 

Muitas vezes o segundo termo na Equação~\eqref{Eq:TeoremaEixosParalelos} é muito maior que $I_{\textrm{CM}}$, devido a um grande valor de $h$. Nesses casos, podemos desprezar o primeiro termo, restando somente
\begin{equation}
  I_P = h^2 M,
\end{equation}
%
que corresponde ao caso de utilizarmos a Equação~\eqref{Eq:DefMomInercia} para uma partícula girando em torno do eixo $P$, com toda a massa concentrada no centro de massa. Esta análise é o que nos permite tratar uma esfera de raio $r_e$ girando a uma distância $d$ em torno de um eixo como se fosse uma partículo, se $d \gg r_e$.

\comment{TODO Exemplo momento de inércia barra fina por integração. TODO aditividade do momento de inércia}

%%%%%%%%%%%%%%%%%%%%%%%%%%%%%%%%%%%%%%%%%%%%%%%%%%%%%%%%%%%%%%%%%%
\section{Rolamento}
%%%%%%%%%%%%%%%%%%%%%%%%%%%%%%%%%%%%%%%%%%%%%%%%%%%%%%%%%%%%%%%%%%

\comment{TODO substituir o ``valor'' da velocidade do centro de massa por algo diferente de $v_CM$, deixar esta notação para a ''variável'' (como por exemplo $v_s$, $v_i$) ou usar $u$, sei lá}

O rolamento é um exemplo de movimento que combina uma \emph{rotação} com uma \emph{translação}. Se analisarmos o movimento de cada ponto de um objeto que rola, teremos uma situação bastante complicada: enquanto o centro de massa descreve uma treta, os demais pontos descrevem \emph{cicloides}. Podemos, no entanto, dividir esse movimento em duas componentes simples: um movimento de translação do centro de massa e um movimento de rotação pura no referencial preso ao centro de massa.
Vamos analisar aqui somente rolamentos que ocorrem se deslisar. 

%%%%%%%%%%%%%%%%%%%%%%%%%%%%%%%%%%%%%%%%%
\subsection{Características do rolamento}
%%%%%%%%%%%%%%%%%%%%%%%%%%%%%%%%%%%%%%%%%

Ao analisarmos o movimento no referencial do solo, verificamos que o fato de não haver deslizamento implica que no ponto de contato entre corpo que rola e a superfície, a velocidade é nula. Já no topo, a velocidade é maior que a do centro de massa, pois durante o rolamento ocorre um deslocamento para frente (em relação ao centro de massa) dessa parte superior. Por outro lado, no referencial do centro de massa, verificamos que as partes superior e inferior se deslocam com a mesma velocidade, porém em sentidos opostos. Podemos relacionar a velocidade angular da roda no referencial do centro de massa à velocidade dos pontos na borda da roda através de
\begin{equation}
  v_b = \omega R.
\end{equation}
%
Ainda no referencial do centro de massa, observamos que o solo se move para trás com uma velocidade igual em módulo e direção, porém com sentido contrário, à velocidade do centro de massa em relação ao solo, cujo módulo é $v_\textrm{CM}$. Se não ocorre deslizamento entre a roda e o ponto de contato, concluímos que
\begin{equation}
  v_{\textrm{CM}} = v_b.
\end{equation}
%
Deste resultado, concluímos que a parte superior da roda (no referencial do centro de massa) se desloca para frente com velocidade $v_s = v_{\textrm{CM}}$, enquanto a inferior se desloca com velocidade $v_i = -v_{\textrm{CM}}$. Neste referencial, a velocidade do centro de massa é, por definição, nula.

Para encontrarmos a velocidade no referencial do solo, basta utilizarmos $\vec{v}_S = \vec{v}_{S'} + \vec{v}_{SS'}$, sabendo que $\vec{v}_{SS'} = v_{\textrm{CM}}$, pois é a velocidade com que o referencial $S'$ se desloca em relação ao referencial do solo. Concluímos então que no referencial do solo
\begin{align}
  v_i &= 0 \\
  v_{\textrm{CM}} &= v_{\textrm{CM}} \\
  v_s &= 2 v_{\textrm{CM}}.
\end{align}
%
Esse resultado pode ser obtido de maneira intuitiva ao se ``somar'' um movimento de rotação pura a um movimento de translação pura, como na Figura ???. \comment{Figura mostrando isso aqui.}

%%%%%%%%%%%%%%%%%%%%%%%%%%%%%%%%%%%%%%%%%%%%%%%%%%%%%%%%%%%%%%%%%%%%%%%%%%%%%%%%%%%
\subsection{Forças no rolamento}
%%%%%%%%%%%%%%%%%%%%%%%%%%%%%%%%%%%%%%%%%%%%%%%%%%%%%%%%%%%%%%%%%%%%%%%%%%%%%%%%%%%

Se ignorarmos a força de arrasto oferecida pelo ar, quando um corpo rola sem deslisar e com velocidade constante, ele está sujeito a uma força resultante nula. No eixo vertical, isso implica que $N = P$. Já no eixo horizontal, concluímos que $f_{at} = 0$, pois não há nenhuma outra força que eventualmente possa equilibrá-la. Se o objeto que rola for uma roda de bicicleta, por exemplo, ao frearmos deve agir sobre algum ponto da roda uma força dirigida para trás, que será responsável por desacelerar o centro de massa do sistema. Claramente essa força será uma força de atrito que atuará no ponto de contato da roda com o solo. Se, por outro lado, o ciclista resolver acelerar a bicicleta, deverá aparecer uma força de atrito no ponto de contato da roda com o solo, dirigida para frente (lembre-se que a força que o ciclista exerce é uma força interna, portanto não pode acelerar o sistema). Concluímos então que \emph{no caso de um rolamento, as forças que aceleram ou desaceleram o centro de massa são as forças de atrito no ponto de contato com o solo.}

Do ponto de vista da rotação, as forças de atrito -- sendo as únicas forças externas que não estão equilibradas e que não atuam em direção ao eixo de rotação -- devem ser responsáveis pela aceleração angular dos corpos que rolam. Além disso, as forças de atrito atuam de forma que o ângulo entre a força e o raio que liga o eixo de rotação ao ponto de aplicação da força é de \grau{90}.

\paragraph{Rolamento em uma Rampa}

Um problema que pode ser tratado a partir das observações acima e que trás um resultado bastante interessante é o de um objeto que rola rampa abaixo. Analizando o corpo do ponto de vista da translação (como se ele fosse um bloco e não pudesse girar), podemos escrever para o eixo $x$ \comment{rever sinais depois de fazer a figura}
\begin{equation}
  f_{at} - Mg \sen\theta = M a_{\textrm{CM}}.
\end{equation}
%
Para o eixo $y$ temos
\begin{equation}
  N_y - P = 0,
\end{equation}
%
mas essa equação não será particularmente útil, pois \textbf{não podemos assumir que $f_{at} = \mu N$}, pois nada garante que o corpo esteja na iminência de deslizar.

Analisando a rotação do sistema, temos
\begin{align}
  \tau &= I\alpha \\
  f_{at} R &=I\alpha,
\end{align}
%
onde usamos que $\tau = R f_{at} \sen\grau{90}$. Como para o rolamento $v_{\textrm{CM}} = \omega R$, temos que 
\begin{equation}
  a_{\textrm{CM}} = \alpha R.
\end{equation}
%
Devido à escolha do sistema de coordenadas, uma aceleração positiva no eixo $x$ implica em uma \emph{aceleração angular negativa} segundo a convenção de que acelerações angulares no sentido antihorário são positivas. Logo, devemos acertar essa diferença de sinais adicionando um sinal negativo:
\begin{equation}
  a_{\textrm{CM}} = -\alpha R.
\end{equation}

Podemos então montar um sistema de equações dado por
\begin{equation}
\left\{ \begin{aligned} f_{at} - Mg\sen\theta = M a_{\textrm{CM}} \\ f_{at}R = I\alpha \\ a_{\textrm{CM}} = -\alpha R \end{aligned}\right.
\end{equation}
%
Resolvendo para a aceleração, obtemos
\begin{equation}
  a_{\textrm{CM}} = -\frac{\sen\theta}{1+I/(MR^2)} g.
\end{equation}
%
Esse resultado é interessante pois ele mostra que a aceleração a que um corpo será submetido ao descer uma rampa executando um rolamento sem deslizar é diferente para cada objeto, dependendo do momento de inércia do objeto. Considerando que o momento de inércia para objetos com seção reta circular pode ser escrito como $I = f MR^2$, onde $f$ é um número menor que 1, temos
\begin{equation}
  a_{\textrm{CM}} = -\frac{\sen\theta}{1+f} g.
\end{equation}
%
Isto é, cada típo de objeto tem uma aceleração diferente, sendo tanto menor quanto maior for o valor de $f$. Se, por exemplo, tomarmos três objetos com formas distintas -- um aro, um cilindro e uma esfera maciça, por exemplo -- e os soltarmos a partir do topo de uma rampa, eles levarão tempos diferentes para percorrer a distância até a base. Considerando os três objetos tomados como exemplo, temos que a ordem de chegada será: esfera, cilindro e aro, devido aos valores de $f$ para esses três objetos: $f_a = 1$, $f_c = \nicefrac{1}{2}$ e $f_e = \nicefrac{2}{5}$. Veja ainda que a aceleração não depende da massa ou do raio do objeto em questão, mas sim do fator $f$ associado à \emph{forma} do objeto.

%%%%%%%%%%%%%%%%%%%%%%%%%%%%%%%%%%%%%%%%%%
\subsection{Energia cinética no rolamento}
%%%%%%%%%%%%%%%%%%%%%%%%%%%%%%%%%%%%%%%%%%

Para um corpo que rola, podemos calcular a energia cinética em torno do ponto de contato com o solo através de
\begin{equation}
  K = \frac{1}{2} I_P \omega_P^2.
\end{equation}
%
A velocidade angular $\omega_P$ em torno de $P$ tem o mesmo valor que a velocidade angular $\omega$ em torno do centro de massa. Podemos perceber isso através de
\begin{equation}
  \omega = \frac{v_b}{R} = \frac{v_{\textrm{CM}}}{R}
\end{equation}
%
e analizando a velocidade angular do ponto superior do corpo e também do centro de massa:
\begin{align}
  \omega_P &= \frac{v_s}{2R} \\
  &= \frac{2v_{\textrm{CM}}}{2R} = \frac{v_{\textrm{CM}}}{R} \\
  \omega_P &= \frac{v_{\textrm{CM}}}{R} \\
  &= \frac{v_{\textrm{CM}}}{R}. \mathnote{aqui substituir pelo valor da vel do cm, não deixar a variável}
\end{align}
%
Portanto, para calcular a energia cinética em torno de $P$, basta calcularmos o momento de inércia em torno do eixo que passa por esse ponto.

Utilizando o teorema dos eixos paralelos, temos $I_P = I_{\textrm{CM}} + h^2M$, de onde obtemos
\begin{align}
  K &= \frac{1}{2} (I_{\textrm{CM}} + h^2M) \omega^2 \\
  &= \frac{1}{2} I_{\textrm{CM}} + \frac{1}{2} M R^2 \omega^2.
\end{align}
%
onde usamos $h = R$. Utilizando ainda $v_{\textrm{CM}} = \omega R$ \mathnote{isso tá deduzido acima?}
temos
\begin{equation}
  K = \frac{1}{2} I_{\textrm{CM}} \omega^2 + \frac{1}{2} m v_{\textrm{CM}}^2.
\end{equation}
%
Na equação acima temos claramente um termo que corresponde à energia cinética de rotação do corpo em torno do centro de massa -- o primeiro -- e um termo que corresponde à energia cinética de translação do objeto -- o segundo--. Concluímos, portanto, que a energia cinética é simplesmente aditiva em suas parcelas translacional e rotacional.

%%%%%%%%%%%%%%%%%%%%%%%%%%%%%%%%%%%%%%%%%%%%%%%%%%%%%%%%%%%%
\section{Caráter vetorial das variáveis da rotação}
%%%%%%%%%%%%%%%%%%%%%%%%%%%%%%%%%%%%%%%%%%%%%%%%%%%%%%%%%%%%

%%%%%%%%%%%%%%%%%%%%%%%%%%%%%%%%%%%%
\subsection{Velocidade e aceleração}
%%%%%%%%%%%%%%%%%%%%%%%%%%%%%%%%%%%%

Quando discutimos as grandezas da translação, concluímos que posição, velocidade e aceleração eram grandezas vetoriais e tinham módulo, direção e sentido. Podemos atribuir um caráter vetorial à velocidade angular e à aceleração angular. Nesses casos, no entanto, a direção do vetor não nos dá a direção do movimento, mas a direção \emph{em torno} da qual o objeto gira.

Para definirmos tal direção de maneira única, utilizamos a regra da mão direita: ``seguramos'' o eixo em torno do qual o objeto gira de forma que os dedos (exceto o polegar) apontem no sentido de rotação. Fazendo isso, o polegar apontará na direção do vetor.

No caso da aceleração, apontamos a direção da variação da velocidade (na direção de $\vec{\omega}$ se o módulo da velocidade angular cresce e na direção contrária se o módulo decresce). Tanto $\vec{\omega}$ quanto $\vec{\alpha}$ obedecem a todos os requisitos para serem denominados vetores, inclusive à soma vetorial.

A posição e -- consequentemente -- o deslocamento angulares, no entanto, não podem ser tratados como vetores. Se tomarmos um livro e realizarmos dois deslocamentos angulares sucessivos de \grau{90} em torno dos eixos $x$ e $y$, a ordem em que eles forem realizados influenciará no resultado final, resultando em estados finais diferentes. Como a soma vetorial de $\vec{a} + \vec{b} = \vec{b} + \vec{a}$, percebemos que os deslocamentos angulares não podem ser tratados como vetores.

\comment{Pra mim isso não explica nada. Como podemos mostrar que duas velocidades angulares podem ser somadas? (Acho que o Teorema de Euler para rotação explica, mas como? ver isso e colocar essa explicação aqui). TODO Por que podemos tratar deslocamentos para pequenos ângulos como vetores e não para grandes ângulos? ver isso com cuidado}

%%%%%%%%%%%%%%%%%%%
\subsection{Torque como o produto vetorial $\vec{r}\times\vec{F}$}
%%%%%%%%%%%%%%%%%%%

Da mesma forma que $\vec{\omega}$ e $\vec{\alpha}$ são grandezas vetoriais, também é possível mostrar que o torque é uma grandeza vetorial. Analisando a expressão para o módulo do produto vetorial entre dois vetores $\vec{a}$ e $\vec{b}$:
\begin{equation}
  |\vec{a}\times\vec{b}| = ab\sen\phi,
\end{equation}
%
onde $\phi$ é o ângulo entre os dois vetores, e comparando-a com a Equação~\eqref{Eq:DefModTorque}, podemos escrever o torque como
\begin{align}
  |\vec{\tau}| &= F d \sen\phi \\
  &= |\vec{F}\times\vec{d}.
\end{align}
%
Na expressão acima, $\vec{F}$ é o vetor que descreve a força que gera o torque, enquanto $\vec{d}$ é o vetor que denota a posição do ponto onde a força é aplicada. A origem do vetor é o próprio ponto em torno do qual o objeto gira. Apesar de utilizarmos $\vec{d}$ até o momento, em geral posições são denotadas por $\vec{r}$. Assim, o torque pode ser definido como: \comment{a origem pode ser um ponto qualquer, porém quando formos tratar de corpos rígidos mais adiante, usaremos sempre um ponto no eixo de rotação ... como explicar isso direito?}
\begin{equation}\label{Eq:DefTorque}
  \vec{\tau} = \vec{r}\times\vec{F}.
\end{equation}

A direção do vetor torque pode ser facilmente compreendida ao se analisar a expressão para a Segunda Lei de Newton para a rotação,
\begin{equation}
  \tau = I \alpha.
\end{equation}
%
Sabendo que $I$ é uma grandeza escalar e que atribuimos um caráter vetorial para a aceleração angular, obrigatoriamente temos que o torque também tem uma caráter vetorial (pois uma das propriedades dos vetores é que a multiplicação de um escalar por um vetor resulta em um vetor). Assim, o torque assume a mesma direção que a aceleração angular:
\begin{equation}
  \vec{\tau} = I\vec{\alpha}.
\end{equation}
%
Dessa conclusão podemos tirar uma observação importante, justificando a escolha da ordem dos vetores na Equação~\eqref{Eq:DefTorque}: devido à regra da mão direita, se temos um eixo em torno do qual um objeto gira e a aceleração angular é positiva, o torque dado pelo produto vetorial~\eqref{Eq:DefTorque} acima deve correspondentemente ser positivo. Para isso, também devemos adotar a regra da mão direita para o produto vetorial. Analisando o diagrama ao lado, percebemos que a ordem do produto vetorial deve ser $\vec{r}\times\vec{F}$, caso contrário o sentido resultante para o torque seria oposto ao sentido da aceleração. Alternativamente podemos usar $\vec{\tau} = -\vec{F}\times\vec{r}$, já que $\vec{a}\times\vec{b} = - \vec{b}\times\vec{a}$.

%%%%%%%%%%%%%%%%%%%%%%%%%%%%%%%%%%%%%%%%%%%%%%%%%%%%%%
\section{Momento angular e Segunda Lei de Newton}
%%%%%%%%%%%%%%%%%%%%%%%%%%%%%%%%%%%%%%%%%%%%%%%%%%%%%%

Da mesma forma que temos o momento linear, para o caso das rotações temos o momento angular, definido como
\begin{equation}\label{Eq:DefMomAngular}
  \vec{\ell} = \vec{r}\times\vec{p},
\end{equation}
%
onde $r$ denota a posição de uma partícula qualquer e $p$ denota seu momento angular. Se derivarmos essa expressão em relação ao tempo, temos
\begin{align}
  \frac{d\vec{\ell}}{dt} &= \frac{d(\vec{r}\times\vec{p})}{dt} \\
  &= \frac{d\vec{r}}{dt}\times\vec{p} + \vec{r}\times\frac{d\vec{p}}{dt},
\end{align}
%
onde usamos a regra da cadeia. Notando que $d\vec{r}/dt = \vec{v}$, $\vec{p} = m \vec{v}$ e $d\vec{p}/dt = \vec{F}$, podemos escrever
\begin{equation}
  \frac{d\vec{\ell}}{dt} = m\vec{v}\times\vec{v} + \vec{r} \times \vec{F}.
\end{equation}
%
Finalmente, notando que o produto vetorial de dois vetores colineares é nulo, temos que $\vec{v}\times\vec{v} = 0$ e, portanto,
\begin{align}
  \frac{d\vec{\ell}}{dt} &= \vec{r} \times \vec{F} \\
  &= \vec{\tau}.
\end{align}

Mais uma vez obtivemos um resultado para o caso das rotações que tem um análogo no caso da translação: a equação acima mostra que a taxa de variação do momento angular no tempo é igual ao torque, o que é análogo à forma $\vec{F} = d\vec{p}/dt$ para a Segunda Lei de Newton. Portanto, temos uma nova forma para a Segunda Lei de Newton para Rotações:
\begin{equation}\label{Eq:SegLeiNewtonRotDLDT}
  \vec{\tau} = \frac{d\vec{\ell}}{dt}.
\end{equation}

\comment{$\tau$ e $\ell$ devem ser definidos em relaçao ao mesmo ponto}

%%%%%%%%%%%%%%%%%%%%%%%%%%%%%%%%%%%%%%%%%%%%%%%%%%%%%%%%%%%%%%%%%%%%%%%%%%%%
\subsection{Momento angular para uma partícula que se desloca em linha reta}
%%%%%%%%%%%%%%%%%%%%%%%%%%%%%%%%%%%%%%%%%%%%%%%%%%%%%%%%%%%%%%%%%%%%%%%%%%%%
\comment{mostrar que $r\sen\phi = d$ ($d$ é a distância mínima entre a reta em que a partícula se desloca e a origem)}

Assim como no caso do torque, o momento angular é calculado em relação a um ponto. Mesmo que a partícula se desloque em uma linha reta, sem executar uma rotação em torno de um ponto, podemos lhe atribuir um momento angular.

%%%%%%%%%%%%%%%%%%%%%%%%%%%%%%%%%%%%%%%%%%%%%%%%%%%%%%%%
\subsection{Momento angular de um sistema de partículas}
%%%%%%%%%%%%%%%%%%%%%%%%%%%%%%%%%%%%%%%%%%%%%%%%%%%%%%%%

O momento angular de um sistema de partículas pode ser calculado somando-se o momento angular das várias partículas que o constituem:
\begin{align}
  \vec{L} &= \vec{\ell}_1 + \vec{\ell}_2 + \vec{\ell}_3 + \dots + \vec{\ell}_N \\
  &= \sum_{i=1}^N \vec{\ell}_i.
\end{align}
%
Esta propriedade é característica dos vetores, e já a utilizamos para definir o momento linear do centro de massa $\vec{P}_{\textrm{CM}}$ como sendo a soma do momento linear das partículas que o constituem.

Se derivarmos a expressão acima em relação ao tempo, temos
\begin{align}
  \frac{d\vec{L}}{dt} &= \frac{d}{dt}\left(\sum_{i=1}^N\vec{\ell}_i\right) \\
  &= \sum_{i=1}^N \frac{d\vec{\ell}_i}{dt}.
\end{align}
%
De acordo com a Equação~\ref{Eq:SegLeiNewtonRotDLDT} para a Segunda Lei de Newton para Rotações, $d\vec{\ell}_i/dt = \vec{\tau}_i$, isto é, o torque que atua sobre a i-ésima partícula. No entanto, para um sistema de partículas que interagem através de forças, os torques devido a forças internas geram um par que se cancela na soma. Dessa forma, restarão somente os torques externos, logo
\begin{equation}\label{Eq:SegLeiNewtonRotSisPartDLDT}
  \vec{\tau}_R^{\textrm{Ext}} = \frac{d\vec{L}}{dt}.
\end{equation}

%%%%%%%%%%%%%%%%%%%%%%%%%%%%%%%%%%%%%%%%%%%%%%%%%%%%%%%%
\subsection{Momento angular de um corpo rígido}
%%%%%%%%%%%%%%%%%%%%%%%%%%%%%%%%%%%%%%%%%%%%%%%%%%%%%%%%
\comment{Figuras para explicar o cálculo do momento angular de um corpo rígido}

Se um corpo gira em torno de um eixo, podemos calcular seu momento angular dividindo-o em várias partes e tratando-o como um sistema de partículas. Na Figura ??? ao lado, o momento angular de uma das partículas que compõe o corpo é mostrado. Segundo a definição do momento angular para uma partícula, temos
\begin{align}
  \vec{\ell}_i &= \vec{r}\times\vec{p} \\
  &= r_i p_i \sen \grau{90} \\
  &= r_i m_i v_i,
\end{align}
%
onde as variáveis $r_i$, $m_i$ e $v_i$ se referem à posição, velocidade e massa da partícula em questão. Usamos ainda o índice $i$ pois calculamos o momento angular de somente uma partícula, porém vamos somar sobre as demais já que a expressão é a mesma para todas elas.

Se o corpo for simétrico e homogêneo, podemos perceber facilmente que para toda partícula $P$ existe uma partícula $P'$ diametralmente oposta à primeira e que tem os mesmos valores para as componentes do momento angular para os eixos $x$ e $y$, porém com sentidos contrários. Logo, ao realizarmos a soma sobre todas as partículas, concluímos que tais componentes resultarão em zero, restando somente a componente $z$ do momento angular. Esta componente pode ser calculada utilizando o ângulo $\theta$ e obtemos
\begin{align}
  \ell_{iz} &= \ell_i \sen\theta \\
  &= r_i m_i v_i \sen\theta.
\end{align}
%
A distância $r_{\perp}$ pode ser escrita como
\begin{equation}
  \ell_{\perp} = r_i\sen\theta.
\end{equation}
%
Consequentemente, 
\begin{equation}
  \ell_{iz} = m_i r_{\perp}^2 \omega,
\end{equation}
%
onde utilizamos $v_i = \omega r_{\perp}$ (lembre-se que $\omega$ é constante e igual para todas as partículas que compôe um corpo rígido). Dessa forma, podemos escrever o momento angular total do corpo como
\begin{equation}
  L_z = \left[\sum_{i=1}^N m_i r_{\perp}^2\right] \omega.
\end{equation}
%
O termo entre colchetes nada mais é que o momento de inércia do corpo. Além disso, sabemos que só resta a componente $z$ do momento angular, porém esta é a mesma direção da velocidade angular. Logo
\begin{equation}
  \vec{L}_z = I \vec{\omega}.
\end{equation}

É importante notar que se o corpo não for simétrico, restará uma componente do plano $xy$ que mudará constantemente de direção. No caso de um objeto assimétrico sofrer uma rotação, portanto, deve haver um torque que é realizado por um agente externo -- pois $\vec{\tau} = d\vec{L} / dt$ e se $\vec{L}$ não é constante, então $\vec{\tau} \neq 0$ --. Se o objeto em questão está prezo por mancais, por exemplo, tais suportes exercem força sobre o corpo que geram torques, possibilitanto que o momento angular varie. No entanto, existem as reações a essas forças, que são exercidas pelo corpo sobre os suportes, sendo responsáveis pelas \emph{vibrações} características de um corpo assimétrico submetido a rotações.

\comment{Aqui entram exemplos (em um ambiente próprio, paragraph)}
%%%%%%%%%%%%%%%%%%%%%%%%%%%%%%%%%%%%%%%%%%%%%%%%%%%%%%%%
\section{Conservação do momento angular}
%%%%%%%%%%%%%%%%%%%%%%%%%%%%%%%%%%%%%%%%%%%%%%%%%%%%%%%%

A partir da Equação~\ref{Eq:SegLeiNewtonRotSisPartDLDT}, percebemos que se $\vec{\tau}_R^{\textrm{Ext}} = 0$, temos que $d\vec{L}/dt = 0$, ou seja,
\begin{equation}
  \vec{L} = \textrm{constante}.
\end{equation}
%
Temos, portanto, uma nova lei de conservação -- a \emph{conservação do momento angular} --. Assim como nos casos da \emph{conservação da energia} e da \emph{conservação do momento linear}, o fato de termos uma lei de conservação envolvendo o momento angular nos permitirá analisar sistemas sem sabem em detalhes o que ocorre entre dois instantes quaisquer. Se um evento ocorre de forma que $\vec{\tau}_R^{\textrm{Ext}} = 0$, temos que o momento angular antes e depois de tal evento é o mesmo:
\begin{equation}
  L_i = L_f.
\end{equation}
%
Logo, se temos informações sobre o sistema antes do evento, podemos relacioná-las ao estado final do sistema sem saber detalhes do que ocorreu durante o evento. Isso será muito útil na análise de várias situações.

%%%%%%%%%%%%%%%%%%%%%%%%%%%%%%%%%%%%%%%%%%%%%%%%%%%%%%%%
\section{Precessão de um giroscópio}
%%%%%%%%%%%%%%%%%%%%%%%%%%%%%%%%%%%%%%%%%%%%%%%%%%%%%%%%
\comment{Fazer figuras,melhorar texto, descrições, o que é um giroscópio, falar como é impressionante, etc.}

A Precessão de um giroscópio é um exemplo claro da razão pela qual o torque é uma grandeza vetorial. Se não fosse esse o caso, o movimento não poderia ser explicado. Na figura ao lado, mostramos um desenho esquemático de um giroscópio, com as forças e torques que atuam sobre ele quando ele está parado. Verificamos que há um torque na direção $y$ e que -- ao liberarmos a movimentação do sistema -- será responsável por girar o giroscópio em torno desse eixo, dotando-o de um momento angular $\vec{L}$ também na direção de $y$.

No caso de o giroscópio já estar girando antes de o soltarmos, já teremos um momento angular inicial $\vec{L}$ na direção do eixo do disco do giroscópio. Sabemos que nesse caso 
\begin{equation}
  \vec{L} = I\vec{\omega}.
\end{equation}
%
Se mantivermos a velocidade do disco constante, temos que o momento angular deve ser constante. Se soltarmos o sistema, o peso continuará exercendo um torque igual ao da situação anterior, na direção de $y$. Como $\vec{\tau}$ é perpendicular a $\vec{L}$, ele não pode mudar o \emph{módulo} do momento angular, porém pode mudar sua \emph{direção}. De fato, sabendo que
\begin{equation}
  \vec{\tau} = \frac{d\vec{L}}{dt},
\end{equation}
%
podemos escrever
\begin{equation}
  d\vec{L} = \vec{\tau} dt,
\end{equation}
%
o que nos indica que a \emph{variação} do vetor momento angular tem a mesma direção que o torque. Logo, após um intervalo de tempo $dt$, temos que o giroscópio aponta em uma nova direção no espaço.

Podemos determinar a velocidade de precessão do giroscópio fazendo a seguinte análise: Sabemos que o módulo do torque é dado, nesse caso, por
\begin{equation}
  \tau = Mgr,
\end{equation}
%
e portanto,
\begin{equation}
  dL = Mgr\,dt.
\end{equation}
%
Além disso, analisando a figura ao lado, temos que o arco $s$ tem comprimento
\begin{equation}
  s = L \phi.
\end{equation}
%
Para um ângulo muito pequeno,
\begin{equation}
  ds = L d\phi.
\end{equation}
%
Logo,
\begin{align}
  d\phi &= \frac{dL}{L} \\
  &= \frac{Mgr\,dt}{I\omega},
\end{align}
%
e, consequentemente,
\begin{equation}
  \Omega \equiv \frac{d\phi}{dt} = \frac{Mgr}{I\omega}.
\end{equation}
