\chapter{Sistemas de partículas e Momento Linear}
\label{Chap:MomentoLinear}
%%%%%%%%%%%%%%%%%%%%%%%%%%%%%%%%%%%%%%%%%
\minitoc

\clearpage

%%%%%%%%%%%%%%%%%%%%%%%%%%%%%%%%%%%%%%%%%
\section{Introdução}
{\it

Intro ...

}

\section{Sistemas de partículas}
\section{Centro de massa de um sistema de partículas}
\subsection{Torque}
\subsection{Cálculo do centro de massa de um conjunto de partículas}
\section{Centro de massa de um corpo extenso}
\subsection{Barra uniforme} % dividida em n fatias
\subsection{Expressão geral para o cálculo do centro de massa}
\subsection{Barra não uniforme}
\section{Técnicas de simetria}
\section{Segunda Lei de Newton para o centro de massa}
\section{Momento linear}
% deriva fórmula do cm e assume a = 0 => F_R = 0
\subsection{Segunda Lei de Newton em termos do momento linear}
\section{Conservação do momento linear}
\section{Impulso}
\section{Colisões}
\subsection{Colisões inelásticas}
\subsection{Colisões elásticas}
\subsection{Colisões bidimensionais}
\subsection{Colisões em série}
