%%%%%%%%%%%%%%%%%%%%
\chapter{Introdução}
%%%%%%%%%%%%%%%%%%%%

%\item Achei isso interessante ([Referência](https://www.washingtonpost.com/posteverything/wp/2016/09/02/meet-the-parents-who-wont-let-their-children-study-literature/?utm_campaign=pockethits&utm_medium=email&utm_source=pocket&utm_term=.46e4274e1f53)):
%>It’s worth remembering that at American universities, the original rationale for majors was not to train students for careers. Rather, the idea was that after a period of broad intellectual exploration, a major was supposed to give students the experience of mastering one subject, in the process developing skills such as discipline, persistence, and how to research, analyze, communicate clearly and think logically.

%Ou seja, o barateamento, a formação do "engenheiro que faz casinha" é completamente fora do razoável. E o negócio é pra ser difícil mesmo, não é pra passar pra cumprir tabela. Se ferrar estudando é um rito, hehe. Bem diz o Campbell, esses bárbaros estão assim por que não têm mais ritos/mitos em suas vidas.

% Se fala em desenvolver habilidades de relacionamento com outras pessoas, de liderança, de correr atrás das coisas. Talvez fosse interessante citar que essas são características que eles deveriam desenvolver na graduação, pois são tendência em ser exigidas pelo mercado de trabalho (e pela própria vida desses viventes):
%Attitude (atitude)
%Communication (comunicação)
%Conflict resolution (resolução de conflitos)
%Creative thinking (pensamento criativo, alternativa: creativity – criatividade)
%Critical thinking (pensamento crítico)
%Decision making (tomada de decisão)
%Empathy (empatia)
%Ethics (ética)
%Flexibility (flexibilidade, alternativa: resilience – resiliência)
%Leadership (liderança)
%Motivation (motivação)
%Networking (rede de contatos ativos)
%Patience (paciência)
%Persuasion (persuasão)
%Positivity (positividade)

Problem solving (solução de problemas)

Teamwork (trabalho em equipe)

Time management (gestão do tempo)

%%%%%%%%%%%%%%%%%%%%%%%%%%%%%%%%%%%%%%%%%%%%%
\section{Objetivos da disciplina de Física 1}
%%%%%%%%%%%%%%%%%%%%%%%%%%%%%%%%%%%%%%%%%%%%%

- Os conteúdos de física 1 são fundamentais para estudar física nos níveis posteriores, mesmo para disciplinas aparentemente não relacionadas, como física 3 (vetores, por exemplo)

- A capacidade de abstração é uma das coisas importantes que começamos a ver nessa disciplina

- Temos uma chance de aplicar conceitos de matemática mostrados no ensino médio, fazemos isso de uma maneira progressiva

- deixar claro que essa não é uma disciplina que mostra tudo na profundidade total do negócio

- falar que a disciplina será desafiadora

- alguns assuntos são apresentados simplesmente por serem interessantes!

- Apesar de física 1 ser básica e todos os seus conteúdos servirem para um entendimento mais completo e profundo de qualquer fenômeno físico, é comum que outras disciplinas possam ser úteis para algumas áreas e não ser para outras. No entanto, isso é uma característica de uma formação geral: não é possível determinar exatamente os conteúdos que cada profissional necessitará durante a vida, mesmo pessoas formadas no mesmo curso. Assim, a formação tende a ser um grupo de conhecimentos que servem de base para outros mais específicos, que podem ser adquiridos em cursos de nível de pós-graduação, ou mesmo diretamente na vida profissional. O que importa é capacitar o formando para aprender sozinho (damos a base necessária).

- Um dos fatores que se procura na formação é, progressivamente, fazer com que o aluno seja autônomo e tenha iniciativa.

- falar de concepções alternativas? O alunos não são tabula rasa, então tem algumas coisas que eles sabem de um jeito errado. Talvez fosse interessante comentar um pouco sobre isso, como precisamos desconstruir essas ideias erradas.

%%%%%%%%%%%%%%%%%%%%%%%%%%%%%%%%%%%%%%%
\section{Métodos de ensino e avaliação}
%%%%%%%%%%%%%%%%%%%%%%%%%%%%%%%%%%%%%%%

- discutir as avaliações, não falar do 5.75, falar como elas são, falar das substitutivas

- pedir para que sejam claros no que estão fazendo: ao dividir o problema em partes, nomeá-las, ao usar uma equação, colocar a fórmula primeiro, depois dizer o que é zero (e por quê). Dizer que isso vai ser progressivamente exigido.

%%%%%%%%%%%%%%%%%%%%%%%%%%%%%%%%
\section{Métodos de aprendizado}
%%%%%%%%%%%%%%%%%%%%%%%%%%%%%%%%

- falar sobre a rotina de estudos, que tem que começar logo

- falar que exige várias horas de estudo semanais

- falar que o ideal é estudar todo dia um pouco, não deixar para a última semana, pois isso impossibilita procurar ajuda; além disso, compreender exije um tempo para digerir

- falar da carga horária gigante da universidade: profissão aluno. A infância acabou, agora você trabalha 12 horas diárias, pq vc não tem direitos garantidos pela clt :P

- Física exige cálculo, capacidade de abstração. Fazer contas exige prática. Mesmo que você entenda o que deve ser feito, pra poder fazer na prova, em um tempo limitado, é necessário prática. Pratique.

\end{itemize}

%%%%%%%%%%%%%%%%%%%%%
\section{Referências}
%%%%%%%%%%%%%%%%%%%%%

Colocar aqui as referências que usei como ``inspiração'' para escrever o texto:
Kleppner,
Landau e Lifshitz, Mechanics
etc.


\vspace{1cm}
\begin{flushright}
Clebson Abati Graeff,\\
Pato Branco, \monthyear.
\end{flushright}

