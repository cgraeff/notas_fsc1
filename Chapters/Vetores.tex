%%%%%%%%%%%%%%%%%%%%%%%%%%%%%%%%%%%%%%%%%%%%%%%%%%%%%%%%%%%%%%%%%%%%%%%%%%%%%%
\chapter{Vetores}
\label{Chap:Vetores}
%%%%%%%%%%%%%%%%%%%%%%%%%%%%%%%%%%%%%%%%%%%%%%%%%%%%%%%%%%%%%%%%%%%%%%%%%%%%%%
%\minitoc

%\clearpage

\begin{fullwidth}
{\it
Para que possamos estender o tratamento do movimento obtido no Capítulo~\ref{Chap:MovimentoUnidimensional} a três dimensões, vamos precisar utilizar vetores. Neste capítulo discutiremos tais objetos e algumas de suas propriedades.
}
\end{fullwidth}

%%%%%%%%%%%%%%%%%%%%
\section{Introdução}
%%%%%%%%%%%%%%%%%%%%

As grandezas físicas podem ser divididas em dois tipos: escalares e vetoriais. As escalares são aquelas que são compostas por um valor numérico e, na maioria das vezes, uma unidade. Podemos citar como exemplos desse tipo de grandeza a temperatura, o tempo e a massa. Se, por exemplo, denotarmos uma temperatura como ``\np[\tcdegree C]{25,0}'' temos uma informação completa.

Já no caso de grandezas vetoriais, temos ---~além de um valor numérico e de uma unidade~--- uma direção no espaço\footnote{Uma direção no espaço é simplesmente uma linha reta.} e um sentido\footnote{A partir de um ponto qualquer de uma reta, podemos nos mover em dois sentidos, um para cada extremidade.}. No movimento retilíneo a linha reta onde ocorre o movimento é a direção, enquanto o sentido denotamos simplesmente pelo sinal. Note, portanto, que um movimento unidimensional tem todas as características de um vetor, apesar de permitir um tratamento simplificado através de uma variável $x$ para a posição que se comporta como um escalar\footnote{Veremos adiante que a posição $x$ é na verdade uma \emph{projeção} de um vetor em uma direção do espaço. De fato, as projeções se comportam como escalares.} Para que possamos trabalhar em duas ou três dimensões, no entanto, precisamos tratar algumas grandezas como vetores propriamente.

As regras de soma, subtração, multiplicação e divisão para grandezas escalares são aquelas da álgebra comum e não nos oferecem dificuldades. Os vetores, por sua vez, possuem propriedades diferentes de soma, subtração, e multiplicação; além disso, não existe o conceito de divisão entre dois vetores. Teremos que verificar exatamente como realizar operações envolvendo vetores e nos acostumar com suas propriedades diferenciadas em relação ao escalares.

% Existe diferença entre bounded e unbounded vectors: o deslocamento é bounded, pq está ligado à posição onde ele está. Outros podem não ser, como o momento angular, podendo ser transladado. De qq forma, as regras da soma precisam dessa translação pra ser explicadas. Ver Quick introduction to tensor analysis.

%%%%%%%%%%%%%%%%%%%%%%%%%%%%%%%%%%%%%%%%%%%%%%
\section{Representação geométrica de um vetor}
%%%%%%%%%%%%%%%%%%%%%%%%%%%%%%%%%%%%%%%%%%%%%%

\begin{marginfigure}
\centering
\begin{tikzpicture}[>=Stealth, rotate=-60]

    \draw[->, thick] (0,0) node[below left]{$A$} -- (0,2) node[above right]{$B$};
    
    \fill[opacity=1] (0,0) circle (1.25pt);
    \fill[opacity=1] (0,2) circle (1.25pt);
     
    \draw [name path=curve 1, dashdotted] (0,0) .. controls (2,1) and (-2,1) .. (0,2);

\end{tikzpicture}
\caption{Ilustração de um deslocamento entre os pontos $A$ e $B$. Por mais que o caminho percorrido seja distinto da ``linha que liga os dois pontos'', o deslocamento é sempre ao longo de tal reta.}
\end{marginfigure}

O exemplo mais simples de vetor é o \emph{deslocamento}. Se ocorre um deslocamento em um plano entre os pontos $A$ e $B$, o vetor deslocamento é geometricamente uma ``flecha'' que liga os dois pontos, partindo de $A$ e apontando para $B$. Veja que \emph{não importa o caminho percorrido}, o vetor liga os pontos em linha reta (Figura~\ref{Fig:Deslocamento_eh_o_vetor_mais_simples}). Nessa definição estão englobadas claramente as três propriedades vetoriais:
\begin{description}
    \item[Direção:] A reta que liga $A$ a $B$;
    \item[Sentido:] De $A$ para $B$;
    \item[Módulo:] O módulo do vetor está associado ao seu comprimento e, para o deslocamento entre $A$ e $B$, é a própria distância em linha reta entre os dois pontos.
\end{description}

\begin{marginfigure}
\centering
\begin{tikzpicture}[>=Stealth, rotate=-60]

    \draw[->, thick] (0,0) node[below left]{$A$} -- (0,2) node[above right]{$B$};
    
    \fill (0,0) circle (1.25pt);
    \fill (0,2) circle (1.25pt);
     
    \draw [name path=curve 1, dashdotted] (0,0) .. controls (2,1) and (-2,1) .. (0,2);
    
    \draw[dashed] (0,-1) -- (0, 3);
    \draw[|-|] (0.3,0) -- node[below]{$d$} (0.3,2);

\end{tikzpicture}
\caption{Destacamos nesta figura a direção do deslocamento através de uma linha reta pontilhada. Além disso, mostramos o valor do deslocamento, que é o próprio valor de distância.\label{Fig:Deslocamento_eh_o_vetor_mais_simples}}
\end{marginfigure}

No caso de outras grandezas ---~como a velocidade, por exemplo~--- também associamos o módulo ao comprimento do vetor, porém não temos uma relação direta entre seu tamanho geométrico e o módulo: podemos adotar que um vetor com \np[cm]{1,0} denote uma velocidade de \np[m/s]{1,0} ou \np[m/s]{20,0}. Mesmo para o deslocamento, ao fazermos um desenho, adotamos esse tipo de fator de escala: \np[cm]{1,0} pode denotar \np[m]{1,0}, ou \np[km]{1,0}, por exemplo. Apesar disso, ao representarmos vários vetores da mesma grandeza, devemos utilizar o mesmo fator de proporcionalidade.\footnote{Veremos que manter essa escala não é uma tarefa fácil: muitas vezes os vetores de uma mesma grandeza podem ter tamanhos inconvenientes se respeitarmos a mesma escala. Em alguns casos, o módulo do vetor pode até mesmo ser desconhecido. Nesses casos, optamos por denotar mais precisamente a direção e o sentido dos vetores, deixando o comprimento das flechas como um indicador qualitativo.}

\begin{marginfigure}
\centering
\begin{tikzpicture}[>=Stealth, rotate=-60]

    \draw[->, thick] (0,0) -- node[above]{$\vec{a}$} (0,2);
    
    \fill (0,0) circle (1.25pt) node[below left]{$A$};
    \fill (0,2) circle (1.25pt) node[above right]{$B$};
     
    \draw[|-|] (0.3,0) -- node[below]{$a$} (0.3,2);

\end{tikzpicture}
\caption{Um vetor $\vec{a}$ qualquer, cujo módulo é denotado por $a$ .\label{Fig:vetor_simples_com_nome}}
\end{marginfigure}

Finalmente, resta notar que matematicamente, um vetor é denotado por uma pequena flecha sobre um símbolo
\begin{equation*}
    \vec{a},
\end{equation*}
%
ou por um símbolo em negrito
\begin{equation*}
    \bm{b}.
\end{equation*}
%
Adotaremos o primeiro tipo de notação, porém a segunda é mais comum em textos mais avançados. Além disso, ao utilizarmos somente a letra usada para denotar o vetor, sem a flecha e sem negrito, entende-se que se trata do módulo do vetor. Alternativamente, podemos usar $|\vec{a}|$ para denotar o módulo\footnote{Os matemáticos costumam olhar com cara feia que fala em ``módulo'' de um vetor. Eles preferem o termo \emph{norma de um vetor}, e usam a notação $\lVert\vec{a}\rVert$. Na Física, no entanto, todos usam o termo ``módulo''.}. Se, por exemplo, temos um vetor $\vec{a}$ como o da Figura~\ref{Fig:vetor_simples_com_nome}, o símbolo $a$ se refere ao módulo de tal vetor.

%%%%%%%%%%%%%%%%%%%%%%%%%%%%%%%%%%%%%%
\section{Operações envolvendo vetores}
%%%%%%%%%%%%%%%%%%%%%%%%%%%%%%%%%%%%%%

Para os vetores, temos as seguintes operações matemáticas possíveis:
\begin{itemize}
    \item soma;
    \item subtração;
    \item multiplicação de um vetor por um escalar;
    \item multiplicação de um vetor por outro, resultando em um escalar;
    \item multiplicação de um vetor por outro, resultando em um vetor.
\end{itemize}

Tais operações diferem daquelas realizadas com escalares: devido ao fato de que os vetores têm direção e sentido, por exemplo, temos que a soma de dois vetores não é dada pela soma de seus módulos. Veremos a seguir como realizar essas operações, exceto as duas últimas que não serão necessárias tão cedo, e as veremos quando for conveniente.

%%%%%%%%%%%%%%%%%
\subsection{Soma}
%%%%%%%%%%%%%%%%%

\begin{marginfigure}[2cm]
\centering
\begin{tikzpicture}[>=Stealth, rotate=0]

    \coordinate (A) at (0,0);
    \coordinate (B) at (0,2.5);
    \coordinate (C) at (2.5,1.25);
    
%    \fill[opacity=1] (A) circle (1.25pt) node[below left]{$A$};
%    \fill[opacity=1] (B) circle (1.25pt) node[above right]{$B$};
%    \fill[opacity=1] (C) circle (1.25pt) node[right]{$C$};

    \draw[->, thick] (A) -- node[left]{$\vec{a}$}(B);
    \draw[->, thick] (B) -- node[above]{$\vec{b}$} (C);
    \draw[->, thick, dashed] (A) -- node[below]{$\vec{c}$} (C);

\end{tikzpicture}
\caption{Soma de dois vetores.\label{Fig:SomaGeometricaDeDoisVetores}}
\end{marginfigure}

Dados dois vetores $\vec{a}$ e $\vec{b}$, a soma $\vec{a}+\vec{b}$ pode ser ``calculada'' geometricamente da seguinte forma (Figura~\ref{Fig:SomaGeometricaDeDoisVetores}): tomamos o segundo vetor ($\vec{b}$) e o transladamos de forma que sua origem coincida com o final do primeiro vetor ($\vec{a}$); traçamos um vetor da origem do primeiro vetor até o final do segundo ($\vec{c}$). Este último vetor ($\vec{c}$) representa a soma dos dois primeiros:
\begin{equation}
    \vec{c} = \vec{a} + \vec{b}.
\end{equation}

%%%%%%%%%%%%%%%%%%%%%%%%%%%%%%%%%%%%%%%%%%%%%%%%%%
\paragraph{Comutatividade, regra do paralelogramo}
%%%%%%%%%%%%%%%%%%%%%%%%%%%%%%%%%%%%%%%%%%%%%%%%%%

Se fizermos o contrário, isto é, transladarmos o primeiro vetor até que sua origem coincida com o final do segundo e traçarmos um vetor do início do segundo até o final do primeiro, veremos que o resultado obtido será o mesmo (Figura~\ref{Fig:RegraParalelogramoSoma}).

\begin{marginfigure}
\centering
\begin{tikzpicture}[>=Stealth, rotate=0]

    \coordinate (A) at (0,0);
    \coordinate (B) at (0,2.5);
    \coordinate (C) at (2.5,1.25);
    \coordinate (Ap) at (2.5,-1.25);
    
%    \fill[opacity=1] (A) circle (1.25pt) node[below left]{$A$};
%    \fill[opacity=1] (B) circle (1.25pt) node[above right]{$B$};
%    \fill[opacity=1] (C) circle (1.25pt) node[right]{$C$};

    \draw[->, thick] (A) -- node[left]{$\vec{a}$}(B);
    \draw[->, thick] (B) -- node[above]{$\vec{b}$} (C);
    \draw[->, thick, dashed] (A) -- node[below]{$\vec{c}$} (C);
    
    \draw[dotted, ->] (Ap) -- node[right]{$\vec{a}$} (C);
    \draw[dotted, ->] (A) -- node[below]{$\vec{b}$} (Ap);

\end{tikzpicture}
\caption{Soma de dois vetores através da ``regra do paralelogramo''. \label{Fig:RegraParalelogramoSoma}}
\end{marginfigure}

Logo, concluímos que a soma é comutativa:
\begin{equation}
  \vec{a} + \vec{b} = \vec{b} + \vec{a}.
\end{equation}
%
Devido ao fato de que os vetores formam a figura de um paralelogramo, é comum que a soma geométrica dos vetores seja feita através da elaboração de tal figura. Por isso, a soma geométrica também é conhecida como ``regra do paralelogramo''.

%%%%%%%%%%%%%%%%%%%%%%%%%%%
\paragraph{Associatividade}
%%%%%%%%%%%%%%%%%%%%%%%%%%%

Graficamente, através da Figura ~\ref{Fig:VetoresAssociatividade}, podemos ver também que a soma de vetores é associativa: se considerarmos a soma dos vetores $\vec{a}$, $\vec{b}$, e $\vec{c}$, percebemos que
\begin{align}
  \vec{a}+\vec{b} + \vec{c} &= \vec{a} + (\vec{b} + \vec{c}) \\
  &= (\vec{a}+\vec{b}) + \vec{c} \\
  &= (\vec{a}+\vec{c}) + \vec{b}.
\end{align}


\begin{marginfigure}
\centering
\begin{tikzpicture}[>=Stealth, scale=1.5,
shortened line/.style={shorten >=-#1,shorten <=-#1},
 shortened line/.default=-0.3mm]

    \fill (1,3) circle (0.5pt);
    \draw[->] (0,0) -- node[below]{$\vec{a}$}(2,1);
    \draw[-{Stealth[right]}] (2, 1) -- node[right]{$\vec{b}$} (2,2);
    \draw[-{Stealth[right]}, shortened line] (2,2) -- node[above]{$\vec{c}$} (1,3);
    
    \draw[-{Stealth[left]}, shortened line] (0,0) -- node[left]{$\vec{d}$}(1,3);
    
    \draw[->] (0,0) -- node[below, sloped]{$\vec{a}+\vec{b}$}(2, 2);
    \draw[-{Stealth[width = 2pt]}, shortened line] (2,1) -- node[below, sloped]{$\vec{b}+\vec{c}$}(1, 3);
    
    \draw[->, dotted] (2,1) -- (1,2);
    \draw[-{Stealth[width = 2pt]}, dotted, shortened line] (1,2) -- (1,3);
    \draw[->] (0,0) -- node[below right, sloped]{$\vec{a} + \vec{c}$} (1,2);

\end{tikzpicture}
\caption{Associatividade: note que $\vec{d} = (\vec{a} + \vec{b}) + \vec{c} = \vec{a} + (\vec{b} + \vec{c}) = (\vec{a} + \vec{c}) + \vec{b}$.\label{Fig:VetoresAssociatividade}}
\end{marginfigure}

%%%%%%%%%%%%%%%%%%%%%%%
\subsection{Vetor nulo}
%%%%%%%%%%%%%%%%%%%%%%%

Na Figura~\ref{Fig:CaminhoFechadoVetorNulo} temos uma série de deslocamentos sucessivos, iniciando pelo vetor $\vec{a}$. Após efetuar todos os delocamentos, voltamos para o ponto inicial, tendo um deslocamento total nulo. Nesse caso, temos que a soma dos vetores é igual ao \emph{vetor nulo}:
\begin{equation}
    \vec{a} + \vec{b} + \vec{c} + \vec{d} + \vec{e} + \vec{f} = \vec{0}.
\end{equation}
%
É comum se omitir a flecha sobre o zero ao se expressar o vetor nulo, pois não existe ambiguidade como no caso de um vetor não-nulo, onde o símbolo sem a flecha é utilizado para denotar o módulo do vetor.

\begin{marginfigure}
\centering
\begin{tikzpicture}[>=Stealth]

    \draw[->] (0,0) -- node[above]{$\vec{a}$} (1,0);
    \draw[->] (1,0) -- node[above]{$\vec{b}$} (2.5,-1);
    \draw[->] (2.5,-1) -- node[right]{$\vec{c}$} (2.2, -2);
    \draw[->] (2.2, -2) -- node[below]{$\vec{d}$} (1.5, -2.5);
    \draw[->] (1.5,-2.5) -- node[below]{$\vec{e}$} (0, -1.5);
    \draw[->] (0, -1.5) -- node[left]{$\vec{f}$} (0,0);
    
\end{tikzpicture}
\caption{A soma de um conjunto de vetores que forma um caminho fechado é um \emph{vetor nulo}.\label{Fig:CaminhoFechadoVetorNulo}}
\end{marginfigure}

%%%%%%%%%%%%%%%%%%%%%%
\subsection{Subtração}
%%%%%%%%%%%%%%%%%%%%%%

Vamos supor que temos dois vetores $\vec{a}$ e $\vec{b}$, de forma que
\begin{equation}
    \vec{a} + \vec{b} = 0.
\end{equation}
%
É evidente nesse caso que o vetor $\vec{b}$ é igual ao vetor $\vec{a}$ em direção e módulo, porém tem o sentido oposto. Vamos definir o operador ``$-$'' como sendo o responsável por essa alteração de sentido. Assim, a expressão acima pode ser denotada como:
\begin{equation}
    \vec{a} - \vec{a} = 0.
\end{equation}

\begin{marginfigure}[-1cm]
\centering
\begin{tikzpicture}[>=Stealth, scale=1.5]

    \draw[->] (0,0) -- node[below]{$\vec{a}$}(2,0);
    \draw[->] (2,0) -- node[below right]{$\vec{b}$} +(1,1);
    \draw[->, dashed] (2,0) -- node[below right]{$-\vec{b}$} +(-1,-1);
    
    \draw[->, dashdotted] (0,0) -- node[below,sloped]{$\vec{a} - \vec{b}$} (1,-1);

\end{tikzpicture}
\caption{Para determinarmos a diferença $\vec{a} - \vec{b}$, determinamos o vetor $-\vec{b}$ e então realizamos a soma $\vec{a} + (-\vec{b})$.\label{Fig:DefSubtracaoVetorial}}
\end{marginfigure}

Assim, a subtração de dois vetores é igual a soma do primeiro com \emph{menos o segundo}, onde ``menos o segundo'' significa que esse vetor será tomado no sentido contrário e somado ao primeiro. A Figura~\ref{Fig:DefSubtracaoVetorial} mostra graficamente tal processo. Note também que
\begin{equation}
    \vec{a} - \vec{b} = -(\vec{b}-\vec{a}).
\end{equation}

%%%%%%%%%%%%%%%%%%%%%%%%%%%%%%%%%%%%%%%%%%%%%%%%%%%%%%%
\paragraph{Subtração através da regra do paralelogramo}
%%%%%%%%%%%%%%%%%%%%%%%%%%%%%%%%%%%%%%%%%%%%%%%%%%%%%%%

Uma observação que nos permite calcular mais facilmente a diferença $\vec{a} - \vec{b}$ entre dois vetores é ilustrada na Figura~\ref{Fig:Macete_subtracao}: podemos transladar o vetor $\vec{b}$ até que seu início coincida com o início do vetor $\vec{a}$ e então desenhamos uma seta iniciando na ponta do vetor que aparece após o sinal de menos ($\vec{b}$) e terminando na ponta do vetor que aparece antes do sinal ($\vec{a}$). Construindo um paralelogramo, verificamos que uma das ``diagonais'' do paralelogramo nos dá a soma e a outra nos dá a diferença entre os vetores.

\begin{marginfigure}[-3cm]
\centering
\begin{tikzpicture}[>=Stealth, scale=1.5, shortened line/.style={shorten >=-#1,shorten <=-#1}, shortened line/.default=-0.3mm]

    \draw[-{Stealth[width=2pt]}] (0,0) -- node[below]{$\vec{a}$}(2,0);
    \draw[->] (2,0) -- node[below right]{$\vec{b}$} +(1,1);
    \draw[->, dashed] (2,0) -- node[below right]{$-\vec{b}$} +(-1,-1);
    
    \draw[->, dashdotted] (0,0) -- node[below,sloped]{$\vec{a} - \vec{b}$} (1,-1);
    
    \draw[->, dotted] (0,0) -- node[above left]{$\vec{b}$} +(1,1);
    \draw[-{Stealth[width=2pt]}, dotted] (1,1) -- node[above, sloped]{$\vec{a} - \vec{b}$} +(1,-1);

    \begin{scope}[shift={(0,-2.5)}]
    
        \fill (3,1) circle (0.5pt);
    
        \draw[-{Stealth[width=2pt]}] (0,0) -- node[below]{$\vec{a}$}(2,0);
        \draw[-{Stealth[right]}, shortened line] (2,0) -- node[below right]{$\vec{b}$} +(1,1);
                
        \draw[->, dotted] (0,0) -- node[above]{$\vec{b}$} +(1,1);
        \draw[-{Stealth[left]}, dotted, shortened line] (1,1) -- node[above]{$\vec{a}$}+(2,0);
        
        \draw[-{Stealth[width=3pt]}, dashdotted] (1,1) -- node[above, sloped, near end]{$\vec{a} - \vec{b}$} +(1,-1);
        \draw[-{Stealth[width=2pt]}, dashdotted, shortened line] (0,0) -- node[above, sloped, near start]{$\vec{a} + \vec{b}$} +(3,1);
    
    \end{scope}
    
\end{tikzpicture}
\caption{Subtração: podemos calcular a subtração ligando as extremidades dos vetores, quando eles partem de uma mesma origem.\label{Fig:Macete_subtracao}}
\end{marginfigure}

%%%%%%%%%%%%%%%%%%%%%%%%%%%%%%%%%%%%%%%%%%%%%%%%
\subsection{Multiplicação e divisão por escalar}
%%%%%%%%%%%%%%%%%%%%%%%%%%%%%%%%%%%%%%%%%%%%%%%%

Podemos multiplicar um vetor por um escalar $\alpha$, obtendo outro vetor:
\begin{equation}
  \vec{c} = \alpha\vec{b}.
\end{equation}
%
O módulo do vetor $\vec{c}$ será dado por $\mod{\vec{c}} = \mod{\alpha}\mod{\vec{b}}$. Note que a multiplicação por um escalar positivo só é capaz de alterar o módulo do vetor, uma vez que escalares não possuem informação de direção e sentido, e por isso essas informações são preservadas. Caso tenhamos uma multiplicação por um escalar negativo, além da alteração no módulo, ocorre uma mudança de sentido.

\begin{marginfigure}
\centering
\begin{tikzpicture}[>=Stealth]
    \draw[->] (0,0) -- node[above]{$\vec{a}$} (1,0);
    
    \draw[->] (0,-0.75) -- node[above]{$\vec{b} = 2 \,\vec{a}$} +(2,0);
    \draw[->] (0,-1.5) -- node[above]{$\vec{c} = 3\,\vec{a}$} +(3,0);
    \draw[->] (0,-2.25) -- node[above]{$\vec{d} = \np{3,5}\,\vec{a}$} +(3.5,0);

\end{tikzpicture}
\caption{Ao multiplicarmos um vetor por um escalar positivo, obtemos um novo vetor que preserva as informações de direção e sentido do vetor inicial, porém que tem um módulo diferente.}
\end{marginfigure}

 \begin{marginfigure}
\centering
\begin{tikzpicture}[>=Stealth]

    \draw[->] (0,0) -- node[above]{$\vec{a}$} (1,0);  
    \draw[->] (0,-0.75) -- node[above]{$\vec{e} = -\vec{a}$} +(-1,0);
    \draw[->] (0,-1.5) -- node[above]{$\vec{f} = -\np{2.3}\,\vec{a}$} +(-2.3,0);

\end{tikzpicture}
\caption{O efeito de multiplicar um vetor por um escalar negativo é o de alterar o seu módulo e o seu sentido.}
\end{marginfigure}

Para determinarmos o efeito de uma divisão por um escalar, como por exemplo
\begin{equation}
    \vec{b} = \frac{\vec{a}}{\alpha},
\end{equation}
%
basta a reescrevermos como um produto pelo inverso do escalar:
\begin{align}
    \vec{b} &= \frac{\vec{a}}{\alpha} \\
    &= \frac{1}{\alpha} \vec{a}.
\end{align}

%%%%%%%%%%%%%%%%%%%%%%%%%%%%%%%%%%%%%%%%
\subsection{Equações envolvendo vetores}
%%%%%%%%%%%%%%%%%%%%%%%%%%%%%%%%%%%%%%%%

Os vetores podem ser escritos em equações e seguem as mesmas regras que os escalares. Se
\begin{equation}
  \vec{c} = \vec{a} + \vec{b},
\end{equation}
%
então podemos escrever
\begin{align}
  \vec{a} = \vec{c} - \vec{b} \\
  \vec{b} = \vec{c} - \vec{a} \\
  \vec{a} + \vec{b} - \vec{c} = 0,
\end{align}
%
isto é, podemos passar um vetor de um membro para o outro de uma equação assim como em uma equação envolvendo escalares. A Figura~\ref{Fig:RelacoesEqVetores} mostra as relações acima através da representação geométrica dos vetores.

\begin{marginfigure}
\centering
\begin{tikzpicture}[>=Stealth, scale=1.5]

    \draw[->] (0,0) -- node[above]{$\vec{a}$} (1,0);
    \draw[-{Stealth[width=3pt]}] (1,0) -- node[right]{$\vec{b}$} +(0,-1);
    \draw[{Stealth[width=3pt]}-, dashed] (1,-1) -- node[below left]{$\vec{c}$} (0,0);
    
    \begin{scope}[shift={(1.5,0)}]
        \draw[->,dashed] (0,0) -- node[above]{$\vec{a}$} (1,0);
        \draw[<-] (1,0) -- node[right]{$-\vec{b}$} +(0,-1);
        \draw[<-] (1,-1) -- node[below left]{$\vec{c}$} (0,0);
    \end{scope}
    
    \begin{scope}[shift={(0,-1.5)}]
        \draw[dotted, ->] (0,0) -- node[above]{$\vec{a}$} (1,0);
        \draw[dotted, -{Stealth[width=3pt]}] (1,0) -- node[right]{$\vec{b}$} +(0,-1);
        \draw[{Stealth[width=3pt]}-] (1,-1) -- node[below left]{$\vec{c}$} (0,0);
        \draw[->] (1,-1) -- node[below]{$-\vec{a}$} +(-1,0);
        \draw[dashed,->] (0,0) -- node[left]{$\vec{b}$}(0,-1); 
    \end{scope}
    
    \begin{scope}[shift={(1.5,-1.5)}]
        \draw[->] (0,0) -- node[above]{$\vec{a}$} (1,0);
        \draw[->] (1,0) -- node[right]{$\vec{b}$} +(0,-1);
        \draw[->] (1,-1) -- node[below left]{$-\vec{c}$} (0,0);
    \end{scope}
    
\end{tikzpicture}
\caption{Relações entre vetores tais que $\vec{c} = \vec{a} + \vec{b}$. \label{Fig:RelacoesEqVetores}}
\end{marginfigure}

%%%%%%%%%%%%%%%%%%%%%%%%%%%%%%%%
\section{Sistemas de referência} 
%%%%%%%%%%%%%%%%%%%%%%%%%%%%%%%%

A análise de fenômenos físicos utilizando a representação gráfica de vetores é muito trabalhosa e suscetível a erros. Por isso, vamos utilizar um \emph{sistema de referência} para denotar os vetores. A grande vantagem de fazer isso é o fato de que podemos tratar eixos perpendiculares de maneira independente um do outro. Além disso, esse tratamento se resume a manipular grandezas que denominamos como \emph{componentes vetoriais} e que se comportam como \emph{escalares} ao invés de vetores.

Grande parte da utilidade de um sistema de referência provém de uma escolha adequada para a direção dos eixos: mesmo que \emph{qualquer} conjunto de eixos não-colineares\footnote{Um vetor ou eixo é colinear a outro se ambos têm a mesma direção.} sejam capazes de descrever as propriedades vetoriais, uma escolha adequada do sistema de referência leva a uma grande simplificação do tratamento matemático

%%%%%%%%%%%%%%%%%%%%%%%%%%%%
\subsection{Bases vetoriais}
%%%%%%%%%%%%%%%%%%%%%%%%%%%%

Se tomarmos dois vetores não-colineares em um plano, por exemplo, podemos escrever qualquer outro vetor em termos desses dois através de
\begin{equation}
  \vec{v} = \alpha \vec{e}_1 + \beta \vec{e}_2.
\end{equation}

\begin{marginfigure}
\centering
\begin{tikzpicture}[>=Stealth, scale=1.5]
    
    % vetores nas direções dos versores
    \draw[->] (0,0) -- (2.25,0) node[below]{$\vec{a}_1$};
    \draw[->] (0,0) -- (0.54,1.08) node[left]{$\vec{a}_2$};
    
    % vetor soma
    \draw[->] (0,0) -- node[above]{$\vec{a}$} (2.79, 1.08);
    
    % linhas pontilhadas
    \draw[dotted] (2.25, 0) -- (2.79,1.08);
    \draw[dotted] (0.54, 1.08) -- (2.79, 1.08);
    
    % versores
    \draw[->, very thick] (0,0) -- node[below]{$\vec{e}_1$} (0.6,0);
    \draw[->, very thick] (0,0) -- node[left]{$\vec{e}_2$} (0.3,0.6);
    
\end{tikzpicture}
\caption{Dados dois vetores não colineares, podemos usar as propriedades de multiplicação de vetor por escalar e de soma para que possamos construir um novo vetor.\label{Fig:VetorObtidoAPartirDeUmaBase}}
\end{marginfigure}

\noindent{}O processo descrito pela equação acima consiste em tomar os vetores $\vec{e}_1$ e $\vec{e}_2$ e os multiplicar por uma grandeza escalar, obtendo dois novos vetores
\begin{align}
    \vec{a}_1 &= \alpha\vec{e}_1 \\
    \vec{a}_2 &= \beta\vec{e}_2
\end{align}
%
e então determinar um novo vetor a partir da soma desses dois últimos:
\begin{equation}
    \vec{a} = \vec{a}_1 + \vec{a}_2.
\end{equation}

Os vetores $\vec{e}_1$ e $\vec{e}_2$ formam o que chamamos de \emph{base vetorial}, enquanto os escalares $\alpha$ e $\beta$ são chamados de \emph{componentes vetoriais}. Dados valores adequados para $\alpha$ e $\beta$, podemos atingir qualquer ponto do plano definido pelos vetores $\vec{e}_1$ e $\vec{e}_2$. Na Figura~\ref{Fig:VetoresObtidosAPartirDeUmaBase} temos alguns exemplos de vetores obtidos através da base da Figura~\ref{Fig:VetorObtidoAPartirDeUmaBase}, com os respectivos valores das constantes escalares. Note que para que possamos varrer todo o espaço bidimensional, também precisamos utilizar valores negativos de $\alpha$ e $\beta$. Na Figura~\ref{Fig:VetorObtidoAPartirDeUmaBaseValoresNegativos} temos um exemplo de vetor determinado dessa forma.

% Realizar operações com vetores geometricamente é possível, porém trabalhoso. Podemos utilizar as propriedades de soma de vetores e de multiplicação por um escalar para definir um sistema de coordenadas em relação a qual o vetores podem ser descritos em termos de componentes.

\begin{figure}
\centering
\begin{tikzpicture}[>=Stealth, scale = 1.5]

    \node at (0.5, -0.5) {$\alpha = \np{1.5}$};
    \node at (0.5, -0.75) {$\beta = \np{1.25}$};
    
    % vetores nas direções dos versores
    \draw[->] (0,0) -- (0.9,0);
    \draw[->] (0,0) -- (0.375,0.75);
    
    % vetor soma
    \draw[->] (0,0) -- (1.275, 0.75);
    
    % linhas pontilhadas
    \draw[dotted] (0.9, 0) -- (1.275,0.75);
    \draw[dotted] (0.375, 0.75) -- (1.275, 0.75);
    
    % versores
    \draw[->, very thick] (0,0) -- node[below]{$\vec{e}_1$} (0.6,0);
    \draw[->, very thick] (0,0) -- node[left]{$\vec{e}_2$} (0.3,0.6);
    
    \begin{scope}[shift = {(2,0)}]
    
    \node at (0.5, -0.5) {$\alpha = \np{2.5}$};
    \node at (0.5, -0.75) {$\beta = \np{1.2}$};
    
    % vetores nas direções dos versores
    \draw[->] (0,0) -- (1.5,0);
    \draw[->] (0,0) -- (0.36,0.72);
    
    % vetor soma
    \draw[->] (0,0) -- (1.86, 0.72);
    
    % linhas pontilhadas
    \draw[dotted] (1.5, 0) -- (1.86,0.72);
    \draw[dotted] (0.36, 0.72) -- (1.86, 0.72);
    
    % versores
    \draw[->, very thick] (0,0) -- node[below]{$\vec{e}_1$} (0.6,0);
    \draw[->, very thick] (0,0) -- node[left]{$\vec{e}_2$} (0.3,0.6);
    
    \end{scope}
    
    \begin{scope}[shift = {(0,-2.5)}]
    
    \node at (0.5, -0.5) {$\alpha = \np{0.5}$};
    \node at (0.5, -0.75) {$\beta = \np{1.75}$};
    
    % vetores nas direções dos versores
    \draw[->] (0,0) -- (0.3,0);
    \draw[->] (0,0) -- (0.525,1.05);
    
    % vetor soma
    \draw[->] (0,0) -- (0.825, 1.05);
    
    % linhas pontilhadas
    \draw[dotted] (0.3, 0) -- (0.825,1.05);
    \draw[dotted] (0.525, 1.05) -- (0.825, 1.05);
    
    % versores
    \draw[->, very thick] (0,0) -- node[below]{$\vec{e}_1$} (0.6,0);
    \draw[->, very thick] (0,0) -- node[left]{$\vec{e}_2$} (0.3,0.6);
    
    \end{scope}
    
    \begin{scope}[shift = {(2,-2.5)}]
    
    \node at (0.5, -0.5) {$\alpha = \np{1.75}$};
    \node at (0.5, -0.75) {$\beta = \np{0.5}$};
    
    % vetores nas direções dos versores
    \draw[->] (0,0) -- (1.05,0);
    \draw[->] (0,0) -- (0.15,0.3);
    
    % vetor soma
    \draw[->] (0,0) -- (1.2, 0.3);
    
    % linhas pontilhadas
    \draw[dotted] (1.05, 0) -- (1.2, 0.3);
    \draw[dotted] (0.15, 0.3) -- (1.2, 0.3);
    
    % versores
    \draw[->, very thick] (0,0) -- node[below]{$\vec{e}_1$} (0.6,0);
    \draw[->, very thick] (0,0) -- node[left]{$\vec{e}_2$} (0.3,0.6);
    
    \end{scope}
    
\end{tikzpicture}
\caption{Escalares $\alpha$ e $\beta$ diferentes resultam em vetores diferentes. \label{Fig:VetoresObtidosAPartirDeUmaBase}}
\end{figure}

\begin{marginfigure}
\centering
\begin{tikzpicture}[>=Stealth, scale=1.5]
    
    % vetores nas direções dos versores
    \draw[->] (0,0) -- (0.6,0);
    \draw[->] (0,0) -- (-0.3,-0.6);
    
    % vetor soma
    \draw[->] (0,0) -- (0.3, -0.6);
    
    % linhas pontilhadas
    \draw[dotted] (0.6, 0) -- (0.3,-0.6);
    \draw[dotted] (-0.3,-0.6) -- (0.3, -0.6);
    
    % versores
    \draw[->, very thick] (0,0) -- node[below]{$\vec{e}_1$} (0.6,0);
    \draw[->, very thick] (0,0) -- node[left]{$\vec{e}_2$} (0.3,0.6);
    
\end{tikzpicture}
\caption{Para que possamos varrer todo o espaço, as constantes escalares devem assumir também valores negativos. Na figura acima, $\alpha = 1$, enquanto $\beta = -1$.\label{Fig:VetorObtidoAPartirDeUmaBaseValoresNegativos}}
\end{marginfigure}

%%%%%%%%%%%%%%%%%%%%%%%%%%%%%%%%%%%%%%%%%%%%%%%%%%%%%%%%%%%%%%%%%
\subsection{Cálculo das componentes vetoriais, vetores unitários}
%%%%%%%%%%%%%%%%%%%%%%%%%%%%%%%%%%%%%%%%%%%%%%%%%%%%%%%%%%%%%%%%%

Vamos supor que tenhamos um vetor $\vec{a}$ conhecido, que faz um ângulo $\theta$ com o vetor $\vec{e}_1$ da base. Como podemos determinar os valores de $\alpha$ e $\beta$ que nos permitem reconstruir o vetor $\vec{a}$ a partir dos vetores da base?\footnote[][-1cm]{Tal processo é conhecido como \emph{decomposição vetorial}, e os escalares $\alpha$ e $\beta$ são denominados como \emph{componentes vetoriais nas direções $\hat{e}_1$ e $\hat{e}_2$}.}

Para determinarmos os valores das constantes escalares, vamos considerar a Figura~\ref{Fig:VetorObtidoAPartirDeUmaBase}. Como assumimos que a base é conhecida, sabemos os vetores $\vec{e}_1$ e $\vec{e}_2$, seus módulos, e o ângulo $\omega$ entre tais vetores\footnote[][-13mm]{É importante destacar aqui que definimos o ângulo $\theta$ entre o vetor $\vec{a}$ e o vetor $\vec{e}_1$ da base como sendo aquele medido no sentido anti-horário a partir do vetor da base. Precisamos tomar esse cuidado pois caso contrário teremos problemas de sinal.}. Se conhecemos o valor do ângulo $\theta$, podemos determinar a \emph{projeção}\footnote[][4cm]{A projeção de um vetor $\vec{a}$ qualquer em uma direção é sempre dada por $a\cos\theta$, onde $a$ é o módulo do vetor e $\theta$ é o ângulo entre o vetor e a direção na qual o estamos projetando. Podemos a interpretar como a ``sombra'' do vetor quando iluminado por um feixe de luz que incide perpendicularmente à direção sobre a qual estamos projetando.} de $\vec{a}$ na direção de $\vec{e}_1$:
\begin{figure}[!h]
\centering
\begin{tikzpicture}[>=Stealth, scale=3]
    
    % vetores nas direções dos versores
    \draw[->] (0,0) coordinate (origin) -- (2.25,0) node[below]{$\vec{a}_1$} coordinate (a1);
    \draw[->] (0,0) -- (0.54,1.08) node[left]{$\vec{a}_2$} coordinate (a2);
    
    % vetor soma
    \draw[->] (0,0) -- node[above]{$\vec{a}$} (2.79, 1.08) coordinate (a);
    
    % linhas pontilhadas
    \draw[dotted] (2.25, 0) -- (2.79,1.08);
    \draw[dotted] (0.54, 1.08) -- (2.79, 1.08);
    \draw[dashed] (2.79, 1.08) -- (2.79, 0) coordinate (quina) -- (2.25,0);
    \draw[|<->|] (2.79,-0.25) -- node[below]{$|\vec{a}_2| \cos\theta$} (2.25,-0.25);
    
    % medida
    \draw[dotted] (0,0) -- (0,-0.5) (2.79,0) -- +(0,-0.5);
    \draw[|-|] (0,-0.5) -- node[below]{$a\cos\theta$} (2.79, -0.5);
    % versores
    \draw[->, very thick] (0,0) -- node[below]{$\vec{e}_1$} (0.6,0);
    \draw[->, very thick] (0,0) -- node[left]{$\vec{e}_2$} (0.3,0.6);
    
    % ângulos
    \pic [draw, "$\omega$", angle eccentricity = 1.2, angle radius = 10mm] {angle = a1--origin--a2};
    \pic [draw, "$\theta$", angle eccentricity = 1.2, angle radius = 12mm] {angle = a1--origin--a};
    \pic [draw, "$\cdot$", angle eccentricity = 0.5, angle radius = 3mm] {angle = a--quina--origin};
    \pic [draw, "$\omega$", angle eccentricity = 2, angle radius = 3mm] {angle = quina--a1--a};
    
\end{tikzpicture}
\caption{Projeção do vetor $\vec{a}$ na direção de $\vec{e}_1$.}
\end{figure}

Note que a projeção $a\cos\theta$ tem o mesmo tamanho que o vetor $\vec{a}_1$, mais a distância tracejada até o ângulo reto. Essa distância, no entanto, pode ser calculada através de 
\begin{displaymath}
    a_2\cos\omega
\end{displaymath}
%
pois o vetor $\vec{a}_2$ pode ser transladado para a posição da linha pontilhada à direita. Assim, temos a relação
\begin{equation}
    a\cos\theta = a_1 + a_2\cos\omega.
\end{equation}

Lembrando que os vetores $\vec{a}_1$ e $\vec{a}_2$ são definidos como
\begin{align*}
    \vec{a}_1 &= \alpha \vec{e}_1 \\
    \vec{a}_2 &= \beta \vec{e}_2,
\end{align*}
%
podemos reescrever essa relação como
\begin{equation}
    \alpha e_1 = a\cos\theta - \beta e_2 \cos\omega.
\end{equation}
%
Veja que mesmo conhecendo os valores de $a$, $\theta$, $e_1$, $e_2$ e $\omega$, não é possível determinar os valores de $\alpha$ e $\beta$, pois temos uma equação com duas incógnitas.

Voltando à figura, podemos perceber ainda que ao fazer a projeção do vetor $\vec{a}$ na direção de $\vec{e}_2$, obtemos:
\begin{figure}[!h]
\centering
\begin{tikzpicture}[>=Stealth, scale=3]
    
    % vetores nas direções dos versores
    \draw[->] (0,0) coordinate (origin) -- (2.25,0) node[below]{$\vec{a}_1$} coordinate (a1);
    \draw[->] (0,0) -- node[left, near end]{$\vec{a}_2$} (0.54,1.08) coordinate (a2);
    
    % vetor soma
    \draw[->] (0,0) -- node[above]{$\vec{a}$} (2.79, 1.08) coordinate (a);
    
    % linhas pontilhadas
    \draw[dotted] (2.25, 0) -- (2.79,1.08);
    \draw[dotted] (0.54, 1.08) -- (2.79, 1.08);
    \draw[dashed] (2.79, 1.08) -- (63.435:2.22) coordinate (quina) -- (a2);
    \draw[|<->|] (65.8:2.226) -- node[sloped, above]{$|\vec{a}_2| \cos\omega$} (0.46,1.12);
    
    % medida
    %\draw[dotted] (0,0) -- (0,-0.5) (2.79,0) -- +(0,-0.5);
    \draw[|-|] (153.435:0.3) -- node[sloped, above]{$a\cos(\omega - \theta)$} +(63.435:2.22);
    
    % versores
    \draw[->, very thick] (0,0) -- node[below]{$\vec{e}_1$} (0.6,0);
    \draw[->, very thick] (0,0) -- node[left]{$\vec{e}_2$} (0.3,0.6);
    
    % ângulos
    \pic [draw, "$\omega$", angle eccentricity = 1.2, angle radius = 10mm] {angle = a1--origin--a2};
    \pic [draw, "$\theta$", angle eccentricity = 1.2, angle radius = 12mm] {angle = a1--origin--a};
    \pic [draw, "$\cdot$", angle eccentricity = 0.5, angle radius = 3mm] {angle = origin--quina--a};
    \pic [draw, "$\omega$", angle eccentricity = 2, angle radius = 3mm] {angle = a--a2--quina};
    \pic [draw, "$\omega - \theta$", angle eccentricity = 1.4, angle radius = 16mm] {angle = a--origin--a2};
\end{tikzpicture}
\caption{Projeção do vetor $\vec{a}$ na direção de $\vec{e}_2$.}
\end{figure}

\noindent{}Desta vez, observando que a distância tracejada entre o final de $\vec{a}_2$ e o ângulo reto é dada por\footnote{Note que a linha pontilhada horizontal é paralela ao vetor $\vec{a}_1$. Logo, o ângulo entre a reta pontilhada e a tracejada também é $\omega$.}
\begin{displaymath}
    a_1\cos\omega,
\end{displaymath}
%
obtemos a relação
\begin{equation}
    a\cos(\omega - \theta) - a_1\cos\omega = a_2,
\end{equation}
%
ou ainda
\begin{equation}
    a\cos(\omega - \theta) - \alpha e_1\cos\omega = \beta e_2,
\end{equation}
%
onde utilizamos as definições de $\vec{a}_1$ e $\vec{a}_2$.

Podemos agora montar um sistema de equações a partir do qual é possível determinar os valores de $\alpha$ e $\beta$:
\begin{equation}
\begin{system}
    \alpha e_1 &= a\cos\theta - \beta e_2 \cos\omega \\
    \beta e_2 &= a\cos(\omega - \theta) - \alpha e_1\cos\omega.
\end{system}
\end{equation}
%
Para simplificar a solução do sistema acima, podemos tomar os vetores da base de forma que seus módulos sejam iguais a 1:
\begin{equation}
    |\vec{e}_1| = |\vec{e}_2| = 1.
\end{equation}
%
Temos portanto \emph{vetores unitários}, também conhecidos como \emph{versores}, sendo que geralmente são representados como
\begin{displaymath}
    \hat{e}_1, \;\hat{e}_2.
\end{displaymath}
%
Note que esse passo não é necessário, uma vez que assumimos que os módulos dos vetores da base são conhecidos, porém ele ajuda a simplificar o processo de determinar os escalares $\alpha$ e $\beta$. Resolvendo o sistema, obtemos:
\begin{align}
    \alpha &= \frac{a[\cos\theta - \cos(\omega-\theta)\cos\omega]}{1-\cos^2\omega} \label{Eq:AlphaEBetaEmFuncaoDeAETheta1} \\
    \beta &= \frac{a[\cos(\omega - \theta) - \cos\theta\cos\omega]}{1-\cos^2\omega} \label{Eq:AlphaEBetaEmFuncaoDeAETheta2}.
\end{align}

%%%%%%%%%%%%%%%%%%%%%%%%%%%%%%%%%%%%%%%%%%%%%%%%%%%%
\subsection{Bases ortogonais, componentes vetoriais}
\label{Sec:ComponentesVetoriais}
%%%%%%%%%%%%%%%%%%%%%%%%%%%%%%%%%%%%%%%%%%%%%%%%%%%%

Apesar de podermos utilizar dois vetores quaisquer ---~desde que eles não sejam colineares~---, é mais simples descrever os vetores em termos de uma \emph{base ortogonal},\footnote[][-5mm]{Uma base ortogonal constituida por vetores unitários é denominada \emph{base ortonormal}.} isto é, uma base em que o ângulo $\omega$ entre os vetores é de \degree{90}.\footnote[][5mm]{Sempre que conhecemos dois vetores que formam uma base (isto é, dois vetores que não são colineares), podemos calcular dois outros vetores que são ortogonais entre si através do processo de ortogonalização de Gram-Schmidt. Você verá isso na disciplina de Álgebra Linear.} Nesse caso, verificamos através das Equações~\eqref{Eq:AlphaEBetaEmFuncaoDeAETheta1} e~\eqref{Eq:AlphaEBetaEmFuncaoDeAETheta2} que $\alpha$ e $\beta$ podem ser obtidos de maneira muito mais simples:
\begin{align}
    \alpha &= \frac{a[\cos\theta - \cos(\np[\tcdegree]{90}-\theta)\cos\np[\tcdegree]{90}]}{1-\cos^2\np[\tcdegree]{90}} \\
    &=a\cos\theta \\
    \beta &= \frac{a[\cos(\np[\tcdegree]{90} - \theta) - \cos\theta\cos\np[\tcdegree]{90}]}{1-\cos^2\np[\tcdegree]{90}} \\
    &= a\sen\theta,
\end{align}
%
onde utilizamos o fato de que $\cos\np[\tcdegree]{90} = 0$ e $\cos(\np[\tcdegree]{90} - \theta) = \sen\theta$. Temos, portanto, uma grande simplificação do processo de determinação das componentes vetoriais.

\begin{marginfigure}[2cm]
\centering
\begin{tikzpicture}[>=Stealth, scale=1.5]

    \draw[dashed] (0,0) -- (2.5,0);
    \draw[dashed] (0,0) -- (0,1.5);

    \draw[->, thick] (0,0) -- node[above]{$\vec{a}$} (2,1);
    
    \draw[dotted] (2,1) -- (2,0);
    \draw[dotted] (2,1) -- (0,1);
    
    \draw[->] (0,0) -- (2,0) node[below]{$\alpha\vec{e}_1$};
    \draw[->] (0,0) -- (0,1) node[left]{$\beta\vec{e}_2$};
    
    \draw[->] (0,0) -- (0.5,0) node[below left]{$\vec{e}_1$};
    \draw[->] (0,0) -- (0,0.5) node[below left]{$\vec{e}_2$};
    
%    \draw[|<->|] (0,-0.3) -- node[below]{$a_x$} (2,-0.3);
%    \draw[|-|] (-0.3,0) -- node[left]{$a_y$} (-0.3, 1);
%    \draw[|<->|] (2.3,0) -- node[right]{$a_y$} (2.3, 1);
    
    \coordinate (O) at (0,0);
    \coordinate (A) at (1,0);
    \coordinate (B) at (2,1);
       
    \pic [draw, "$\theta$", angle eccentricity=1.5] {angle = A--O--B};

\end{tikzpicture}
\caption{Decomposição de um vetor $\vec{a}$ em uma base ortogonal. \label{Fig:DecomposicaoEixosOrto}}
\end{marginfigure}

Devemos notar ainda que para um sistema de referência composto por uma base ortogonal, \emph{as projeções do vetor $\vec{a}$ nas direções dos vetores da base são as próprias componentes do vetor em termos de tais vetores unitários}. Isso permite que a determinação das componentes seja sempre feita utilizando funções as funções trigonométricas seno e cosseno, de uma forma bastante simples: se tomarmos o vetor $\vec{a}$ na Figura~\ref{Fig:DecomposicaoEixosOrto}, percebemos que ele é dado por
\begin{equation}\label{eq:vec_a}
    \vec{a} = \alpha\vec{e}_1 + \beta\vec{e}_2.
\end{equation}
%
Novamente, da figura, observamos que\sidenote{Note que as expressões para as projeções do vetor $\vec{a}$ podem ser calculadas através das definições das funções trigonométricas %
\begin{minipage}{\linewidth} %
\begin{align*} %
    \sen \theta &= \frac{C_o}{h} & \cos \theta &= \frac{C_a}{h}, %
\end{align*} %
\vspace{0.1mm}
\end{minipage} %
%
onde $C_o$ e $C_a$ representam os catetos oposto e adjacente, respectivamente, enquanto $h$ representa a hipotenusa. %
}
\begin{align}
    |\alpha\vec{e}_1| &= a\cos\theta \\
    |\beta\vec{e}_2| &= a\sen\theta.
\end{align}
%
Como escolhemos $\vec{e}_1$ e $\vec{e}_2$ como unitários, então é claro que
\begin{align}
    \alpha &= a\cos\theta \\
    \beta &= a\sen\theta.
\end{align}

Em Física\footnote{Não cremos em bases não-ortonormais, mas que elas existem, existem.} geralmente utilizamos uma base ortogonal com vetores unitários para descrever os vetores. Os eixos de referência utilizados são denominados como eixos $x$ e $y$, e contam com vetores unitários denominados $\versi$ e $\versj$, respectivamente\footnote{Veja que o que chamamos de \emph{eixos de referência} são simplesmente as direções dos próprios vetores unitários. Ao empregarmos essa nomenclatura, no entanto, assumimos sempre que se trata de uma base ortogonal.}. Já para as componentes, ao invés de utilizarmos $\alpha$ e $\beta$, utilizamos simplesmente $a_x$ e $a_y$. Considerando essa notação, temos que a Equação~\eqref{eq:vec_a} para o vetor $\vec{a}$ pode ser reescrita como (veja a Figura~\ref{Fig:sis_ref_orto_xy_e_versi_versj})
\begin{equation}
    \vec{a} = a_x \versi + a_y \versj,
\end{equation}

\begin{marginfigure}
\centering
\begin{tikzpicture}[>=Stealth, scale=1.5]

    \draw[->] (0,0) -- (2.5,0) node[below left]{$x$};
    \draw[->] (0,0) -- (0,1.5) node[below left]{$y$};

    \draw[->, thick] (0,0) -- node[above]{$\vec{a}$} (2,1);
    \draw[->, thick] (0,0) -- (0.5,0)node[below left]{$\versi$};
    \draw[->, thick] (0,0) -- (0,0.5) node[below left]{$\versj$};
    
    \draw[dotted] (2,1) -- (2,0);
    \draw[dotted] (2,1) -- (0,1);
    
    \draw[|-|] (0,-0.3) -- node[below]{$a_x$} (2,-0.3);
    \draw[|-|] (-0.3,0) -- node[left]{$a_y$} (-0.3, 1);
    
    \coordinate (O) at (0,0);
    \coordinate (A) at (1,0);
    \coordinate (B) at (2,1);
    \coordinate (C) at (0,1);
       
    \pic [draw, "$\theta$", angle eccentricity=1.25, angle radius = 10mm] {angle = A--O--B};

\end{tikzpicture}
\caption{Sistema de referência ortonormal com vetores unitários $\versi$ e $\versj$.\label{Fig:sis_ref_orto_xy_e_versi_versj}}
\end{marginfigure}

onde
\begin{align}
    a_x &= a\cos\theta \label{Eq:CompX}\\
    a_y &= a\sen\theta. \label{Eq:CompY}
\end{align}
%
Essa escolha é tão comum que muitas vezes simplesmente descrevemos um vetor em termos de suas componentes em um sistema de referência, como em
\begin{equation}
    \vec{a} = (a_x,a_y),
\end{equation}
ou
\begin{equation}
  \vec{a} = \begin{pmatrix} a_x \\ a_y \end{pmatrix}.
\end{equation}

\begin{marginfigure}
\centering
\begin{tikzpicture}[>=Stealth, scale=1.5]

    \draw[->] (0,0) -- (2.5,0) node[below left]{$x$};
    \draw[->] (0,0) -- (0,1.5) node[below left]{$y$};

    \draw[->, thick] (0,0) -- node[above]{$\vec{a}$} (2,1);
    
    \draw[dotted] (2,1) -- (2,0);
    \draw[dotted] (2,1) -- (0,1);
    
    \draw[|-|] (0,-0.3) -- node[below]{$a_x$} (2,-0.3);
    \draw[|-|] (-0.3,0) -- node[left]{$a_y$} (-0.3, 1);
    
    \coordinate (O) at (0,0);
    \coordinate (A) at (1,0);
    \coordinate (B) at (2,1);
    \coordinate (C) at (0,1);
       
    \pic [draw, "$\theta$", angle eccentricity=1.5] {angle = A--O--B};
    \pic [draw, "$\sigma$", angle eccentricity=1.5, angle radius = 4mm] {angle = B--O--C};

\end{tikzpicture}
\caption{Decomposição de vetores usando o ângulo $\sigma$ entre o vetor e o eixo $y$.\label{Fig:sis_ref_orto_angulo_eixo_y}}
\end{marginfigure}

Apesar de ser mais comum termos as informações de ângulo em relação ao eixo $x$, também podemos ter informações acerca do ângulo entre o vetor e o eixo $y$ (Figura~\ref{Fig:sis_ref_orto_angulo_eixo_y}). O tratamento para a obtenção das componentes é igualmente válido, e também consiste em decompor o vetor através das funções trigonométricas:
\begin{align}
    a_x &= a\sen\sigma \\
    a_y &= a\cos\sigma.
\end{align}
%
Ao analisar uma situação como a resolução de um problema, devemos estar atentos e utilizar a decomposição correta.\footnote{O melhor a se fazer é verificar quais são as expressões para o cálculo das componentes utilizando a própria definição das funções trigonométricas seno e cosseno.}

%%%%%%%%%%%%%%%%%%%%%%%%%%%%%%%%%%
\paragraph{Notação módulo-ângulo: coordenadas polares}
%%%%%%%%%%%%%%%%%%%%%%%%%%%%%%%%%%

Da seção anterior, fica evidente que além de podermos descrever um vetor $\vec{a}$ através de suas componentes nos eixos $x$ e $y$, podemos defini-lo completamente em duas dimensões através do ângulo $\theta$ entre o vetor e um dos eixos de referência, e do módulo $a$ do vetor. Um sistema de referências desse tipo é denominado \emph{sistema de coordenadas polares}. Nesta notação, ainda estamos nos valemos da definição dos eixos coordenados, uma vez que eles são utilizados para definir os ângulos. 

Em duas dimensões, podemos observar a partir das Equações~\eqref{Eq:CompX} e~\eqref{Eq:CompY} que
\begin{align}
  \frac{a_y}{a_x} &=\frac{a\sen\theta}{a\cos\theta} \\
  &= \tan\theta,
\end{align}
%
de onde podemos escrever\footnote{A função arco tangente ($\arctan$) é a função inversa da função tangente. Seu efeito é determinar o ângulo correspondente a um dado valor de tangente. Se tomarmor $\arctan(\tan \theta)$, obtemos simplesmente o ângulo $\theta$.}
\begin{equation}\label{Eq:AngaPartirDeComp}
  \theta = \arctan\frac{a_y}{a_x}.
\end{equation}
%
Equivalentemente, podemos verificar esse resultado utilizando relações trigonométricas através da  Figura~\ref{Fig:sis_ref_orto_angulo_eixo_y}.

Também podemos encontrar uma relação entre as componentes e o módulo do vetor através de
\begin{align}
  a_ x^2 + a_y^2 &= (a\cos\theta) ^2 + (a\sen\theta)^2 \\
  &= a^2\cos^2\theta + a^2\sen^2\theta \\
  &= a^2(\sen^2\theta + \cos^2\theta) \\
  &= a^2,
\end{align}
%
onde usamos o fato de que
\begin{equation}
    \sen^2\theta + \cos^2\theta = 1,
\end{equation}
%
o que nos permite escrever o módulo $a$ do vetor em termos de suas componentes como
\begin{equation}\label{Eq:ModAPartirDeComp}
    a = \sqrt{a_x^2 + a_y^2}.
\end{equation}
%
Novamente, esse resultado pode ser obtido geometricamente ao empregarmos o teorema de Pitágoras em qualquer um dos triângulos mostrados na Figura~\ref{Fig:sis_ref_orto_angulo_eixo_y}.

Portanto, se conhecemos o vetor em termos de duas componentes, podemos calcular seu módulo e o ângulo que ele faz com o eixo horizontal e vice-versa. Concluímos então que as duas notações são completamente equivalentes.

%%%%%%%%%%%%%%%%%%%%%%%%%%%%%%%%%%%%%%%%%%%%
\paragraph{Exemplo: Decomposição de vetores}
%%%%%%%%%%%%%%%%%%%%%%%%%%%%%%%%%%%%%%%%%%%%

\begin{quote}
    Dados três vetores deslocamento $\vec{a}$, $\vec{b}$, e $\vec{c}$, decomponha-os em um sistema de referência ortogonal, composto por um eixo horizontal $x$ e um eixo vertical $y$, sabendo que seus módulos são
    \begin{align*}
        a &= \np[m]{3} & b &= \np[m]{4,5} & c &= \np[m]{2.50}
    \end{align*}
    %
    e que fazem ângulos $\theta_a$, $\theta_b$, e $\theta_c$ com o eixo horizontal dados por
    \begin{align*}
        \theta_a &= \np[\tcdegree]{20} & \theta_b &= \np[\tcdegree]{36} & \theta_c &= \np[\tcdegree]{58}.
    \end{align*}
\end{quote}

\begin{marginfigure}[-3cm]
\centering
\begin{tikzpicture}[>=Stealth]

    \draw[->] (0,0) coordinate (O) -- (4,0) coordinate (X) node[below left]{$x$};
    \draw[->] (0,0) -- (0,3) node[below left]{$y$};
    
    \draw[->] (0,0) -- (20:3) coordinate (A) node[above]{$\vec{a}$};
    \draw[->] (0,0) -- (36:4.5) coordinate (B) node[above]{$\vec{b}$};
    \draw[->] (0,0) -- (58:2.5) coordinate (C) node[above]{$\vec{c}$};
    
    \pic [draw, dotted, angle eccentricity = 1.1, angle radius = 20mm] {angle = X--O--A};
    \pic [draw, dotted, angle eccentricity = 1.2, angle radius = 15mm] {angle = X--O--B};
    \pic [draw, dotted, angle eccentricity = 1.3, angle radius = 10mm] {angle = X--O--C};
    
\end{tikzpicture}
\caption{Vetores em relação ao sistema de referência. \label{Fig:Ex:Decomp}}
\end{marginfigure}

Podemos ilustrar os vetores no sistema de referência através da Figura~\ref{Fig:Ex:Decomp}. Para determinarmos as componentes, basta utilizarmos as Expressões~\eqref{Eq:CompX} e~\eqref{Eq:CompY}:
\begin{align*}
    a_x &= a\cos\theta_a & a_y &= a\sen\theta_a \\
    &= (\np[m]{3})\cdot(\cos\np[\tcdegree]{20}) & &= (\np[m]{3})\cdot(\sen\np[\tcdegree]{20}) \\
    &= \np[m]{2,82} & &= \np[m]{1,03} \\
    \\
    b_x &= a\cos\theta_a & b_y &= a\sen\theta_a \\
    &= (\np[m]{4.5})\cdot(\cos\np[\tcdegree]{36}) & &= (\np[m]{4.5})\cdot(\sen\np[\tcdegree]{36}) \\
    &= \np[m]{3.64} & &= \np[m]{2.65} \\
    \\
    c_x &= a\cos\theta_a & c_y &= a\sen\theta_a \\
    &= (\np[m]{2.5})\cdot(\cos\np[\tcdegree]{58}) & &= (\np[m]{2.5})\cdot(\sen\np[\tcdegree]{58}) \\
    &= \np[m]{1.32} & &= \np[m]{2.12}.
\end{align*}

%%%%%%%%%%%%%%%%%%%%%%%%%%%%%%%%%%%%%%%%%%%%%%%%%%%%%%%%%%%%%%%%%%%%%%%%%%
\paragraph{Exemplo: Decomposição de vetores em um referencial rotacionado}
%%%%%%%%%%%%%%%%%%%%%%%%%%%%%%%%%%%%%%%%%%%%%%%%%%%%%%%%%%%%%%%%%%%%%%%%%%

\begin{quote}
    A Figura~\ref{Fig:Ex:DecompRefRot} mostra dois sistemas de referência ortogonais ---~um constituido por um eixo $x$ horizontal e um eixo $y$ vertical, e outro constituido por dois eixos $x'$ e $y'$~---, sendo que o segundo foi obtido a partir de uma rotação do primeiro sistema de referência por um ângulo $\omega = \np[\tcdegree]{30}$. Se um vetor $\vec{a}$ tem módulo $a = \np[m]{2.5}$ e faz um ângulo $\theta = \np[\tcdegree]{45}$ em relação ao eixo horizontal, quais são as componentes do vetor nos eixos do sistema rotacionado?
\end{quote}

\begin{marginfigure}[-3.5cm]
\centering
\begin{tikzpicture}[>=Stealth, scale=0.8]

    \draw[->] (0,0) coordinate (O) -- (4,0) coordinate (X) node[below left]{$x$};
    \draw[->] (0,0) -- (0,3) node[below left]{$y$};
    
    \draw[->] (0,0) -- (30:4) coordinate (Xl) node[below] {$x'$};
    \draw[->] (0,0) -- (120:3) node[left]{$y'$};
    
    \draw[->, thick] (0,0) -- (45:2.5) coordinate (A) node[above]{$\vec{a}$};
    
    \pic [draw, "$\theta$", angle eccentricity = 1.2, angle radius = 13mm]{angle = X--O--A};
    \pic [draw, "$\omega$", angle eccentricity = 1.5, angle radius = 5mm]{angle = X--O--Xl};
    
\end{tikzpicture}
\caption{O vetor $\vec{a}$ em relação aos dois sistemas de referência. \label{Fig:Ex:DecompRefRot}}
\end{marginfigure}

\begin{marginfigure}[5mm]
\centering
\begin{tikzpicture}[>=Stealth, scale = 0.8]

    \draw[->, densely dotted] (0,0) coordinate (O) -- (4,0) coordinate (X) node[below left]{$x$};
    \draw[->, densely dotted] (0,0) -- (0,3) node[below left]{$y$};
    
    \draw[->] (0,0) -- (30:4) coordinate (Xl) node[below] {$x'$};
    \draw[->] (0,0) -- (120:3) node[left]{$y'$};
    
    \draw[->, thick] (0,0) -- (45:2.5) coordinate (A) node[above]{$\vec{a}$};
    
    \pic [gray,draw, densely dotted, "$\theta$", angle eccentricity = 1.2, angle radius = 10mm]{angle = X--O--A};
    \pic [gray,draw, densely dotted, "$\omega$", angle eccentricity = 1.5, angle radius = 5mm]{angle = X--O--Xl};
    \pic [draw, "$\theta'$", angle eccentricity = 1.2, angle radius = 14mm]{angle = Xl--O--A};
    
\end{tikzpicture}
\caption{As componentes do vetor $\vec{a}$ no sistema de referência rotacionado estão ligadas ao valor do ângulo $\theta'$. \label{Fig:Ex:DecompRefRotThetaP}}
\end{marginfigure}

Para que possamos determinar as componentes do vetor $\vec{a}$ nos eixos $x'$ e $y'$, precisamos somente determinar o ângulo $\theta'$ que o vetor faz com o eixo $x'$, uma vez que o módulo do vetor é constante. A partir da figura, podemos verificar facilmente que tal ângulo corresponde a
\begin{align}
    \theta' &= \theta - \omega \\
    &= \np[\tcdegree]{45} - \np[\tcdegree]{30} \\
    &= \np[\tcdegree]{15}.
\end{align}
%
Assim, temos que
\begin{align}
    a_{x'} &= a\cos\theta' & a_{y'} &= a\sen\theta' \\
    &= (\np[m]{2.5})\cdot(\cos\np[\tcdegree]{15}) & &= (\np[m]{2.5})\cdot(\sen\np[\tcdegree]{15}) \\
    &\approx \np[m]{2.415} & &\approx \np[m]{0.647}
\end{align}

%%%%%%%%%%%%%%%%%%%%%%%%%%%%%%%%%%%%%%%%%%%%%%%%%%%%%%%%%%%%%%%%%%%%%%%%%%
\paragraph{Exemplo: Decomposição de vetores em um referencial rotacionado}
%%%%%%%%%%%%%%%%%%%%%%%%%%%%%%%%%%%%%%%%%%%%%%%%%%%%%%%%%%%%%%%%%%%%%%%%%%

\begin{quote}
    A Figura~\ref{Fig:Ex:DecompRefRot} mostra dois sistemas de referência ortogonais ---~um constituido por um eixo $x$ horizontal e um eixo $y$ vertical, e outro constituido por dois eixos $x'$ e $y'$~---, sendo que o segundo foi obtido a partir de uma rotação do primeiro sistema de referência por um ângulo $\omega = \np[\tcdegree]{20}$. Se um vetor $\vec{a}$ tem componentes $a_x = \np[m]{2}$ e $a_y = \np[m]{1}$, quais são as componentes do vetor nos eixos do sistema rotacionado?
\end{quote}

\begin{marginfigure}[3cm]
\centering
\begin{tikzpicture}[>=Stealth, scale = 0.8]

    \draw[->] (0,0) coordinate (O) -- (4,0) coordinate (X) node[below left]{$x$};
    \draw[->] (0,0) -- (0,3) node[below left]{$y$};
    
    \draw[->] (0,0) -- (20:4) coordinate (Xl) node[below] {$x'$};
    \draw[->] (0,0) -- (105:3) node[left]{$y'$};
    
    \draw[->, thick] (0,0) -- (26.565:2.2361) coordinate (A) node[above]{$\vec{a}$};
    
    \pic [draw, "$\theta$", angle eccentricity = 1.2, angle radius = 15mm]{angle = X--O--A};
    \pic [draw, "$\omega$", angle eccentricity = 1.2, angle radius = 10mm]{angle = X--O--Xl};
    
\end{tikzpicture}
\caption{O vetor $\vec{a}$ em relação aos dois sistemas de referência. \label{Fig:Ex:DecompRefRotModAng}}
\end{marginfigure}

A situação do problema acima difere do exemplo da seção anterior simplesmente no fato de que não temos o módulo $a$ e o ângulo $\theta$, mas sim as componentes $a_x$ e $a_y$ do vetor $\vec{a}$. Para que possamos determinar as componentes do vetor $\vec{a}$ nos eixos $x'$ e $y'$, tanto o ângulo $\theta'$ que o vetor faz com o eixo $x'$, quanto o módulo do vetor devem ser determinados.

Podemos determinar o módulo através da Equação~\eqref{Eq:ModAPartirDeComp}:
\begin{align}
    a &= \sqrt{a_x^2 + a_y^2} \\
    &= \sqrt{(\np[m]{2})^2 + (\np[m]{1})^2} \\
    &= \sqrt{(\np[m^2]{4}) + (\np[m^2]{1}} \\
    &= \sqrt{\np[m^2]{5}} \\
    &\approx \np[m]{2.236}.
\end{align}
%
Para determinarmos o ângulo $\theta'$, precisamos antes determinar o ângulo $\theta$. A partir da Equação~\eqref{Eq:AngaPartirDeComp}, temos
\begin{align}
    \theta &= \arctan\left(\frac{a_y}{a_x}\right) \\
    &= \arctan\left(\frac{(\np[m]{1})}{(\np[m]{2})}\right) \\
    &=\arctan\left(\frac{1}{2}\right) \\
    &\approx \np[\tcdegree]{26.565}.
\end{align}
%
Assim, temos para $\theta'$
\begin{align}
    \theta' &= \theta - \omega \\
    &= \np[\tcdegree]{26.565} - \np[\tcdegree]{20} \\
    &= \np[\tcdegree]{6.565}.
\end{align}
%
Finalmente, temos que
\begin{align}
    a_{x'} &= a\cos\theta' & a_{y'} &= a\sen\theta' \\
    &= (\np[m]{2.236})\cdot(\cos\np[\tcdegree]{6.565}) & &= (\np[m]{2.236})\cdot(\sen\np[\tcdegree]{6.565}) \\
    &\approx \np[m]{2.221} & &\approx \np[m]{0.256}
\end{align}

%%%%%%%%%%%%%%%%%%%%%%%%%%%%%%%%%%%%%%%%%%%%%%%%%%%%
\subsection{Sistema de referência em três dimensões}
%%%%%%%%%%%%%%%%%%%%%%%%%%%%%%%%%%%%%%%%%%%%%%%%%%%%

\begin{marginfigure}[3cm]
\centering
\begin{tikzpicture}[>=Stealth, scale=2.0]

    \draw[->, dashed] (0,0,0) -- (1.5,0,0) node[below left]{$y$};
    \draw[->, dashed] (0,0,0) -- (0,1.5,0) node[below left]{$z$};
    \draw[->, dashed] (0,0,0) -- (0,0,2) node[below]{$x$};
    
    \draw[->] (0,0,0) -- (0.3,0,0) node[below left]{$\versj$};
    \draw[->] (0,0,0) -- (0,0.3,0) node[below left]{$\versk$};
    \draw[->] (0,0,0) -- (0,0,0.3) node[below]{$\versi$};
      
    \draw[->, thick] (0,0,0) -- node[above]{$\vec{a}$} (1,1,1);
    
    \draw[dotted] (1,1,1) -- (1,1,0) (1,1,1) -- (1,0,1) (1,1,1) -- (0,1,1) (1,1,0) -- (1,0,0) (1,1,0) -- (0,1,0) (0,1,1) -- (0,0,1) (0,1,1) -- (0,1,0) (1,0,1) -- (1,0,0) (1,0,1) -- (0,0,1);
    
    \draw[<->] (1.1,0,0) -- node[right]{$a_z$} (1.1,1,0);
    \draw[very thin] (1.05,1,0) -- (1.15,1,0);
    \draw[<->] (1.15,0,0) -- node[right]{$a_x$} (1.15,0,1);
    \draw[very thin] (1.10,0,1) -- (1.23,0,1);
    \draw[<->] (1,0,1.2) -- node[below]{$a_y$} (0,0,1.2);
    \draw[very thin] (1,0,1.15) -- (1,0,1.3);
    
\end{tikzpicture}
\caption{Sistema de referência tridimensional.\label{Fig:vetores_unitários}}
\end{marginfigure}

Para que possamos descrever o caso tridimensional, precisamos de uma base composta de três vetores não-colineares. Da mesma maneira que no caso bidimensional, o tratamento utilizando uma base ortogonal, com vetores unitários\footnote[][1cm]{Esse tipo de base é denominada \emph{ortonormal}.}, é mais simples. Portanto, adicionamos mais um eixo de referência, denominado $z$, perpendicular aos eixos $x$ e $y$, sendo que o vetor unitário que aponta em sua direção é denominado $\versk$. Assim, um vetor $\vec{a}$ qualquer pode ser descrito em termos das componentes nos eixos $x$, $y$, e $z$ como
\begin{equation}
    \vec{a} = a_x \,\versi + a_y\,\versj + a_z\,\versk.
\end{equation}
%
Um exemplo disso é o vetor $\vec{a} = \np{1,0} \,\versi + \np{1,0}\,\versj + \np{1,0}\,\versk$ mostrado na Figura~\ref{Fig:vetores_unitários}\footnote{Note que em uma representação bidimensional de um sistema tridimensional a visualização das componentes pode ser mais difícil. O ideal seria girar a figura e a visualizar em diversos ângulos.}

%%%%%%%%%%%%%%%%%%%%%%%%%%%%%%%%%
\paragraph{Notação módulo-ângulo: coordenadas esféricas}
%%%%%%%%%%%%%%%%%%%%%%%%%%%%%%%%%

Assim como no caso bidimensional, um vetor pode ser denotado em relação a uma base através dos valores numéricos das componentes, ou através do módulo do vetor e de ângulos em relação a eixos de referência. Como temos uma base tridimensional, precisamos de mais um ângulo entre o vetor e um dos eixos. Em geral, se utiliza um ângulo  $\theta$ entre o vetor e o eixo $z$, e um ângulo $\phi$ entre a projeção do vetor no plano $xy$ e o eixo $x$. Tal sistema de referência é conhecido como \emph{sistema de coordenadas esféricas}.

% Segue figura retirada de: http://www.texample.net/tikz/examples/the-3dplot-package/
% com alterações

\begin{marginfigure}[-1.5cm]
\centering
%% Copyright 2009 Jeffrey D. Hein
%
% This work may be distributed and/or modified under the
% conditions of the LaTeX Project Public License, either version 1.3
% of this license or (at your option) any later version.
% The latest version of this license is in
%   http://www.latex-project.org/lppl.txt
% and version 1.3 or later is part of all distributions of LaTeX
% version 2005/12/01 or later.
%
% This work has the LPPL maintenance status `maintained'.
% 
% The Current Maintainer of this work is Jeffrey D. Hein.
%
% This work consists of the files 3dplot.sty and 3dplot.tex

%Description
%-----------
%3dplot.tex - an example file demonstrating the use of the 3dplot.sty package.

%Created 2009-11-07 by Jeff Hein.  Last updated: 2009-11-09
%----------------------------------------------------------

%Update Notes
%------------

%2009-11-07: Created file along with 3dplot.sty package

%Angle Definitions
%-----------------

%set the plot display orientation
%synatax: \tdplotsetdisplay{\theta_d}{\phi_d}
\tdplotsetmaincoords{60}{110}

%define polar coordinates for some vector
%TODO: look into using 3d spherical coordinate system
\pgfmathsetmacro{\rvec}{.8}
\pgfmathsetmacro{\thetavec}{30}
\pgfmathsetmacro{\phivec}{60}

%start tikz picture, and use the tdplot_main_coords style to implement the display 
%coordinate transformation provided by 3dplot
\begin{tikzpicture}[scale=3,tdplot_main_coords, >=Stealth]

%set up some coordinates 
%-----------------------
\coordinate (O) at (0,0,0);

%determine a coordinate (P) using (r,\theta,\phi) coordinates.  This command
%also determines (Pxy), (Pxz), and (Pyz): the xy-, xz-, and yz-projections
%of the point (P).
%syntax: \tdplotsetcoord{Coordinate name without parentheses}{r}{\theta}{\phi}
\tdplotsetcoord{P}{\rvec}{\thetavec}{\phivec}

%draw figure contents
%--------------------

%draw the main coordinate system axes
\draw[thick,->] (0,0,0) -- (1,0,0) node[anchor=north east]{$x$};
\draw[thick,->] (0,0,0) -- (0,1,0) node[anchor=north west]{$y$};
\draw[thick,->] (0,0,0) -- (0,0,1) node[anchor=south]{$z$};

%draw a vector from origin to point (P) 
\draw[-stealth,thick] (O) -- (P) node[below right]{$\vec{a}$};

%draw projection on xy plane, and a connecting line
\draw[dashed] (O) -- (Pxy);
\draw[dashed] (P) -- (Pxy);

%draw the angle \phi, and label it
%syntax: \tdplotdrawarc[coordinate frame, draw options]{center point}{r}{angle}{label options}{label}
\tdplotdrawarc{(O)}{0.2}{0}{\phivec}{anchor=north}{$\phi$}


%set the rotated coordinate system so the x'-y' plane lies within the
%"theta plane" of the main coordinate system
%syntax: \tdplotsetthetaplanecoords{\phi}
\tdplotsetthetaplanecoords{\phivec}

%draw theta arc and label, using rotated coordinate system
\tdplotdrawarc[tdplot_rotated_coords]{(0,0,0)}{0.5}{0}{\thetavec}{anchor=south west}{$\theta$}

\end{tikzpicture}
\caption{Ângulos em um sistema de coordenadas esféricas.\label{Fig:ExemploVetor3D}}
\end{marginfigure}

Podemos determinar as componentes $a_x$, $a_y$, e $a_z$ em termos de $a$, $\theta$, e $\phi$ através das definições das funções trigonométricas:
\begin{description}
    \item[Componente $a_x$:] Podemos determinar o comprimento $a_{xy}$ da projeção do vetor $\vec{a}$ no plano $xy$ fazendo
    \begin{equation}
        a_{xy} = a \sen\theta.
    \end{equation}
    A projeção $a_x$ é então dada por
    \begin{align}
        a_x &= a_{xy} \cos\phi \\
        &= a \sen\theta\cos\phi.
    \end{align}
    
    \item[Componente $a_y$:] Podemos obter $a_y$ usando de $a_{xy}$ através de
    \begin{align}
        a_y &= a_{xy} \sen\phi \\
        &= a \sen\theta\sen\phi.
    \end{align}
    
    \item[Componente $a_x$:] A componente no eixo $z$ pode ser obtida diretamente através de
    \begin{equation}
        a_z = a \cos \theta.
    \end{equation}
\end{description}

Para obtermos os ângulos através das componentes, basta utilizar a função arco tangente (função inversa da tangente), de onde obtemos:
\begin{align}
    \theta &= \arccos \frac{a_z}{a} \\
    \phi &= \arctan \frac{a_y}{a_x}.
\end{align}
%
Para obtermos uma relação para o módulo, podemos utilizar as expressões deduzidas acima para escrever
\begin{align}
    a_x^2 + a_y^2 + a_z^2 &= a^2 \sen^2\theta\cos^2\phi + a^2 \sen^2\theta \sen^2\phi + a^2 \cos^2 \\
    &= a^2 [\sen^2\theta(\sen^2\phi + \cos^2\phi) + \cos^2\theta] \\
    &= a^2,
\end{align}
%
onde usamos a propriedade de que $\sen^2\alpha + \cos^2\alpha = 1$ para qualquer ângulo $\alpha$. Finalmente, podemos escrever
\begin{equation}
    a = \sqrt{a_x^2 + a_y^2 + a_z^2}.
\end{equation}

%%%%%%%%%%%%%%%%%%%%%%%%%%%%%%%%
\subsection{Projeções completas}
%%%%%%%%%%%%%%%%%%%%%%%%%%%%%%%%

Na Figura~\ref{Fig:ProjCompleta}, mostramos a projeção do vetor $\vec{a}$ em dois sistemas de referências diferentes. No sistema composto pelos eixos $x$ e $y$, verificamos que as componentes são
\begin{align}
    a_x &= a \cos\theta \\
    a_y &= a \sen\theta.
\end{align}
%
Quais são as componentes do vetor $\vec{a}$ nos eixos $x'$ e $y'$?

\begin{marginfigure}[4cm]
\centering
\begin{tikzpicture}[>=Stealth, scale = 1.3]

    \draw[->, dashed] (0,0) coordinate (origin) -- (2,0) node[below left]{$x$} coordinate (x);
    \draw[->, dashed] (0,0) -- (0,2) node[below left]{$y$};
    
    \draw[->, thick] (0,0) -- node[above, sloped]{$\vec{a}$}(35:1) coordinate (a);
    \draw[dotted] (35:1) -- +(0,-0.5736);
    \draw[dotted] (35:1) -- +(-0.8192,0);
    
    \draw[|-|] (0,-0.2) -- node[below]{$a_x$} +(0.8192,0);
    \draw[|-|] (-0.2,0) -- node[left]{$a_y$} +(0,0.5736);
    
    \draw[->, densely dotted] (0,0) -- (35:2) node[right]{$x'$};
    \draw[->, densely dotted] (0,0) -- (125:2) node[left]{$y'$};
    
    \pic[draw, "$\theta$", angle eccentricity = 1.5] {angle = x--origin--a};
    
\end{tikzpicture}
\caption{Projeção de um vetor $\vec{a}$ em dois sistemas de referência diferentes. No sistema de referência $x'-y'$ as componentes são dadas por $a_{x'} = a$ e $a_{y'} = 0$.\label{Fig:ProjCompleta}}
\end{marginfigure}

Verificamos através da figura que o vetor $\vec{a}$ aponta na mesma direção do eixo $x'$, portanto, podemos utilizar a decomposição com um ângulo de \np[\tcdegree]{0}:
\begin{align}
    a_{x'} &= a \cos\np[\tcdegree]{0} \\
    a_{y'} &= a \sen\np[\tcdegree]{0},
\end{align}
%
o que resulta em
\begin{align}
    a_{x'} &= a \\
    a_{y'} &= 0.
\end{align}
%
Concluímos, portanto, que se um vetor aponta na mesma direção que um eixo de referência, a projeção de tal vetor nesse eixo é igual ao próprio módulo do vetor. Por outro lado, se o vetor é perpendicular ao eixo, a projeção nele é nula.

%%%%%%%%%%%%%%%%%%%%%%%%%%%%%%%%
\subsection{Projeções negativas}
%%%%%%%%%%%%%%%%%%%%%%%%%%%%%%%%

\begin{marginfigure}[3cm]
\centering
\begin{tikzpicture}

    \draw[-Stealth] (0,0) -- (3,0) node[below left]{$x$};
    \draw[-Stealth] (0,0) -- (0,2.5) node[below left]{$y$};
    
    \draw[-Stealth, thick] (2.25,0.5) coordinate (ai) -- node[above right]{$\vec{a}$} (1,2) coordinate (af);
    \draw[dotted] (2.25, 0) -- (2.25,0.5) -- (0,0.5);
    \draw[dotted] (1,0) -- (1,2) -- (0,2);
    
    \draw[thick, -Stealth] (2.25,0) -- (1,0);
    \draw[thick, -Stealth] (0,0.5) coordinate (ayi) -- (0,2);
    
    \draw[|<->|] (2.25,-0.3) -- node[below]{$a_x$} (1,-0.3);
    \draw[|<->|] (-0.3,0.5) -- node[left]{$a_y$} (-0.3,2);
    
    \draw[loosely dotted] (ai) -- ++(1,0) coordinate (eixop);
    
    \pic[draw, "$\theta$", angle eccentricity = 1.9, angle radius = 2.5mm] {angle = eixop--ai--af};
    \pic[draw, "$\phi$", angle eccentricity = 1.5] {angle = af--ai--ayi}; 
    
\end{tikzpicture}
\caption{Podemos utilizar tanto o ângulo $\phi$, quanto o ângulo $\theta$ em relação ao eixo $x$ para determinar as componentes de um vetor, porém devemos estar atentos ao sinal. \label{Fig:DecompVetorCompNegativa}}
\end{marginfigure}

Algo que é bastante importante notarmos é o fato de que muitos vetores têm uma ou mais componentes negativas. Isso está ligado ao simples fato de que para os gerar a partir dos vetores da base escolhida, precisamos que as componentes tenham valores negativos. Por isso, é importante que ao decompor um vetor em termos de uma base/sistema de referência, saibamos identificar se tais sinais são necessários. Se determinarmos os valores das componentes através de
\begin{align}
    a_x &= a \cos\theta \\
    a_y &= a\sen\theta,
\end{align}
%
onde o ângulo $\theta$ é medido a partir do eixo $x$, no sentido anti-horário, os sinais serão ``gerados'' apropriadamente pelas próprias funções trigonométricas.

Existem dois casos, no entanto, em que simplesmente utilizar as expressões acima não será suficiente. O primeiro deles está relacionado ao fato de que muitas vezes não temos, ou não é conveniente utilizar o ângulo $\theta$ como definido acima.\footnote{Em geral a informação que temos, ou que é fácil de se verificar, é um ângulo entre o vetor e um dos eixos do sistema de referência, sendo que costumeiramente se escolhe um ângulo que seja menor que \degree{90}.} Se, por exemplo, tivermos o ângulo $\phi$ na Figura~\ref{Fig:DecompVetorCompNegativa}, podemos determinar os \emph{valores numéricos} das projeções do vetor ao usar tal ângulo, porém não obteremos os sinais adequados. Na figura, claramente a componente $a_x$ é negativa, uma vez que $\degree{90} < \theta < \degree{180}$, logo\footnote{Lembre-se que $a$ representa o módulo do vetor, o que é sempre positivo.}
\begin{equation}
    a_x = a\cos\theta < 0.
\end{equation}
%
Por outro lado, podemos verificar que
\begin{equation}
    |a_x| = a \cos\phi,
\end{equation}
%
o que pode ser obtido através das relações trigonométricas. Como temos que $\degree{0} < \phi < \np[\tcdegree]{90}$, o que implica que $a \cos\phi > 0$, utilizamos a notação de módulo, pois sabemos que $a_x < 0$. Logo
\begin{equation}\label{Eq:ComponenteComSinalAdicionadoManualmente}
    a_x = - a \cos\phi,
\end{equation}
%
onde precisamos inserir o sinal ``manualmente'' para que a componente tenha o sinal negativo necessário. Isso se deve ao fato de que estamos utilizando um método geométrico para determinar a componente ---~estamos usando a projeção~---, porém ele não é capaz de descrever o seu sinal.

A maneira mais simples e intuitiva de verificar o sinal adequado para a componente do vetor quando utilizamos um ângulo diferente do $\theta$ na Figura~\ref{Fig:DecompVetorCompNegativa} é observarmos a ``sombra'' da projeção no eixo que estamos analisando. Se ela aponta no sentido negativo do eixo, sabemos que a componente deve ser negativa; se ela aponta no sentido positivo, sabemos que ela é positiva. O sinal negativo deve ser então introduzido ``manualmente'', da mesma maneira que fizemos na Equação~\eqref{Eq:ComponenteComSinalAdicionadoManualmente} acima.

O segundo caso em que devemos ter atenção à questão do sinal é quando temos um sistema ortogonal ``heterodoxo''. Rigorosamente, devemos ter que o semieixo positivo $y$ aponta em uma direção e sentido que faz um ângulo de \degree{90} em relação ao semieixo $x$ positivo, sendo que o ângulo deve ser medido no sentido anti-horário. Muitas vezes, no entanto, adotamos eixos que não respeitam essa regra. Nesses casos, mesmo que utilizemos as funções trigonométricas para o ângulo $\theta$ medido a partir do eixo $x$, em sentido anti-horário, podemos ter problemas de sinal. 

\begin{marginfigure}
\centering
\begin{tikzpicture}[>=Stealth]

    \draw[->] (0,0) coordinate (O) -- (3,0) node[below left]{$x$} coordinate (X);
    \draw[->] (0,1.5) -- (0,-1.5) node[above left]{$y$};
    
    \draw[thick, ->] (0,0) -- (30:2) node[below]{$\vec{a}$} coordinate (A);
    
    \pic [draw, "$\gamma$", angle eccentricity = 1.5, angle radius = 5mm]{angle = X--O--A};
    
\end{tikzpicture}
\caption{Vetor $\vec{a}$ em relação aos eixos do sistema de referência. \label{Fig:Ex:DecompNegativaNaoOrtodoxa}}
\end{marginfigure}

Na Figura~\ref{Fig:Ex:DecompNegativaNaoOrtodoxa} temos um sistema de referência onde adotamos o eixo vertical $y$ apontando para baixo. Nesse caso, ao calcularmos o valor de $a_y$ utilizando o ângulo $\gamma$ mostrado, obtemos
\begin{equation}
    a_y = a \cos\gamma > 0,
\end{equation}
%
porém a projeção do vetor no eixo $y$ aponta no sentido negativo. A expressão acima, portanto, está incorreta, pois resulta em uma projeção com o sinal inadequado. Devemos inserir o sinal manualmente:
\begin{equation}
    a_y = -a \cos\gamma > 0.
\end{equation}
%
Novamente, a maneira mais simples de determinar a projeção corretamente é utilizar um ângulo entre \degree{0} e \degree{90}, utilizar a função trigonométrica adequada e então verificar o sinal através da ``sombra'' do vetor no eixo.


%%%%%%%%%%%%%%%%%%%%%%%%%%%%%%%%%%%%%%%%%%%%%%%%%%%%%%%%%%%%%%%%%%%%%%%%%
\paragraph{Exemplo: Decomposição de um vetor com uma componente negativa}
%%%%%%%%%%%%%%%%%%%%%%%%%%%%%%%%%%%%%%%%%%%%%%%%%%%%%%%%%%%%%%%%%%%%%%%%%

\begin{quote}
    O vetor $\vec{a}$ mostrado na Figura~\ref{Fig:Ex:DecompNegativa} tem módulo \np[m]{2,0} e aponta em uma direção que faz um ângulo $\phi = \np[\tcdegree]{30}$ abaixo do eixo $x$. Determine as componentes $a_x$ e $a_y$ do vetor.
\end{quote}

\begin{marginfigure}
\centering
\begin{tikzpicture}[>=Stealth]

    \draw[->] (0,0) coordinate (O) -- (3,0) node[below left]{$x$} coordinate (X);
    \draw[->] (0,-1.5) -- (0,1.5) node[below left]{$y$};
    
    \draw[thick, ->] (0,0) -- (-30:2) node[below]{$\vec{a}$} coordinate (A);
    
    \pic [draw, "$\phi$", angle eccentricity = 1.5, angle radius = 5mm]{angle = A--O--X};
    
\end{tikzpicture}
\caption{Vetor $\vec{a}$ em relação aos eixos do sistema de referência. \label{Fig:Ex:DecompNegativa}}
\end{marginfigure}

Podemos determinar o ``tamanho'' das projeções através das funções trigonométricas seno e cosseno, utilizando o ângulo $\phi$. Obtemos então,
\begin{align}
    a_x &= a\cos\phi \\
    a_y &= a\sen\phi.
\end{align}
%
Sabemos, porém, que a componente $a_y$ deve ser negativa, uma vez que a projeção aponta no sentido negativo do eixo. Como o módulo de $a$ é maior que zero, assim como o seno de \np[\tcdegree]{30}, devemos inserir o sinal ``manualmente'':
\begin{equation}
    a_y = -a\sen\phi.
\end{equation}
%
Calculando os valores numéricos, obtemos
\begin{align}
    a_x &= a\cos\phi & a_y &= -a\sen\phi \\
    &= (\np[m]{2,0})\cdot(\cos\np[\tcdegree]{30}) & &= -(\np[m]{2,0})\cdot(\sen\np[\tcdegree]{30}) \\
    &\approx \np[m]{1.732} & &= \np[m]{-1,0}
\end{align}

\begin{marginfigure}
\centering
\begin{tikzpicture}[>=Stealth]

    \draw[->] (0,0) coordinate (O) -- (3,0) node[below left]{$x$} coordinate (X);
    \draw[->] (0,-1.5) -- (0,1.5) node[below left]{$y$};
    
    \draw[thick, ->] (0,0) -- (-30:2) node[below]{$\vec{a}$} coordinate (A);
    
    \pic [draw, "$\phi$", angle eccentricity = 1.5, angle radius = 5mm]{angle = A--O--X};
    \pic [draw, "$\theta$", angle eccentricity = 1.5, angle radius = 3mm]{angle = X--O--A};
    
\end{tikzpicture}
\caption{Vetor $\vec{a}$ em relação aos eixos do sistema de referência, mostrando também o ângulo $\theta$ entre o vetor e o eixo $x$, medido no sentido anti-horário. \label{Fig:Ex:DecompNegativaAngTheta}}
\end{marginfigure}

Alternativamente, podemos simplesmente determinar o valor do ângulo $\theta$ (veja a Figura~\ref{Fig:Ex:DecompNegativaAngTheta}) e utilizar as Expressões~\eqref{Eq:CompX} e~\eqref{Eq:CompY}:
\begin{align}
    \theta &= \np[\tcdegree]{360} - \phi \\
    &= \np[\tcdegree]{360} - \np[\tcdegree]{30} \\
    &= \np[\tcdegree]{330},
\end{align}
%
de onde obtemos
\begin{align}
    a_x &= a\cos\phi & a_y &= -a\sen\phi \\
    &= (\np[m]{2,0})\cdot(\cos\np[\tcdegree]{330}) & &= (\np[m]{2,0})\cdot(\sen\np[\tcdegree]{330}) \\
    &\approx \np[m]{1.732} & &= \np[m]{-1,0}
\end{align}

%%%%%%%%%%%%%%%%%%%%%%%%%%%%%%%%%%%%%%%%%%%%%%%%%%%%%%%%%%%%%
\paragraph{Exemplo: Projeção de um vetor em um eixo qualquer}
%%%%%%%%%%%%%%%%%%%%%%%%%%%%%%%%%%%%%%%%%%%%%%%%%%%%%%%%%%%%%

\begin{quote}
    Na Figura~\ref{Fig:Ex:ProjecaoEixoQQ} temos um vetor $\vec{a}$, cujas coordenadas em relação ao sistema de referência $xy$ mostrado são $a_x = \np{-0.866}$ e $a_y = \np{0.5}$. Determine a projeção do vetor na direção do eixo tracejado mostrado na figura, sabendo que o ângulo $\alpha$ indicado é de \np[\tcdegree]{30}.
\end{quote}

\begin{marginfigure}
\centering
\begin{tikzpicture}[>=Stealth,scale = 1.5]

    \draw[->] (-1,0) -- (1,0) node[below left]{$x$};
    \draw[->] (0,-1) -- (0,1) node[below left]{$y$};
    
    \draw[dashed] (0,0) coordinate (O) -- (120:2) coordinate (P) -- +(0,-1) coordinate (Q);
    
    \pic [draw, "$\alpha$", angle eccentricity = 1.5, angle radius = 5mm]{angle = Q--P--O};
    
    \draw[->, thick] (0,0) --  node[below]{$\vec{a}$} (150:1);
\end{tikzpicture}
\caption{Disposição do vetor $\vec{a}$ em relação ao sistema de coordenadas e direção do eixo sobre o qual queremos determinar a projeção de tal vetor.\label{Fig:Ex:ProjecaoEixoQQ}}
\end{marginfigure}

Para que possamos determinar a projeção do vetor na direção desejada precisamos de duas informações: o módulo do vetor e o ângulo formado entre ele e tal direção. A determinação do módulo é bastante simples, bastando utilizar a Expressão~\eqref{Eq:ModAPartirDeComp}:
\begin{align}
    a &= \sqrt{a_x^2 + a_y^2} \\
    &= \sqrt{(-\np{0,866})^2 + \np{0.5}^2} \\
    &\approx 1.
\end{align}

\begin{marginfigure}[2cm]
\centering
\begin{tikzpicture}[>=Stealth, scale = 1.5]

    \draw[->] (-1,0) -- (1,0) coordinate (X) node[below left]{$x$};
    \draw[->] (0,-1) -- (0,1) node[below left]{$y$};
    
    \draw[dashed] (0,0) coordinate (O) -- (120:2) coordinate (P) -- +(0,-1) coordinate (Q);
    
    \draw[->, thick] (0,0) --  node[below]{$\vec{a}$} (150:1) coordinate (A);
        
    \pic [draw, "$\alpha$", angle eccentricity = 1.5, angle radius = 5mm]{angle = Q--P--O};
    \pic [draw, "$\theta$", angle eccentricity = 1.5, angle radius = 3mm]{angle = X--O--A};
    \pic [draw, "$\gamma$", angle eccentricity = 1.3, angle radius = 7mm]{angle = X--O--P};
    \pic [draw, "$\beta$", angle eccentricity = 1.3, angle radius = 9mm]{angle = P--O--A};
 
\end{tikzpicture}
\caption{Definição dos ângulos $\beta$ e $\gamma$.\label{Fig:Ex:ProjecaoEixoQQ2}}
\end{marginfigure}

Observando a Figura~\ref{Fig:Ex:ProjecaoEixoQQ2}, verificamos que podemos obter o ângulo $\beta$ entre o vetor e a reta tracejada pela diferença entre o ângulo $\theta$ e o ângulo $\gamma$, onde o primeiro é o ângulo entre o vetor e o eixo $x$ positivo e o segundo o ângulo entre a reta tracejada e o eixo $x$. Para determinarmos o ângulo $\theta$, basta utilizarmos a Expressão~\eqref{Eq:AngaPartirDeComp}:
\begin{align}
    \theta &= \arctan \left(\frac{a_y}{a_x}\right) \\
    &= \arctan \left(\frac{\np{-0.866}}{\np{0.5}}\right) \\
    &\approx \np[\tcdegree]{-30} \\
    &= \np[\tcdegree]{150}.
\end{align}
%
Note que ao utilizarmos uma calculadora para determinar o ângulo, é comum que a função $\arctan$ retorne um valor negativo para o ângulo, o que equivale nesse caso ao ângulo entre o eixo $x$ e a continuação da reta pontilhada no quarto quadrante, medido no sentido horário ---~veja a Figura~\ref{Fig:Ex:AngNegativoConverter}~---. Para determinarmos o ângulo $\theta$ que desejamos, basta somar \np[\tcdegree]{180}.

\begin{marginfigure}
\centering
\begin{tikzpicture}[>=Stealth, scale = 1.75]

    \draw[-Stealth] (-1,0) -- (1,0) coordinate (X) node[below left]{$x$};
    \draw[-Stealth] (0,-1) -- (0,1) node[below left]{$y$};
    
    \draw[dashed] (120:1) coordinate (A) -- (-60:1) coordinate (B);
    
    \coordinate (O) at (0,0);
        
    \pic [->, draw, "$\alpha$", angle eccentricity = 1.4, angle radius = 4mm]{angle = X--O--A};
    \pic [<-, draw, "$\beta$", angle eccentricity = 1.4, angle radius = 6mm]{angle = B--O--X};
 
\end{tikzpicture}
\caption{Os ângulos $\alpha$ e $\beta$ estão relacionados através de $\alpha - \beta = \np[\tcdegree]{180}$. Note que $\beta$ é um ângulo negativo, pois é medido no sentido horário.\label{Fig:Ex:AngNegativoConverter}}
\end{marginfigure}

Basta agora determinarmos o ângulo $\gamma$. Através da Figura~\ref{Fig:Ex:ProjecaoEixoQQ2}, podemos notar que o ângulo entre a reta tracejada e o eixo $y$ é igual a $\alpha$ ---~pois são alternos internos, veja a Figura~\ref{Fig:Ex:AlternosInternos}~---. Como o ângulo entre os eixos $x$ e $y$ é de \np[\tcdegree]{90}, temos que
\begin{equation}
    \gamma = \np[\tcdegree]{120}.
\end{equation}

\begin{marginfigure}
\centering
\begin{tikzpicture}[>=Stealth]

    \draw[name path = reta1] (-0.75,-2) coordinate (inf) -- (-0.75,2);
    \draw[name path = reta2] (0.75,-2) -- (0.75,2) coordinate (sup);
   
    \draw[name path = transversal, dashed] (120:2) -- (-60:2);
        
    \path[name intersections={of=reta1 and transversal}] (intersection-1) coordinate (a);
    \path[name intersections={of=reta2 and transversal}] (intersection-1) coordinate (b);
    
    \pic [draw, "$\alpha$", angle eccentricity = 1.4, angle radius = 5mm]{angle = inf--a--b};
    \pic [draw, "$\alpha$", angle eccentricity = 1.4, angle radius = 5mm]{angle = sup--b--a};
 
\end{tikzpicture}
\caption{Ângulos alternos internos são iguais.\label{Fig:Ex:AlternosInternos}}
\end{marginfigure}

\noindent{}Concluímos, portanto, que
\begin{equation}
    \beta = \np[\tcdegree]{30}.
\end{equation}

Finalmente, podemos determinar a projeção do vetor $\vec{a}$ na direção do eixo tracejado:
\begin{align}
    a_e &= a\cos\np[\tcdegree]{30} \\
    &\approx \np{0,866}.
\end{align}

%%%%%%%%%%%%%%%%%%%%%%%%%%%%%%%%%%%%%%%%%%%%%
\subsection{Operações através de componentes}
\label{Sec:OpAtravesDeComp}
%%%%%%%%%%%%%%%%%%%%%%%%%%%%%%%%%%%%%%%%%%%%%

Verificaremos agora uma propriedade importante dos vetores e que facilita o tratamento de sistemas que precisam ser descritos vetorialmente: a independência dos eixos\footnote[][-1cm]{Para que isso seja verdade, devemos considerar um sistema de referência ortogonal, que é o que sempre vamos fazer.} para as operações de soma, subtração, e multiplicação por escalar. Verificaremos que sempre que uma dessas três operações precisa ser feita entre dois vetores quaisquer, basta que façamos tal operação para as componentes do vetor em cada um dos eixos de referência.

\begin{marginfigure}[1cm]
\centering
\begin{tikzpicture}[>=Stealth, scale=1.0]

    \draw[->] (0,0) -- (3.5,0) node[below left]{$x$};
    \draw[->] (0,0) -- (0,2.5) node[below left]{$y$};

    \draw[->, thick] (0,0) -- node[below]{$\vec{a}$} (1.3,0.8);
    \draw[-{Stealth[right]}, thick] (1.3,0.8) -- node[below]{$\vec{b}$} (2,2);
    \draw[-{Stealth[left]}, thick] (0,0) -- node[above]{$\vec{c}$} (2,2);
    
    \draw[dotted] (1.3,0.8) -- (1.3,0);
    \draw[dotted] (1.3,0.8) -- (0,0.8);
    \draw[dotted] (2,2) -- (2,0);
    \draw[dotted] (2,2) -- (0,2);
    
    \draw[|-|] (0,-0.3) -- node[below]{$a_x$} (1.3,-0.3);
    \draw[|-|] (-0.3,0) -- node[left]{$a_y$} (-0.3, 0.8);
    
    \draw[-|] (1.3,-0.3) -- node[below]{$b_x$} (2,-0.3);
    \draw[-|] (-0.3, 0.8) -- node[left]{$b_y$} (-0.3,2);
    
    \draw[|-|] (0, -0.8) -- node[below]{$c_x$} (2,-0.8);
    \draw[|-|] (-0.8, 0) -- node[left]{$c_y$} (-0.8,2);
    
\end{tikzpicture}
\caption{Soma através de componentes vetoriais.\label{Fig:Soma_comp}}
\end{marginfigure}

Na Figura~\ref{Fig:Soma_comp}, temos uma soma geométrica de vetores. Temos também um sistema de referência sobre o qual projetamos as componentes dos vetores $\vec{a}$ e $\vec{b}$. Se tomarmos o vetor $\vec{c}$ vemos que
\begin{align}
  c_x &= a_x + b_x \\
  c_y &= a_y + b_y,
\end{align}
%
ou seja, podemos simplesmente somar as componentes dos vetores nos eixos para poder calcular as componentes do vetor resultante. Isto nos dá uma forma muito mais simples para realizar somas e subtrações de vetores. Se considerarmos ainda que a subtração nada mais é do que a soma de um vetor cujo sentido é invertido pelo sinal, concluímos que a subtração de dois vetores também pode ser realizada através das componentes, de forma que se
\begin{equation}
    \vec{d} = \vec{a} - \vec{b},
\end{equation}
%
então
\begin{align}
    d_x &= a_x - b_x \\
    d_y &= a_y - b_y.
\end{align}

Finalmente, ao multiplicarmos um vetor por um escalar, sabemos que o vetor resultante tem a mesma direção e sentido que o vetor original, porém seu módulo é alterado. Assim, se
\begin{equation}
    \vec{e} = \alpha \vec{a},
\end{equation}
%
então
\begin{align}
    e_x &= e \cos\theta \\
    e_y &= e \sen\theta,
\end{align}
%
porém temos que
\begin{equation}
    e = \alpha a,
\end{equation}
%
o que nos permite escrever
\begin{align}
    e_x &= \alpha a \cos\theta \\
    e_y &= \alpha a \sen\theta.
\end{align}
%
Na expressão acima, $a\cos\theta$ e $a\sen\theta$ são as próprias componentes do vetor $\vec{a}$ projetadas nos eixos de referência. Portanto, concluímos que
\begin{align}
    e_x &= \alpha a_x \\
    e_y &= \alpha a_y,
\end{align}
%
isto é, ao multiplicarmos um vetor por um escalar, podemos simplesmente multiplicar todas as suas componentes por tal escalar.

Note que verificamos nas expressões acima que as operações envolvendo as componentes de cada dimensão (eixo) ocorrem de forma que não é necessário saber informações acerca das demais dimensões. Isso significa que se os eixos são ortogonais, eles atuam independentemente uns dos outros.\footnote{Isso não é verdade para equações que envolvam o \emph{produto vetorial entre dois vetores}. Veremos tal produto somente no Capítulo~\ref{Chap:MomentoAngular}}. Isso será fundamental para que possamos simplificar a interpretação de movimentos bi e tridimensionais no Capítulo~\ref{Chap:MovimentoBidimensional}, mas tal resultado não se limita à cinemática.

%%%%%%%%%%%%%%%%%%%%%%%%%%%%%%%%%%%%%%%%%%%%%%%%%%%%%%%%%%%%%%
\subsection{Operações e equações através de vetores unitários}
%%%%%%%%%%%%%%%%%%%%%%%%%%%%%%%%%%%%%%%%%%%%%%%%%%%%%%%%%%%%%%

A grande vantagem de escrever os vetores em termos das componentes vetoriais é que podemos realizar cálculos de uma maneira bastante cômoda. Se temos a soma de dois vetores 
\begin{align}
  \vec{a} &= a_x \versi + a_y \versj + a_z \versk \\
  \vec{b} &= b_x \versi + b_y \versj + b_z \versk,
\end{align}
%
podemos escrever
\begin{equation}
  \vec{c} = a_x \versi + a_y \versj + a_z \versk + b_x \versi + b_y \versj + b_z \versk.
\end{equation}
%
Devido ao vetor unitário, podemos somar as componente,s colocando-o em evidência, obtendo
\begin{equation}
  \vec{c} = (a_x + b_x) \versi + (a_y + b_y) \versj + (a_z + b_z)\versk.
\end{equation}
%
Em casos mais complexos, como veremos mais adiante, a praticidade da notação de versores unitários se tornará mais evidente.
% TODO Legal seria um exemplo ou dois mostrando que é melhor

%%%%%%%%%%%%%%%%%%%%%%%%%%%%%%%%%%%%%%%%%%%%%%%%
\paragraph{Exemplo: Soma e subtração de vetores}
%%%%%%%%%%%%%%%%%%%%%%%%%%%%%%%%%%%%%%%%%%%%%%%%

\begin{quote}
    Dados dois vetores $\vec{a}$ e $\vec{b}$ tais que
        \begin{align}
            \vec{a} &= \np[m]{2,45}~\versi - \np[m]{4,32}~\versj \\
            \vec{b} &= \np[m]{1,67}~\versi + \np[m]{3,33}~\versj,
        \end{align}
    %
    determine as componentes, o módulo e o ângulo em relação ao eixo $x$ dos vetores $\vec{c}$ e $\vec{d}$, dados por
        \begin{align}
            \vec{c} &= \vec{a} + \vec{b} \\
            \vec{d} &= \vec{a} - \vec{b}.
        \end{align}
\end{quote}

Podemos determinar os vetores simplesmente realizando as operações de acordo com a notação de vetores unitários:
\begin{align}
    \vec{c} &= (\np[m]{2,45}~\versi - \np[m]{4,32}~\versj) + (\np[m]{1,67}~\versi + \np[m]{3,33}~\versj) \\
    &= (\np[m]{2,45} + \np[m]{1,67})\;\versi + (\np[m]{-4,32} + \np[m]{3,33})\;\versj \\
    &= \np[m]{4.12}~\versi - \np[m]{0,99}~\versj \\
    \\
    \vec{d} &= (\np[m]{2,45}~\versi - \np[m]{4,32}~\versj) - (\np[m]{1,67}~\versi + \np[m]{3,33}~\versj) \\
    &= (\np[m]{2,45} - \np[m]{1,67})\;\versi + (\np[m]{-4,32} - \np[m]{3,33})\;\versj \\
    &= \np[m]{0.78}~\versi - \np[m]{7.65}~\versj.
\end{align}
%
Note que as componentes dos vetores no eixo $x$ são os próprios valores numéricos que multiplicam o vetor unitário $\versi$, enquanto as componentes no eixo $y$ são os números que multiplicam o vetor unitário $\versj$.

Os módulos dos vetores são calculados através da Expressão~\eqref{Eq:ModAPartirDeComp} resultando em
\begin{align}
    c &= \sqrt{c_x^2 + c_y^2} \\
    &= \sqrt{(\np[m]{4.12})^2 +(- \np[m]{0,99})^2} \\
    &\approx \np[m]{4.24}
\\
    d &= \sqrt{d_x^2 + d_y^2} \\
    &= \sqrt{(\np[m]{0.78})^2 +(- \np[m]{7.65})^2} \\
    &\approx \np[m]{7.69}
\end{align}
%
Finalmente, os ângulos são dados a partir da Expressão~\eqref{Eq:AngaPartirDeComp}:
\begin{align}
    \theta_c &= \arctan\left(\frac{c_y}{c_x}\right) \\
    &= \arctan\left(\frac{- \np[m]{0,99}}{\np[m]{4.12}}\right) \\
    &\approx \np[\tcdegree]{-13,51}
    \\
    \theta_d &= \arctan\left(\frac{d_y}{d_x}\right) \\
    &= \arctan\left(\frac{- \np[m]{7.65}}{\np[m]{0.78}}\right) \\
    &\approx \np[\tcdegree]{-84,18}.
\end{align}
%
Note que ambos os ângulos são negativos, o que significa simplesmente que eles são medidos a partir do eixo $x$, no sentido horário.

%%%%%%%%%%%%%%%%%%%%%%
%\section{Questionário}
%%%%%%%%%%%%%%%%%%%%%%

%\begin{question}[type={exam}]
%Uma questão.
%\end{question}
