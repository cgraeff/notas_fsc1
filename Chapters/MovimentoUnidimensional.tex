%%%%%%%%%%%%%%%%%%%%%%%%%%%%%%%%%%
\chapter{Movimento Unidimensional}
\label{Chap:MovimentoUnidimensional}
%%%%%%%%%%%%%%%%%%%%%%%%%%%%%%%%%%

%\minitoc

%\clearpage

\begin{fullwidth}
{\it
O primeiro passo para que possamos estudar a mecânica é a definição das variáveis físicas que descrevem o movimento dos corpos e a caracterização de tais grandezas como funções do tempo. Vamos definir precisamente posição, velocidade e aceleração, estudando a relação entre tais grandezas em casos simples. 

Nos capítulos seguintes, veremos que tais grandezas são vetoriais, ou seja, têm um valor, uma direção e um sentido. Além disso, veremos que a aceleração está ligada à força a que um corpo está sujeito o que nos dará uma forma de prever seu movimento a partir de observações gerais acerca das circustâncias em que o corpo está inserido.
}
\end{fullwidth}

%%%%%%%%%%%%%%%%%%%%
\section{Introdução}
%%%%%%%%%%%%%%%%%%%%

Diversos sistemas físicos interessantes exibem movimento. A \emph{cinemática} é a área da física que se preocupa em descrever o movimento sem se preocupar com a causa de tal fenômeno: Estamos interessados em descrever a série de posições diferentes ocupadas por um corpo qualquer no espaço conforme o tempo progride. Além disso, também estamos interessados em descrever como é essa evolução temporal da posição. Para isso, basta definir três grandezas principais --~a posição, a velocidade, e a aceleração~--.

Apesar de termos uma noção cotidiana de tais grandezas, veremos que em alguns casos teremos definições que diferem dessas noções devido ao fato de que elas não são úteis ou precisas. Como estamos interessados em uma descrição \emph{quantitativa}, utilizaremos a linguagem matemática como base para tais definições, nos valendo de vários objetos matemáticos (números, equações, funções, gráficos, limites, vetores, etc.), bem como de técnicas para trabalhar com tais objetos.

O tratamento matemático utilizado neste texto é propositalmente simplificado, sendo adequado a um nível de Física Básica do ensino superior. Sendo assim, o texto considera que o leitor não tem conhecimento de Cálculo Diferencial e Integral, mas que está cursando uma disciplina sobre o assunto. Tendo isso em mente, ferramentas de Cálculo serão apresentadas e empregadas aos poucos, porém sempre no sentido de apresentar resultados importantes, não sendo exigidos como um conhecimento do aluno\footnote{Algumas seções opcionais podem exigir um conhecimento sólido em Cálculo.}. Neste primeiro capítulo, são exigidos como conhecimento prévio \emph{operações básicas, equações, funções, gráficos, e área de figuras planas} somente.

%%%%%%%%%%%%%%%%%%%%%%%%%%%%%%%%%%
\section{Movimento unidimensional}
%%%%%%%%%%%%%%%%%%%%%%%%%%%%%%%%%%

Definimos como sendo unidimensional o movimento que ocorre ao longo de uma reta, que denominamos como \emph{direção} do movimento. Essa definição é útil por ser simples e -- como veremos nos capítulos seguintes -- é capaz de fornecer uma descrição geral ao simplesmente adicionarmos mais dois eixos de movimento.

Trataremos em todos os capítulos apenas movimentos de \emph{corpos rígidos}, isto é, corpos cujas partes que o constituem não se movem em relação umas às outras. Para tais corpos, podemos separar o movimento em uma translação do \emph{centro de massa} e uma rotação em torno do centro massa\footnote[][-2cm]{Essa separação é conhecida como Teorema de Mozzi-Chasles.}. O centro de massa é um ponto que substitui o sistema para fins de determinação da translação do corpo, sendo que para corpos simétricos e de densidade uniforme ele se localiza no centro do corpo, como veremos no Capítulo~\ref{Chap:CentroDeMassaEMomentoLinear}. Trataremos as rotações somente no Capítulo~\ref{Chap:Rotacoes}, nos preocupando somente com a translação do centro de massa até lá.\footnote[][-3cm]{Uma maneira equivalente é tratar o corpo como uma partícula --~isto é, um corpo de dimensões desprezíveis~--, o que efetivamente elimina a rotação do corpo.}

%%%%%%%%%%%%%%%%%%%%%%%%%%%%%%%%
\section{Posição e Deslocamento}
%%%%%%%%%%%%%%%%%%%%%%%%%%%%%%%%

%%%%%%%%%%%%%%%%%%%%
\subsection{Posição}
%%%%%%%%%%%%%%%%%%%%

O primeiro passo para que possamos determinar a posição de um corpo  é verificar qual é a \emph{direção} onde ocorre o movimento. Podemos então colocar um objeto em um ponto qualquer de tal reta\footnote[][-3cm]{Como estamos tratando de um movimento unidimensional, ou seja, o movimento ao longo de uma reta, é natural que a direção do espaço seja simplesmente uma linha reta. Verificaremos no entanto que no caso tridimensional, poderemos descrever o movimento através de vetores, sendo que a direção nesse caso será uma das propriedades dos vetores.}:

\begin{figure}
\centering
\begin{tikzpicture}[
     interface/.style={
        % superfície
        postaction={draw,decorate,decoration={border,angle=-45,
                    amplitude=0.2cm,segment length=2mm}}},
    ]
    % Interface
    \draw[line width=.5pt,interface](-4,0)--(4,0);
    
    % bloco
    \draw[pattern=north west lines, pattern color=gray] (2.5,0) rectangle (3.5,1);

\end{tikzpicture}
\caption{Corpo que ocupa uma posição qualquer ao longo de uma reta.}
\end{figure}

\noindent{}Claramente, tal descrição é insuficiente. Para determinar a posição do corpo, precisamos de um \emph{ponto de referência}. A partir desse ponto, podemos então determinar a posição medindo a distância entre ele e o corpo:

\begin{figure}
\centering
\begin{tikzpicture}[
     interface/.style={
        % superfície
        postaction={draw,decorate,decoration={border,angle=-45,
                    amplitude=0.2cm,segment length=2mm}}},
    ]
    % Interface
    \draw[line width=.5pt,interface](-4,0)--(4,0);
    
    % bloco
    \draw[pattern=north west lines, pattern color=gray] (2.5,0) rectangle (3.5,1);
    \fill (3,0.5) circle (1pt);

    % origem
    \draw[fill] (0,0) circle (2pt);
        
    % distância
    \draw[|-|] (0,1.2)--node[above]{$D$}(3,1.2);
\end{tikzpicture}
\caption{Podemos utilizar um ponto de referência para ajudar a determinar a posição de um objeto.}
\end{figure}

Tal descrição ainda é insuficiente, pois podemos ter outro objeto que pode estar à mesma distância da origem:

\begin{figure}
\centering
\begin{tikzpicture}[
     interface/.style={
        % superfície
        postaction={draw,decorate,decoration={border,angle=-45,
                    amplitude=0.2cm,segment length=2mm}}},
    ]
    % Interface
    \draw[line width=.5pt,interface](-4,0)--(4,0);
    
    % bloco
    \draw[pattern=north west lines, pattern color=gray] (2.5,0) rectangle (3.5,1);
    \fill (3,0.5) circle (1pt);
    
    \draw[pattern=north west lines, pattern color=gray, dotted] (-2.5,0) rectangle (-3.5,1);
    \fill[gray] (-3,0.5) circle (1pt);

    % origem
    \draw[fill] (0,0) circle (2pt);
        
    % distância
    \draw[|-|] (0,1.2)--node[above]{$D$}(3,1.2);
    \draw[|-|,gray] (0,1.2)--node[above]{$D$}(-3,1.2);
\end{tikzpicture}
\caption{Somente as informações de direção e de distância não são suficientes para determinar a posição.}
\end{figure}

\noindent{}Podemos definir dois \emph{sentidos} na figura acima: à esquerda da origem, ou à direita dela. Com essas três informações -- direção, módulo\footnote{Módulo se refere ao valor numérico da medida de distância entre a origem e a posição do corpo.}, e sentido -- podemos determinar com exatidão a posição de um corpo qualquer.

\pagebreak
Podemos denotar o sentido por um sinal se adotarmos a \emph{reta real} para descrever a posição:

\begin{figure}
\centering
\begin{tikzpicture}[
     >=Stealth,
     interface/.style={
        % superfície
        postaction={draw,decorate,decoration={border,angle=-45,
                    amplitude=0.2cm,segment length=2mm}}},
    ]
    % Interface
    \draw[line width=.5pt,interface](-4,0)--(4,0);
    
    % bloco
    \draw[pattern=north west lines, pattern color=gray] (2.5,0) rectangle (3.5,1);
    \fill (3,0.5) circle (1pt);
    
    \draw[pattern=north west lines, pattern color=gray] (-2.5,0) rectangle (-3.5,1);
    \fill (-3,0.5) circle (1pt);

    % origem
    \draw[fill] (0,0) circle (2pt);
        
    % reta real
    \draw[->] (-4.5,-0.5)--(4.5,-0.5) node[below]{$x$};
    \foreach \x in {0,...,4}
        \draw (\x cm,-0.4) -- (\x cm,-0.6)
            node[below]{\small$\x$};
    
    \foreach \x in {1,...,4}
        \draw (-\x cm,-0.4) -- (-\x cm,-0.6)
            node[below]{\small{$\mathllap{-}\x$}};
\end{tikzpicture}
\caption{Podemos utilizar a reta real para descrever a posição de um corpo. Desta forma, podemos diferencias posições nos diferentes sentidos do eixo através do \emph{sinal positivo ou negativo}.\label{Fig:dois_blocos}}
\end{figure}

\noindent{}onde temos que as posições dos blocos são dadas por
\begin{align}
    x_1 &= \np[m]{-3} \\
    x_2 &= \np[m]{3}.
\end{align}

Uma reta numerada como a mostrada na figura acima é denominada como \emph{eixo coordenado}. A descrição de movimentos bi e tridimensional necessita de mais do que um eixo coordenado, por isso é comum que eles sejam diferenciados através das denominações $x$, $y$ e $z$. Para um deslocamento unidimensional, em geral denominamos tal eixo como um \emph{eixo $x$}. A direção do eixo é arbitrária, podendo ser horizontal, vertical\footnote{Ao tratarmos de movimentos unidimensionais verticais, por exemplo, podemos utilizar $x$. Quando trabalhamos em duas dimensões, no entanto, é preferível que o eixo vertical seja denominado $y$.} ou mesmo inclinada, bastando ser na direção do movimento unidimensional. O sentido positivo do eixo também é arbitrário, e podemos fazer essa escolha livremente.

Em alguns casos, podemos utilizar a distância até a origem para expressar a posição mesmo para um movimento que não é retilíneo, caso não haja ambiguidade em relação à definição da localização. Um exemplo disso são estradas nas quais se utilizam marcadores de distância. Se necessitamos declarar o endereço de uma propriedade ao longo de uma rodovia, podemos utilizar a distância em relação a um marco inicial. Apesar de esse claramente não ser um caso unidimensional, pois o deslocamento não será em uma linha reta, podemos marcar um ponto de maneira única através da distância \emph{ao longo} da estrada até o marco inicial.

Como visto no Capítulo anterior, a maioria das medidas físicas têm uma dimensão. No caso da posição, como ela é descrita através de uma medida de distância entre a origem e a posição do corpo, tempos que a dimensão é a de \emph{comprimento} e -- no Sistema Internacional -- suas unidades são o metro.

%%%%%%%%%%%%%%%%%%%%%%%%%
\subsection{Deslocamento}
%%%%%%%%%%%%%%%%%%%%%%%%%

Vamos considerar um deslocamento do bloco da direita na Figura~\ref{Fig:dois_blocos} para a posição $x = -\np[m]{1,0}$. Podemos medir seu deslocamento entre a posição inicial e a final utilizando uma trena e obteríamos um deslocamento de \np[m]{4} para a esquerda ao longo da reta, porém se sabemos os valores numéricos associados às posições inicial e final no eixo $x$, podemos calcular esse valor facilmente fazendo\footnote{A notação usando $\Delta$ representa a \emph{variação} de uma grandeza qualquer. Vamos utilizá-la para posição em vários eixos ($\Delta x$, $\Delta y$, $\Delta z$), tempo ($\Delta t$), vetores ($\Delta\vec{r}$), etc.}
\begin{align}
  \Delta x &= x_f - x_i \\
  &= (-\np[m]{1,0}) - (\np[m]{3,0}) \\
  &= -\np[m]{4,0}.
\end{align}
%
O que dizer sobre o sinal negativo? Esse sinal significa que o deslocamento se deu no \emph{sentido negativo do eixo}\footnote{Lembre-se que o sentido do eixo é arbitrário. Nesse caso o sentido positivo é para a diteita e o negativo, consequentemente, para a esquerda}. Ao medirmos, o valor da medida não é suficiente para descrevermos o deslocamento. Temos que declarar que o deslocamento foi -- nesse caso -- para a esquerda. Portanto, o deslocamento tem um módulo (\np[m]{4,0}), uma direção (ao longo do eixo $x$) e um sentido (para a esquerda, ou no sentido negativo do eixo)\footnote{Veremos mais adiante que essas propriedades são características de vetores e serão muito importantes para descrevermos o movimento em duas e três dimensões.}, da mesma forma que a posição. Se o deslocamento fosse no sentido positivo do eixo, o resultado do cálculo de $\Delta x$ seria positivo. 

O deslocamento é dado através da diferença entre posições. Como vimos no capítulo anterior, só podemos somar, subtratir e igualar termos que têm a mesma dimensão. Logo, concluímos que o deslocamento tem dimensão de \emph{comprimento} e suas unidades são o metro no SI, assim como a posição.

Claramente temos que se as posições inicial e final são iguais, o deslocamento será zero. Apesar de a utilidade de tal definição ser pouco evidente agora, veremos adiante que isso faz sentido para as grandezas físicas, pois no caso de uma força conservativa -- por exemplo -- temos que o trabalho é nulo quando o deslocamento é zero.

%%%%%%%%%%%%%%%%%%%%%%%%%%%%%%%%%
\subsection{Deslocamento escalar}
%%%%%%%%%%%%%%%%%%%%%%%%%%%%%%%%%

Algo importante a se notar é que o deslocamento é a diferença de posição entre duas posições quaisquer ocupadas por um corpo. Consequentemente, para um veículo que se desloca durante um dia de trabalho, por exemplo, os valores de deslocamento em relação à posição inicial -- a garagem, por exemplo -- será diferente para cada momento do dia. Quando o veículo retorna à garagem, seu deslocamento será nulo, pois as posições inicial e final são a mesma. Se verificarmos o hodômetro do veículo, no entanto, veremos um valor diferente de zero. Este valor pode ser denominado de \emph{deslocamento escalar}\footnote{Apesar de ser algo mais ligado à nossa experiência cotidiana de deslocamento, o deslocamento escalar será de pouca utilidade.} e é calculado pela soma do \emph{módulo} de todos os deslocamentos efetuados pelo veículo:
\begin{equation}
  d_s = |\Delta x_1| + |\Delta x_2| + |\Delta x_3| + \dots + |\Delta x_n|.
\end{equation}

Novamente, temos que a dimensão é de \emph{comprimento} e as unidades no SI são metros, uma vez que o deslocamento escalar é determinado a partir de uma equação e da soma de termos com tais dimensões.


%%%%%%%%%%%%%%%%%%%%%%%%%%%%%%%%%%%%%%%%%
\subsection{Posição como função do tempo}
%%%%%%%%%%%%%%%%%%%%%%%%%%%%%%%%%%%%%%%%%

Se ocorre movimento, podemos dizer que a cada instante de tempo $t$, temos um valor de posição $x$ diferente. Se nos lembrarmos do conceito de funções, temos que dados dois grupos de números, uma função é a operação matemática que liga elementos do primeiro grupo a elementos do segundo\footnote[][-2cm]{Lembre-se de que dois elementos do grupo $t$ podem levar a um mesmo elemento do grupo $x$, porém um elemento de $t$ não pode levar a dois elementos de $x$. Fisicamente isso equivale ao fato de que um objeto não pode se encontrar em dois lugares ao mesmo tempo, porém pode em dois momentos diferentes estar em um mesmo lugar.}:
\begin{figure}\forcerectofloat
\centering
\begin{tikzpicture}
\draw (0,0.5) node[above] {$t$};
\draw (3.1,0.5) node[above] {$x$};
\draw (0,-1) ellipse [x radius=12pt, y radius=40pt];
\draw (3.1,-1) ellipse [x radius=12pt, y radius=40pt];
\node [circle,draw,fill,scale=0.3] (A){};
\node [circle,draw,fill,scale=0.3] (B) [right=3cm of A] {};
\node [circle,draw,fill,scale=0.3] (C) [below=of A] {};
\node [circle,draw,fill,scale=0.3] (D) [right=3cm of C] {};
\node [circle,draw,fill,scale=0.3] (E) [below=of C] {};
\node [circle,draw,fill,scale=0.3] (F) [right=3cm of E] {};
\draw [thick, arrows={ - Stealth}]
(A) edge [bend left=45] node[above]{$x = x(t)$}(B)
(C) edge [bend left=45] (D)
(E) edge [bend left=45] (F);
\end{tikzpicture}
\caption{A cada valor de tempo $t$ temos um valor de posição $x$ associado. A função $x(t)$ é a operação que descreve a relação entre essas duas variáveis.}
\end{figure}

Dessa forma, podemos denotar o conjunto de instantes de tempo $t$ e o conjunto de posições $x$ correspondente como uma função:
\begin{equation}
    x: t \mapsto x(t).
\end{equation}
%
Com isso podemos elaborar gráficos que mostram, por exemplo, a variação temporal da posição. Na Figura~\ref{Fig:Graf_posicao_func_tempo} temos três curvas que representam funções \emph{distintas}, isto é, \emph{formas diferentes} de relacionar a posição com o tempo. Cada forma das curvas está relacionada a um tipo de movimento diferente. Em particular, para as formas mostradas, temos:
\begin{description}
    \item[Linha cheia:] Movimento com posição constante, ou seja, o corpo se encontra em repouso.
    \item[Linha tracejada:] Movimento com \emph{velocidade} constante.
    \item[Linha ponto-tracejada:] Movimento com \emph{aceleração} constante.
\end{description}
%
No primeiro caso é fácil verificar que a linha indica que a posição se mantém a mesma para todos os valores de $t$. Para os outros dois casos, verificaremos a justificativa nas seções adiante.

\begin{marginfigure}
\centering
\begin{tikzpicture}[>=Stealth, extended line/.style={shorten >=-#1,shorten <=-#1},
 extended line/.default=3mm]] % talvez fosse melhor amplicar com scale=1.5
    % Draw axes: acho que o |- é pra desenhar um "canto", um L
    \draw [<->,thick] (0,3) node (yaxis) [below left] {$x$}
        |- (4.3,0) node (xaxis) [below left] {$t$};
    % Desenhar função:
    \draw[smooth,name path=plota,samples=1000,domain=0:3.5]
    plot(\x,{2});
    
    \draw[smooth, densely dashed, name path=plotb,samples=1000,domain=0:3.5]
    plot(\x,{0.5*\x + 0.5});

    \draw[smooth, dash dot, name path=plotc,samples=1000,domain=0:3.5]
    plot(\x,{0.15*\x^2});
     
\end{tikzpicture}
\caption{Gráficos que exemplificam possíveis formas para os gráficos da função posição $x(t)$.\label{Fig:Graf_posicao_func_tempo}}
\end{marginfigure}

%%%%%%%%%%%%%%%%%%%%%%%%%%%%%%%%
\section{Velocidade}
%%%%%%%%%%%%%%%%%%%%%%%%%%%%%%%%

%%%%%%%%%%%%%%%%%%%%%%%%%%%%%
\subsection{Velocidade média}
%%%%%%%%%%%%%%%%%%%%%%%%%%%%%

Se considerarmos que um deslocamento sempre leva um tempo para ser efetuado, podemos calcular uma grandeza de grande interesse associada a ele: a \emph{velocidade}. Definimos a velocidade média\footnote{O símbolo $\mean{~}$ denota o valor médio de uma grandeza.} como
\begin{equation}
  \mean{v} = \frac{\Delta x}{\Delta t}.
\end{equation}
%
Temos agora outra variável que descreve o movimento. 

Podemos determinar a dimensão da velocidade através de
\begin{align}
    [\mean{v}] &= \left[\frac{\Delta x}{\Delta t}\right] \\
    &= \frac{[\Delta x]}{[\Delta t]} \\
    &= \frac{\rm{L}}{\rm{T}}.
\end{align}
%
Consequentemente, no SI, a velocidade tem unidades de $\rm{m}/\rm{s}$.

\begin{marginfigure}[-3cm]
\centering
\begin{tikzpicture}[>=Stealth, extended line/.style={shorten >=-#1,shorten <=-#1},
 extended line/.default=3mm]]
    % Draw axes: acho que o |- é pra desenhar um "canto", um L
    \draw [<->,thick] (0,3) node (yaxis) [below left] {$x$}
        |- (4.3,0) node (xaxis) [below left] {$t$};
    % Desenhar função:
    \draw[smooth,name path=plot,samples=1000,domain=0.5:3.5]
    plot(\x,{0.4*\x^2 + 1.1 - 0.9*\x});
    
    % linhas dos eixos à curva
    \node[below](a)at(1.2,0){$t_i$};
    \node[below](b)at(3,0){$t_f$};
    \path[name path=froma](a)--+(0,4);
    \path[name path=fromb](b)--+(0,4);
    \draw[dotted, thin,name intersections={of=froma and plot}](a)--(intersection-1) coordinate (plot-a-intersection)--(0,0|-intersection-1)node[left]{$x_1$};
    \draw[dotted, thin, name intersections={of=fromb and plot}](b)--(intersection-1) coordinate (plot-b-intersection)--(0,0|-intersection-1)node[left]{$x_2$};
    
    % linha entre as duas interseções (as coordenadas foram salvas acima)
    \draw[extended line, dashed] (plot-a-intersection) -- (plot-b-intersection);

    % pontos nas interseções
    \fill [opacity=1] (plot-a-intersection) circle (2pt);
    \fill [opacity=1] (plot-b-intersection) circle (2pt);
   
\end{tikzpicture}
\caption{Gráfico da posição em função do tempo. Podemos interpretar a velocidade média graficamente ao ligarmos os pontos da curva que representam os instantes/posições inicial e final.\label{Fig:Interp_graf_vel_med}}
\end{marginfigure}

Podemos conferir uma interpretação gráfica à velocidade média. Para isso, vamos tomar a Figura~\ref{Fig:Interp_graf_vel_med}, onde marcamos dois pontos que correspondem à posição $x_i$ no instante $t_i$ e à posição $x_f$ no instante $t_f$. Ligamos esses dois pontos por uma reta.


Traçando uma reta horizontal e uma vertical, podemos completar um triângulo retângulo (Figura~\ref{Fig:Interp_graf_vel_med_b}). Nesse triângulo, temos que o tamanho da lateral direita é igual a $x_f - x_i$, ou seja, corresponde a $\Delta x$. Já a parte inferior é igual a $t_f - t_i$, correspondendo a $\Delta t$.
\begin{marginfigure}
\centering
\begin{tikzpicture}[>=Stealth, extended line/.style={shorten >=-#1,shorten <=-#1},
 extended line/.default=3mm]]
    % Draw axes: acho que o |- é pra desenhar um "canto", um L
    \draw [<->,thick] (0,3) node (yaxis) [below left] {$x$}
        |- (4.3,0) node (xaxis) [below left] {$t$};
    % Desenhar função:
    \draw[smooth, densely dotted, name path=plot,samples=1000,domain=0.5:3.5]
    plot(\x,{0.4*\x^2 + 1.1 - 0.9*\x});
    
    % linhas dos eixos à curva
    \node[below](a)at(1.2,0){$t_i$};
    \node[below](b)at(3,0){$t_f$};
    \path[name path=froma](a)--+(0,4);
    \path[name path=fromb](b)--+(0,4);
    \draw[dotted, thin,name intersections={of=froma and plot}](a)--(intersection-1) coordinate (plot-a-intersection)--(0,0|-intersection-1)node[left]{$x_1$};
    \draw[dotted, thin, name intersections={of=fromb and plot}](b)--(intersection-1) coordinate (plot-b-intersection)--(0,0|-intersection-1)node[left]{$x_2$};
    
    % linha entre as duas interseções (as coordenadas foram salvas acima)
    \draw[extended line, dashed] (plot-a-intersection) -- (plot-b-intersection);

    % pontos nas interseções
    \fill [opacity=1] (plot-a-intersection) circle (2pt);
    \fill [opacity=1] (plot-b-intersection) circle (2pt);
    
    % linhas complementares
    \path[name path=horiz-a] (plot-a-intersection)--+(4,0);
    \path[name path=vert-b] (plot-b-intersection)--+(0,-2);
    \draw[thin, name intersections={of=horiz-a and vert-b}] (plot-b-intersection) -- (intersection-1) coordinate (ang-reto);
    
    \draw[thin, name intersections={of=fromb and horiz-a}] (intersection-1) -- (plot-a-intersection);
    
    % marcar os ângulos
    \tkzMarkRightAngle(plot-a-intersection,ang-reto,plot-b-intersection);
    \tkzLabelAngle[pos=.15](plot-a-intersection,ang-reto,plot-b-intersection){$\cdot$};
    \path pic[draw, angle radius=5mm, pic text=$\theta$,angle eccentricity=1.3] {angle = ang-reto--plot-a-intersection--plot-b-intersection};
    
    % medidas
    \draw[|-|] ($ (plot-b-intersection) + (0.2,0) $) -- node[right]{$\Delta x$} ($ (ang-reto) + (0.2,0) $);
    
    \draw[|-|] ($ (plot-a-intersection) + (0, -0.2) $) -- node[below]{$\Delta t$} ($ (ang-reto) + (0, -0.2) $);
   
\end{tikzpicture}
\caption{Triângulo formado pela reta que liga os pontos e as linhas horizontal e vertical.\label{Fig:Interp_graf_vel_med_b}}
\end{marginfigure}

\noindent{}Ao calcularmos a tangente do ângulo $\theta$, temos
\begin{align}
    \tan\theta &= \frac{\Delta x}{\Delta t} \\
    & = \mean{v},
\end{align}
%
isto é, \emph{a inclinação da reta que une os pontos correspondentes aos instantes/posições inicial e final está relacionada à velocidade média através de}:\footnote{Um cálculo preciso necessitaria que levássemos em conta a escala do gráfico, porém estamos mais interessados na interpretação qualitativa.}
\begin{equation}
    \mean{v} = \tan\theta.
\end{equation}
%
Assim, se a inclinação entre um par de pontos é maior que entre outro par, temos que a velocidade média é maior no primeiro caso (Figura~\ref{Fig:Interp_graf_vel_med_diff}).

\begin{marginfigure}
\centering
\begin{tikzpicture}[>=Stealth, extended line/.style={shorten >=-#1,shorten <=-#1},
 extended line/.default=3mm]] % talvez fosse melhor amplicar com scale=1.5
    % Draw axes: acho que o |- é pra desenhar um "canto", um L
    \draw [<->,thick] (0,3) node (yaxis) [below left] {$x$}
        |- (4.3,0) node (xaxis) [below left] {$t$};
    % Desenhar função:
    \draw[smooth,name path=plot,samples=1000,domain=0.5:3.5]
    plot(\x,{0.4*\x^2 + 1.1 - 0.9*\x});
    
    % linhas dos eixos à curva
    \coordinate (a) at (1.2,0);
    \coordinate (b) at (3,0);
    \coordinate (c) at (0.7,0);
    \coordinate (d) at (2.3, 0);

    \path[name path=froma](a)--+(0,4);
    \path[name path=fromb](b)--+(0,4);
    \path[name path=fromc](c)--+(0,4);
    \path[name path=fromd](d)--+(0,4);


    \path[name intersections={of=froma and plot}](a)--(intersection-1) coordinate (plot-a-intersection)--(0,0|-intersection-1);
    \path[name intersections={of=fromb and plot}](b)--(intersection-1) coordinate (plot-b-intersection)--(0,0|-intersection-1);
    \path[name intersections={of=fromc and plot}](c)--(intersection-1) coordinate (plot-c-intersection)--(0,0|-intersection-1);
    \path[name intersections={of=fromd and plot}](d)--(intersection-1) coordinate (plot-d-intersection)--(0,0|-intersection-1);

    % linha entre as duas interseções (as coordenadas foram salvas acima)
    \draw[extended line, dashed] (plot-a-intersection) -- (plot-b-intersection);
    \draw[extended line, dashed] (plot-c-intersection) -- (plot-d-intersection);

    % pontos nas interseções
    \fill[opacity=1] (plot-a-intersection) circle (2pt) node[below right]{$A$};
    \fill[opacity=1] (plot-b-intersection) circle (2pt) node[below right]{$B$};
    \fill[opacity=1] (plot-c-intersection) circle (2pt) node[above right]{$C$};
    \fill[opacity=1] (plot-d-intersection) circle (2pt) node[below right]{$D$};
    
\end{tikzpicture}
\caption{As retas $\overline{AB}$ e $\overline{CD}$ representam valores de velocidade média diferentes, como pode ser visto devido às diferentes inclinações.\label{Fig:Interp_graf_vel_med_diff}}
\end{marginfigure}

%%%%%%%%%%%%%%%%%%%%%%%%%%%%%%%%%%%
\subsection{Velocidade instantânea}
%%%%%%%%%%%%%%%%%%%%%%%%%%%%%%%%%%%

Na Figura~\ref{Fig:Interp_graf_vel_med}, se tomarmos intervalos sucessivamente menores de tempo, podemos definir o que chamamos de \emph{velocidade instantânea}. Se estamos interessados em calcular a velocidade em um ponto $P$ (Figura~\ref{Fig:Interp_graf_vel_med_lim}), podemos tomar pares $(t_i, x_i)$, $(t_f, x_f)$ sucessivamente mais próximos até que a distância entre eles seja desprezível, ou seja, \emph{tenda a zero}. Nesse momento, a reta que liga os dois pontos passa a ser uma \emph{reta tangente à curva no ponto $P$}, isto é, uma reta que toca a curva $x(t)$ somente no ponto $P$.

\begin{marginfigure}
\centering
\begin{tikzpicture}[>=Stealth, extended line/.style={shorten >=-#1,shorten <=-#1},
 extended line/.default=3mm]] % talvez fosse melhor amplicar com scale=1.5
    % Draw axes: acho que o |- é pra desenhar um "canto", um L
    \draw [<->,thick] (0,3) node (yaxis) [below left] {$x$}
        |- (4.3,0) node (xaxis) [below left] {$t$};
    % Desenhar função:
    \draw[smooth,name path=plot,samples=1000,domain=0.5:3.5]
    plot(\x,{0.4*\x^2 + 1.1 - 0.9*\x});
    
    % linhas dos eixos à curva
    \coordinate (a) at (1.2,0);
    \coordinate (b) at (3,0);
    \coordinate (c) at (0.7,0);
    \coordinate (d) at (3.3, 0);
    \coordinate (e) at (1.8,0);
    \coordinate (f) at (2.8,0);
    \coordinate (p) at (2.3,0);
    \path[name path=froma](a)--+(0,4);
    \path[name path=fromb](b)--+(0,4);
    \path[name path=fromc](c)--+(0,4);
    \path[name path=fromd](d)--+(0,4);
    \path[name path=frome](e)--+(0,4);
    \path[name path=fromf](f)--+(0,4);
    \path[name path=fromp](p)--+(0,4);
    \path[name intersections={of=froma and plot}](a)--(intersection-1) coordinate (plot-a-intersection)--(0,0|-intersection-1);
    \path[name intersections={of=fromb and plot}](b)--(intersection-1) coordinate (plot-b-intersection)--(0,0|-intersection-1);
    \path[name intersections={of=fromc and plot}](c)--(intersection-1) coordinate (plot-c-intersection)--(0,0|-intersection-1);
    \path[name intersections={of=fromd and plot}](d)--(intersection-1) coordinate (plot-d-intersection)--(0,0|-intersection-1);
    \path[name intersections={of=frome and plot}](e)--(intersection-1) coordinate (plot-e-intersection)--(0,0|-intersection-1);
    \path[name intersections={of=fromf and plot}](f)--(intersection-1) coordinate (plot-f-intersection)--(0,0|-intersection-1);
    \path[name intersections={of=fromp and plot}](p)--(intersection-1) coordinate (plot-p-intersection)--(0,0|-intersection-1);
    
    % linha entre as duas interseções (as coordenadas foram salvas acima)
    \draw[extended line, densely dotted] (plot-a-intersection) -- (plot-b-intersection);
    \draw[extended line, densely dotted] (plot-c-intersection) -- (plot-d-intersection);
    \draw[extended line, densely dotted] (plot-e-intersection) -- (plot-f-intersection);

    % pontos nas interseções
    \fill[opacity=1, gray] (plot-a-intersection) circle (2pt);
    \fill[opacity=1, gray] (plot-b-intersection) circle (2pt);
    \fill[opacity=1, gray] (plot-c-intersection) circle (2pt);
    \fill[opacity=1, gray] (plot-d-intersection) circle (2pt);
    \fill[opacity=1, gray] (plot-e-intersection) circle (2pt);
    \fill[opacity=1, gray] (plot-f-intersection) circle (2pt);
    
    \fill[opacity=1] (plot-p-intersection) circle (2pt);
   
    % tangente
    \draw[smooth, densely dashed, name path=deriv,samples=1000,domain=1.5:3.1]
    plot(\x,{0.94*\x - 1.016});
     
\end{tikzpicture}
\caption{Gráfico da posição em função do tempo onde mostramos o processo em que tomamos o limite $\Delta t \to 0$.\label{Fig:Interp_graf_vel_med_lim}}
\end{marginfigure}

Temos então que, graficamente, podemos interpretar a velocidade instantânea como a inclinação de uma reta tangente à curva $x(t)$ no ponto em que estamos interessados em calculá-la, isto é no ponto $(t,x)$.

Esse processo de aproximações sucessivas em que fazemos $\Delta t$ progressivamente menor é o que chamamos -- quando tomamos \emph{$\Delta t$ tendendo a zero} -- de \emph{limite}. Denotamos esse processo como
\begin{equation}
  v = \lim_{\Delta t \to 0} \frac{\Delta x}{\Delta t}.
\end{equation}
%
A velocidade instantânea é o valor de velocidade no momento considerado. É o que é mostrado pelo velocímetro de um carro, por exemplo\footnote{Isso é verdade somente se o sistema de medidas é analógico. Para medidas digitais, o valor é uma média de uma série de medidas tomadas em um curto intervalo de tempo}.

Além disso, se $\theta=0$, temos que a velocidade é nula em tal ponto, mesmo que momentaneamente. Em um gráfico que mostre a posição em função do tempo para um objeto lançado verticalmente para cima -- por exemplo --, esse ponto corresponderia à posição de máxima altura, onde o objeto pára momentaneamente.

Finalmente, se o ângulo $\theta$ está abaixo da horizontal, verificamos que a velocidade é negativa (veja a reta tangente ao ponto $P_1$ na Figura~\ref{Fig:Interp_graf_vel_med_signs}, pois nos deslocamos no sentido negativo do eixo -- $\Delta x$ é negativo, portanto -- e, nesse caso, verificamos que a tangente de $\theta$ é negativa\footnote{Lembre-se do círculo trigonométrico: para um ângulo como o da reta tangente a $P_1$, a tangente está abaixo do eixo horizontal.}.

%%%%%%%%%%%%%%%%%%%%%%%%%%%%%%%%%%%%%%%%
\paragraph{Discussão: Cálculo do limite}
%%%%%%%%%%%%%%%%%%%%%%%%%%%%%%%%%%%%%%%%

\begin{marginfigure}
\centering
\begin{tikzpicture}[>=Stealth, extended line/.style={shorten >=-#1,shorten <=-#1},
 extended line/.default=3mm]] % talvez fosse melhor amplicar com scale=1.5
    % Draw axes: acho que o |- é pra desenhar um "canto", um L
    \draw [<->,thick] (0,3) node (yaxis) [below left] {$x$}
        |- (4.3,0) node (xaxis) [below left] {$t$};
    % Desenhar função:
    \draw[smooth,name path=plot,samples=1000,domain=0.2:3.5]
    plot(\x,{0.4*\x^2 + 1.1 - 0.9*\x});
    
    % linhas dos eixos à curva
    \coordinate (p) at (1.125,0);
    \path[name path=fromp](p)--+(0,4);
    \path[name intersections={of=fromp and plot}](p)--(intersection-1) coordinate (plot-p-intersection)--(0,0|-intersection-1);
    
    % pontos nas interseções   
    \fill[opacity=1] (plot-p-intersection) circle (2pt) node[above]{$P$};
   
    % tangente   
    \draw[smooth, densely dashed, name path=deriv,samples=1000,domain=0.6:1.8]
    plot(\x,{0.59375});
     
\end{tikzpicture}
\caption{No ponto $P$ temos que o ângulo de inclinação da reta tangente é nulo, portanto temos que momentaneamente a velocidade é nula.\label{Fig:Interp_graf_vel_med_zero}}
\end{marginfigure}

\begin{marginfigure}
\centering
\begin{tikzpicture}[>=Stealth, extended line/.style={shorten >=-#1,shorten <=-#1},
 extended line/.default=3mm]] % talvez fosse melhor amplicar com scale=1.5
    % Draw axes: acho que o |- é pra desenhar um "canto", um L
    \draw [<->,thick] (0,3) node (yaxis) [below left] {$x$}
        |- (4.3,0) node (xaxis) [below left] {$t$};
    % Desenhar função:
    \draw[smooth,name path=plot,samples=1000,domain=0.2:3.5]
    plot(\x,{0.4*\x^2 + 1.1 - 0.9*\x});
    
    % linhas dos eixos à curva
    \coordinate (o) at (0.6,0);
    \coordinate (p) at (2.3,0);
    \path[name path=fromp](p)--+(0,4);
    \path[name path=fromo](o)--+(0,4);
    \path[name intersections={of=fromp and plot}](p)--(intersection-1) coordinate (plot-p-intersection)--(0,0|-intersection-1);
    \path[name intersections={of=fromo and plot}](o)--(intersection-1) coordinate (plot-o-intersection)--(0,0|-intersection-1);
    
    % pontos nas interseções   
    \fill[opacity=1] (plot-p-intersection) circle (2pt) node[above]{$P_2$};
    \fill[opacity=1] (plot-o-intersection) circle (2pt) node[above]{$P_1$};
   
    % tangente
    \draw[smooth, densely dashed, name path=deriv,samples=1000,domain=1.8:2.8]
    plot(\x,{0.94*\x - 1.016});
    
%    \draw[smooth, densely dashed, name path=deriv,samples=1000,domain=0.6:1.8]
%    plot(\x,{0.59375});
    
    \draw[smooth, densely dashed, name path=deriv,samples=1000,domain=0.05:1.2]
    plot(\x,{0.956 - 0.42*\x});
     
\end{tikzpicture}
\caption{As inclinações das retas tangentes indicam que em $P_1$ a velocidade é negativa, enquanto em $P_2$ temos uma velocidade positiva.\label{Fig:Interp_graf_vel_med_signs}}
\end{marginfigure}

Ao utilizarmos a definição
\begin{equation*}
  v = \lim_{\Delta t \to 0} \frac{\Delta x}{\Delta t}
\end{equation*}
%
temos algo um tanto quanto estranho: estamos calculando a razão entre dois números que estão indo a zero, já que se $\Delta t$ vai a zero, o deslocamento $\Delta x$ correspondente também vai a zero. À primeira vista, poderíamos pensar que a razão seria \emph{zero sobre zero}, o que é uma indeterminação matemática.

Vamos utilizar um exemplo para mostrar que não é esse o caso. Vamos tomar uma função para a posição que é dada por
\begin{equation}\label{Eq:FormaFuncCalcDeriv}
    x(t) = \alpha t^2 + \beta
\end{equation}
%
e calcular a velocidade instantânea para um instante de tempo $t = t'$. Devemos tomar um intervalo de tempo que contenha o valor $t'$ para que possamos calcular a o valor de tal velocidade. Uma maneira simples de fazer isso é adotar o início do intervalo como sendo o próprio valor $t'$:
\begin{equation}
    t_i = t'.
\end{equation}
%
O final do intervalo --~isto é, o valor de $t_f$~-- pode ser escrito em termos do valor de $t_i$ e da própria largura $\Delta t$ do intervalo:
\begin{equation}
    t_f = t_i + \Delta t.
\end{equation}
%
Assim, podemos escrever
\begin{align}
    v & = \lim_{\Delta t \to 0} \frac{\Delta x}{\Delta t} \\
    &= \lim_{\Delta t \to 0} \frac{x_f - x_i}{\Delta t} \\
    &= \lim_{\Delta t \to 0} \frac{x(t_f) - x(t_i)}{\Delta t} \\
    &= \lim_{\Delta t \to 0} \frac{x(t_i + \Delta t) - x(t_i)}{\Delta t}.
\end{align}

Agora podemos substituir a Expressão~\eqref{Eq:FormaFuncCalcDeriv} para a posição, obtendo
\begin{align}
    v & = \lim_{\Delta t \to 0} \frac{[\alpha (t_i + \Delta t)^2 + \beta]-[\alpha (t_i)^2 + \beta]}{\Delta t} \\
    &= \lim_{\Delta t \to 0} \frac{[\alpha (t_i^2 + \Delta t^2 + 2t_i\Delta t) + \beta]-[\alpha t_i^2 + \beta]}{\Delta t} \\
    &= \lim_{\Delta t \to 0} \frac{\alpha t_i^2 + \alpha\Delta t^2 + 2\alpha t_i\Delta t + \beta - \alpha t_i^2 - \beta}{\Delta t}.
\end{align}
%
Note que os termos $\beta$ se cancelam, assim como os termos $\alpha t_i^2$. Portanto, resta a expressão
\begin{align}
    v &= \lim_{\Delta t \to 0} \frac{\alpha\Delta t_i^2 + 2\alpha t_i\Delta t}{\Delta t} \\
    &= \lim_{\Delta t \to 0} \alpha\Delta t + 2\alpha t_i
\end{align}

Finalmente, ao tomarmos o limite $\Delta t \to 0$ na expressão acima, temos que o primeiro termo irá a zero, \emph{porém o segundo não}. Como temos que $t_i = t'$, o valor da velocidade no instante $t'$ será dado por
\begin{equation}
    v = 2\alpha t'.
\end{equation}

O processo discutido acima é o que chamamos de \emph{derivada}. Temos, portanto, que  o cálculo da derivada de uma função nada mais é do que uma maneira de determinar a \emph{taxa de variação} de tal função. Por isso, temos que a velocidade instantânea --~que é a taxa de variação da posição~-- pode ser calculada através da derivada da função posição. Devemos destacar que a apresentação dada acima ao conceito de derivada não é geral, nem ampla e rigorosa. Um estudo mais aprofundado dos conceitos de limite e derivada serão apresentados na disciplina de Cálculo.

%%%%%%%%%%%%%%%%%%%%%%%%%%%%%%%%%%%%%%%%%%%%%%%%%%%%%%
\subsection{Velocidades escalares média e instantânea}
%%%%%%%%%%%%%%%%%%%%%%%%%%%%%%%%%%%%%%%%%%%%%%%%%%%%%%

Se viajamos de uma cidade a outra e voltamos, temos um deslocamento nulo. Consequentemente, a velocidade média durante esse percurso será também nula. No entanto, podemos tomar o deslocamento escalar e dividi-lo pelo tempo transcorrido e definir uma \emph{velocidade escalar média}:
\begin{equation}
  \mean{v}_s = \frac{d_s}{\Delta t}.
\end{equation}
%
A velocidade escalar média é o que o computador de bordo de um carro verifica como velocidade média em um trajeto. Apesar de ela corresponder a nossa intuição de ``velocidade média'', ela não é uma grandeza vetorial -- como veremos adiante -- e não será de grande interesse para a descrição de fenômenos físicos.

Se tomarmos o limite com $\Delta t \to 0$, podemos dizer que o deslocamento nesse pequeno intervalo de tempo não sofre alteração de direção, portanto única diferença diferença possível entre a o deslocamento o deslocamento escalar é um sinal. Nesse caso, teremos que a velocidade escalar instantânea será igual ao módulo da velocidade instantânea:
\begin{equation}
  v_s = |v|.
\end{equation}

%%%%%%%%%%%%%%%%%%%%%%%%%%%%%%%%%%%%%%%%%%%%
\subsection{Velocidade como função do tempo}
%%%%%%%%%%%%%%%%%%%%%%%%%%%%%%%%%%%%%%%%%%%%

De maneira análoga ao caso da evolução temporal da posição, podemos dizer que para cada instante de tempo $t_i$ temos uma velocidade $v_i$ associada. Assim, podemos denotar o conjunto de instantes de tempo $t$ e o conjunto de posições $x$ correspondente como uma função:

\begin{marginfigure}
\centering
\begin{tikzpicture}[>=Stealth, extended line/.style={shorten >=-#1,shorten <=-#1},
 extended line/.default=3mm]] % talvez fosse melhor amplicar com scale=1.5
    % Draw axes: acho que o |- é pra desenhar um "canto", um L
    \draw [<->,thick] (0,3) node (yaxis) [below left] {$v$}
        |- (4.3,0) node (xaxis) [below left] {$t$};
    % Desenhar função:
    \draw[smooth,name path=plota,samples=1000,domain=0:3.5]
    plot(\x,{2});
    
    \draw[smooth, densely dashed, name path=plotb,samples=1000,domain=0:3.5]
    plot(\x,{0.5*\x + 0.5});

    \draw[smooth, dash dot, name path=plotc,samples=1000,domain=0:3.5]
    %plot(\x,{0.15*\x^2 + 0.1 + 0.1*\x*sin(10*\x r)});
     plot(\x,{0.5*sin(10*\x r)});
     
\end{tikzpicture}
\caption{Gráficos que exemplificam possíveis formas para os gráficos da velocidade $v(t)$.\label{Fig:Graf_vel_func_tempo}}
\end{marginfigure}

\begin{equation}
    x: t \mapsto x(t),
\end{equation}
%
o que corresponde a
\begin{figure}\forcerectofloat
\centering
\begin{tikzpicture}
\draw (0,0.5) node[above] {$t$};
\draw (3.1,0.5) node[above] {$v$};
\draw (0,-1) ellipse [x radius=12pt, y radius=40pt];
\draw (3.1,-1) ellipse [x radius=12pt, y radius=40pt];
\node [circle,draw,fill,scale=0.3] (A){};
\node [circle,draw,fill,scale=0.3] (B) [right=3cm of A] {};
\node [circle,draw,fill,scale=0.3] (C) [below=of A] {};
\node [circle,draw,fill,scale=0.3] (D) [right=3cm of C] {};
\node [circle,draw,fill,scale=0.3] (E) [below=of C] {};
\node [circle,draw,fill,scale=0.3] (F) [right=3cm of E] {};
\draw [thick, arrows={ - Stealth}]
(A) edge [bend left=45] node[above]{$v = v(t)$}(B)
(C) edge [bend left=45] (D)
(E) edge [bend left=45] (F);
\end{tikzpicture}
\caption{A cada valor de tempo $t$ temos um valor de velocidade $v$ associado. A função $v(t)$ descreve a relação entre essas duas variáveis.\label{Fig:GrafVelFuncTempoDiversasCurvas}}
\end{figure}


Novamente, o fato de que podemos escrever a velocidade como uma função do tempo nos permite a elaboração de gráficos. Assim como no caso dos gráficos de posição, cada sistema físico tem uma forma de curva diferente, o que permite que seja possível determinar propriedades importantes rapidamente, através de uma simples conferência visual. Na Figura~\ref{Fig:GrafVelFuncTempoDiversasCurvas}, por exemplo, temos que a linha cheia representa um movimento com velocidade constante, a linha tracejada representa um movimento com aceleração constante, e a linha ponto-tracejada representa um movimento oscilatório, como o de um pêndulo.

%%%%%%%%%%%%%%%%%%%%%%%%%%%%%%%%%%%%%%%%%%%%%%%%%%%%%%%%%%%%%%%%%%%%%%%%%%%%%
\subsection{Evolução temporal da posição para o caso de velocidade constante}
%%%%%%%%%%%%%%%%%%%%%%%%%%%%%%%%%%%%%%%%%%%%%%%%%%%%%%%%%%%%%%%%%%%%%%%%%%%%%

Se conhecemos a velocidade média, podemos então descrever a distância percorrida em função do tempo como
\begin{equation}
  \Delta x = \mean{v} \Delta t,
\end{equation}
%
ou
\begin{equation}
    x_f = x_i + \mean{v} \Delta t.
\end{equation}
%
Em especial, se a velocidade é constante, então $v = \mean{v}$, e obtemos
\begin{equation}
    x_f = x_i + v \Delta t.
\end{equation}
%
Como podemos zerar um cronômetro e iniciar a medida de tempo a partir do valor zero no início de um experimento, podemos escolher $t_i = 0$ e $t_f = t$, logo
\begin{equation}\label{Eq:XX0AT}
  x_f = x_i + vt.\mathnote{Evolução temporal da posição para velocidade constante.}
\end{equation}

Comparando a equação acima com uma \emph{equação da reta},
\begin{equation}
    y = A + Bx,
\end{equation}
%
percebemos que o gráfico da posição em função da velocidade deve seguir uma reta:

\begin{figure}
\centering
\begin{tikzpicture}[>=Stealth]
    \draw[->] (0,0) -- (5,0) node[below left]{$t$};
    \draw[->] (0,0) -- (0,3) node[below left]{$x$};
    
    \draw[thick] (-0.5,1) -- (4.5, 2.5);
    
\end{tikzpicture}
\caption{Para o caso de \emph{velocidade constante} temos que o gráfico da velocidade é uma reta.}
\end{figure}

%%%%%%%%%%%%%%%%%%%%%%%%%%%%%%%%%%%%%%%%%%%%%%%%%%%%%%%%%%%%
\paragraph{Exemplo: Tempo para que um veículo alcance outro}
%%%%%%%%%%%%%%%%%%%%%%%%%%%%%%%%%%%%%%%%%%%%%%%%%%%%%%%%%%%%

\begin{quote}
	Em um certo instante, dois carros trafegam por uma rua retilínea de forma que suas velocidades são constantes. A distância entre eles é de \np[m]{800}. Se as velocidades são $v_1 = \np[m/s]{25,2}$ e $v_2 = \np[m/s]{30,5}$,
	\begin{itemize}
		\item[(a)] quanto tempo transcorre até que o primeiro veículo seja alcançado pelo segundo?
		\item[(b)] qual é a distância percorrida por cada um dos veículos entre o instante inicial e o instante onde o segundo alcança o primeiro?
	\end{itemize}
\end{quote}

Para determinarmos o tempo transcorrido entre o instante inicial e o instante em que o primeiro carro é alcançado pelo segundo, podemos estabelecer um eixo coordenado na direção da estrada. Tomamos o sentido para o qual ambos os veículos se deslocam como o sentido positivo. Na Figura~\ref{Fig:EsboçoPerseguicaoCarros} temos um esboço do problema.

\begin{marginfigure}
\centering
\begin{tikzpicture}[>=Stealth,
     interface/.style={
        % superfície
        postaction={draw,decorate,decoration={border,angle=-45,
                    amplitude=0.2cm,segment length=2mm}}},
    ]
    
	\draw[interface] (0,0) -- (4.5,0);

	\draw[pattern = north west lines] (0.5,0) rectangle +(0.5,0.5) (3,0) rectangle +(0.5,0.5);
	\draw[->] (0.75,0.65) +(-0.2, 0) -- node[above]{$v_2$} +(0.2,0);
	\draw[->] (3.25,0.65) +(-0.15, 0) -- node[above]{$v_1$} +(0.15,0);

	\draw[->, dotted] (0,0.25) -- (4.5,0.25) node[above left]{$x$};
\end{tikzpicture}
\caption{Esboço do problema no instante em que começamos a analisá-lo. \label{Fig:EsboçoPerseguicaoCarros}}
\end{marginfigure}

Podemos descrever a posição como função do tempo para cada um dos veículos como\footnote{Assumimos que $t_i = 0$ e $t_f = t$, o que resulta em $\Delta t = t$.}
\begin{align}
	x_1(t) &= x_1^i + v_1 t \\
	x_2(t) &= x_2^i + v_2 t.
\end{align}
%
Note que podemos escolher a origem do sistema de coordenadas para coincidir com a posição do segundo veículo. Assim, temos que
\begin{align}
	x_1^i &= \np[m]{800} \\
	x_2^i &= 0,
\end{align}
%
o que nos leva a
\begin{align}
	x_1(t) &= (\np[m]{800}) + v_1 t \\
	x_2(t) &= v_2 t.
\end{align}
%
A Figura~\ref{Fig:GraficoPosicaoPerseguicao} mostra um gráfico das posições $x_1(t)$ e $x_2(t)$.

\begin{marginfigure}
\centering
\begin{tikzpicture}[>=Stealth]

	\draw[->] (0,0) -- (4,0) node[below left]{$t$};
	\draw[->] (0,0) -- (0,3) node[below left]{$x$};

	\draw[smooth, dashdotted, name path=plotc,samples=1000,domain=0:3.75]
        plot(\x,{0.15*(1.6 + 6*0.63*\x )}) node[below right]{$x_1$};
	\draw[smooth, name path=plotc,samples=1000,domain=0:3.75]
        plot(\x,{0.15*6*0.7625*\x }) node[above right]{$x_2$};

\end{tikzpicture}
\caption{Gráficos da posição em função do tempo para os dois veículos. Veja que eventualmente as retas se cruzam, o que indica que ambos ocupam a mesma posição no eixo $x$, em um mesmo tempo $t$.\label{Fig:GraficoPosicaoPerseguicao}}
\end{marginfigure}

Para determinarmos o tempo transcorrido entre o instante inicial e o instante onde os carros se encontram, basta determinarmos $t$ ao igualar os dois valores de posição:
\begin{align}
	x_1(t) &= x_2(t) \\
	(\np[m]{800}) + v_1 t &= v_2 t \\
	(\np[m]{800}) &= (v_2 - v_1) t \\
	t &= \frac{(\np[m]{800})}{v_2-v_1},
\end{align}
%
o que resulta em
\begin{equation}
	t = \np[s]{150,9}.
\end{equation}

Para determinar as distâncias percorridas pelo veículos, basta lembrar que
\begin{equation}
	\Delta x = v t,
\end{equation}
%
o que resulta em
\begin{align}
	\Delta x_1 &= (\np[m/s]{25,2})\cdot(\np[s]{150,9}) \\
	&= \np[m]{3803,8} \\
	\Delta x_2 &= (\np[m/s]{30,5})\cdot(\np[s]{150,9}) \\
	&= \np[m]{4603,8}.
\end{align}

%%%%%%%%%%%%%%%%%%%%%%%%%%%%%%%%%%%%%%%%%%%%%%%%%%%%%%%%%%%%%%%%%%%%%%%%%%%%%%%%%
\paragraph{Exemplo: Colisão entre dois veículos que trafegam em sentidos opostos}
%%%%%%%%%%%%%%%%%%%%%%%%%%%%%%%%%%%%%%%%%%%%%%%%%%%%%%%%%%%%%%%%%%%%%%%%%%%%%%%%%

\begin{quote}
	Dois trens trafegam em sentidos opostos na mesma linha férrea retilínea. Em certo instante, a distância entre eles é de \np[km]{40,0}. Se as velocidades são $v_1 = \np[km/h]{25,0}$ e $v_2 = \np[km/h]{35,0}$ e são constantes, em que ponto da linha férrea --~em relação à posição inicial do primeiro trem~-- eles colidirão?
\end{quote}

Para determinarmos o ponto onde ocorre a colisão, devemos determinar as expressões que descrevem a posição como função do tempo, e então considerar que ao colidirem, os dois trens têm a mesma posição, para o mesmo valor de tempo. Na Figura~\ref{Fig:EsboçoColisaoTrens} temos um esboço do problema.

\begin{marginfigure}
\centering
\begin{tikzpicture}[>=Stealth,
     interface/.style={
        % superfície
        postaction={draw,decorate,decoration={border,angle=-45,
                    amplitude=0.2cm,segment length=2mm}}},
    ]
    
	\draw[interface] (0,0) -- (4.5,0);

	\draw[pattern = north west lines] (0.5,0) rectangle +(0.5,0.5) (3,0) rectangle +(0.5,0.5);
	\draw[->] (0.75,0.65) +(-0.12, 0) -- node[above]{$v_1$} +(0.12,0);
	\draw[<-] (3.25,0.65) +(-0.22, 0) -- node[above]{$v_2$} +(0.22,0);

	\draw[->, dotted] (0,0.25) -- (4.5,0.25) node[above left]{$x$};
\end{tikzpicture}
\caption{Esboço do problema no instante em que começamos a analisá-lo. \label{Fig:EsboçoColisaoTrens}}
\end{marginfigure}

Devemos escrever as expressões para a posição em função do tempo utilizando
\begin{equation}
	x(t) = x_i + vt.
\end{equation}
%
Vamos assumir que o eixo coordenado $x$ aponta na direção da linha férrea, sendo que o sentido positivo é o mesmo que o da velocidade do primeiro trem. Vamos considerar ainda que a origem do eixo está na posição inicial do primeiro trem. Assim,
\begin{align}
	x_1^i &= 0 \\
	x_2^i &= \np[km]{40,0} \\  
	v_1 &= \np[km/h]{25,0} \\
	v_2 &= \np[km/h]{-35,0}.
\end{align}
%
Temos, portanto, que as expressões para a posição em função do tempo são dadas por
\begin{align}
	x_1(t) &= (\np[km/h]{25,0})\cdot t \\
	x_2(t) &= (\np[km]{40,0}) + (-\np[km/h]{35,0})\cdot t.
\end{align}

Igualando as expressões acima, obtemos o valor de $t$ para o qual acontece a colisão:
\begin{align}
	(\np[km/h]{25,0})\cdot t &= (\np[km]{40,0}) - (\np[km/h]{35,0})\cdot t \\
	(\np[km/h]{25,0} - \np[km/h]{35,0})\cdot t &= (\np[km]{40,0}) \\
	t &=\frac{(\np[km]{40,0})}{(\np[km/h]{25,0} - \np[km/h]{35,0})} \\
	t &= \np[h]{0,67}.
\end{align}
%
Finalmente, podemos encontra a posição onde os trens colidem ao calcular a posição do primeiro trem no instante da colisão:
\begin{align}
	x_1 &= (\np[km/h]{25,0})\cdot (\np[h]{0,67}) \\
	&= \np[km]{16,67}.
\end{align}

%%%%%%%%%%%%%%%%%%%%%%%%%%%%%%%%
\section{Aceleração}
%%%%%%%%%%%%%%%%%%%%%%%%%%%%%%%%

%%%%%%%%%%%%%%%%%%%%%%%%%%%%%
\subsection{Aceleração média}
%%%%%%%%%%%%%%%%%%%%%%%%%%%%%

Da mesma forma que podemos ter variações de posição em dados intervalos de tempo, implicando na definição da velocidade, podemos ter variações da velocidade. Tais variações resultam na definição da aceleração. Portanto, se temos uma variação de velocidade em um intervalo de tempo, temos que a aceleração média será dada por
\begin{equation}
  \mean{a} = \frac{\Delta v}{\Delta t}.
\end{equation}
%
Fazendo a análise dimensional temos
\begin{align}
	[\mean{a}] &= \left[\frac{\Delta v}{\Delta t}\right] \\
		&= \frac{[\Delta v]}{[\Delta t]} \\
		&= \frac{\rm{L}}{\rm{T}^2}.
\end{align}
%
Logo, no Sistema Internacional, a aceleração é dada em $\rm{m}/\rm{s}^2$.

Assim como pudemos dar uma interpretação gráfica para a velocidade média $\mean{v}$, em um gráfico $x \times t$, podemos fazer o mesmo para a aceleração média. Observando a Figura~\ref{Fig:Interp_graf_acel_med_b}, temos que
\begin{equation}
	\mean{a} = \tan \theta = \frac{\Delta v}{\Delta t},
\end{equation}
%
isto é, a aceleração média está relacionada à inclinação da reta que liga os pontos $(t_i, v_i)$ e $(t_f, v_f)$.

\begin{marginfigure}[-4cm]
\centering
\begin{tikzpicture}[>=Stealth, extended line/.style={shorten >=-#1,shorten <=-#1},
 extended line/.default=4mm]]
    % Draw axes: acho que o |- é pra desenhar um "canto", um L
    \draw [<->,thick] (0,3) node (yaxis) [below left] {$v$}
        |- (4.3,0) node (xaxis) [below left] {$t$};
    % Desenhar função:
    \draw[densely dotted,smooth, name path=plot,samples=1000,domain=0.5:3.5]
    plot(\x,{0.4*\x^2 + 1.1 - 0.9*\x + 0.4*sin((2*\x - 0.5) r)});
    
    % linhas dos eixos à curva
    \node[below, gray](a)at(1.2,0){$t_i$};
    \node[below, gray](b)at(3,0){$t_f$};
    \path[name path=froma](a)--+(0,4);
    \path[name path=fromb](b)--+(0,4);
    \draw[dotted, thin,name intersections={of=froma and plot}](a)--(intersection-1) coordinate (plot-a-intersection)--(0,0|-intersection-1)node[left]{$x_1$};
    \draw[dotted, thin, name intersections={of=fromb and plot}](b)--(intersection-1) coordinate (plot-b-intersection)--(0,0|-intersection-1)node[left]{$x_2$};
    
    % linha entre as duas interseções (as coordenadas foram salvas acima)
    \draw[extended line, dashed] (plot-a-intersection) -- (plot-b-intersection);

    % pontos nas interseções
    \fill [opacity=1] (plot-a-intersection) circle (2pt);
    \fill [opacity=1] (plot-b-intersection) circle (2pt);
    
    % linhas complementares
    \path[name path=horiz-a] (plot-a-intersection)--+(4,0);
    \path[name path=vert-b] (plot-b-intersection)--+(0,-2);
    \draw[thin, name intersections={of=horiz-a and vert-b}] (plot-b-intersection) -- (intersection-1) coordinate (ang-reto);
    
    \draw[thin, name intersections={of=fromb and horiz-a}] (intersection-1) -- (plot-a-intersection);
    
    % marcar os ângulos
    \tkzMarkRightAngle(plot-a-intersection,ang-reto,plot-b-intersection);
    \tkzLabelAngle[pos=.15](plot-a-intersection,ang-reto,plot-b-intersection){$\cdot$};
    \path pic[draw, angle radius=7mm, pic text=$\theta$,angle eccentricity=1.3] {angle = ang-reto--plot-a-intersection--plot-b-intersection};
    
    % medidas
    \draw[|-|] ($ (plot-b-intersection) + (0.2,0) $) -- node[right]{$\Delta v$} ($ (ang-reto) + (0.2,0) $);
    
    \draw[|-|] ($ (plot-a-intersection) + (0, -0.2) $) -- node[below]{$\Delta t$} ($ (ang-reto) + (0, -0.2) $);
   
\end{tikzpicture}
\caption{Triângulo formado pela reta que liga os pontos e as linhas horizontal e vertical.\label{Fig:Interp_graf_acel_med_b}}
\end{marginfigure}

%%%%%%%%%%%%%%%%%%%%%%%%%%%%%%%%%%%
\subsection{Aceleração instantânea}
%%%%%%%%%%%%%%%%%%%%%%%%%%%%%%%%%%%

Podemos definir a aceleração instantânea como
\begin{equation}
  a = \lim_{\Delta t \to 0} \frac{\Delta v}{\Delta t}.
\end{equation}
%
Novamente em analogia com a velocidade, graficamente tal limite pode ser interpretado como a inclinação da reta tangente à curva $v(t)$ no ponto $P = (t,v)$ em que estamos interessados em calcular a aceleração, Figura~\ref{Fig:Interp_graf_acel_med_lim}.

\begin{marginfigure}[-4cm]
\centering
\begin{tikzpicture}[>=Stealth, extended line/.style={shorten >=-#1,shorten <=-#1},
 extended line/.default=3mm]] % talvez fosse melhor amplicar com scale=1.5
    % Draw axes: acho que o |- é pra desenhar um "canto", um L
    \draw [<->,thick] (0,3) node (yaxis) [below left] {$v$}
        |- (4.3,0) node (xaxis) [below left] {$t$};
    % Desenhar função:
    \draw[smooth,name path=plot,samples=1000,domain=0.5:3.5]
    plot(\x,{0.4*\x^2 + 1.1 - 0.9*\x + 0.4*sin((2*\x - 0.5) r)});
    
    % linhas dos eixos à curva
    \coordinate (a) at (1.2,0);
    \coordinate (b) at (3,0);
    \coordinate (c) at (0.7,0);
    \coordinate (d) at (3.3, 0);
    \coordinate (e) at (1.8,0);
    \coordinate (f) at (2.8,0);
    \coordinate (p) at (2.3,0);
    \path[name path=froma](a)--+(0,4);
    \path[name path=fromb](b)--+(0,4);
    \path[name path=fromc](c)--+(0,4);
    \path[name path=fromd](d)--+(0,4);
    \path[name path=frome](e)--+(0,4);
    \path[name path=fromf](f)--+(0,4);
    \path[name path=fromp](p)--+(0,4);
    \path[name intersections={of=froma and plot}](a)--(intersection-1) coordinate (plot-a-intersection)--(0,0|-intersection-1);
    \path[name intersections={of=fromb and plot}](b)--(intersection-1) coordinate (plot-b-intersection)--(0,0|-intersection-1);
    \path[name intersections={of=fromc and plot}](c)--(intersection-1) coordinate (plot-c-intersection)--(0,0|-intersection-1);
    \path[name intersections={of=fromd and plot}](d)--(intersection-1) coordinate (plot-d-intersection)--(0,0|-intersection-1);
    \path[name intersections={of=frome and plot}](e)--(intersection-1) coordinate (plot-e-intersection)--(0,0|-intersection-1);
    \path[name intersections={of=fromf and plot}](f)--(intersection-1) coordinate (plot-f-intersection)--(0,0|-intersection-1);
    \path[name intersections={of=fromp and plot}](p)--(intersection-1) coordinate (plot-p-intersection)--(0,0|-intersection-1);
    
    % linha entre as duas interseções (as coordenadas foram salvas acima)
    \draw[extended line, densely dotted] (plot-a-intersection) -- (plot-b-intersection);
    \draw[extended line, densely dotted] (plot-c-intersection) -- (plot-d-intersection);
    \draw[extended line, densely dotted] (plot-e-intersection) -- (plot-f-intersection);

    % pontos nas interseções
    \fill[opacity=1, gray] (plot-a-intersection) circle (2pt);
    \fill[opacity=1, gray] (plot-b-intersection) circle (2pt);
    \fill[opacity=1, gray] (plot-c-intersection) circle (2pt);
    \fill[opacity=1, gray] (plot-d-intersection) circle (2pt);
    \fill[opacity=1, gray] (plot-e-intersection) circle (2pt);
    \fill[opacity=1, gray] (plot-f-intersection) circle (2pt);
    
    \fill[opacity=1] (plot-p-intersection) circle (2pt) node[below]{$P$};
   
    % tangente
    \draw[smooth, densely dashed, name path=deriv,samples=1000,domain=1.5:3.1]
    plot(\x,{0.480140842*\x - 0.285634781});
     
\end{tikzpicture}
\caption{Gráfico da velocidade em função do tempo onde mostramos o processo em que tomamos o limite $\Delta t \to 0$.\label{Fig:Interp_graf_acel_med_lim}}
\end{marginfigure}


%%%%%%%%%%%%%%%%%%%%%%%%%%%%%%%%%%%%%%%%%%%%
\subsection{Aceleração como função do tempo}
%%%%%%%%%%%%%%%%%%%%%%%%%%%%%%%%%%%%%%%%%%%%

Assim como podemos descrever a posição e a velocidade como funções do tempo, podemos fazer o mesmo para a aceleração:
\begin{figure}[h]\forcerectofloat
\centering
\begin{tikzpicture}
\draw (0,0.5) node[above] {$t$};
\draw (3.1,0.5) node[above] {$a$};
\draw (0,-1) ellipse [x radius=12pt, y radius=40pt];
\draw (3.1,-1) ellipse [x radius=12pt, y radius=40pt];
\node [circle,draw,fill,scale=0.3] (A){};
\node [circle,draw,fill,scale=0.3] (B) [right=3cm of A] {};
\node [circle,draw,fill,scale=0.3] (C) [below=of A] {};
\node [circle,draw,fill,scale=0.3] (D) [right=3cm of C] {};
\node [circle,draw,fill,scale=0.3] (E) [below=of C] {};
\node [circle,draw,fill,scale=0.3] (F) [right=3cm of E] {};
\draw [thick, arrows={ - Stealth}]
(A) edge [bend left=45] node[above]{$a = a(t)$}(B)
(C) edge [bend left=45] (D)
(E) edge [bend left=45] (F);
\end{tikzpicture}
\caption{A cada valor de tempo $t$ temos um valor de aceleração $a$ associado. A função $a(t)$ descreve a relação entre essas duas variáveis.}
\end{figure}

\noindent{}Podemos ter formas complicadas para a aceleração, porém, para que possamos trabalhar situações mais simples, nos limitaremos a movimentos com \emph{aceleração constante}. Faremos isso pois além de simplificarmos o tratamento, temos um caso importante de aceleração constante, a \emph{aceleração da gravidade} próximo à superfície da Terra.

\begin{marginfigure}[-2cm]
\centering
\begin{tikzpicture}[>=Stealth, extended line/.style={shorten >=-#1,shorten <=-#1},
 extended line/.default=3mm]] % talvez fosse melhor amplicar com scale=1.5
    % Draw axes: acho que o |- é pra desenhar um "canto", um L
    \draw[->] (0,-1.3) -- (0,1.5) node[below left] {$a$};
	\draw[->] (0,0) -- (4,0) node[below left] {$t$};

    % Desenhar função:
    \draw[smooth, dashdotted, name path=plot,samples=1000,domain=0:3.5]
    plot(\x,{sin((3 * \x) r)});

	\end{tikzpicture}
\caption{Em um sistema \emph{massa-mola}, um corpo oscila devido à força exercida pela mola e devido à sua própria inércia. Nesse sistema, a aceleração não é constante, variando de acordo com $a(t) = A\omega^2\sin(\omega t)$, onde $A$ representa a amplitude de oscilação e a frequência angular $\omega$ está relaciona à frequência de oscilação.\label{Fig:Exemplo_acel_complicada}}
\end{marginfigure}

%%%%%%%%%%%%%%%%%%%%%%%%%%%%%%%%%%%%%%%%%%%%%%%%%%%%%%%%%%%%%%%%%%%%%%%%%%%%%%%%
\subsection{Evolução temporal da velocidade para o caso de aceleração constante}
%%%%%%%%%%%%%%%%%%%%%%%%%%%%%%%%%%%%%%%%%%%%%%%%%%%%%%%%%%%%%%%%%%%%%%%%%%%%%%%%

Se conhecemos a aceleração média, podemos então descrever a variação da velocidade em função do tempo como
\begin{equation}
  \Delta v = \mean{a} \Delta t,
\end{equation}
%
ou
\begin{equation}
    v_f = v_i + \mean{a} \Delta t.
\end{equation}
%
Em especial, se a aceleração é constante, então $a = \mean{a}$, e obtemos
\begin{equation}
    v_f = v_i + a \Delta t.
\end{equation}
%
Como podemos zerar um cronômetro e iniciar a medida de tempo a partir do valor zero no início de um experimento, podemos escolher $t_i = 0$ e $t_f = t$, logo
\begin{equation}\label{Eq:VV0AT}
  v_f = v_i + at.\mathnote{Evolução temporal da velocidade para aceleração constante.}
\end{equation}
%
Assim como no caso da posição em função do tempo para velocidade constante, o gráfico da velocidade em função do tempo para o caso de aceleração constante também é uma reta.

%%%%%%%%%%%%%%%%%%%%%%%%%%%%%%%%%%%%%%%%%%%%%%%%%%%%%%%%%%%%%%%%%%%%%%%%%%%%%
\section{Sentidos dos eixos de referência e sinais das variáveis cinemáticas}
\label{Sec:Sinais}
%%%%%%%%%%%%%%%%%%%%%%%%%%%%%%%%%%%%%%%%%%%%%%%%%%%%%%%%%%%%%%%%%%%%%%%%%%%%%

Ao adotarmos a reta real para descrever a posição, utilizamos o \emph{sinal} para denotar o sentido. Ao calcularmos o deslocamento $\Delta x = x_f - x_i$, temos que se o deslocamento é no sentido positivo do eixo, ele será positivo; se for no sentido negativo do eixo, então o deslocamento é negativo.

Devido à própria definição da velocidade $\mean{v} = \Delta x / \Delta t$, sabendo que $\Delta t$ é sempre positivo, verificamos que se o movimento é no sentido positivo do eixo, então ela tem valores positivos. Caso a velocidade seja no sentido negativo do eixo, então seu valor é negativo.

Para a aceleração, no entanto, é mais complicado definirmos o sinal apropriado. Se temos um deslocamento para a direita, por exemplo, temos uma velocidade positiva enquanto ele ocorre. A aceleração, porém, pode ser positiva, negativa, ou nula, sem que haja mudança no sinal do deslocamento, ou da velocidade. Isso ocorre pois a aceleração descreve \emph{alterações na velocidade}. Um movimento no sentido positivo do eixo pode ocorrer de forma que a velocidade aumente, diminua, ou permaneça constante. As acelerações, nesses três casos seriam maior, menor, e igual a zero, respectivamente. Para percebermos o porque, basta verificarmos o sinal da variação da velocidade: no primeiro caso, $\Delta v > 0$, no segundo $\Delta v < 0$, no terceiro $\Delta v = 0$.

\begin{figure}[!h]
\centering
\begin{tikzpicture}[>=Stealth,
     interface/.style={
        % superfície
        postaction={draw,decorate,decoration={border,angle=-45,
                    amplitude=0.2cm,segment length=2mm}}},
    ]
    
    % Divisão
    \draw (-5,0) -- (5,0);
    \draw (0,5.5) -- (0,-5.75);

    % Figura superior esquerda
    \draw[interface](-4.5,4)--(-0.5,4);
    \draw[->] (-4.5,3.7) -- (-0.5,3.7) node[below left]{$x$};
    
    \draw[pattern = north west lines] (-4,4) rectangle +(0.5,0.5);
    
    \draw[->] (-4,4.7) -- node[above]{$v_i = \np[m/s]{1}$} +(0.5,0);
    
    \draw[interface](-4.5,2)--(-0.5,2);
    \draw[->] (-4.5,1.7) -- (-0.5,1.7) node[below left]{$x$};
    
    \draw[pattern = north west lines] (-2,2) rectangle +(0.5,0.5);
    
    \draw[->] (-2,2.7) -- node[above]{$v_f = \np[m/s]{4}$} +(1,0);
    
    \path (-4.5,0.75) circle (1.25pt) node[right] {$\Delta v = \np[m/s]{3} \ \Rightarrow \ a > 0$};
    
    % Figura superior direita
    \draw[interface](0.5,4)--(4.5,4);
    \draw[->] (0.5,3.7) -- (4.5,3.7) node[below left]{$x$};
    
    \draw[pattern = north west lines] (1,4) rectangle +(0.5,0.5);
    
    \draw[->] (1,4.7) -- node[above]{$v_i = \np[m/s]{5}$} +(1.25,0);
    
    \draw[interface](0.5,2)--(4.5,2);
    \draw[->] (0.5,1.7) -- (4.5,1.7) node[below left]{$x$};
    
    \draw[pattern = north west lines] (3,2) rectangle +(0.5,0.5);
    
    \draw[->] (3,2.7) -- node[above]{$v_f = \np[m/s]{2}$} +(0.5,0);
    
    \path (0.5,0.75) circle (1.25pt) node[right] {$\Delta v = \np[m/s]{-3} \ \Rightarrow \ a < 0$};
     % Figura inferior esquerda
    \draw[interface](-4.5,-2)--(-0.5,-2);
    \draw[<-] (-4.5,-2.3) node[below right]{$x$} -- (-0.5,-2.3);
    
    \draw[pattern = north west lines] (-4,-2) rectangle +(0.5,0.5);
    
    \draw[->] (-4,-1.3) -- node[above]{$v_i = \np[m/s]{-1}$} +(0.5,0);
    
    \draw[interface](-4.5,-4)--(-0.5,-4);
    \draw[<-] (-4.5,-4.3) node[below right]{$x$} -- (-0.5,-4.3);
    
    \draw[pattern = north west lines] (-2,-4) rectangle +(0.5,0.5);
    
    \draw[->] (-2,-3.3) -- node[above]{$v_f = \np[m/s]{-4}$} +(1,0);
    
    \path (-4.5,-5.25) circle (1.25pt) node[right] {$\Delta v = \np[m/s]{-3} \ \Rightarrow \ a < 0$};
    % Figura inferior direita
    \draw[interface](0.5,-2)--(4.5,-2);
    \draw[<-] (0.5,-2.3) node[below right]{$x$} -- (4.5,-2.3);
    
    \draw[pattern = north west lines] (1,-2) rectangle +(0.5,0.5);
    
    \draw[->] (1,-1.3) -- node[above]{$v_i = \np[m/s]{-5}$} +(1.25,0);
    
    \draw[interface](0.5,-4)--(4.5,-4);
    \draw[<-] (0.5,-4.3) node[below right]{$x$} -- (4.5,-4.3);
    
    \draw[pattern = north west lines] (3,-4) rectangle +(0.5,0.5);
    
    \draw[->] (3,-3.3) -- node[above]{$v_f = \np[m/s]{-2}$} +(0.5,0);
    
    \path (0.5,-5.25) circle (1.25pt) node[right] {$\Delta v = \np[m/s]{3} \ \Rightarrow \ a > 0$};
\end{tikzpicture}
\caption{A orientação do eixo também é responsável pelo sinal da aceleração. Note que nem sempre uma diminuição do valor absoluto da velocidade implica em uma aceleração negativa.}
\end{figure}

No caso de termos uma velocidade no sentido negativo, temos uma situação análoga. No entanto, quando verificamos um aumento do valor numérico da velocidade, mantendo o sentido negativo, temos que $\Delta v < 0$. Isso representa o oposto da situação em que o deslocamento é no sentido positivo, onde tinhamos um valor positivo quando o valor da velocidade aumentava. Além disso, no deslocamento no sentido negativo, se o valor da velocidade diminui, então $\Delta v > 0$. Novamente, isso é o oposto do que acontece em um deslocamento no sentido positivo.

Podemos agrupar essas observações acerca da aceleração nos seguintes casos:
\begin{itemize}
    \item Se não há variação da velocidade, então a aceleração é nula.
    \item Se a velocidade aumenta em valor, então a aceleração tem o mesmo sinal que a velocidade.
    \item Se a velocidade diminui em valor, então a aceleração tem o sinal oposto ao da velocidade.
\end{itemize}

Finalmente, devemos nos lembrar de que a escolha do sentido positivo do eixo é arbitrária. Podemos escolher de maneira que seja mais conveniente, o que em geral significa minimizar o número de grandezas com sinal negativo. Uma vez escolhido um sentido positivo, devemos nos ater a tal escolha, de maneira a garantir que a descrição do movimento seja consistente.

%%%%%%%%%%%%%%%%%%%%%%%%%%%%%%%%%%%%%%%%%%%%%%%%%%%%%%%%%%%%%%%%%%%%%%%%%
\section{Interpretação da área de um gráfico $v \times t$ e $a \times t$}
%%%%%%%%%%%%%%%%%%%%%%%%%%%%%%%%%%%%%%%%%%%%%%%%%%%%%%%%%%%%%%%%%%%%%%%%%

\begin{marginfigure}
\centering
\begin{tikzpicture}[>=Stealth, extended line/.style={shorten >=-#1,shorten <=-#1},
 extended line/.default=3mm]] % talvez fosse melhor amplicar com scale=1.5
    % Draw axes: acho que o |- é pra desenhar um "canto", um L
    \draw [<->,thick] (0,3) node (yaxis) [below left] {$v$}
        |- (4.3,0) node (xaxis) [below left] {$t$};
    % Desenhar função:
    \draw[smooth,name path=plota,samples=1000,domain=0:3]
    plot(\x,{2});
    
     \fill [pattern=north west lines, pattern color=gray, domain=0.5:2.5, variable=\x]
      (0.5, 0) node[below]{$t_i$}
      -- plot ({\x}, {2})
      -- (2.5, 0) node[below]{$t_f$}
      -- cycle;
      
      \draw[dashed] (0.5, 0) -- (0.5, 2);
      \draw[dashed] (2.5, 0) -- (2.5, 2);
      \path (0, 2) node[left]{$v_0$};
      
      \draw[|-|] (3.2, 0) -- node[right]{$v_0$} (3.2, 2);
      \draw[|-|] (0.5, -0.6) -- node[below]{$\Delta t$} (2.5, -0.6);
     
\end{tikzpicture}
\caption{A área hachurada está relacionada ao deslocamento em um movimento com velocidade $v_0$ no intervalo de tempo destacado.\label{Fig:Graf_area_graf_v}}
\end{marginfigure}

Se temos que um objeto se move com velocidade constante, a distância percorrida por ele será
\begin{equation}
  \Delta x = v \Delta t.
\end{equation}
%
Ao fazer um gráfico de $v\times t$, percebemos que a equação acima determina a \emph{área} delimitada pela curva $v(t)$, o eixo $t$ e os eixos verticais que passam por $t_1$ e $t_2$. Se tivéssemos uma situação mais complicada, com uma velocidade $v(t)$ que variasse de uma maneira mais complexa, poderíamos determinar a distância percorrida entre dois instantes $t_1$ e $t_2$ simplesmente calculando a área entre a curva, o eixo $x$ e os eixos verticais passando por $t_1$ e $t_2$. Se a curva $v(t)$ está abaixo do eixo $t$, isso significa que a velocidade é negativa, ou seja, nesta região o objeto estará ``voltando'' e o deslocamento será, consequentemente, negativo.

\begin{marginfigure}
\centering
\begin{tikzpicture}[>=Stealth, extended line/.style={shorten >=-#1,shorten <=-#1},
 extended line/.default=3mm]] % talvez fosse melhor amplicar com scale=1.5
    % Draw axes: acho que o |- é pra desenhar um "canto", um L
    \draw [<->,thick] (0,3) node (yaxis) [below left] {$v$}
        |- (4.3,0) node (xaxis) [below left] {$t$};
    % Desenhar função:
    \draw[smooth,name path=plota,samples=1000,domain=0:2.8]
    plot(\x,{1.5 - 3.4*\x + 8.3 * \x*\x - 5.9 * \x*\x*\x + 1.2 * \x*\x*\x*\x});

    \coordinate (a) at (0.25,0);
    \coordinate (b) at (2.75,0);
    \path[name path=froma](a)--+(0,3);
    \path[name path=fromb](b)--+(0,3);
    \draw[dashed, name intersections={of=froma and plota}](a) node[below]{$t_i$} -- (intersection-1);
	\draw[dashed, name intersections={of=fromb and plota}](b) node[below]{$t_f$} -- (intersection-1);

    \fill [pattern=north west lines, pattern color=gray, domain=0.25:2.75, variable=\x]
     	  (0.25, 0)
    	  -- plot ({\x}, {1.5 - 3.4*\x + 8.3 * \x*\x - 5.9 * \x*\x*\x + 1.2 * \x*\x*\x*\x})
          -- (2.75, 0)
          -- cycle;
\end{tikzpicture}
\caption{Podemos utilizar a área para determinar o deslocamento em um caso mais complexo, onde a velocidade varia arbitrariamente.\label{Fig:Graf_area_graf_v_complicado}}
\end{marginfigure}

Para determinar o valor numérico do deslocamento através da área, podemos dividir a região hachurada em várias barras de uma largura arbitrária $\Delta t$ e altura dada pela própria curva $v(t)$. Somando os valores obtidos para cada uma das barras, podemos determinar -- pelo menos de forma aproximada -- o deslocamento total. Se tomarmos intervalos $\Delta t$ sucessivamente menores, eventualmente conseguiremos calcular a área com grande precisão e verificaremos que nesse caso temos exatamente a área ``abaixo'' da curva.

\begin{marginfigure}
\centering
\begin{tikzpicture}[>=Stealth, extended line/.style={shorten >=-#1,shorten <=-#1},
 extended line/.default=3mm]] % talvez fosse melhor amplicar com scale=1.5
    % Draw axes: acho que o |- é pra desenhar um "canto", um L
    \draw [<->,thick,gray] (0,3) node (yaxis) [below left] {$v$}
        |- (4.3,0) node (xaxis) [below left] {$t$};
    % Desenhar função:
    \draw[smooth,name path=plota,samples=1000,domain=0:2.8]
    plot(\x,{1.5 - 3.4*\x + 8.3 * \x*\x - 5.9 * \x*\x*\x + 1.2 * \x*\x*\x*\x});

	\coordinate (a) at (0.25,0);
    \coordinate (b) at (0.5,0);
    \coordinate (c) at (0.75,0);
    \coordinate (d) at (1,0);
    \coordinate (e) at (1.25,0);
    \coordinate (f) at (1.5,0);
    \coordinate (g) at (1.75,0);
    \coordinate (h) at (2,0);
    \coordinate (i) at (2.25,0);
    \coordinate (j) at (2.5,0);
    \coordinate (k) at (2.75,0);
    \path[name path=froma](a)--+(0,3);
    \path[name path=fromb](b)--+(0,3);
    \path[name path=fromc](c)--+(0,3);
    \path[name path=fromd](d)--+(0,3);
    \path[name path=frome](e)--+(0,3);
    \path[name path=fromf](f)--+(0,3);
    \path[name path=fromg](g)--+(0,3);
    \path[name path=fromh](h)--+(0,-0.6);
    \path[name path=fromi](i)--+(0,-0.6);
    \path[name path=fromj](j)--+(0,-0.6);
    \path[name path=fromk](k)--+(0,3);
    \path[dashed, name intersections={of=froma and plota}](a) -- (intersection-1) coordinate (int-a);
	\path[dashed, name intersections={of=fromb and plota}](b) -- (intersection-1) coordinate (int-b);
	\path[dashed, name intersections={of=fromc and plota}](c) -- (intersection-1) coordinate (int-c);
	\path[dashed, name intersections={of=fromd and plota}](d) -- (intersection-1) coordinate (int-d);
	\path[dashed, name intersections={of=frome and plota}](e) -- (intersection-1) coordinate (int-e);
	\path[dashed, name intersections={of=fromf and plota}](f) -- (intersection-1) coordinate (int-f);
	\path[dashed, name intersections={of=fromg and plota}](g) -- (intersection-1) coordinate (int-g);
	\path[dashed, name intersections={of=fromh and plota}](h) -- (intersection-1) coordinate (int-h);
	\path[dashed, name intersections={of=fromi and plota}](i) -- (intersection-1) coordinate (int-i);
	\path[dashed, name intersections={of=fromj and plota}](j) -- (intersection-1) coordinate (int-j);
	\path[dashed, name intersections={of=fromk and plota}](k) -- (intersection-1) coordinate (int-k);

	\draw[very thin, pattern=north west lines, pattern color=gray] (b) rectangle (int-a);
	\draw[very thin, pattern=north west lines, pattern color=gray] (c) rectangle (int-b);
	\draw[very thin, pattern=north west lines, pattern color=gray] (d) rectangle (int-c);
	\draw[very thin, pattern=north west lines, pattern color=gray] (e) rectangle (int-d);
	\draw[very thin, pattern=north west lines, pattern color=gray] (f) rectangle (int-e);
	\draw[very thin, pattern=north west lines, pattern color=gray] (g) rectangle (int-f);
	\draw[very thin, pattern=north west lines, pattern color=gray] (h) rectangle (int-g);
	\draw[very thin, pattern=north west lines, pattern color=gray] (i) rectangle (int-h);
	\draw[very thin, pattern=north west lines, pattern color=gray] (j) rectangle (int-i);
	\draw[very thin, pattern=north west lines, pattern color=gray] (k) rectangle (int-j);

	\draw[dashed] (a) node[below]{$t_i$} -- (int-a);
	\draw[dashed] (k) node[below right]{$t_f$} -- (int-k);
\end{tikzpicture}
\caption{Para determinar o valor da área, basta dividirmos a região em barras com uma largura $\Delta t$ arbitrária e uma altura $v(t)$.\label{Fig:Graf_area_graf_v_complicado_barras}}
\end{marginfigure}

\begin{marginfigure}
\centering
\begin{tikzpicture}[>=Stealth, extended line/.style={shorten >=-#1,shorten <=-#1},
 extended line/.default=3mm]] % talvez fosse melhor amplicar com scale=1.5
    % Draw axes: acho que o |- é pra desenhar um "canto", um L
    \draw [<->,thick,gray] (0,3) node (yaxis) [below left] {$v$}
        |- (4.3,0) node (xaxis) [below left] {$t$};
    % Desenhar função:
    \draw[smooth,name path=plota,samples=1000,domain=0:2.8]
    plot(\x,{1.5 - 3.4*\x + 8.3 * \x*\x - 5.9 * \x*\x*\x + 1.2 * \x*\x*\x*\x});

	\coordinate (a) at (0.25,0);
    \coordinate (b) at (0.35,0);
    \coordinate (c) at (0.45,0);
    \coordinate (d) at (0.55,0);
    \coordinate (e) at (0.65,0);
    \coordinate (f) at (0.75,0);
    \coordinate (g) at (0.85,0);
    \coordinate (h) at (0.95,0);
    \coordinate (i) at (1.05,0);
    \coordinate (j) at (1.15,0);
    \coordinate (k) at (1.25,0);
    \coordinate (l) at (1.35,0);
    \coordinate (m) at (1.45,0);
    \coordinate (n) at (1.55,0);
    \coordinate (o) at (1.65,0);
    \coordinate (p) at (1.75,0);
    \coordinate (q) at (1.85,0);
    \coordinate (r) at (1.95,0);
    \coordinate (s) at (2.05,0);
    \coordinate (t) at (2.15,0);
    \coordinate (u) at (2.25,0);
    \coordinate (v) at (2.35,0);
    \coordinate (x) at (2.45,0);
    \coordinate (w) at (2.55,0);
    \coordinate (y) at (2.65,0);
    \coordinate (z) at (2.75,0);
    \path[name path=froma](a)--+(0,3);
    \path[name path=fromb](b)--+(0,3);
    \path[name path=fromc](c)--+(0,3);
    \path[name path=fromd](d)--+(0,3);
    \path[name path=frome](e)--+(0,3);
    \path[name path=fromf](f)--+(0,3);
    \path[name path=fromg](g)--+(0,3);
    \path[name path=fromh](h)--+(0,3);
    \path[name path=fromi](i)--+(0,3);
    \path[name path=fromj](j)--+(0,3);
    \path[name path=fromk](k)--+(0,3);
    \path[name path=froml](l)--+(0,3);
    \path[name path=fromm](m)--+(0,3);
    \path[name path=fromn](n)--+(0,3);
    \path[name path=fromo](o)--+(0,3);
    \path[name path=fromp](p)--+(0,3);
    \path[name path=fromq](q)--+(0,3);
    \path[name path=fromr](r)--+(0,3);
    \path[name path=froms](s)--+(0,-0.6);
    \path[name path=fromt](t)--+(0,-0.6);
    \path[name path=fromu](u)--+(0,-0.6);
    \path[name path=fromv](v)--+(0,-0.7);
    \path[name path=fromx](x)--+(0,-0.6);
    \path[name path=fromw](w)--+(0,-0.6);
    \path[name path=fromy](y)--+(0,3);
    \path[name path=fromz](z)--+(0,3);
    \path[dashed, name intersections={of=froma and plota}](a) -- (intersection-1) coordinate (int-a);
	\path[dashed, name intersections={of=fromb and plota}](b) -- (intersection-1) coordinate (int-b);
	\path[dashed, name intersections={of=fromc and plota}](c) -- (intersection-1) coordinate (int-c);
	\path[dashed, name intersections={of=fromd and plota}](d) -- (intersection-1) coordinate (int-d);
	\path[dashed, name intersections={of=frome and plota}](e) -- (intersection-1) coordinate (int-e);
	\path[dashed, name intersections={of=fromf and plota}](f) -- (intersection-1) coordinate (int-f);
	\path[dashed, name intersections={of=fromg and plota}](g) -- (intersection-1) coordinate (int-g);
	\path[dashed, name intersections={of=fromh and plota}](h) -- (intersection-1) coordinate (int-h);
	\path[dashed, name intersections={of=fromi and plota}](i) -- (intersection-1) coordinate (int-i);
	\path[dashed, name intersections={of=fromj and plota}](j) -- (intersection-1) coordinate (int-j);
	\path[dashed, name intersections={of=fromk and plota}](k) -- (intersection-1) coordinate (int-k);
	\path[dashed, name intersections={of=froml and plota}](l) -- (intersection-1) coordinate (int-l);
	\path[dashed, name intersections={of=fromm and plota}](m) -- (intersection-1) coordinate (int-m);
	\path[dashed, name intersections={of=fromn and plota}](n) -- (intersection-1) coordinate (int-n);
	\path[dashed, name intersections={of=fromo and plota}](o) -- (intersection-1) coordinate (int-o);
	\path[dashed, name intersections={of=fromp and plota}](p) -- (intersection-1) coordinate (int-p);
	\path[dashed, name intersections={of=fromq and plota}](q) -- (intersection-1) coordinate (int-q);
	\path[dashed, name intersections={of=fromr and plota}](r) -- (intersection-1) coordinate (int-r);
	\path[dashed, name intersections={of=froms and plota}](s) -- (intersection-1) coordinate (int-s);
	\path[dashed, name intersections={of=fromt and plota}](t) -- (intersection-1) coordinate (int-t);
	\path[dashed, name intersections={of=fromu and plota}](u) -- (intersection-1) coordinate (int-u);
	\path[dashed, name intersections={of=fromv and plota}](v) -- (intersection-1) coordinate (int-v);
	\path[dashed, name intersections={of=fromx and plota}](x) -- (intersection-1) coordinate (int-x);
	\path[dashed, name intersections={of=fromw and plota}](w) -- (intersection-1) coordinate (int-w);
	\path[dashed, name intersections={of=fromy and plota}](y) -- (intersection-1) coordinate (int-y);
	\path[dashed, name intersections={of=fromz and plota}](z) -- (intersection-1) coordinate (int-z);

	\draw[very thin, pattern=north west lines, pattern color=gray] (b) rectangle (int-a);
	\draw[very thin, pattern=north west lines, pattern color=gray] (c) rectangle (int-b);
	\draw[very thin, pattern=north west lines, pattern color=gray] (d) rectangle (int-c);
	\draw[very thin, pattern=north west lines, pattern color=gray] (e) rectangle (int-d);
	\draw[very thin, pattern=north west lines, pattern color=gray] (f) rectangle (int-e);
	\draw[very thin, pattern=north west lines, pattern color=gray] (g) rectangle (int-f);
	\draw[very thin, pattern=north west lines, pattern color=gray] (h) rectangle (int-g);
	\draw[very thin, pattern=north west lines, pattern color=gray] (i) rectangle (int-h);
	\draw[very thin, pattern=north west lines, pattern color=gray] (j) rectangle (int-i);
	\draw[very thin, pattern=north west lines, pattern color=gray] (k) rectangle (int-j);
	\draw[very thin, pattern=north west lines, pattern color=gray] (l) rectangle (int-k);
	\draw[very thin, pattern=north west lines, pattern color=gray] (m) rectangle (int-l);
	\draw[very thin, pattern=north west lines, pattern color=gray] (n) rectangle (int-m);
	\draw[very thin, pattern=north west lines, pattern color=gray] (o) rectangle (int-n);
	\draw[very thin, pattern=north west lines, pattern color=gray] (p) rectangle (int-o);
	\draw[very thin, pattern=north west lines, pattern color=gray] (q) rectangle (int-p);
	\draw[very thin, pattern=north west lines, pattern color=gray] (r) rectangle (int-q);
	\draw[very thin, pattern=north west lines, pattern color=gray] (s) rectangle (int-r);
	\draw[very thin, pattern=north west lines, pattern color=gray] (t) rectangle (int-s);
	\draw[very thin, pattern=north west lines, pattern color=gray] (u) rectangle (int-t);
	\draw[very thin, pattern=north west lines, pattern color=gray] (v) rectangle (int-u);
	\draw[very thin, pattern=north west lines, pattern color=gray] (x) rectangle (int-v);
	\draw[very thin, pattern=north west lines, pattern color=gray] (w) rectangle (int-x);
	\draw[very thin, pattern=north west lines, pattern color=gray] (y) rectangle (int-w);
	\draw[very thin, pattern=north west lines, pattern color=gray] (z) rectangle (int-y);

	\draw[dashed] (a) node[below]{$t_i$} -- (int-a);
	\draw[dashed] (z) node[below right]{$t_f$} -- (int-z);
\end{tikzpicture}
\caption{Podemos melhorar a aproximação diminuindo a largura das barras, obtendo um erro tão pequeno quanto necessário.\label{Fig:Graf_area_graf_v_complicado_barras_estreito}}
\end{marginfigure}

Para o caso de um gráfico de $a \times t$, temos uma situação análoga: se a aceleração for constante, a área entre a curva, o eixo $x$, e os eixos verticais passando por $t_1$ e $t_2$ será igual à variação da velocidade $\Delta v = a \Delta t$. Desenvolvendo um raciocínio análogo ao caso anterior para o cálculo da área entre a curva $a(t)$ e o eixo $x$, concluímos que a variação da velocidade para casos em que a aceleração não é constante pode ser calculada através da área ``abaixo'' da curva. Devemos, novamente, subtrair a área das regiões abaixo do eixo $t$.

%%%%%%%%%%%%%%%%%%%%%%%%%%%%%%%%%%%%%%%%%%%%%%%%%%%%%%%%%%%%%%%%%%%%%%%
\section{Equações cinemáticas para movimentos com aceleração constante}
%%%%%%%%%%%%%%%%%%%%%%%%%%%%%%%%%%%%%%%%%%%%%%%%%%%%%%%%%%%%%%%%%%%%%%%

Apesar de ser perfeitamente aceitável tratar uma situação em que a aceleração varia, isso não é uma tarefa muito fácil utilizando as ferramentas matemáticas que temos disponíveis. Por isso, vamos tratar com mais detalhes o caso da aceleração constante. Um exemplo de movimento de tal tipo é o caso de movimentos submetidos à aceleração da gravidade, que veremos neste capítulo para movimentos exclusivamente verticais, mas que serão vistos em duas dimensões no Capítulo~\ref{Chap:MovimentoBidimensional}. 

Quando um objeto cai livremente próximo da superfície da Terra, ele sofre uma aceleração para baixo com módulo\footnote{Esse valor não é o mesmo em todos os pontos da superfície da Terra, porém vamos utilizar \np[m/s^2]{9,8} como um valor aproximado para qualquer ponto.} \np[m/s^2]{9,8}. Essa aceleração é comum a todos os objetos, independentemente de suas massas, caso a \emph{força de arrasto}\footnote{Esta força é a resistência ao deslocamento em um meio fluido, como o ar, e será discutida em mais detalhes no Capítulo~\ref{Chap:Dinâmica}.} seja desprezível. A existência dessa aceleração se deve à força fundamental da natureza denominada \emph{força gravitacional}, responsável pela atração entre corpos como --~por exemplo~-- um objeto qualquer e a Terra, a Terra e a Lua, o Sol e a Terra, etc. Veremos adiante que essa força tem uma dependência direta na massa dos corpos, o que resulta na independência da aceleração gravitacional em relação à massa do corpo que é atraído pela Terra.

A aceleração da gravidade próximo da superfície da terrestre é a principal justificativa para o estudo de movimentos com aceleração constante. Em geral, não há razões para supor que um objeto qualquer que se movimente esteja sujeito a uma aceleração constante, exceto se ele estiver claramente sob efeito da força gravitacional, sendo as demais forças desprezíveis. Nas próximas seções verificaremos expressões que relacionarão as variáveis cinemáticas, nos permitindo realizar previsões teóricas acerca do movimento em diversas situações.

%%%%%%%%%%%%%%%%%%%%%%%%%%%%%%%%%%%%%%%%%%%%%%%%%%%%%%%%%%%%%%%%%%%%%%%%%%%%%%%%%%%
\paragraph{Sinal da aceleração em movimentos envolvendo a aceleração gravitacional}
%%%%%%%%%%%%%%%%%%%%%%%%%%%%%%%%%%%%%%%%%%%%%%%%%%%%%%%%%%%%%%%%%%%%%%%%%%%%%%%%%%%

Como discutido na Seção\ref{Sec:Sinais}, a determinação do sinal da aceleração depende de o módulo da velocidade aumentar ou diminuir, e do próprio sinal da velocidade. Aplicando tal raciocínio ao caso da aceleração gravitacional percebemos que os casos podem ser reduzidos a dois: ou o sobe e a velocidade diminui, ou ele desce e ela aumenta. 

No primeiro caso, sabemos que a aceleração tem o sinal oposto ao da velocidade. Se o eixo de referência aponta para cima, então a velocidade é positiva e a aceleração é negativa. Se o eixo de referência aponta para baixo, então a velocidade é negativa e a aceleração é positiva.

No segundo caso, a aceleração tem o mesmo sinal que a velocidade. Logo, se o eixo de referência aponta para cima, então a velocidade é negativa e a aceleração também é negativa. Se o eixo aponta para baixo, a velocidade é positiva e a aceleração também é positiva.

Em ambos os casos, podemos afirmar que \emph{se o eixo aponta para cima}, então \emph{a aceleração é negativa}, e que \emph{se o eixo aponta para baixo} então \emph{a aceleração é positiva}. Verificaremos no Capítulo~\ref{Chap:MovimentoBidimensional} que essa relação pode ser entendida de maneira muito mais simples e intuitiva através de uma análise da aceleração do ponto de vista de \emph{vetores}. Mais adiante, no Capítulo~\ref{Chap:Dinamica}, verificaremos que a relação entre aceleração e força, o que deve tornar a determinação do sinal da aceleração gravitacional ainda mais intuitivo.

%%%%%%%%%%%%%%%%%%%%%%%%%%%%%%%%%%%%%%
\subsection{Equação para a velocidade}
%%%%%%%%%%%%%%%%%%%%%%%%%%%%%%%%%%%%%%

Verificamos anteriormente que se a aceleração é constante, temos que $\mean{a} = a$ e, portanto,
\begin{equation*}
  v_f = v_i + at.
\end{equation*}
%
Temos portanto uma relação \emph{linear} entre a velocidade e o tempo, ou seja, existe uma proporção direta entre o valor da velocidade e o tempo transcorrido sob efeito da aceleração. Note que as demais expressões que serão deduzidas adiante \emph{não serão lineares}, portanto não poderemos utilizar ``leis de proporção''\footnote{Regra de três.} para as outras variáveis no caso de movimento com aceleração constante.

%%%%%%%%%%%%%%%%%%%%%%%%%%%%%%%%%%%%
\subsection{Equações para a posição}
%%%%%%%%%%%%%%%%%%%%%%%%%%%%%%%%%%%%

\begin{marginfigure}[2cm]
\centering
\begin{tikzpicture}[>=Stealth, extended line/.style={shorten >=-#1,shorten <=-#1},
 extended line/.default=3mm]] % talvez fosse melhor amplicar com scale=1.5
    % Draw axes: acho que o |- é pra desenhar um "canto", um L
    \draw [<->,thick,gray] (0,3) node (yaxis) [below left] {$v$}
        |- (4.3,0) node (xaxis) [below left] {$t$};
    % Desenhar função:
    \draw[smooth,name path=plot,samples=1000,domain=0:2.8]
    plot(\x,{0.7 + 0.5 * \x});

    \coordinate (a) at (0.5,0);
    \coordinate (b) at (2.5,0);
    \path[name path=froma](a)--+(0,3);
    \path[name path=fromb](b)--+(0,3);

 	\draw[dashed, name intersections={of=froma and plot}](a) node[below]{$t_i$} --(intersection-1) coordinate (plot-a-intersection)--(0,0|-intersection-1) node[left]{$v_i$};
 	\draw[dashed, name intersections={of=fromb and plot}](b) node[below]{$t_f$} --(intersection-1) coordinate (plot-b-intersection)--(0,0|-intersection-1) node[left]{$v_f$};

    \fill [pattern=north west lines, pattern color=gray, domain=0.5:2.5, variable=\x]
     	  (0.5, 0)
    	  -- plot ({\x}, {0.7 + 0.5 * \x})
          -- (2.5, 0)
          -- cycle;

	\draw[dashed](plot-a-intersection) -- +(2,0);

	\node[circle, fill=white, scale=0.7] (anode) at (1.5,0.5) {$A_1$};
	\node[circle, fill=white, scale=0.7] (anode) at (2,1.3) {$A_2$};

\end{tikzpicture}
\caption{Para o caso de aceleração constante, podemos calcular a área a dividindo em um retângulo e um triângulo.\label{Fig:Graf_area_acel_const}}
\end{marginfigure}

Podemos calcular uma expressão para a evolução temporal da posição se considerarmos a Figura~\ref{Fig:Graf_area_acel_const}. Se a aceleração é constante, vimos que a velocidade deve ser descrita por uma reta em um gráfico $v\times t$. Sabemos ainda que o deslocamento é dado pela área abaixo da curva $v(t)$. Logo, temos que
\begin{align}
	\Delta x &= A \\
			 &= A_1 + A_2.
\end{align}
%
A área $A_1$ é dada por
\begin{equation}
	A_1 = v_i \Delta t,
\end{equation}
%
enquanto $A_2$ é dada por
\begin{equation}
	A_2 = \frac{(v_f - v_i)\Delta t}{2}.
\end{equation}
%
Logo,
\begin{equation}\label{Eq:DeltaComoSomaAreas}
  \Delta x = v_i\Delta t + \frac{(v_f - v_i)\Delta t}{2}.
\end{equation}
%
Utilizando a equação $v_f = v_i + at$, e fazendo ainda $t_i = 0$ e $t_f = t$, temos
\begin{equation}
  \Delta x = v_i t + \frac{(v_i + at - v_i) t}{2}
\end{equation}
%
e, finalmente,
\begin{equation}\label{Eq:XX0V0TAT22}
  x_f = x_i + v_i t +\frac{at^2}{2}.\mathnote{Evolução temporal da posição para aceleração constante (1\textordfeminine~Equação).}
\end{equation}

Caso não haja informação sobre a velocidade inicial, a equação acima pode ser reescrita com o auxílio da $v_f = v_i + at$:
\begin{align}
  x_f &= x_i + (v_f - at) t + \frac{at^2}{2} \\
  &= x_i + v_f t + \frac{at^2 - 2at^2}{2},
\end{align}
%
resultando em
\begin{equation}
  x_f = x_i + v_f t - \frac{at^2}{2}.\mathnote{Evolução temporal da posição para aceleração constante (2\textordfeminine~Equação).}
\end{equation}

Novamente considerando que para o caso especial de uma aceleração constante, temos que a velocidade é uma reta, e com o auxílio da Equação~\eqref{Eq:DeltaComoSomaAreas} podemos escrever a velocidade média como
\begin{align}
  \mean{v} &= \frac{\Delta x}{\Delta t} \\
  &= \frac{v_i \Delta t + [(v_f - v_i)/2] \Delta t}{\Delta t} \\
  &= \frac{v_i + v_f}{2}. \label{Eq:VelMediaComoMedAritParaAcelConst}
\end{align}
%
Note que a velocidade média é dada pela média aritmética entre a velocidade inicial $v_i$ e a velocidade final $v_f$, \emph{porém isso só é verdade para casos onde a aceleração é constante}.

Utilizando a própria definição da velocidade média, podemos escrever
\begin{align}
    \Delta x &= \mean{v} \Delta t \\
    x_f - x_i &= \mean{v} \Delta t \\
    x_f &= x_i + \mean{v} \Delta t.
\end{align}
%
Se tomarmos $t_i = 0$ e $t_f = t$, podemos utilizar o resultado para a velocidade média dado pela Equação~\eqref{Eq:VelMediaComoMedAritParaAcelConst} acima, para escrever
\begin{equation}
  x_f = x_i + \frac{v_i + v_f}{2} t.\mathnote{Evolução temporal da posição para aceleração constante (3\textordfeminine~Equação).}
\end{equation}

%%%%%%%%%%%%%%%%%%%%%%%%%%%%%%%%%%
\subsection{Equação de Torricelli}
%%%%%%%%%%%%%%%%%%%%%%%%%%%%%%%%%%

A partir da Equação~\ref{Eq:VV0AT}, podemos isolar o tempo e obter
\begin{equation}
  t = \frac{v_f-v_i}{a}.
\end{equation}
%
Substituindo esta expressão na Equação~\ref{Eq:XX0V0TAT22}, obtemos
\begin{align}
  x_f - x_i &= v_i \left(\frac{v_f-v_i}{a}\right) + \frac{a}{2} \left(\frac{v_f-v_i}{a}\right)^2 \\
  &= \frac{v_f v_i - v_i^2}{a} + \frac{v_f^2 + v_i^2-2v_fv_i}{2a}.
\end{align}
%
multiplicando os dois lados da equação por $2a$, temos
\begin{equation}
  2a\Delta x = 2v_i v_f - 2v_i^2 + v_f^2 +v_i^2 - 2v_f v_i.
\end{equation}
%
Eliminando o primeiro e o quarto termos à direita e somando os restantes, obtemos
\begin{equation}
  v_f^2 = v_i^2 + 2 a \Delta x. \mathnote{Equação de Torricelli.}
\end{equation}

%%%%%%%%%%%%%%%%%%%%%%%%%%%%%%%%%%%%%%%%%%%%%%%
\subsection{Variáveis ausentes em cada equação}
%%%%%%%%%%%%%%%%%%%%%%%%%%%%%%%%%%%%%%%%%%%%%%%

As cinco equações obtidas para a cinemática com aceleração constante envolvem as variáveis $x_i$, $x_f$, $v_i$, $v_f$, $a$ e $t$. Porém cada uma das equações deixa algum desses parâmetros de fora. Isso pode ser usado para a solução de problemas quando tal informação não é conhecida. A Tabela~\ref{Tab:EqsCinematicasVarAusentes} apresenta as equações e destaca a variável ausente em cada uma delas.
\begin{table}[!h]
\centering
\begin{tabular}{cc}
\toprule
Equação & Variável ausente\\
\midrule
$v_f = v_i + at$ & $\Delta x$ \\
$x_f = x_i + v_i t + at^2 / 2$ & $v_f$ \\
$x_f = x_i + v_f t - at^2 / 2$ & $v_i$ \\
$x_f = x_i + (v_i + v_f) t / 2$ & $a$ \\
$v_f^2 = v_i^2 + 2 a \Delta x$ & $t$ \\
\bottomrule
\end{tabular}
\caption{Relação das equações para a cinemática unidimensional e a variável ausente em cada uma delas. \label{Tab:EqsCinematicasVarAusentes}}
\end{table}

%%%%%%%%%%%%%%%%%%%%%%%%%%%%%%%%%%%%%%%%
\paragraph{Exemplo: Lançamento vertical}
%%%%%%%%%%%%%%%%%%%%%%%%%%%%%%%%%%%%%%%%

\begin{quote}
    Uma pedra é lançada verticalmente para cima, sendo que após \np[s]{4,0} ela se encontra a uma altura de \np[m]{9,6} em relação ao ponto de lançamento. Qual era a velocidade inicial da pedra?
\end{quote}

\begin{marginfigure}
\centering
\begin{tikzpicture}[>=Stealth]
    \draw[pattern = north west lines, dashed] (0,-1.5) circle (3mm);
    \draw[pattern = north west lines] (0.5,0) circle (3mm);
    
    \draw[->] (0,-1.2) -- +(0,0.5) node[below right]{$v_i$};
    
    \draw[->] (-0.5,-2) -- (-0.5,1) node[below left]{$x$};
    \draw[|-|] (-0.5,-1.5) node[left]{$x_i$} -- (-0.5,0) node[left]{$x_f$};
\end{tikzpicture}
\caption{Corpo lançado verticalmente.}
\end{marginfigure}

Sabemos que a pedra sofre um deslocamento $\Delta y = \np[m]{9,6}$ em relação ao ponto de lançamento, porém não sabemos se tal deslocamento se refere ao movimento de subida ou de descida. Sabemos, no entanto, que o tempo necessário para que o movimento aconteça é de \np[s]{4,0}.

Nas equações para movimentos com aceleração constante, verificamos que existe uma expressão que relaciona as informações que temos (deslocamento, tempo e aceleração) e a informação que desejamos (velocidade inicial):
\begin{align}
    \Delta x &= v_i t + \frac{a}{2}t^2 \\
    &= v_i t - \frac{g}{2}t^2.
\end{align}
%
Veja que a aceleração aparece com um sinal negativo pois adotamos um eixo vertical $y$ que aponta para cima. Isolando a velocidade, obtemos
\begin{equation}
    v_i = \frac{\Delta x}{t} + \frac{gt}{2},
\end{equation}
%
o que resulta em
\begin{equation}
    v_i = \np[m/s]{22,0}.
\end{equation}

%%%%%%%%%%%%%%%%%%%%%%%%%%%%%%%%%%%%%%%%%%%%%%%%%%%%%%%%%%%%%%%%%%%%%%%%%%%%%
\paragraph{Exemplo: Distância percorrida durante um movimento de queda livre}
%%%%%%%%%%%%%%%%%%%%%%%%%%%%%%%%%%%%%%%%%%%%%%%%%%%%%%%%%%%%%%%%%%%%%%%%%%%%%

\begin{quote}
	Um corpo cai de uma altura de \np[m]{100,0}. Determine qual é a distância que ele percorre nos primeiros 10\% e nos últimos 10\% do tempo de queda. Despreze os efeitos da resistência do ar.
\end{quote}

Em primeiro lugar, devemos determinar qual é o tempo necessário para que o corpo percorra a distância de \np[m]{100,0}. Para isso, basta utilizarmos
\begin{equation}
	\Delta x = vt + \frac{at^2}{2},
\end{equation}
%
sendo que $v = 0$, uma vez que o corpo cai e --~portanto~-- subentende-se que partiu do repouso. Para descrever a queda, podemos utilizar um eixo que aponta verticalmente para baixo, assim temos que a aceleração da gravidade é dada por
\begin{align}
	a &= +g \\
	&= \np[m/s^2]{9,8}.
\end{align}
%
Assim,
\begin{equation}
	\Delta x = \frac{gt^2}{2},
\end{equation}
%
de onde obtemos
\begin{align}
	t &= \sqrt{\frac{2 \Delta x}{g}} \\
	&= \np[s]{4,52}.
\end{align}

Para determinarmos a distância percorrida nos primeiros 10\% do tempo, basta utilizarmos a expressão para $\Delta x$ acima:
\begin{align}
	\Delta x_{10\%} &= \frac{gt^2}{2} \\
	&= \frac{(\np[m/s^2]{9,8}) \cdot (\np[s]{0,452})^2}{2} \\
	&= \np[m]{1,0},
\end{align}
%
onde \np[s]{0,452} corresponde a 10\% do tempo total transcorrido. Para determinar a distância percorrida nos últimos 10\%, basta utilizarmos a expressão acima para calcular qual é a distância percorrida no período inicial que corresponde a 90\% do tempo, depois subtrair o resultado da distância total. Logo,
\begin{align}
	\Delta x_{90\%} &= \frac{gt^2}{2} \\
	&= \frac{(\np[m/s^2]{9,8}) \cdot (\np[s]{4,066})^2}{2} \\
	&= \np[m]{81,0}
\end{align}
%
e verificamos que nos últimos 10\% são percorridos
\begin{equation}
	\Delta x_{100\%} - \Delta x_{90\%} = \np[m]{19,0}.
\end{equation}
%
Em outras palavras, verificamos que nos primeiros 10\% do tempo, o corpo percorre 1\% da distância, enquanto nos últimos 10\% do tempo, ele percorre 19\% da distância.

%%%%%%%%%%%%%%%%%%%%%%%%%%%%%%%%%%%%%%%%%%%%%%%%%%%%%%%%%%%%%%%%%%%%
\paragraph{Exemplo: Distância percorrida no último segundo de queda}
%%%%%%%%%%%%%%%%%%%%%%%%%%%%%%%%%%%%%%%%%%%%%%%%%%%%%%%%%%%%%%%%%%%%

\begin{quote}
    Um corpo é solto a partir do repouso, caindo uma distância de \np[m]{100} até atingir o solo. Qual é a distância percorrida no último segundo de queda? Desconsidere a resistência do ar ao movimento.
\end{quote}

Podemos determinar a distância percorrida no último segundo de queda através de
\begin{equation}
    \Delta x = v_f t - a\frac{t^2}{2},
\end{equation}
%
bastanto determinar a velocidade final através de
\begin{equation}
    v_f^2 = v_i^2 + 2 a \Delta x.
\end{equation}
%
Em ambos os casos, a aceleração $a$ será a própria aceleração gravitacional, com o sinal positivo, já que adotamos um eixo de referência que aponta para baixo (Figura~\ref{Fig:DeslUltimoSegQueda}).

\begin{marginfigure}
\centering
\begin{tikzpicture}[>=Stealth,
    interface/.style={
        % superfície
        postaction={draw,decorate,decoration={border,angle=-45,
                    amplitude=0.2cm,segment length=2mm}}},
    ]
    \draw[interface](-0.5,0) -- (0.5,0);
    
    \draw[fill] (0, 3) circle (2pt);
    
    \draw[densely dotted] (0,1.18) circle (2pt);
    
    \draw[->] (-0.75, 3.5) -- (-0.75, -0.5) node[above left]{$x$};
    \draw[|-|] (-0.75, 3) -- (-0.75,1.18);
    \draw[|<->|] (-0.75, 1.18) -- node[left]{$\Delta x$} (-0.75,0);
    
\end{tikzpicture}
\caption{Desejamos determinar a distância $\Delta x$ percorrida no último segundo de queda.\label{Fig:DeslUltimoSegQueda}}
\end{marginfigure}

Como a velocidade inicial é zero, temos
\begin{align}
    v_f &= \sqrt{2 a \Delta x} \\
    &= \sqrt{2 \cdot (\np[m/s^2]{9,8}) \cdot (\np[m]{100})} \\
    &= \np[m/s]{44,27}
\end{align}
%
Substituindo tal resultado na expressão para o deslocamento, temos
\begin{align}
    \Delta x &= v_f t - a\frac{t^2}{2} \\
    &= (\np[m/s]{44,27})\cdot(\np[s]{1,0}) - \frac{(\np[m/s^2]{9,8})\cdot(\np[s]{1,0})^2}{2} \\
    &= \np[m]{39,37}.
\end{align}
%
Note que utilizamos como tempo a própria duração do intervalo de tempo em que ocorre o deslocamento $\Delta x$, uma vez que o valor de $t$ que aparece na expressão para o cálculo do deslocamento é o tempo transcorrido entre o corpo passar pelo ponto inicial $x_i$ e o ponto final $x_f$.\footnote{Lembre-se que para obter a expressão $\Delta x = v_ft-at^2/2$ e as expressões $\Delta x = v_it+at^2/2$ e $v_f = v_i + a t$, escolhemos ``zerar o cronômetro'' no início do intervalo, utilizando $t_i = 0$ e $t_f = t$.}

%%%%%%%%%%%%%%%%%%%%%%%%%%%%%%%%%%%%%%%%%%%%%%%%%%%%%%%%%%%%%%
\paragraph{Exemplo: Velocidade inicial em uma pista retilínea}
%%%%%%%%%%%%%%%%%%%%%%%%%%%%%%%%%%%%%%%%%%%%%%%%%%%%%%%%%%%%%%

\begin{quote}
    Em uma corrida de carros, na reta final, um competidor cruza a linha de chegada com uma velocidade de \np[km/h]{268,3}. Sabendo que a reta tem um comprimento de \np[m]{300,0} e que o tempo necessário para que o veículo a percorresse foi de \np[s]{4,82}, determine a velocidade com que o competidor entrou na reta final.
\end{quote}

Sabemos que a velocidade final é dada por
\begin{equation*}
    \np[km/h]{268,3} = \np[m/s]{74,53}.
\end{equation*}
%
Não sabemos qual é o valor da aceleração, por isso não podemos determinar a velocidade inicial a partir da velocidade final e do tempo. No entanto, temos a distância percorrida, logo, podemos utilizar a expressão
\begin{equation}
    \Delta x = \frac{v_i + v_f}{2} t,
\end{equation}
%
de onde obtemos
\begin{align}
    v_i &= 2\frac{\Delta x}{t} - v_f \\
    &= 2 \frac{(\np[m]{300,0})}{(\np[s]{4,82})} - (\np[m/s]{74,53}) \\
    &= \np[m/s]{49,95},
\end{align}
%
ou, em quilômetros por hora
\begin{equation}
    v_i = \np[km/h]{179,82}.
\end{equation}

%%%%%%%%%%%%%%%%%%%%%%%%%%%%%%%%%%%%%%%%%%%%%%%%%%%%%
\paragraph{Exemplo: Distância entre gotas sucessivas}
%%%%%%%%%%%%%%%%%%%%%%%%%%%%%%%%%%%%%%%%%%%%%%%%%%%%%

\begin{quote}
	Determine uma expressão para a distância entre duas gotas que caem de uma torneira, sendo o tempo entre a queda de cada gota dado por $\tau$.
\end{quote}

Podemos descrever a posição de cada gota de acordo com
\begin{equation}
	x_f = x_i + vt + \frac{at^2}{2}.
\end{equation}
%
Como as gotas caem, temos que a velocidade inicial é nula. Escolhendo um eixo $x$ vertical, com sentido positivo para baixo, temos que $a = g$. Também podemos assumir que a origem do eixo está na posição de onde caem as gotas, o que nos permite escolher $x_i = 0$ e $x_f = x$. Assim,
\begin{equation}
	x = \frac{gt^2}{2}.
\end{equation}

Note que na expressão acima já temos implicitamente a escolha $t_i = 0$ e $t_f = t$, o que implica que estamos registrando o tempo a partir do momento em que a gota cai. Para que possamos determinar a distância entre as gotas, devemos perceber que quando a segunda gota cai, a primeira gota já está se movendo há um tempo $\tau$. Assim, se escrevermos a as expressões em termos do tempo no ``cronômetro da segunda gota'' temos para as posições:
\begin{align}
	x_1 &= \frac{g(t + \tau)^2}{2} \\
	x_2 &= \frac{gt^2}{2}.
\end{align}
%
Temos entaão para a distância entre as gotas
\begin{align}
	x_1 - x_2 &= \frac{g(t + \tau)^2}{2} - \frac{gt^2}{2}\\
			  &= \frac{g}{2} \left[(t + \tau)^2 - t^2\right] \\
			  &= \frac{g}{2} \left[t^2 + \tau^2 + 2\tau t - t^2\right] \\
			  &= \frac{g}{2} [\tau^2 + 2\tau t],
\end{align}
%
de onde podemos verificar que a distância entre as gotas aumenta linearmente com o tempo.

%%%%%%%%%%%%%%%%%%%%%%%%%%%%%%%%%%%%%%%%%%%%
\paragraph{Exemplo: Profundidade de um poço}
%%%%%%%%%%%%%%%%%%%%%%%%%%%%%%%%%%%%%%%%%%%%

\begin{quote}
Ao soltarmos uma pedra dentro de um poço, percebemos que o som é ouvido após \np[s]{4,2}. Se a velocidade do som no ar é de \np[m/s]{343,4} e é constante, qual é a profundidade do poço?
\end{quote}

Se soltamos a pedra, sabemos que sua velocidade inicial é nula. Assim, temos que a distância percorrida por ela até chegar à água é dada por
\begin{equation}
	\Delta x = \frac{gt^2}{2},
\end{equation}
%
onde utilizamos $a = g$ pois escolhemos um eixo $x$ que aponta verticalmente para baixo. Temos então que o tempo $t_q$ para que a pedra atinja a água é dado por
\begin{equation}
	t_q = \sqrt{\frac{2 \Delta x}{g}}.
\end{equation}
%
O som se propaga com velocidade constante, portanto temos que o tempo que ele demora parapercorrer a altura do poço pode ser calculado através de 
\begin{equation}
	\Delta x = v t,
\end{equation}
%
resultando em
\begin{equation}
	t_s = \frac{\Delta x}{v_s}.
\end{equation}

Sabemos que a soma dos dois tempos é igual a \np[s]{4,2}:
\begin{align}
	t_q + t_s &= (\np[s]{4,2}) \\
	\sqrt{\frac{2 \Delta x}{g}} + \frac{\Delta x}{v_s} &= (\np[s]{4,2}).
\end{align}
%
Fazendo a mudança de variáveis
\begin{equation}
	\Delta x = \lambda^2,
\end{equation}
%
na expressão acima, temos
\begin{equation}
	\sqrt{\frac{2}{g}} \lambda + \frac{1}{v_s} \lambda^2 - (\np[s]{4,2}) = 0.
\end{equation}
%
Temos, portanto, uma equação de segundo grau, substituindo os valores da aceleração da gravidade e da velocidade do som, temos
\begin{equation}
	(\np[s/m]{2.9121E-3})\cdot \lambda^2 + (\np[s/m^{\nicefrac{1}{2}}]{0.45175})\cdot \lambda - (\np[s]{4,2}) = 0,
\end{equation}
%
o que resulta em
\begin{align}
	\lambda' &= \np[m^{\nicefrac{1}{2}}]{-163.93} \\
	\lambda''&= \np[m^{\nicefrac{1}{2}}]{8.798}.
\end{align}
%
Vamos descartar o primeiro valor para $\lambda$ pois sabemos que ele corresponde à raiz quadrada de uma distância e deve ser positivo. Determinando $\Delta x$, temos
\begin{equation}
	\Delta x = \np[m]{77.40}.
\end{equation}

%%%%%%%%%%%%%%%%%%%%%%%%%%
\section{Seções opcionais}
%%%%%%%%%%%%%%%%%%%%%%%%%%

%%%%%%%%%%%%%%%%%%%%%%%
\subsection{Acelerador}
%%%%%%%%%%%%%%%%%%%%%%%

Um erro bastante comum é imaginar que o acelerador de um carro determina a aceleração do veículo. Certamente existe uma relação entre o estado de aceleração do veículo e o quanto pressionamos o pedal, porém essa relação não é simples. A função do pedal do acelerador é controlar a quantidade de ar que é admitida no motor, bem como a quantidade de combustível que é injetada nos cilindros ou nos dutos de admissão, o que determina a \emph{potência} desenvolvida pelo motor\footnote{Os números de potência divulgados pelo fabricante se referem à potência máxima que o motor é capaz de desenvolver. A potência que é efetivamente desenvolvida depende da velocidade de rotação, da quantidade de ar admitido e da quantidade de combustível injetada.}.

Se um veículo parte do repouso, ao pressionarmos o pedal do acelerador, percebemos que ele começa a ganhar velocidade. No entanto, essa situação não passa de um caso específico. Podemos analisar algumas situações que mostram que a aceleração não está diretamente ligada ao deslocamento do pedal:
\begin{itemize}
	\item Se um carro se desloca em um trecho plano de uma rodovia com velocidade constante, a aceleração é nula, pois qualquer valor de aceleração implica em uma alteração no valor da velocidade. Sabemos que para que o veículo se mantenha com tal velocidade, precisamos manter o pedal pressionado em certa posição. Se pressionarmos mais o pedal, o carro passa a ganhar velocidade, ou seja, ele passa a acelerar. Por outro lado, se ao invés de pressionarmos mais o pedal, pressionarmos \emph{menos}, percebemos que a velocidade do veículo passa a \emph{diminuir}. Isto é, o pedal do acelerador, sob certas condições, pode causar uma desaceleração do veículo.
	\item Se o carro chega a uma subida íngreme, se o pedal do acelerador for mantido na mesma posição, a velocidade passará a diminuir. Isso se deve ao fato de que em uma subida, existe uma componente da força peso que aponta na direção oposta ao movimento, fazendo com que ocorra uma desaceleração do veículo. Em muitos casos, para que possamos manter a velocidade constante, basta pressionarmos ainda mais o pedal do acelerador. Se a subida for suficientemente íngreme, isso não vai bastar: mesmo que pressionemos o pedal até o fim, a velocidade continuará a diminuir.
	\item Se após um trecho plano o carro chega a uma descida, se mantivermos o pedal na mesma posição, o carro passará a ganhar velocidade (ou seja, ele passará a acelerar mesmo que não tenhamos sinalizado tal intenção, o que seria feito ao se pressionar o pedal). Dependendo da inclinação da descida, podemos fazer com que o carro passe a se mover com velocidade constante ao diminuirmos a pressão sobre o pedal, permitindo que ele volte um pouco. No entanto, se a inclinação for grande, podemos tirar completamente o pé do pedal e ainda assim continuar ganhando velocidade.
\end{itemize}

Em outros idiomas o pedal do acelerador tem nomes diferentes, que não dão margem a uma interpretação equivocada: em inglês, o pedal do acelerador se chama \emph{throttle}, o que literalmente significa \emph{estrangular} e se refere ao fato de que o pedal regula a quantidade de ar admitida no motor. Em alemão, o pedal se chama \emph{gaspedal}, e tem um significado similar. Talvez um termo mais adequado para o pedal do acelerador seria \emph{pedal de potência}.% ou pedal de ``atitude''?

Outro comando que pode suscitar erros são os freios. Em geral associamos os freios a uma desaceleração de um veículo, o que em geral é verdade. No entanto, em alguns casos podemos ter uma aceleração de um veículo \emph{apesar de aplicarmos os freios}. Se temos uma descida íngrime, ao tirarmos o pé do pedal do acelerador, podemos continuar \emph{acelerando}, como discutimos acima. Um recurso que podemos utilizar para controlar a velocidade é utilizar os freios, porém --~dependendo da intensidade da frenagem~-- podemos ter
\begin{itemize}
	\item Uma simples diminuição da aceleração, caso a intensidade da frenagem seja pequena;
	\item Se a intensidade da frenagem compensar exatamente a componente da força peso que tende a acelerar o veículo na descida, passaremos a nos mover com velocidade constante;
	\item Finalmente, se a intesidade da frenagem for maior do que aquela que mantém o carro com velocidade constante, então teremos uma diminuição da velocidade do veículo.
\end{itemize}

Um motorista experiente é capaz de utilizar os recursos de ``aceleração/potência'' e de frenagem do veículo para garantir uma condução suave:
\begin{itemize}
	\item ao fazer com que o veículo ganhe velocidade, quando esta se aproxima do valor que ele julga ser adequado para o trecho de pista em que transita, ele deve diminuir progressivamente a pressão no pedal do acelerador, fazendo com que a aceleração real do veículo diminua e eventualmente se atinja a velocidade constante desejada;
	\item Em uma frenagem, os freios devem começar a ser aplicados com pouca intensidade e com antecedência. Após esse período de frenagem suave, ao se aproximar do ponto de parada, os freios devem ser aplicados com mais intensidade, sendo que o aumento deve ser progressivo. Finalmente, quando o veículo estiver próximo de parar, a pressão no pedal do freio deve ser aliviada, porém não completamente: isso evita o ``chacoalhão'' de uma parada brusca.\footnote{Em uma frenagem de emergência, no entanto, estamos interessados na maior diminuição possível da velocidade, no menor espaço possível. Nesse caso os freios devem ser aplicados com toda a intensidade possível, porém sem deixar que as rodas se travem --~pois isso fará com que se perca o controle direcional do veículo~--. Nesse tipo de frenagem, o sistema ABS é muito útil.}
\end{itemize}

Finalmente, note que uma apreciação mais profunda das situações discutidas aqui exigem conhecimentos de \emph{dinâmica} e de \emph{energia e potência}. Tais conteúdos serão discutidos nos capítulos posteriores e poderemos revisitar essas questões posteriormente.

%%%%%%%%%%%%%%%%%%%%%%
%\section{Questionário}
%%%%%%%%%%%%%%%%%%%%%%

%\begin{question}[type={exam}]
%Uma questão.
%\end{question}

%\begin{question}[type={exam}]
%Dois trens andam na mesma linha férrea, no mesmo sentido, porém a velocidade do segundo é maior. Num dado instante, estão a uma dada distância. Além disso, há um desvio a uma certa distância, sendo que o primeiro trem tomará um dos ramos da estrada férrea e o outro tomará o outro ramo. Qual deve ser a distância máxima entre o primeiro trem e o desvio para que não haja colisão?
%\end{question}
