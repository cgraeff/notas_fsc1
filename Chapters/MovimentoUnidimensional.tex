%%%%%%%%%%%%%%%%%%%%%%%%%%%%%%%%%%
\chapter{Movimento Unidimensional}
\label{Chap:MovimentoUnidimensional}
%%%%%%%%%%%%%%%%%%%%%%%%%%%%%%%%%%

%\minitoc

%\clearpage

%%%%%%%%%%%%%%%%%%%%
\section{Introdução}
%%%%%%%%%%%%%%%%%%%%

{\it
O primeiro passo para que possamos estudar a mecânica é a definição das variáveis físicas que descrevem o movimento dos corpos. Vamos então definir precisamente posição, velocidade e aceleração. Nos capítulos seguintes, veremos que esta última está ligada à força a que o corpo está sujeito e que nos dará uma forma simples de prever seu movimento.
}

%%%%%%%%%%%%%%%%%%%%%%%%%%%%%%%%%%
\section{Movimento unidimensional}
%%%%%%%%%%%%%%%%%%%%%%%%%%%%%%%%%%

Definimos como sendo unidimensional o movimento que ocorre ao longo de uma reta, que denominamos como \emph{direção} do movimento. Essa definição é útil por ser simples e -- como veremos nos capítulos seguintes -- é capaz de fornecer uma descrição geral ao simplesmente adicionarmos mais dois eixos de movimento.

Trataremos em todos os capítulos apenas movimentos de \emph{corpos rígidos}, isto é, corpos cujas partes que o constituem não se movem em relação umas às outras. Para tais corpos, podemos separar o movimento em uma translação do \emph{centro de massa} e uma rotação em torno do centro massa\footnote{Essa separação é conhecida como Teorema de Mozzi-Chasles.}. O centro de massa é um ponto que substitui o sistema para fins de determinação da translação do corpo, sendo que para corpos simétricos e de densidade uniforme ele se localiza no centro do corpo, como veremos no Capítulo~\ref{Chap:CentroDeMassaEMomentoLinear}. Trataremos as rotações somente no Capítulo~\ref{Chap:Rotacoes}, nos preocupando somente com a translação do centro de massa até lá.\footnote{Uma maneira equivalente é tratar o corpo como uma partícula -- isto é, um corpo cujas dimensões são desprezíveis --, o que efetivamente elimina a rotação do corpo.}

%%%%%%%%%%%%%%%%%%%%%%%%%%%%%%%%
\section{Posição e Deslocamento}
%%%%%%%%%%%%%%%%%%%%%%%%%%%%%%%%

%%%%%%%%%%%%%%%%%%%%
\subsection{Posição}
%%%%%%%%%%%%%%%%%%%%

O primeiro passo para que possamos determinar a posição de um corpo  é verificar qual é a direção onde ocorre o movimento. Podemos então colocar um objeto em um ponto qualquer de tal reta:

\begin{figure}
\centering
\begin{tikzpicture}[
     interface/.style={
        % superfície
        postaction={draw,decorate,decoration={border,angle=-45,
                    amplitude=0.2cm,segment length=2mm}}},
    ]
    % Interface
    \draw[gray,line width=.5pt,interface](-4,0)--(4,0);
    
    % bloco
    \draw[pattern=north west lines, pattern color=gray] (2.5,0) rectangle (3.5,1);

\end{tikzpicture}
\end{figure}

\pagebreak
\noindent{}Claramente, tal descrição é insuficiente. Para determinar a posição do corpo, precisamos de um \emph{ponto de referência}. A partir desse ponto, podemos então determinar a posição medindo a distância entre ele e o corpo:

\begin{figure}\forceversofloat
\centering
\begin{tikzpicture}[
     interface/.style={
        % superfície
        postaction={draw,decorate,decoration={border,angle=-45,
                    amplitude=0.2cm,segment length=2mm}}},
    ]
    % Interface
    \draw[gray,line width=.5pt,interface](-4,0)--(4,0);
    
    % bloco
    \draw[pattern=north west lines, pattern color=gray] (2.5,0) rectangle (3.5,1);
    \fill (3,0.5) circle (1pt);

    % origem
    \draw[fill] (0,0) circle (2pt);
        
    % distância
    \draw[|-|] (0,1.2)--node[above]{$D$}(3,1.2);
\end{tikzpicture}
\caption{Podemos utilizar um ponto de referência para ajudar a determinar a posição de um objeto.}
\end{figure}

Tal descrição ainda é insuficiente, pois podemos ter outro objeto que pode estar à mesma distância da origem:

\begin{figure}\forceversofloat
\centering
\begin{tikzpicture}[
     interface/.style={
        % superfície
        postaction={draw,decorate,decoration={border,angle=-45,
                    amplitude=0.2cm,segment length=2mm}}},
    ]
    % Interface
    \draw[gray,line width=.5pt,interface](-4,0)--(4,0);
    
    % bloco
    \draw[pattern=north west lines, pattern color=gray] (2.5,0) rectangle (3.5,1);
    \fill (3,0.5) circle (1pt);
    
    \draw[pattern=north west lines, pattern color=gray, dotted] (-2.5,0) rectangle (-3.5,1);
    \fill[gray] (-3,0.5) circle (1pt);

    % origem
    \draw[fill] (0,0) circle (2pt);
        
    % distância
    \draw[|-|] (0,1.2)--node[above]{$D$}(3,1.2);
    \draw[|-|,gray] (0,1.2)--node[above]{$D$}(-3,1.2);
\end{tikzpicture}
\caption{Somente as informações de direção e de distância não são suficientes para determinar a posição.}
\end{figure}

\noindent{}Podemos definir dois \emph{sentidos} na figura acima: à esquerda da origem, ou à direita dela. Com essas três informações -- direção, módulo, e sentido -- podemos determinar com exatidão a posição de um corpo qualquer.

Podemos denotar o sentido por um sinal se adotarmos a \emph{reta real} para descrever a posição:

\begin{figure}\forceversofloat
\centering
\begin{tikzpicture}[
     >=Stealth,
     interface/.style={
        % superfície
        postaction={draw,decorate,decoration={border,angle=-45,
                    amplitude=0.2cm,segment length=2mm}}},
    ]
    % Interface
    \draw[gray,line width=.5pt,interface](-4,0)--(4,0);
    
    % bloco
    \draw[pattern=north west lines, pattern color=gray] (2.5,0) rectangle (3.5,1);
    \fill (3,0.5) circle (1pt);
    
    \draw[pattern=north west lines, pattern color=gray] (-2.5,0) rectangle (-3.5,1);
    \fill (-3,0.5) circle (1pt);

    % origem
    \draw[fill] (0,0) circle (2pt);
        
    % reta real
    \draw[->] (-4.5,-0.5)--(4.5,-0.5) node[below]{$x$};
    \foreach \x in {-4,...,4}
        \draw (\x cm,-0.4) -- (\x cm,-0.6)
            node[below]{\small$\x$};
\end{tikzpicture}
\caption{Podemos utilizar a reta real para descrever a posição de um corpo. Desta forma, podemos diferencias posições nos diferentes sentidos do eixo através do \emph{sinal positivo ou negativo}.\label{Fig:dois_blocos}}
\end{figure}

\noindent{}onde temos que as posições dos blocos são dadas por
\begin{align}
    x_1 &= \np[m]{-3} \\
    x_2 &= \np[m]{4}.
\end{align}

Para um deslocamento unidimensional, isto é, o deslocamento ao longo de uma reta, denominamos tal eixo como \emph{eixo $x$}. A direção do eixo é arbitrária, podendo ser horizontal, vertical\footnote{Ao tratarmos de movimentos unidimensionais verticais, por exemplo, podemos utilizar $x$. Quando trabalhamos em duas dimensões, no entanto, é preferível que o eixo vertical seja denominado $y$.} ou mesmo inclinada, bastando ser na direção do movimento unidimensional. O sentido positivo do eixo também é arbitrário, e podemos fazer essa escolha livremente.

Em alguns casos, podemos utilizar a distância até a origem para expressar a posição mesmo para um movimento que não é retilíneo, caso não haja ambiguidade em relação à definição da localização. Um exemplo disso são estradas nas quais se utilizam marcadores de distância. Se necessitamos declarar o endereço de uma propriedade ao longo de uma rodovia, podemos utilizar a distância em relação a um marco inicial. Apesar de esse claramente não ser um caso unidimensional, pois o deslocamento não será em uma linha reta, podemos marcar um ponto de maneira única através da distância \emph{ao longo} da estrada até o marco inicial.

Como visto no capítulo anterior, a maioria das medidas físicas tem uma dimensão. No caso da posição, como ela é descrita através de uma medida de distância entre a origem e a posição do corpo, tempos que a dimensão é a de \emph{comprimento} e -- no Sistema Internacional -- suas unidades são o metro.

%%%%%%%%%%%%%%%%%%%%%%%%%
\subsection{Deslocamento}
%%%%%%%%%%%%%%%%%%%%%%%%%

Vamos considerar um deslocamento do bloco da direita na Figura~\ref{Fig:dois_blocos} para a posição $x = -\np[m]{1,0}$. Podemos medir seu deslocamento entre a posição inicial e a final utilizando uma trena e obteríamos um deslocamento de \np[m]{5} para a esquerda ao longo da reta, porém se sabemos os valores numéricos associados às posições inicial e final no eixo $x$, podemos calcular esse valor facilmente fazendo\footnote{A notação usando $\Delta$ representa a variação de uma variável qualquer. Vamos utilizá-la para posição em vários eixos ($\Delta x$, $\Delta y$, $\Delta z$), tempo ($\Delta t$), vetores ($\Delta\vec{r}$), etc.}
\begin{align}
  \Delta x &= x_f - x_i \\
  &= (-\np[m]{1,0}) - (\np[m]{4,0}) \\
  &= -\np[m]{5,0}.
\end{align}
%
O que dizer sobre o sinal negativo? Esse sinal significa que o deslocamento se deu no \emph{sentido negativo do eixo}\footnote{Lembre-se que o sentido do eixo é arbitrário. Nesse caso o sentido positivo é para a diteita e o negativo, consequentemente, para a esquerda}. Ao medirmos, o valor da medida não é suficiente para descrevermos o deslocamento. Temos que declarar que o deslocamento foi -- nesse caso -- para a esquerda. Portanto, o deslocamento tem um módulo (\np[m]{5,0}), uma direção (ao longo do eixo $x$) e um sentido (para a esquerda, ou no sentido negativo do eixo)\footnote{Veremos mais adiante que essas propriedades são características de vetores e serão muito importantes para descrevermos o movimento em duas e três dimensões.}, da mesma forma que a posição. Se o deslocamento fosse no sentido positivo do eixo, o resultado do cálculo de $\Delta x$ seria positivo. 

O deslocamento é dado através da diferença entre posições. Como vimos no capítulo anterior, só podemos somar, subtratir e igualar termos que têm a mesma dimensão. Logo, concluímos que o deslocamento tem dimensão de \emph{comprimento} e suas unidades são o metro no SI, assim como a posição.

Claramente temos que se as posições inicial e final são iguais, o deslocamento será zero. Apesar de a utilidade de tal definição ser pouco evidente agora, veremos adiante que isso faz sentido para as grandezas físicas, pois no caso de uma força conservativa -- por exemplo -- temos que o trabalho é nulo quando o deslocamento é zero.

%%%%%%%%%%%%%%%%%%%%%%%%%%%%%%%%%
\subsection{Deslocamento escalar}
%%%%%%%%%%%%%%%%%%%%%%%%%%%%%%%%%

Algo importante a se notar é que o deslocamento é a diferença de posição entre duas posições quaisquer ocupadas por um corpo. Consequentemente, para um veículo que se desloca durante um dia de trabalho, por exemplo, os valores de deslocamento em relação à posição inicial -- a garagem, por exemplo -- será diferente para cada momento do dia. Quando o veículo retorna à garagem, seu deslocamento será nulo, pois as posições inicial e final são a mesma. Se verificarmos o hodômetro do veículo, no entanto, veremos um valor diferente de zero. Este valor pode ser denominado de \emph{deslocamento escalar}\footnote{Apesar de ser algo mais ligado à nossa experiência cotidiana de deslocamento, o deslocamento escalar será de pouca utilidade.} e é calculado pela soma do \emph{módulo} de todos os deslocamentos efetuados pelo veículo:
\begin{equation}
  d_s = |\Delta x_1| + |\Delta x_2| + |\Delta x_3| + \dots + |\Delta x_n|.
\end{equation}

Novamente, temos que a dimensão é de \emph{comprimento} e as unidades no SI são metros, uma vez que o deslocamento escalar é determinado a partir de uma equação e da soma de termos com tais dimensões.


%%%%%%%%%%%%%%%%%%%%%%%%%%%%%%%%%%%%%%%%%
\subsection{Posição como função do tempo}
%%%%%%%%%%%%%%%%%%%%%%%%%%%%%%%%%%%%%%%%%

Se ocorre movimento, podemos dizer que a cada instante de tempo $t$, temos um valor de posição $x$ diferente. Se nos lembrarmos do conceito de funções, temos que dados dois grupos de números, uma função é a operação matemática que liga elementos do primeiro grupo a elementos do segundo\footnote[][-2cm]{Lembre-se de que dois elementos do grupo $x$ podem levar a um mesmo elemento do grupo $y$, porém um elemento de $x$ não pode levar a dois elementos de $y$.}:
\begin{figure}
\centering
\begin{tikzpicture}
\draw (0,0.5) node[above] {$t$};
\draw (3.1,0.5) node[above] {$x$};
\draw (0,-1) ellipse [x radius=12pt, y radius=40pt];
\draw (3.1,-1) ellipse [x radius=12pt, y radius=40pt];
\node [circle,draw,fill,scale=0.3] (A){};
\node [circle,draw,fill,scale=0.3] (B) [right=3cm of A] {};
\node [circle,draw,fill,scale=0.3] (C) [below=of A] {};
\node [circle,draw,fill,scale=0.3] (D) [right=3cm of C] {};
\node [circle,draw,fill,scale=0.3] (E) [below=of C] {};
\node [circle,draw,fill,scale=0.3] (F) [right=3cm of E] {};
\draw [thick, arrows={ - Stealth}]
(A) edge [bend left=45] node[above]{$x = x(t)$}(B)
(C) edge [bend left=45] (D)
(E) edge [bend left=45] (F);
\end{tikzpicture}
\caption{A cada valor de tempo $t$ temos um valor de posição $x$ associado. A função $x(t)$ é a operação que descreve a relação entre essas duas variáveis.}
\end{figure}

Dessa forma, podemos denotar o conjunto de instantes de tempo $t$ e o conjunto de posições $x$ correspondente como uma função:
\begin{equation}
    x: t \mapsto x(t).
\end{equation}

Sabemos que a posição de uma partícula pode variar conforme o tempo passa, e isso nos permitirá uma descrição mais completa do movimento ao definir a \emph{velocidade} e a \emph{aceleração} mais adiante. Essas duas também podem variar conforme o tempo passa, portanto conferiremos um caráter especial ao tempo na descrição do movimento. Com isso podemos elaborar gráficos que mostram, por exemplo, a variação temporal da posição (vide Figura~\ref{Fig:Graf_posicao_func_tempo}).
\begin{marginfigure}
\centering
\begin{tikzpicture}[>=Stealth, extended line/.style={shorten >=-#1,shorten <=-#1},
 extended line/.default=3mm]] % talvez fosse melhor amplicar com scale=1.5
    % Draw axes: acho que o |- é pra desenhar um "canto", um L
    \draw [<->,thick] (0,3) node (yaxis) [below left] {$x$}
        |- (4.3,0) node (xaxis) [below left] {$t$};
    % Desenhar função:
    \draw[smooth,name path=plota,samples=1000,domain=0:3.5]
    plot(\x,{2});
    
    \draw[smooth, densely dashed, name path=plotb,samples=1000,domain=0:3.5]
    plot(\x,{0.5*\x + 0.5});

    \draw[smooth, dash dot, name path=plotc,samples=1000,domain=0:3.5]
    plot(\x,{0.15*\x^2});
     
\end{tikzpicture}
\caption{Gráficos que exemplificam possíveis formas para os gráficos da função posição $x(t)$.\label{Fig:Graf_posicao_func_tempo}}
\end{marginfigure}

%%%%%%%%%%%%%%%%%%%%%%%%%%%%%%%%
\section{Velocidade}
%%%%%%%%%%%%%%%%%%%%%%%%%%%%%%%%

%%%%%%%%%%%%%%%%%%%%%%%%%%%%%
\subsection{Velocidade média}
%%%%%%%%%%%%%%%%%%%%%%%%%%%%%

Se considerarmos que um deslocamento sempre leva um tempo para ser efetuado, podemos calcular uma grandeza de grande interesse associada a ele: a \emph{velocidade}. Definimos a velocidade média como
\begin{equation}
  \mean{v} = \frac{\Delta x}{\Delta t}.
\end{equation}
%
Temos agora outra variável que descreve o movimento. Se conhecemos a velocidade média, podemos então descrever a distância percorrida em função do tempo como
\begin{equation}
  \Delta x = \mean{v} \Delta t,
\end{equation}
%
ou
\begin{equation}
    x_f = x_i + \mean{v} \Delta t.
\end{equation}
%
\begin{marginfigure}[-2cm]
\centering
\begin{tikzpicture}[>=Stealth, extended line/.style={shorten >=-#1,shorten <=-#1},
 extended line/.default=3mm]]
    % Draw axes: acho que o |- é pra desenhar um "canto", um L
    \draw [<->,thick] (0,3) node (yaxis) [below left] {$x$}
        |- (4.3,0) node (xaxis) [below left] {$t$};
    % Desenhar função:
    \draw[smooth,name path=plot,samples=1000,domain=0.5:3.5]
    plot(\x,{0.4*\x^2 + 1.1 - 0.9*\x});
    
    % linhas dos eixos à curva
    \node[below](a)at(1.2,0){$t_i$};
    \node[below](b)at(3,0){$t_f$};
    \path[name path=froma](a)--+(0,4);
    \path[name path=fromb](b)--+(0,4);
    \draw[dotted, thin,name intersections={of=froma and plot}](a)--(intersection-1) coordinate (plot-a-intersection)--(0,0|-intersection-1)node[left]{$x_1$};
    \draw[dotted, thin, name intersections={of=fromb and plot}](b)--(intersection-1) coordinate (plot-b-intersection)--(0,0|-intersection-1)node[left]{$x_2$};
    
    % linha entre as duas interseções (as coordenadas foram salvas acima)
    \draw[extended line, dashed] (plot-a-intersection) -- (plot-b-intersection);

    % pontos nas interseções
    \fill [opacity=1] (plot-a-intersection) circle (2pt);
    \fill [opacity=1] (plot-b-intersection) circle (2pt);
   
\end{tikzpicture}
\caption{Gráfico da posição em função do tempo. Podemos interpretar a velocidade média graficamente ao ligarmos os pontos da curva que representam os instantes/posições inicial e final.\label{Fig:Interp_graf_vel_med}}
\end{marginfigure}

\noindent{}Em especial, se a velocidade é constante, então $v = \mean{v}$, e obtemos
\begin{equation}
    x_f = x_i + v \Delta t.
\end{equation}
%
Como podemos zerar um cronômetro e iniciar a medida de tempo a partir do valor zero no início de um experimento, podemos escolher $t_i = 0$ e $t_f = t$, logo
\begin{equation}\label{Eq:VV0AT}
  x_f = x_i + vt.\mathnote{Evolução temporal da posição para velocidade constante.}
\end{equation}


Podemos determinar a dimensão da velocidade através de
\begin{align}
    [\mean{v}] &= \left[\frac{\Delta x}{\Delta t}\right] \\
    &= \frac{[\Delta x]}{[\Delta t]} \\
    &= \frac{\rm{L}}{\rm{T}}.
\end{align}
%
Consequentemente, no SI, a velocidade tem unidades de $\rm{m}/\rm{s}$.

Podemos conferir uma interpretação gráfica à velocidade média. Para isso, vamos tomar a Figura~\ref{Fig:Interp_graf_vel_med}, onde marcamos dois pontos que correspondem à posição $x_i$ no instante $t_i$ e à posição $x_f$ no instante $t_f$. Ligamos esses dois pontos por uma reta.


Traçando uma reta horizontal e uma vertical, podemos completar um triângulo retângulo (Figura~\ref{Fig:Interp_graf_vel_med_b}). Nesse triângulo, temos que o tamanho da lateral direita é igual a $x_f - x_i$, ou seja, corresponde a $\Delta x$. Já a parte inferior é igual a $t_f - t_i$, correspondendo a $\Delta t$.
\begin{marginfigure}[-4cm]
\centering
\begin{tikzpicture}[>=Stealth, extended line/.style={shorten >=-#1,shorten <=-#1},
 extended line/.default=3mm]]
    % Draw axes: acho que o |- é pra desenhar um "canto", um L
    \draw [<->,thick, gray] (0,3) node (yaxis) [below left] {$x$}
        |- (4.3,0) node (xaxis) [below left] {$t$};
    % Desenhar função:
    \draw[smooth,gray, name path=plot,samples=1000,domain=0.5:3.5]
    plot(\x,{0.4*\x^2 + 1.1 - 0.9*\x});
    
    % linhas dos eixos à curva
    \node[below, gray](a)at(1.2,0){$t_i$};
    \node[below, gray](b)at(3,0){$t_f$};
    \path[name path=froma](a)--+(0,4);
    \path[name path=fromb](b)--+(0,4);
    \draw[dotted, thin,name intersections={of=froma and plot}](a)--(intersection-1) coordinate (plot-a-intersection)--(0,0|-intersection-1)node[left]{$x_1$};
    \draw[dotted, thin, name intersections={of=fromb and plot}](b)--(intersection-1) coordinate (plot-b-intersection)--(0,0|-intersection-1)node[left]{$x_2$};
    
    % linha entre as duas interseções (as coordenadas foram salvas acima)
    \draw[extended line, dashed] (plot-a-intersection) -- (plot-b-intersection);

    % pontos nas interseções
    \fill [opacity=1, gray] (plot-a-intersection) circle (2pt);
    \fill [opacity=1, gray] (plot-b-intersection) circle (2pt);
    
    % linhas complementares
    \path[name path=horiz-a] (plot-a-intersection)--+(4,0);
    \path[name path=vert-b] (plot-b-intersection)--+(0,-2);
    \draw[thin, name intersections={of=horiz-a and vert-b}] (plot-b-intersection) -- (intersection-1) coordinate (ang-reto);
    
    \draw[thin, name intersections={of=fromb and horiz-a}] (intersection-1) -- (plot-a-intersection);
    
    % marcar os ângulos
    \tkzMarkRightAngle(plot-a-intersection,ang-reto,plot-b-intersection);
    \path pic[draw, angle radius=4mm, pic text=$\theta$,angle eccentricity=1.3] {angle = ang-reto--plot-a-intersection--plot-b-intersection};
    
    % medidas
    \draw[|-|] ($ (plot-b-intersection) + (0.2,0) $) -- node[right]{$\Delta x$} ($ (ang-reto) + (0.2,0) $);
    
    \draw[|-|] ($ (plot-a-intersection) + (0, -0.2) $) -- node[below]{$\Delta t$} ($ (ang-reto) + (0, -0.2) $);
   
\end{tikzpicture}
\caption{Triângulo formado pela reta que liga os pontos e as linhas horizontal e vertical.\label{Fig:Interp_graf_vel_med_b}}
\end{marginfigure}

\noindent{}Ao calcularmos a tangente do ângulo $\theta$, temos
\begin{align}
    \tan\theta &= \frac{\Delta x}{\Delta t} \\
    & = \mean{v},
\end{align}
%
isto é, \emph{a inclinação da reta que une os pontos correspondentes aos instantes/posições inicial e final está relacionada à velocidade média através de}:\footnote[][-2cm]{Um cálculo preciso necessitaria que levássemos em conta a escala do gráfico, porém estamos mais interessados na interpretação qualitativa.}
\begin{equation}
    \mean{v} = \tan\theta.
\end{equation}
%
Assim, se a inclinação entre um par de pontos é maior que entre outro par, temos que a velocidade média é maior no primeiro caso (Figura~\ref{Fig:Interp_graf_vel_med_diff}).

\begin{marginfigure}
\centering
\begin{tikzpicture}[>=Stealth, extended line/.style={shorten >=-#1,shorten <=-#1},
 extended line/.default=3mm]] % talvez fosse melhor amplicar com scale=1.5
    % Draw axes: acho que o |- é pra desenhar um "canto", um L
    \draw [<->,thick] (0,3) node (yaxis) [below left] {$x$}
        |- (4.3,0) node (xaxis) [below left] {$t$};
    % Desenhar função:
    \draw[smooth,name path=plot,samples=1000,domain=0.5:3.5]
    plot(\x,{0.4*\x^2 + 1.1 - 0.9*\x});
    
    % linhas dos eixos à curva
    \coordinate (a) at (1.2,0);
    \coordinate (b) at (3,0);
    \coordinate (c) at (0.7,0);
    \coordinate (d) at (2.3, 0);

    \path[name path=froma](a)--+(0,4);
    \path[name path=fromb](b)--+(0,4);
    \path[name path=fromc](c)--+(0,4);
    \path[name path=fromd](d)--+(0,4);


    \path[name intersections={of=froma and plot}](a)--(intersection-1) coordinate (plot-a-intersection)--(0,0|-intersection-1);
    \path[name intersections={of=fromb and plot}](b)--(intersection-1) coordinate (plot-b-intersection)--(0,0|-intersection-1);
    \path[name intersections={of=fromc and plot}](c)--(intersection-1) coordinate (plot-c-intersection)--(0,0|-intersection-1);
    \path[name intersections={of=fromd and plot}](d)--(intersection-1) coordinate (plot-d-intersection)--(0,0|-intersection-1);

    % linha entre as duas interseções (as coordenadas foram salvas acima)
    \draw[extended line, dashed] (plot-a-intersection) -- (plot-b-intersection);
    \draw[extended line, dashed] (plot-c-intersection) -- (plot-d-intersection);

    % pontos nas interseções
    \fill[opacity=1] (plot-a-intersection) circle (2pt) node[below right]{$A$};
    \fill[opacity=1] (plot-b-intersection) circle (2pt) node[below right]{$B$};
    \fill[opacity=1] (plot-c-intersection) circle (2pt) node[above right]{$C$};
    \fill[opacity=1] (plot-d-intersection) circle (2pt) node[below right]{$D$};
    
\end{tikzpicture}
\caption{As retas $\overline{AB}$ e $\overline{CD}$ representam valores de velocidade média diferentes, como pode ser visto devido às diferentes inclinações.\label{Fig:Interp_graf_vel_med_diff}}
\end{marginfigure}

%%%%%%%%%%%%%%%%%%%%%%%%%%%%%%%%%%%
\subsection{Velocidade instantânea}
%%%%%%%%%%%%%%%%%%%%%%%%%%%%%%%%%%%

Na Figura~\ref{Fig:Interp_graf_vel_med}, se tomarmos intervalos sucessivamente menores de tempo, podemos definir o que chamamos de \emph{velocidade instantânea}. Se estamos interessados em calcular a velocidade em um ponto $P$ (Figura~\ref{Fig:Interp_graf_vel_med_lim}), podemos tomar pares $(t_i, x_i)$, $(t_f, x_f)$ sucessivamente mais próximos até que a distância entre eles seja desprezível, ou seja, \emph{tenda a zero}. Nesse momento, a reta que liga os dois pontos passa a ser uma \emph{reta tangente à curva no ponto $P$}, isto é, uma reta que toca a curva $x(t)$ somente no ponto $P$.

Temos então que, graficamente, podemos interpretar a velocidade instantânea como a inclinação de uma reta tangente à curva $x(t)$ no ponto em que estamos interessados em calculá-la, isto é no ponto $(t,x)$.

Esse processo de aproximações sucessivas em que fazemos $\Delta t$ progressivamente menor é o que chamamos -- quando tomamos \emph{$\Delta t$ tendendo a zero} -- de \emph{limite}. Denotamos esse processo como
\begin{equation}
  v = \lim_{\Delta t \to 0} \frac{\Delta x}{\Delta t}.
\end{equation}
%
\begin{marginfigure}[-8cm]
\centering
\begin{tikzpicture}[>=Stealth, extended line/.style={shorten >=-#1,shorten <=-#1},
 extended line/.default=3mm]] % talvez fosse melhor amplicar com scale=1.5
    % Draw axes: acho que o |- é pra desenhar um "canto", um L
    \draw [<->,thick] (0,3) node (yaxis) [below left] {$x$}
        |- (4.3,0) node (xaxis) [below left] {$t$};
    % Desenhar função:
    \draw[smooth,name path=plot,samples=1000,domain=0.5:3.5]
    plot(\x,{0.4*\x^2 + 1.1 - 0.9*\x});
    
    % linhas dos eixos à curva
    \coordinate (a) at (1.2,0);
    \coordinate (b) at (3,0);
    \coordinate (c) at (0.7,0);
    \coordinate (d) at (3.3, 0);
    \coordinate (e) at (1.8,0);
    \coordinate (f) at (2.8,0);
    \coordinate (p) at (2.3,0);
    \path[name path=froma](a)--+(0,4);
    \path[name path=fromb](b)--+(0,4);
    \path[name path=fromc](c)--+(0,4);
    \path[name path=fromd](d)--+(0,4);
    \path[name path=frome](e)--+(0,4);
    \path[name path=fromf](f)--+(0,4);
    \path[name path=fromp](p)--+(0,4);
    \path[name intersections={of=froma and plot}](a)--(intersection-1) coordinate (plot-a-intersection)--(0,0|-intersection-1);
    \path[name intersections={of=fromb and plot}](b)--(intersection-1) coordinate (plot-b-intersection)--(0,0|-intersection-1);
    \path[name intersections={of=fromc and plot}](c)--(intersection-1) coordinate (plot-c-intersection)--(0,0|-intersection-1);
    \path[name intersections={of=fromd and plot}](d)--(intersection-1) coordinate (plot-d-intersection)--(0,0|-intersection-1);
    \path[name intersections={of=frome and plot}](e)--(intersection-1) coordinate (plot-e-intersection)--(0,0|-intersection-1);
    \path[name intersections={of=fromf and plot}](f)--(intersection-1) coordinate (plot-f-intersection)--(0,0|-intersection-1);
    \path[name intersections={of=fromp and plot}](p)--(intersection-1) coordinate (plot-p-intersection)--(0,0|-intersection-1);
    
    % linha entre as duas interseções (as coordenadas foram salvas acima)
    \draw[extended line, densely dotted] (plot-a-intersection) -- (plot-b-intersection);
    \draw[extended line, densely dotted] (plot-c-intersection) -- (plot-d-intersection);
    \draw[extended line, densely dotted] (plot-e-intersection) -- (plot-f-intersection);

    % pontos nas interseções
    \fill[opacity=1, gray] (plot-a-intersection) circle (2pt);
    \fill[opacity=1, gray] (plot-b-intersection) circle (2pt);
    \fill[opacity=1, gray] (plot-c-intersection) circle (2pt);
    \fill[opacity=1, gray] (plot-d-intersection) circle (2pt);
    \fill[opacity=1, gray] (plot-e-intersection) circle (2pt);
    \fill[opacity=1, gray] (plot-f-intersection) circle (2pt);
    
    \fill[opacity=1] (plot-p-intersection) circle (2pt);
   
    % tangente
    \draw[smooth, densely dashed, name path=deriv,samples=1000,domain=1.5:3.1]
    plot(\x,{0.94*\x - 1.016});
     
\end{tikzpicture}
\caption{Gráfico da posição em função do tempo onde mostramos o processo em que tomamos o limite $\Delta t \to 0$.\label{Fig:Interp_graf_vel_med_lim}}
\end{marginfigure}
A velocidade instantânea é o valor de velocidade no momento considerado. É o que é mostrado pelo velocímetro de um carro, por exemplo\footnote[][-15mm]{Isso é verdade somente para caros em que o sistema de medidas é analógico. Para medidas digitais, o valor é uma média de uma série de medidas tomadas em um curto intervalo de tempo}.

Além disso, se $\theta=0$, temos que a velocidade é nula em tal ponto, mesmo que momentaneamente. Em um gráfico que mostre a posição em função do tempo para um objeto lançado verticalmente para cima -- por exemplo --, esse ponto corresponderia à posição de máxima altura, onde o objeto pára momentaneamente.

\begin{marginfigure}[-1cm]
\centering
\begin{tikzpicture}[>=Stealth, extended line/.style={shorten >=-#1,shorten <=-#1},
 extended line/.default=3mm]] % talvez fosse melhor amplicar com scale=1.5
    % Draw axes: acho que o |- é pra desenhar um "canto", um L
    \draw [<->,thick] (0,3) node (yaxis) [below left] {$x$}
        |- (4.3,0) node (xaxis) [below left] {$t$};
    % Desenhar função:
    \draw[smooth,name path=plot,samples=1000,domain=0.2:3.5]
    plot(\x,{0.4*\x^2 + 1.1 - 0.9*\x});
    
    % linhas dos eixos à curva
    \coordinate (p) at (1.125,0);
    \path[name path=fromp](p)--+(0,4);
    \path[name intersections={of=fromp and plot}](p)--(intersection-1) coordinate (plot-p-intersection)--(0,0|-intersection-1);
    
    % pontos nas interseções   
    \fill[opacity=1] (plot-p-intersection) circle (2pt) node[above]{$P$};
   
    % tangente   
    \draw[smooth, densely dashed, name path=deriv,samples=1000,domain=0.6:1.8]
    plot(\x,{0.59375});
     
\end{tikzpicture}
\caption{No ponto $P$ temos que o ângulo de inclinação da reta tangente é nulo, portanto temos que momentaneamente a velocidade é nula.\label{Fig:Interp_graf_vel_med_zero}}
\end{marginfigure}

Finalmente, se o ângulo $\theta$ está abaixo da horizontal, verificamos que a velocidade é negativa (veja a reta tangente ao ponto $P_1$ na Figura~\ref{Fig:Interp_graf_vel_med_signs}, pois nos deslocamos no sentido negativo do eixo -- $\Delta x$ é negativo, portanto -- e, nesse caso, verificamos que a tangente de $\theta$ é negativa\footnote[][2cm]{Lembre-se do círculo trigonométrico: para um ângulo como o da reta tangente a $P_1$, a tangente está abaixo do eixo horizontal.}

\begin{marginfigure}
\centering
\begin{tikzpicture}[>=Stealth, extended line/.style={shorten >=-#1,shorten <=-#1},
 extended line/.default=3mm]] % talvez fosse melhor amplicar com scale=1.5
    % Draw axes: acho que o |- é pra desenhar um "canto", um L
    \draw [<->,thick] (0,3) node (yaxis) [below left] {$x$}
        |- (4.3,0) node (xaxis) [below left] {$t$};
    % Desenhar função:
    \draw[smooth,name path=plot,samples=1000,domain=0.2:3.5]
    plot(\x,{0.4*\x^2 + 1.1 - 0.9*\x});
    
    % linhas dos eixos à curva
    \coordinate (o) at (0.6,0);
    \coordinate (p) at (2.3,0);
    \path[name path=fromp](p)--+(0,4);
    \path[name path=fromo](o)--+(0,4);
    \path[name intersections={of=fromp and plot}](p)--(intersection-1) coordinate (plot-p-intersection)--(0,0|-intersection-1);
    \path[name intersections={of=fromo and plot}](o)--(intersection-1) coordinate (plot-o-intersection)--(0,0|-intersection-1);
    
    % pontos nas interseções   
    \fill[opacity=1] (plot-p-intersection) circle (2pt) node[above]{$P_2$};
    \fill[opacity=1] (plot-o-intersection) circle (2pt) node[above]{$P_1$};
   
    % tangente
    \draw[smooth, densely dashed, name path=deriv,samples=1000,domain=1.8:2.8]
    plot(\x,{0.94*\x - 1.016});
    
%    \draw[smooth, densely dashed, name path=deriv,samples=1000,domain=0.6:1.8]
%    plot(\x,{0.59375});
    
    \draw[smooth, densely dashed, name path=deriv,samples=1000,domain=0.05:1.2]
    plot(\x,{0.956 - 0.42*\x});
     
\end{tikzpicture}
\caption{As inclinações das retas tangentes indicam que em $P_1$ a velocidade é negativa, enquanto em $P_2$ temos uma velocidade positiva.\label{Fig:Interp_graf_vel_med_signs}}
\end{marginfigure}

%%%%%%%%%%%%%%%%%%%%%%%%%%%%%%%%%%%%%%%%%%%%%%%%%%%%%%
\subsection{Velocidades escalares média e instantânea}
%%%%%%%%%%%%%%%%%%%%%%%%%%%%%%%%%%%%%%%%%%%%%%%%%%%%%%

Se viajamos de uma cidade a outra e voltamos, temos um deslocamento nulo. Consequentemente, a velocidade média durante esse percurso será também nula. No entanto, podemos tomar o deslocamento escalar e dividi-lo pelo tempo transcorrido e definir uma \emph{velocidade escalar média}:
\begin{equation}
  \mean{v}_s = \frac{d_s}{\Delta t}.
\end{equation}
%
A velocidade escalar média é o que o computador de bordo de um carro verifica como velocidade média em um trajeto. Apesar de ela corresponder a nossa intuição de ``velocidade média'', ela não é uma grandeza vetorial -- como veremos adiante -- e não será de grande interesse para a descrição de fenômenos físicos.

Se tomarmos o limite com $\Delta t \to 0$, podemos dizer que o deslocamento nesse pequeno intervalo de tempo não sofre alteração de direção, portanto única diferença diferença possível entre a o deslocamento o deslocamento escalar é um sinal. Nesse caso, teremos que a velocidade escalar instantânea será igual ao módulo da velocidade instantânea:
\begin{equation}
  v_s = |v|.
\end{equation}

\pagebreak
%%%%%%%%%%%%%%%%%%%%%%%%%%%%%%%%%%%%%%%%%%%%
\subsection{Velocidade como função do tempo}
%%%%%%%%%%%%%%%%%%%%%%%%%%%%%%%%%%%%%%%%%%%%

De maneira análoga ao caso da evolução temporal da posição, podemos dizer que para cada instante de tempo $t_i$ temos uma velocidade $v_i$ associada. Assim, podemos denotar o conjunto de instantes de tempo $t$ e o conjunto de posições $x$ correspondente como uma função:
\begin{equation}
    x: t \mapsto x(t),
\end{equation}
%
o que corresponde a
\begin{figure}
\centering
\begin{tikzpicture}
\draw (0,0.5) node[above] {$t$};
\draw (3.1,0.5) node[above] {$v$};
\draw (0,-1) ellipse [x radius=12pt, y radius=40pt];
\draw (3.1,-1) ellipse [x radius=12pt, y radius=40pt];
\node [circle,draw,fill,scale=0.3] (A){};
\node [circle,draw,fill,scale=0.3] (B) [right=3cm of A] {};
\node [circle,draw,fill,scale=0.3] (C) [below=of A] {};
\node [circle,draw,fill,scale=0.3] (D) [right=3cm of C] {};
\node [circle,draw,fill,scale=0.3] (E) [below=of C] {};
\node [circle,draw,fill,scale=0.3] (F) [right=3cm of E] {};
\draw [thick, arrows={ - Stealth}]
(A) edge [bend left=45] node[above]{$v = v(t)$}(B)
(C) edge [bend left=45] (D)
(E) edge [bend left=45] (F);
\end{tikzpicture}
\caption{A cada valor de tempo $t$ temos um valor de velocidade $v$ associado. A função $v(t)$ descreve a relação entre essas duas variáveis.}
\end{figure}

\begin{marginfigure}[1cm]
\centering
\begin{tikzpicture}[>=Stealth, extended line/.style={shorten >=-#1,shorten <=-#1},
 extended line/.default=3mm]] % talvez fosse melhor amplicar com scale=1.5
    % Draw axes: acho que o |- é pra desenhar um "canto", um L
    \draw [<->,thick] (0,3) node (yaxis) [below left] {$v$}
        |- (4.3,0) node (xaxis) [below left] {$t$};
    % Desenhar função:
    \draw[smooth,name path=plota,samples=1000,domain=0:3.5]
    plot(\x,{2});
    
    \draw[smooth, densely dashed, name path=plotb,samples=1000,domain=0:3.5]
    plot(\x,{0.5*\x + 0.5});

    \draw[smooth, dash dot, name path=plotc,samples=1000,domain=0:3.5]
    plot(\x,{0.15*\x^2 + 0.1 + 0.1*\x*sin(10*\x r)});
     
\end{tikzpicture}
\caption{Gráficos que exemplificam possíveis formas para os gráficos da velocidade $v(t)$. Note que não precisamos nos restringir a formas funcionais simples.\label{Fig:Graf_posicao_func_tempo}}
\end{marginfigure}


%%%%%%%%%%%%%%%%%%%%%%%%%%%%%%%%
\section{Aceleração}
%%%%%%%%%%%%%%%%%%%%%%%%%%%%%%%%

%%%%%%%%%%%%%%%%%%%%%%%%%%%%%
\subsection{Aceleração média}
%%%%%%%%%%%%%%%%%%%%%%%%%%%%%

Da mesma forma que podemos ter variações de posição em dados intervalos de tempo, implicando na definição da velocidade, podemos ter variações da velocidade. Tais variações resultam na definição da aceleração. Portanto, se temos uma variação de velocidade em um intervalo de tempo, temos que a aceleração média será dada por
\begin{equation}
  \mean{a} = \frac{\Delta v}{\Delta t}.
\end{equation}
%
Fazendo a análise dimensional temos
\begin{align}
	[\mean{a}] &= \left[\frac{\Delta v}{\Delta t}\right] \\
		&= \frac{[\Delta v]}{[\Delta t]} \\
		&= \frac{\rm{L}}{\rm{T}^2}.
\end{align}
%
Logo, no Sistema Internacional, a aceleração é dada em $\rm{m}/\rm{s}^2$

Assim como pudemos dar uma interpretação gráfica para a velocidade média $\mean{v}$, em um gráfico $x \times t$, podemos fazer o mesmo para a aceleração média. Observando a Figura~\ref{Fig:Interp_graf_acel_med_b}, temos que
\begin{equation}
	\mean{a} = \tan \theta = \frac{\Delta v}{\Delta t},
\end{equation}
%
isto é, a aceleração média está relacionada à inclinação da reta que liga os pontos $(t_i, v_i)$ e $(t_f, v_f)$.

\begin{marginfigure}[-4cm]
\centering
\begin{tikzpicture}[>=Stealth, extended line/.style={shorten >=-#1,shorten <=-#1},
 extended line/.default=4mm]]
    % Draw axes: acho que o |- é pra desenhar um "canto", um L
    \draw [<->,thick, gray] (0,3) node (yaxis) [below left] {$v$}
        |- (4.3,0) node (xaxis) [below left] {$t$};
    % Desenhar função:
    \draw[smooth,gray, name path=plot,samples=1000,domain=0.5:3.5]
    plot(\x,{0.4*\x^2 + 1.1 - 0.9*\x + 0.4*sin((2*\x - 0.5) r)});
    
    % linhas dos eixos à curva
    \node[below, gray](a)at(1.2,0){$t_i$};
    \node[below, gray](b)at(3,0){$t_f$};
    \path[name path=froma](a)--+(0,4);
    \path[name path=fromb](b)--+(0,4);
    \draw[dotted, thin,name intersections={of=froma and plot}](a)--(intersection-1) coordinate (plot-a-intersection)--(0,0|-intersection-1)node[left]{$x_1$};
    \draw[dotted, thin, name intersections={of=fromb and plot}](b)--(intersection-1) coordinate (plot-b-intersection)--(0,0|-intersection-1)node[left]{$x_2$};
    
    % linha entre as duas interseções (as coordenadas foram salvas acima)
    \draw[extended line, dashed] (plot-a-intersection) -- (plot-b-intersection);

    % pontos nas interseções
    \fill [opacity=1, gray] (plot-a-intersection) circle (2pt);
    \fill [opacity=1, gray] (plot-b-intersection) circle (2pt);
    
    % linhas complementares
    \path[name path=horiz-a] (plot-a-intersection)--+(4,0);
    \path[name path=vert-b] (plot-b-intersection)--+(0,-2);
    \draw[thin, name intersections={of=horiz-a and vert-b}] (plot-b-intersection) -- (intersection-1) coordinate (ang-reto);
    
    \draw[thin, name intersections={of=fromb and horiz-a}] (intersection-1) -- (plot-a-intersection);
    
    % marcar os ângulos
    \tkzMarkRightAngle(plot-a-intersection,ang-reto,plot-b-intersection);
    \path pic[draw, angle radius=4mm, pic text=$\theta$,angle eccentricity=1.3] {angle = ang-reto--plot-a-intersection--plot-b-intersection};
    
    % medidas
    \draw[|-|] ($ (plot-b-intersection) + (0.2,0) $) -- node[right]{$\Delta v$} ($ (ang-reto) + (0.2,0) $);
    
    \draw[|-|] ($ (plot-a-intersection) + (0, -0.2) $) -- node[below]{$\Delta t$} ($ (ang-reto) + (0, -0.2) $);
   
\end{tikzpicture}
\caption{Triângulo formado pela reta que liga os pontos e as linhas horizontal e vertical.\label{Fig:Interp_graf_acel_med_b}}
\end{marginfigure}

%%%%%%%%%%%%%%%%%%%%%%%%%%%%%%%%%%%
\subsection{Aceleração instantânea}
%%%%%%%%%%%%%%%%%%%%%%%%%%%%%%%%%%%

Podemos definir a aceleração instantânea como
\begin{equation}
  a = \lim_{\Delta t \to 0} \frac{\Delta v}{\Delta t}.
\end{equation}
%
Novamente em analogia com a velocidade, tal limite graficamente pode ser interpretada como a inclinação da reta tangente à curva $v(t)$ no ponto $P = (t,v)$ em que estamos interessados em calcular a aceleração, Figura~\ref{Fig:Interp_graf_acel_med_lim}.

\begin{marginfigure}[-4cm]
\centering
\begin{tikzpicture}[>=Stealth, extended line/.style={shorten >=-#1,shorten <=-#1},
 extended line/.default=3mm]] % talvez fosse melhor amplicar com scale=1.5
    % Draw axes: acho que o |- é pra desenhar um "canto", um L
    \draw [<->,thick] (0,3) node (yaxis) [below left] {$v$}
        |- (4.3,0) node (xaxis) [below left] {$t$};
    % Desenhar função:
    \draw[smooth,name path=plot,samples=1000,domain=0.5:3.5]
    plot(\x,{0.4*\x^2 + 1.1 - 0.9*\x + 0.4*sin((2*\x - 0.5) r)});
    
    % linhas dos eixos à curva
    \coordinate (a) at (1.2,0);
    \coordinate (b) at (3,0);
    \coordinate (c) at (0.7,0);
    \coordinate (d) at (3.3, 0);
    \coordinate (e) at (1.8,0);
    \coordinate (f) at (2.8,0);
    \coordinate (p) at (2.3,0);
    \path[name path=froma](a)--+(0,4);
    \path[name path=fromb](b)--+(0,4);
    \path[name path=fromc](c)--+(0,4);
    \path[name path=fromd](d)--+(0,4);
    \path[name path=frome](e)--+(0,4);
    \path[name path=fromf](f)--+(0,4);
    \path[name path=fromp](p)--+(0,4);
    \path[name intersections={of=froma and plot}](a)--(intersection-1) coordinate (plot-a-intersection)--(0,0|-intersection-1);
    \path[name intersections={of=fromb and plot}](b)--(intersection-1) coordinate (plot-b-intersection)--(0,0|-intersection-1);
    \path[name intersections={of=fromc and plot}](c)--(intersection-1) coordinate (plot-c-intersection)--(0,0|-intersection-1);
    \path[name intersections={of=fromd and plot}](d)--(intersection-1) coordinate (plot-d-intersection)--(0,0|-intersection-1);
    \path[name intersections={of=frome and plot}](e)--(intersection-1) coordinate (plot-e-intersection)--(0,0|-intersection-1);
    \path[name intersections={of=fromf and plot}](f)--(intersection-1) coordinate (plot-f-intersection)--(0,0|-intersection-1);
    \path[name intersections={of=fromp and plot}](p)--(intersection-1) coordinate (plot-p-intersection)--(0,0|-intersection-1);
    
    % linha entre as duas interseções (as coordenadas foram salvas acima)
    \draw[extended line, densely dotted] (plot-a-intersection) -- (plot-b-intersection);
    \draw[extended line, densely dotted] (plot-c-intersection) -- (plot-d-intersection);
    \draw[extended line, densely dotted] (plot-e-intersection) -- (plot-f-intersection);

    % pontos nas interseções
    \fill[opacity=1, gray] (plot-a-intersection) circle (2pt);
    \fill[opacity=1, gray] (plot-b-intersection) circle (2pt);
    \fill[opacity=1, gray] (plot-c-intersection) circle (2pt);
    \fill[opacity=1, gray] (plot-d-intersection) circle (2pt);
    \fill[opacity=1, gray] (plot-e-intersection) circle (2pt);
    \fill[opacity=1, gray] (plot-f-intersection) circle (2pt);
    
    \fill[opacity=1] (plot-p-intersection) circle (2pt) node[below]{$P$};
   
    % tangente
    \draw[smooth, densely dashed, name path=deriv,samples=1000,domain=1.5:3.1]
    plot(\x,{0.480140842*\x - 0.285634781});
     
\end{tikzpicture}
\caption{Gráfico da velocidade em função do tempo onde mostramos o processo em que tomamos o limite $\Delta t \to 0$.\label{Fig:Interp_graf_acel_med_lim}}
\end{marginfigure}


%%%%%%%%%%%%%%%%%%%%%%%%%%%%%%%%%%%%%%%%%%%%
\subsection{Aceleração como função do tempo}
%%%%%%%%%%%%%%%%%%%%%%%%%%%%%%%%%%%%%%%%%%%%

Assim como podemos descrever a posição e a velocidade como funções do tempo, podemos fazer o mesmo para a aceleração:
\begin{figure}[h]
\centering
\begin{tikzpicture}
\draw (0,0.5) node[above] {$t$};
\draw (3.1,0.5) node[above] {$a$};
\draw (0,-1) ellipse [x radius=12pt, y radius=40pt];
\draw (3.1,-1) ellipse [x radius=12pt, y radius=40pt];
\node [circle,draw,fill,scale=0.3] (A){};
\node [circle,draw,fill,scale=0.3] (B) [right=3cm of A] {};
\node [circle,draw,fill,scale=0.3] (C) [below=of A] {};
\node [circle,draw,fill,scale=0.3] (D) [right=3cm of C] {};
\node [circle,draw,fill,scale=0.3] (E) [below=of C] {};
\node [circle,draw,fill,scale=0.3] (F) [right=3cm of E] {};
\draw [thick, arrows={ - Stealth}]
(A) edge [bend left=45] node[above]{$a = a(t)$}(B)
(C) edge [bend left=45] (D)
(E) edge [bend left=45] (F);
\end{tikzpicture}
\caption{A cada valor de tempo $t$ temos um valor de aceleração $a$ associado. A função $a(t)$ descreve a relação entre essas duas variáveis.}
\end{figure}

\noindent{}Podemos ter formas complicadas para a aceleração, porém, para que possamos trabalhar situações mais simples, nos limitaremos a movimentos com \emph{aceleração constante}. Faremos isso pois além de simplificarmos o tratamento, temos um caso importante de aceleração constante, a \emph{aceleração da gravidade} próximo à superfície da Terra.

\begin{marginfigure}[-4cm]
\centering
\begin{tikzpicture}[>=Stealth, extended line/.style={shorten >=-#1,shorten <=-#1},
 extended line/.default=3mm]] % talvez fosse melhor amplicar com scale=1.5
    % Draw axes: acho que o |- é pra desenhar um "canto", um L
    \draw[->] (0,-1.3) -- (0,1.5) node[below left] {$a$};
	\draw[->] (0,0) -- (4,0) node[below left] {$t$};

    % Desenhar função:
    \draw[smooth, dashdotted, name path=plot,samples=1000,domain=0:3.5]
    plot(\x,{sin((3 * \x) r)});

	\end{tikzpicture}
\caption{Em um sistema \emph{massa-mola}, um corpo oscila devido à força exercida pela mola e devido à sua própria inércia. Nesse sistema, a aceleração não é constante, variando de acordo com $a(t) = A\omega^2\sin(\omega t)$.\label{Fig:Exemplo_acel_complicada}}
\end{marginfigure}


%%%%%%%%%%%%%%%%%%%%%%%%%%%%%%%%%%%%%%%%%%%%%%%%%%%%%%%%%%%%%%%%%%
\section{Sentidos dos eixos de referência e sinais das variáveis cinemáticas}
%%%%%%%%%%%%%%%%%%%%%%%%%%%%%%%%%%%%%%%%%%%%%%%%%%%%%%%%%%%%%%%%%%

Ao adotarmos a reta real para descrever a posição, utilizamos o \emph{sinal} para denotar o sentido: posições à direita da origem são positivas, enquanto posições à esquerda são negativas. Ao calcularmos o deslocamento $\Delta x = x_f - x_i$, temos que se o deslocamento é no sentido positivo do eixo, ele será positivo; se for no sentido negativo do eixo, então o deslocamento é negativo.

Devido à própria definição da velocidade $\mean{v} = \Delta x / \Delta t$, sabendo que $\Delta t$ é sempre positivo, verificamos que se a velocidade é no sentido positivo do eixo, então ela tem valores positivos. Caso a velocidade seja no sentido negativo do eixo, então seu valor é negativo.

Para a aceleração, no entanto, é mais complicado definirmos o sinal apropriado. Se temos um deslocamento para a direita, por exemplo, temos uma velocidade positiva enquanto ele ocorre. A aceleração, porém, pode ser positiva, negativa, ou nula, sem que haja mudança no sinal do deslocamento, ou da velocidade. Isso ocorre pois a aceleração descreve \emph{alterações na velocidade}. Um movimento no sentido positivo do eixo pode ocorrer de forma que a velocidade aumente, diminua, ou permaneça constante. As acelerações, nesses três casos seriam maior, menor, e igual a zero, respectivamente. Para percebermos o porque, basta verificarmos o sinal da variação da velocidade: no primeiro caso, $\Delta v > 0$, no segundo $\Delta v < 0$, no terceiro $\Delta v = 0$.

No caso de termos uma velocidade no sentido negativo, temos uma situação análoga. No entanto, quando verificamos um aumento do valor numérico da velocidade, mantendo o sentido negativo, temos que $\Delta v < 0$. Isso representa o oposto da situação em que o deslocamento é no sentido positivo, onde tinhamos um valor positivo quando o valor da velocidade aumentava. Além disso, no deslocamento no sentido negativo, se o valor da velocidade diminui, então $\Delta v > 0$. Novamente, isso é o oposto do que acontece em um deslocamento no sentido positivo.

Podemos agrupar essas observações acerca da aceleração nos seguintes casos:
\begin{itemize}
    \item Se não há variação da velocidade, então a aceleração é nula.
    \item Se a velocidade aumenta em valor, então a aceleração tem o mesmo sinal que a velocidade.
    \item Se a velocidade diminui em valor, então a aceleração tem o sinal oposto ao da velocidade.
\end{itemize}

Finalmente, devemos nos lembrar de que a escolha do sentido positivo do eixo é arbitrária. Podemos escolher de maneira que seja mais conveniente, o que em geral significa minimizar o número de grandezas com sinal negativo. Uma vez escolhido um sentido positivo, devemos nos ater a tal escolha, de maneira a garantir que a descrição do movimento seja consistente.

%%%%%%%%%%%%%%%%%%%%%%%%%%%%%%%%%%%%%%%%%%%%%%%%%%%%%%%%%%%%%%%%%%%%%%%%%
\section{Interpretação da área de um gráfico $v \times t$ e $a \times t$}
%%%%%%%%%%%%%%%%%%%%%%%%%%%%%%%%%%%%%%%%%%%%%%%%%%%%%%%%%%%%%%%%%%%%%%%%%

\begin{marginfigure}
\centering
\begin{tikzpicture}[>=Stealth, extended line/.style={shorten >=-#1,shorten <=-#1},
 extended line/.default=3mm]] % talvez fosse melhor amplicar com scale=1.5
    % Draw axes: acho que o |- é pra desenhar um "canto", um L
    \draw [<->,thick,gray] (0,3) node (yaxis) [below left] {$v$}
        |- (4.3,0) node (xaxis) [below left] {$t$};
    % Desenhar função:
    \draw[smooth,name path=plota,samples=1000,domain=0:3]
    plot(\x,{2});
    
     \fill [pattern=north west lines, pattern color=gray, domain=0.5:2.5, variable=\x]
      (0.5, 0) node[below]{$t_i$}
      -- plot ({\x}, {2})
      -- (2.5, 0) node[below]{$t_f$}
      -- cycle;
      
      \draw[dashed] (0.5, 0) -- (0.5, 2);
      \draw[dashed] (2.5, 0) -- (2.5, 2);
      \path (0, 2) node[left]{$v_0$};
      
      \draw[|-|] (3.2, 0) -- node[right]{$v_0$} (3.2, 2);
      \draw[|-|] (0.5, -0.6) -- node[below]{$\Delta t$} (2.5, -0.6);
     
\end{tikzpicture}
\caption{A área hachurada está relacionada ao deslocamento em um movimento com velocidade $v_0$ no intervalo de tempo destacado.\label{Fig:Graf_area_graf_v}}
\end{marginfigure}

Se temos que um objeto se move com velocidade constante, a distância percorrida por ele será
\begin{equation}
  \Delta x = v \Delta t.
\end{equation}
%
Ao fazer um gráfico de $v\times t$, percebemos que a equação acima determina a \emph{área} delimitada pela curva $v(t)$, o eixo $t$ e os eixos verticais que passam por $t_1$ e $t_2$. Se tivéssemos uma situação mais complicada, com uma velocidade $v(t)$ que variasse de uma maneira mais complexa, poderíamos determinar a distância percorrida entre dois instantes $t_1$ e $t_2$ simplesmente calculando a área entre a curva, o eixo $x$ e os eixos verticais passando por $t_1$ e $t_2$. Se a curva $v(t)$ está abaixo do eixo $t$, isso significa que a velocidade é negativa, ou seja, nesta região o objeto estará ``voltando'' e o deslocamento será, consequentemente, negativo.

\begin{marginfigure}
\centering
\begin{tikzpicture}[>=Stealth, extended line/.style={shorten >=-#1,shorten <=-#1},
 extended line/.default=3mm]] % talvez fosse melhor amplicar com scale=1.5
    % Draw axes: acho que o |- é pra desenhar um "canto", um L
    \draw [<->,thick,gray] (0,3) node (yaxis) [below left] {$v$}
        |- (4.3,0) node (xaxis) [below left] {$t$};
    % Desenhar função:
    \draw[smooth,name path=plota,samples=1000,domain=0:2.8]
    plot(\x,{1.5 - 3.4*\x + 8.3 * \x*\x - 5.9 * \x*\x*\x + 1.2 * \x*\x*\x*\x});

    \coordinate (a) at (0.25,0);
    \coordinate (b) at (2.75,0);
    \path[name path=froma](a)--+(0,3);
    \path[name path=fromb](b)--+(0,3);
    \draw[dashed, name intersections={of=froma and plota}](a) node[below]{$t_i$} -- (intersection-1);
	\draw[dashed, name intersections={of=fromb and plota}](b) node[below]{$t_f$} -- (intersection-1);

    \fill [pattern=north west lines, pattern color=gray, domain=0.25:2.75, variable=\x]
     	  (0.25, 0)
    	  -- plot ({\x}, {1.5 - 3.4*\x + 8.3 * \x*\x - 5.9 * \x*\x*\x + 1.2 * \x*\x*\x*\x})
          -- (2.75, 0)
          -- cycle;
\end{tikzpicture}
\caption{Podemos utilizar a área para determinar o deslocamento em um caso mais complexo, onde a velocidade varia arbitrariamente.\label{Fig:Graf_area_graf_v_complicado}}
\end{marginfigure}

Para determinar o valor numérico do deslocamento através da área, podemos dividir a região hachurada em várias barras de uma largura arbitrária $\Delta t$ e altura dada pela própria curva $v(t)$. Somando os valores obtidos para cada uma das barras, podemos determinar -- pelo menos de forma aproximada -- o deslocamento total. Se tomarmos intervalos $\Delta t$ sucessivamente menores, eventualmente conseguiremos calcular a área com grande precisão e verificaremos que nesse caso temos exatamente a área ``abaixo'' da curva.

\begin{marginfigure}
\centering
\begin{tikzpicture}[>=Stealth, extended line/.style={shorten >=-#1,shorten <=-#1},
 extended line/.default=3mm]] % talvez fosse melhor amplicar com scale=1.5
    % Draw axes: acho que o |- é pra desenhar um "canto", um L
    \draw [<->,thick,gray] (0,3) node (yaxis) [below left] {$v$}
        |- (4.3,0) node (xaxis) [below left] {$t$};
    % Desenhar função:
    \draw[smooth,name path=plota,samples=1000,domain=0:2.8]
    plot(\x,{1.5 - 3.4*\x + 8.3 * \x*\x - 5.9 * \x*\x*\x + 1.2 * \x*\x*\x*\x});

	\coordinate (a) at (0.25,0);
    \coordinate (b) at (0.5,0);
    \coordinate (c) at (0.75,0);
    \coordinate (d) at (1,0);
    \coordinate (e) at (1.25,0);
    \coordinate (f) at (1.5,0);
    \coordinate (g) at (1.75,0);
    \coordinate (h) at (2,0);
    \coordinate (i) at (2.25,0);
    \coordinate (j) at (2.5,0);
    \coordinate (k) at (2.75,0);
    \path[name path=froma](a)--+(0,3);
    \path[name path=fromb](b)--+(0,3);
    \path[name path=fromc](c)--+(0,3);
    \path[name path=fromd](d)--+(0,3);
    \path[name path=frome](e)--+(0,3);
    \path[name path=fromf](f)--+(0,3);
    \path[name path=fromg](g)--+(0,3);
    \path[name path=fromh](h)--+(0,-0.6);
    \path[name path=fromi](i)--+(0,-0.6);
    \path[name path=fromj](j)--+(0,-0.6);
    \path[name path=fromk](k)--+(0,3);
    \path[dashed, name intersections={of=froma and plota}](a) -- (intersection-1) coordinate (int-a);
	\path[dashed, name intersections={of=fromb and plota}](b) -- (intersection-1) coordinate (int-b);
	\path[dashed, name intersections={of=fromc and plota}](c) -- (intersection-1) coordinate (int-c);
	\path[dashed, name intersections={of=fromd and plota}](d) -- (intersection-1) coordinate (int-d);
	\path[dashed, name intersections={of=frome and plota}](e) -- (intersection-1) coordinate (int-e);
	\path[dashed, name intersections={of=fromf and plota}](f) -- (intersection-1) coordinate (int-f);
	\path[dashed, name intersections={of=fromg and plota}](g) -- (intersection-1) coordinate (int-g);
	\path[dashed, name intersections={of=fromh and plota}](h) -- (intersection-1) coordinate (int-h);
	\path[dashed, name intersections={of=fromi and plota}](i) -- (intersection-1) coordinate (int-i);
	\path[dashed, name intersections={of=fromj and plota}](j) -- (intersection-1) coordinate (int-j);
	\path[dashed, name intersections={of=fromk and plota}](k) -- (intersection-1) coordinate (int-k);

	\draw[very thin, pattern=north west lines, pattern color=gray] (b) rectangle (int-a);
	\draw[very thin, pattern=north west lines, pattern color=gray] (c) rectangle (int-b);
	\draw[very thin, pattern=north west lines, pattern color=gray] (d) rectangle (int-c);
	\draw[very thin, pattern=north west lines, pattern color=gray] (e) rectangle (int-d);
	\draw[very thin, pattern=north west lines, pattern color=gray] (f) rectangle (int-e);
	\draw[very thin, pattern=north west lines, pattern color=gray] (g) rectangle (int-f);
	\draw[very thin, pattern=north west lines, pattern color=gray] (h) rectangle (int-g);
	\draw[very thin, pattern=north west lines, pattern color=gray] (i) rectangle (int-h);
	\draw[very thin, pattern=north west lines, pattern color=gray] (j) rectangle (int-i);
	\draw[very thin, pattern=north west lines, pattern color=gray] (k) rectangle (int-j);

	\draw[dashed] (a) node[below]{$t_i$} -- (int-a);
	\draw[dashed] (k) node[below right]{$t_f$} -- (int-k);
\end{tikzpicture}
\caption{Para determinar o valor da área, basta dividirmos a região em barras com uma largura $\Delta t$ arbitrária e uma altura $v(t)$.\label{Fig:Graf_area_graf_v_complicado_barras}}
\end{marginfigure}

\begin{marginfigure}
\centering
\begin{tikzpicture}[>=Stealth, extended line/.style={shorten >=-#1,shorten <=-#1},
 extended line/.default=3mm]] % talvez fosse melhor amplicar com scale=1.5
    % Draw axes: acho que o |- é pra desenhar um "canto", um L
    \draw [<->,thick,gray] (0,3) node (yaxis) [below left] {$v$}
        |- (4.3,0) node (xaxis) [below left] {$t$};
    % Desenhar função:
    \draw[smooth,name path=plota,samples=1000,domain=0:2.8]
    plot(\x,{1.5 - 3.4*\x + 8.3 * \x*\x - 5.9 * \x*\x*\x + 1.2 * \x*\x*\x*\x});

	\coordinate (a) at (0.25,0);
    \coordinate (b) at (0.35,0);
    \coordinate (c) at (0.45,0);
    \coordinate (d) at (0.55,0);
    \coordinate (e) at (0.65,0);
    \coordinate (f) at (0.75,0);
    \coordinate (g) at (0.85,0);
    \coordinate (h) at (0.95,0);
    \coordinate (i) at (1.05,0);
    \coordinate (j) at (1.15,0);
    \coordinate (k) at (1.25,0);
    \coordinate (l) at (1.35,0);
    \coordinate (m) at (1.45,0);
    \coordinate (n) at (1.55,0);
    \coordinate (o) at (1.65,0);
    \coordinate (p) at (1.75,0);
    \coordinate (q) at (1.85,0);
    \coordinate (r) at (1.95,0);
    \coordinate (s) at (2.05,0);
    \coordinate (t) at (2.15,0);
    \coordinate (u) at (2.25,0);
    \coordinate (v) at (2.35,0);
    \coordinate (x) at (2.45,0);
    \coordinate (w) at (2.55,0);
    \coordinate (y) at (2.65,0);
    \coordinate (z) at (2.75,0);
    \path[name path=froma](a)--+(0,3);
    \path[name path=fromb](b)--+(0,3);
    \path[name path=fromc](c)--+(0,3);
    \path[name path=fromd](d)--+(0,3);
    \path[name path=frome](e)--+(0,3);
    \path[name path=fromf](f)--+(0,3);
    \path[name path=fromg](g)--+(0,3);
    \path[name path=fromh](h)--+(0,3);
    \path[name path=fromi](i)--+(0,3);
    \path[name path=fromj](j)--+(0,3);
    \path[name path=fromk](k)--+(0,3);
    \path[name path=froml](l)--+(0,3);
    \path[name path=fromm](m)--+(0,3);
    \path[name path=fromn](n)--+(0,3);
    \path[name path=fromo](o)--+(0,3);
    \path[name path=fromp](p)--+(0,3);
    \path[name path=fromq](q)--+(0,3);
    \path[name path=fromr](r)--+(0,3);
    \path[name path=froms](s)--+(0,-0.6);
    \path[name path=fromt](t)--+(0,-0.6);
    \path[name path=fromu](u)--+(0,-0.6);
    \path[name path=fromv](v)--+(0,-0.7);
    \path[name path=fromx](x)--+(0,-0.6);
    \path[name path=fromw](w)--+(0,-0.6);
    \path[name path=fromy](y)--+(0,3);
    \path[name path=fromz](z)--+(0,3);
    \path[dashed, name intersections={of=froma and plota}](a) -- (intersection-1) coordinate (int-a);
	\path[dashed, name intersections={of=fromb and plota}](b) -- (intersection-1) coordinate (int-b);
	\path[dashed, name intersections={of=fromc and plota}](c) -- (intersection-1) coordinate (int-c);
	\path[dashed, name intersections={of=fromd and plota}](d) -- (intersection-1) coordinate (int-d);
	\path[dashed, name intersections={of=frome and plota}](e) -- (intersection-1) coordinate (int-e);
	\path[dashed, name intersections={of=fromf and plota}](f) -- (intersection-1) coordinate (int-f);
	\path[dashed, name intersections={of=fromg and plota}](g) -- (intersection-1) coordinate (int-g);
	\path[dashed, name intersections={of=fromh and plota}](h) -- (intersection-1) coordinate (int-h);
	\path[dashed, name intersections={of=fromi and plota}](i) -- (intersection-1) coordinate (int-i);
	\path[dashed, name intersections={of=fromj and plota}](j) -- (intersection-1) coordinate (int-j);
	\path[dashed, name intersections={of=fromk and plota}](k) -- (intersection-1) coordinate (int-k);
	\path[dashed, name intersections={of=froml and plota}](l) -- (intersection-1) coordinate (int-l);
	\path[dashed, name intersections={of=fromm and plota}](m) -- (intersection-1) coordinate (int-m);
	\path[dashed, name intersections={of=fromn and plota}](n) -- (intersection-1) coordinate (int-n);
	\path[dashed, name intersections={of=fromo and plota}](o) -- (intersection-1) coordinate (int-o);
	\path[dashed, name intersections={of=fromp and plota}](p) -- (intersection-1) coordinate (int-p);
	\path[dashed, name intersections={of=fromq and plota}](q) -- (intersection-1) coordinate (int-q);
	\path[dashed, name intersections={of=fromr and plota}](r) -- (intersection-1) coordinate (int-r);
	\path[dashed, name intersections={of=froms and plota}](s) -- (intersection-1) coordinate (int-s);
	\path[dashed, name intersections={of=fromt and plota}](t) -- (intersection-1) coordinate (int-t);
	\path[dashed, name intersections={of=fromu and plota}](u) -- (intersection-1) coordinate (int-u);
	\path[dashed, name intersections={of=fromv and plota}](v) -- (intersection-1) coordinate (int-v);
	\path[dashed, name intersections={of=fromx and plota}](x) -- (intersection-1) coordinate (int-x);
	\path[dashed, name intersections={of=fromw and plota}](w) -- (intersection-1) coordinate (int-w);
	\path[dashed, name intersections={of=fromy and plota}](y) -- (intersection-1) coordinate (int-y);
	\path[dashed, name intersections={of=fromz and plota}](z) -- (intersection-1) coordinate (int-z);

	\draw[very thin, pattern=north west lines, pattern color=gray] (b) rectangle (int-a);
	\draw[very thin, pattern=north west lines, pattern color=gray] (c) rectangle (int-b);
	\draw[very thin, pattern=north west lines, pattern color=gray] (d) rectangle (int-c);
	\draw[very thin, pattern=north west lines, pattern color=gray] (e) rectangle (int-d);
	\draw[very thin, pattern=north west lines, pattern color=gray] (f) rectangle (int-e);
	\draw[very thin, pattern=north west lines, pattern color=gray] (g) rectangle (int-f);
	\draw[very thin, pattern=north west lines, pattern color=gray] (h) rectangle (int-g);
	\draw[very thin, pattern=north west lines, pattern color=gray] (i) rectangle (int-h);
	\draw[very thin, pattern=north west lines, pattern color=gray] (j) rectangle (int-i);
	\draw[very thin, pattern=north west lines, pattern color=gray] (k) rectangle (int-j);
	\draw[very thin, pattern=north west lines, pattern color=gray] (l) rectangle (int-k);
	\draw[very thin, pattern=north west lines, pattern color=gray] (m) rectangle (int-l);
	\draw[very thin, pattern=north west lines, pattern color=gray] (n) rectangle (int-m);
	\draw[very thin, pattern=north west lines, pattern color=gray] (o) rectangle (int-n);
	\draw[very thin, pattern=north west lines, pattern color=gray] (p) rectangle (int-o);
	\draw[very thin, pattern=north west lines, pattern color=gray] (q) rectangle (int-p);
	\draw[very thin, pattern=north west lines, pattern color=gray] (r) rectangle (int-q);
	\draw[very thin, pattern=north west lines, pattern color=gray] (s) rectangle (int-r);
	\draw[very thin, pattern=north west lines, pattern color=gray] (t) rectangle (int-s);
	\draw[very thin, pattern=north west lines, pattern color=gray] (u) rectangle (int-t);
	\draw[very thin, pattern=north west lines, pattern color=gray] (v) rectangle (int-u);
	\draw[very thin, pattern=north west lines, pattern color=gray] (x) rectangle (int-v);
	\draw[very thin, pattern=north west lines, pattern color=gray] (w) rectangle (int-x);
	\draw[very thin, pattern=north west lines, pattern color=gray] (y) rectangle (int-w);
	\draw[very thin, pattern=north west lines, pattern color=gray] (z) rectangle (int-y);

	\draw[dashed] (a) node[below]{$t_i$} -- (int-a);
	\draw[dashed] (z) node[below right]{$t_f$} -- (int-z);
\end{tikzpicture}
\caption{Podemos melhorar a aproximação diminuindo a largura das barras, obtendo um erro tão pequeno quanto necessário.\label{Fig:Graf_area_graf_v_complicado_barras_estreito}}
\end{marginfigure}

Para o caso de um gráfico de $a \times t$, temos uma situação análoga: se a aceleração for constante, a área entre a curva, o eixo $x$, e os eixos verticais passando por $t_1$ e $t_2$ será igual à variação da velocidade $\Delta v = a \Delta t$. Desenvolvendo um raciocínio análogo ao caso anterior para o cálculo da área entre a curva $a(t)$ e o eixo $x$, concluímos que a variação da velocidade para casos em que a aceleração não é constante pode ser calculada através da área ``abaixo'' da curva. Devemos, novamente, subtrair a área das regiões abaixo do eixo $t$.

%%%%%%%%%%%%%%%%%%%%%%%%%%%%%%%%%%%%%%%%%%%%%%%%%%%%%%%%%%%%%%%%%%%%%%%%%%
\section{Equações cinemáticas para movimentos com aceleração constante}
%%%%%%%%%%%%%%%%%%%%%%%%%%%%%%%%%%%%%%%%%%%%%%%%%%%%%%%%%%%%%%%%%%%%%%%%%%
Apesar de ser perfeitamente aceitável tratar uma situação em que aceleração varia, isso não é uma tarefa muito fácil. Por isso, vamos tratar com mais detalhes o caso da aceleração constante. Um exemplo de movimento com aceleração constante é o caso de movimentos submetidos à aceleração da gravidade, que veremos neste capítulo para movimentos exclusivamente verticais, mas que serão vistos em duas dimensões no Capítulo~\ref{Chap:MovimentoBidimensional}. Antes vamos deduzir as fórmulas para aceleração constante


%%%%%%%%%%%%%%%%%%%%%%%
\subsection{Velocidade}
%%%%%%%%%%%%%%%%%%%%%%%

Se a aceleração é constante, temos que $\mean{a} = a$ e, portanto,
\begin{equation}
  a = \frac{\Delta v}{\Delta t}.
\end{equation}
%
Podemos escrever então
\begin{equation}
  a (t_f - t_i) = (v_f - v_i).
\end{equation}
%
É muito comum, em equações de cinemática, utilizar $t_i = 0$ e $t_f = t$, o que corresponde a iniciar a cronometragem do tempo no início do evento físico que se está estudando. Dessa forma, podemos escrever
\begin{equation}\label{Eq:VV0AT}
  v_f = v_i + at.\mathnote{Evolução temporal da velocidade para aceleração constante.}
\end{equation}

%%%%%%%%%%%%%%%%%%%%%%%%%%%
\subsection{Posição}
%%%%%%%%%%%%%%%%%%%%%%%%%%%

Podemos calcular uma expressão para a evolução temporal da posição se considerarmos a Figura~\ref{Fig:Graf_area_acel_const}. Se a aceleração é constante, vimos que a velocidade deve ser descrita por uma reta em um gráfico $v\times t$. Sabemos ainda que o deslocamento é dado pela área abaixo da curva $v(t)$. Logo, temos que
\begin{align}
	\Delta x &= A \\
			 &= A_1 + A_2.
\end{align}
%
A área $A_1$ é dada por
\begin{equation}
	A_1 = v_i \Delta t,
\end{equation}
%
enquanto $A_2$ é dada por
\begin{equation}
	A_2 = \frac{(v_f - v_i)\Delta t}{2}.
\end{equation}

\begin{marginfigure}
\centering
\begin{tikzpicture}[>=Stealth, extended line/.style={shorten >=-#1,shorten <=-#1},
 extended line/.default=3mm]] % talvez fosse melhor amplicar com scale=1.5
    % Draw axes: acho que o |- é pra desenhar um "canto", um L
    \draw [<->,thick,gray] (0,3) node (yaxis) [below left] {$v$}
        |- (4.3,0) node (xaxis) [below left] {$t$};
    % Desenhar função:
    \draw[smooth,name path=plot,samples=1000,domain=0:2.8]
    plot(\x,{0.7 + 0.5 * \x});

    \coordinate (a) at (0.5,0);
    \coordinate (b) at (2.5,0);
    \path[name path=froma](a)--+(0,3);
    \path[name path=fromb](b)--+(0,3);

 	\draw[dashed, name intersections={of=froma and plot}](a) node[below]{$t_i$} --(intersection-1) coordinate (plot-a-intersection)--(0,0|-intersection-1) node[left]{$v_i$};
 	\draw[dashed, name intersections={of=fromb and plot}](b) node[below]{$t_f$} --(intersection-1) coordinate (plot-b-intersection)--(0,0|-intersection-1) node[left]{$v_f$};

    \fill [pattern=north west lines, pattern color=gray, domain=0.5:2.5, variable=\x]
     	  (0.5, 0)
    	  -- plot ({\x}, {0.7 + 0.5 * \x})
          -- (2.5, 0)
          -- cycle;

	\draw[dashed](plot-a-intersection) -- +(2,0);

	\node[circle, fill=white, scale=0.7] (anode) at (1.5,0.5) {$A_1$};
	\node[circle, fill=white, scale=0.7] (anode) at (2,1.3) {$A_2$};

\end{tikzpicture}
\caption{Para o caso de aceleração constante, podemos calcular a área a dividindo em um retângulo e um triângulo.\label{Fig:Graf_area_acel_const}}
\end{marginfigure}

\noindent{}Logo,
\begin{equation}
  \Delta x = v_i\Delta t + \frac{(v_f - v_i)\Delta t}{2}.
\end{equation}
%
Utilizando a equação $v_f = v_i + at$, e fazendo ainda $t_i = 0$ e $t_f = t$, temos
\begin{equation}
  \Delta x = v_i t + \frac{(v_i + at - v_i) t}{2}
\end{equation}
%
e, finalmente,
\begin{equation}\label{Eq:XX0V0TAT22}
  x_f = x_i + v_i t +\frac{at^2}{2}.\mathnote{Evolução temporal da posição para aceleração constante (1\textordfeminine~Equação).}
\end{equation}

Caso não haja informação sobre a velocidade inicial, a equação acima pode ser reescrita com o auxílio da $v_f = v_i + at$:
\begin{align}
  x_f &= x_i + (v_f - at) t + \frac{at^2}{2} \\
  &= x_i + v_f t + \frac{at^2 - 2at^2}{2},
\end{align}
%
resultando em
\begin{equation}
  x_f = x_i + v_f t - \frac{at^2}{2}.\mathnote{Evolução temporal da posição para aceleração constante (2\textordfeminine~Equação).}
\end{equation}

% Isso vai lá pra definição de velocidade.
%Também podemos escrever
%\begin{equation}
%\mean{v} = \frac{x_f - x_i}{t_f - t_i},
%\end{equation}
%
%ou, com $t_i = 0$ e $t_f = t$,
%\begin{equation}
%  x_f = x_i + \mean{v}t.
%\end{equation}
%
Novamente considerando que para o caso especial de uma aceleração constante, temos que a velocidade é uma reta, podemos escrever a velocidade média como
\begin{align}
  \mean{v} &= \frac{\Delta x}{\Delta t} \\
  &= \frac{v_i \Delta t + [(v_f - v_i)/2] \Delta t}{\Delta t} \\
  &= \frac{v_i + v_f}{2},
\end{align}
%
onde calculamos $\Delta x$ através da área de um triângulo e de um quadrado.
Temos então
\begin{equation}
  x_f = x_i + \frac{v_i + v_f}{2} t.\mathnote{Evolução temporal da posição para aceleração constante (3\textordfeminine~Equação).}
\end{equation}

%%%%%%%%%%%%%%%%%%%%%%%%%%%%%%%%%%
\subsection{Equação de Torricelli}
%%%%%%%%%%%%%%%%%%%%%%%%%%%%%%%%%%

A partir da Equação~\ref{Eq:VV0AT}, podemos isolar o tempo e obter
\begin{equation}
  t = \frac{v_f-v_i}{a}.
\end{equation}
%
Substituindo esta expressão na Equação~\ref{Eq:XX0V0TAT22}, obtemos
\begin{align}
  x_f - x_i &= v_i \left(\frac{v_f-v_i}{a}\right) + \frac{1}{2} a \left(\frac{v_f-v_i}{a}\right)^2 \\
  &= \frac{v_f v_i - v_i^2}{a} + \frac{v_f^2 + v_i^2-2v_fv_i}{2a}.
\end{align}
%
multiplicando os dois lados da equação por $2a$, temos
\begin{equation}
  2a\Delta x = 2v_i v_f - 2v_i^2 + v_f^2 +v_i^2 - 2v_f v_i.
\end{equation}
%
Eliminando o primeiro e o quarto termos à direita e somando os restantes, obtemos
\begin{equation}
  v_f^2 = v_i^2 + 2 a \Delta x. \mathnote{Equação de Torricelli.}
\end{equation}

%%%%%%%%%%%%%%%%%%%%%%%%%%%%%%%%%%%%%%%%%%%%%%%
\subsection{Variáveis ausentes em cada equação}
%%%%%%%%%%%%%%%%%%%%%%%%%%%%%%%%%%%%%%%%%%%%%%%

As cinco equações obtidas para a cinemática com aceleração constante envolvem as variáveis $x_i$, $x_f$, $v_i$, $v_f$, $a$ e $t$. Porém cada uma das equações deixa algum desses parâmetros de fora. Isso pode ser usado para a solução de problemas quando tal informação não é conhecida. A Tabela~\ref{Tab:EqsCinematicasVarAusentes} apresenta as equações e destaca a variável ausente em cada uma delas.
\begin{table}[!h]
\centering
\begin{tabular}{cc}
\toprule
Equação & Variável ausente\\
\midrule
$v_f = v_i + at$ & $\Delta x$ \\
$x_f = x_i + v_i t + at^2 / 2$ & $v_f$ \\
$x_f = x_i + v_f t - at^2 / 2$ & $v_i$ \\
$x_f = x_i + (v_i + v_f) t / 2$ & $a$ \\
$v_f^2 = v_i^2 + 2 a \Delta x$ & $t$ \\
\bottomrule
\end{tabular}
\caption{Relação das equações para a cinemática unidimensional e a variável ausente em cada uma delas. \label{Tab:EqsCinematicasVarAusentes}}
\end{table}

%%%%%%%%%%%%%%%%%%%%%%%%%%%%%%%%%%%%
\subsection{Aceleração da gravidade}
%%%%%%%%%%%%%%%%%%%%%%%%%%%%%%%%%%%%

Quando um objeto cai livremente próximo da superfície da Terra\footnote{Esse valor não é o mesmo em todos os pontos da superfície da Terra, porém vamos utilizar \np[m/s^2]{9,8} como um valor aproximado para qualquer ponto.}, ele sofre uma aceleração para baixo com módulo \np[m/s^2]{9,8}. Essa aceleração é comum a todos os objetos, independentemente de suas massas, caso a \emph{força de arrasto}\footnote{Esta força é a resistência ao deslocamento em um meio fluido, como o ar, e será discutida em detalhes no Capítulo~\ref{Chap:Dinâmica}.} seja desprezível. A existência dessa aceleração se deve à força fundamental da natureza denominada \emph{força gravitacional}, responsável pela atração entre corpos como um objeto qualquer e a Terra, a Terra e a Lua, ou o Sol e a Terra. Veremos adiante que essa força tem uma dependência direta na massa dos corpos, o que resulta na independência da aceleração gravitacional em relação à massa do corpo que é atraído pela Terra.

A aceleração da gravidade próximo da superfície da Terra é a principal justificativa para o estudo de movimentos com aceleração constante. Em geral, não há razões para supor que um objeto qualquer (um veículo, por exemplo) esteja sujeito a uma aceleração constante, exceto no caso em que ele esteja sujeito à aceleração gravitacional. Além do módulo da aceleração, devemos destacar sua direção -- vertical -- e seu sentido -- para baixo --. Vamos nos ater, por enquanto, ao caso de movimentos de queda livre e de lançamentos verticais, isto é, movimentos que ocorrem somente na vertical.

%%%%%%%%%%%%%%%%%%%%%%
\section{Questionário}
%%%%%%%%%%%%%%%%%%%%%%

\begin{question}[type={exam}]
Uma questão de cinemática.
\end{question}

